\documentclass[a4paper,12pt]{article}
\usepackage[top=2.5cm,bottom=2.5cm,left=2.5cm,right=2.5cm]{geometry}
%% \usepackage[brazil,brazilian]{babel}
\usepackage{amsmath, nccmath}
\usepackage{amsfonts}
\usepackage{amssymb}
\usepackage{bm}
\usepackage{multirow}
\usepackage{natbib}
\usepackage[colorlinks,citecolor=blue,urlcolor=blue]{hyperref}
\usepackage[utf8]{inputenc}
\usepackage{graphicx}  
\usepackage{float}
\usepackage{pdflscape} % horizontal page
\usepackage{tabularx}
\usepackage[english,vlined,ruled]{algorithm2e}
\usepackage{booktabs}
\usepackage{tikz}
\usetikzlibrary{positioning,shapes,arrows}
\usepackage{adjustbox}
\usepackage{blkarray}
\usepackage{tikz}
\usetikzlibrary{backgrounds}
\usetikzlibrary{arrows}
\usetikzlibrary{shapes,shapes.geometric,shapes.misc}

% this style is applied by default to any tikzpicture included via \tikzfig
\tikzstyle{tikzfig}=[baseline=-0.25em,scale=0.5]

% these are dummy properties used by TikZiT, but ignored by LaTex
\pgfkeys{/tikz/tikzit fill/.initial=0}
\pgfkeys{/tikz/tikzit draw/.initial=0}
\pgfkeys{/tikz/tikzit shape/.initial=0}
\pgfkeys{/tikz/tikzit category/.initial=0}

% standard layers used in .tikz files
\pgfdeclarelayer{edgelayer}
\pgfdeclarelayer{nodelayer}
\pgfsetlayers{background,edgelayer,nodelayer,main}

% style for blank nodes
\tikzstyle{none}=[inner sep=0mm]

% include a .tikz file
\newcommand{\tikzfig}[1]{%
{\tikzstyle{every picture}=[tikzfig]
\IfFileExists{#1.tikz}
  {\input{#1.tikz}}
  {%
    \IfFileExists{./figures/#1.tikz}
      {\input{./figures/#1.tikz}}
      {\tikz[baseline=-0.5em]{\node[draw=red,font=\color{red},fill=red!10!white] {\textit{#1}};}}%
  }}%
}

% the same as \tikzfig, but in a {center} environment
\newcommand{\ctikzfig}[1]{%
\begin{center}\rm
  \tikzfig{#1}
\end{center}}

% fix strange self-loops, which are PGF/TikZ default
\tikzstyle{every loop}=[]

\usepackage{setspace}
\onehalfspacing

\title{

  List of responses to the comments for the author of:
  A multinomial generalized linear mixed model for clustered competing
  risks data

}

\begin{document}
\maketitle

\section*{Associate Editor}

The manuscript has been read by two independent reviewers and myself. It
proposes a GLMM approach to handle clustered competing risks with a
within-cluster dependence structure.

As you can see below, the manuscript has been positively evaluated, but
there are still some issues that must be addressed. In particular, the
authors should specify the differences with the existing literature and,
especially, with the work by Cederkivst et.al. (2019) that seems to have
proposed a very similar model.

\subsection*{Author's response}

We are very grateful for the positive evaluation and clarified in the
paper the differences between our work and \cite{SCHEIKE}, these
clarifications were made in the introduction and discussion sections.

In summary, the differences are: we follow \cite{SCHEIKE} and work with
the Cumulative Incidence Function (CIF) specification proposed by
them. However, there, to be able to model and infer the CIF parameters
and within-cluster dependence, an elaborated composite likelihood
approach was proposed together with an Adaptative Gaussian Quadrature
(AGQ) scheme to marginalize the composite likelihood. Extra steps were
needed to take to be able to quantify the uncertainty around the
parameters. We, on the other side, modeled the same CIF specification
via a much simpler and known formulation - a generalized linear mixed
model (GLMM) formulation. Besides the simpler formulation (which also
provides a full and proper likelihood function) we used a Laplace
approximation scheme, simpler, in contrast to the used AGQ. To make this
much simpler framework feasible, we took advantage of state-of-the-art
numerical algorithms implementations and computational
libraries. Basically, is the same CIF specification but with different
likelihood functions/formulations and parameter estimation
routines/schemes.

\section*{Reviewer #1}

This is an interesting topic, and the authors reports an extended
simulation study while proposing an estimation methods for competing
risks with clustered data following the proposal by Cedervisk et al
(2019). The author should specify with more detail differences, if they
are present, between their proposal and the original one not only from
the computational side: they declare to propose a new model but it seems
the one proposed by Cedervisk et al.

I suggest adding some short motivations could add applicative values,
examples on the usage of this class of models that could help readers to
figure out in a clearer way the possible applications.

\subsection*{Author's response}

We are very grateful for the insightful comments of the reviewer. We use
the same CIF specification of \cite{SCHEIKE} but with a different
likelihood formulation (a full likelihood approach instead of a
composite one) and parameter estimation routine (a Laplace approximation
instead an expensive AGQ), taking advantage of state-of-the-art
computational libraries and algorithm implementations. We stress out
these points in the paper, as recommended. In terms of motivational
applications, it was also added to the introduction.

\subsection*{Reviewer #1}

The simulations results underline some situation that can be
problematic. This is very interesting because in the original proposal
by Cedervisk et al only a real data set and a single simulation study
were considered, I suggest stressing this point and add some comparisons
with the previous proposal.  Especially for problematic situations, have
they tried to consider the original computational method proposed by
Cedervisk et al?  Can they give more insight for simulation cases where
the latent effects didn't work well?

\subsection*{Author's response}

{\color{red} We are very grateful for the insightful comments of the
  reviewer. We had access to \cite{SCHEIKE}'s code through
  GitHub. Running their code we've not been able to reproduce their
  results. By adapting it to perform some extra simulation studies like
  the ones that we did with our model, in the majority of the scenarios
  convergence has not been reached. Given that and the facts that 1) our
  likelihood functions, formulations, and estimation procedures are very
  different and not directly comparable; 2) our model runs in parallel
  and its running time is quite fast considering its complexity and
  sample size, we end up choosing not to make any statement about these
  differences in the results section.}

\subsection*{Reviewer #1}

Minor

Page 8 row 18, please specify better what is yijt
Page 3 row 52 and page 29 row 1 is Vaupel not Valpel

\subsection*{Author's response}

We are very grateful for the insightful comments of the reviewer. The
clarification and the correction were made.

\section*{Reviewer #2}

This paper proposes a multinomial mixed model (similar to a pattern
mixture model) for the cumulative incidence functions of clustered
multivariate competing risks data. Inference is based on the
(Laplace-approximated) maximum likelihood. The methods are rigorously
developed and could be useful in practice. I have a few minor comments
to help the authors improve the paper.

1. In simple language, what is the main difference of the proposed
approach with Cederkvist et al. (2019)?

\subsection*{Author's response}

We are very grateful for the insightful comments of the reviewer. We use
the same CIF specification of \cite{SCHEIKE} but with a different
likelihood formulation (a full likelihood approach instead of a
composite one) and parameter estimation routine (a Laplace approximation
instead an expensive AGQ), taking advantage of state-of-the-art
computational libraries and algorithm implementations. We stress out
these points in the paper, as recommended.

\subsection*{Reviewer #2}

2. Recently, Ahn et al. (2022) has proposed a robuts approach to
semiparametric regression of multivariate clustered competing risks
data. Can you comment on the similarities and differences with the
proposed method (a parametric one if I understand it correctly)?

\subsection*{Author's response}

We are very grateful for the insightful comments of the reviewer.

\subsection*{Reviewer #2}

3. The Laplace-approximated maximum likelihood looks similar to the EM
algorithm with the latent variables u_i treated as missing data. Can you
clarify the difference?

\subsection*{Author's response}

We are very grateful for the insightful comments of the reviewer.

\bibliographystyle{dcu} \bibliography{references}

\end{document}
