\documentclass[a4paper,12pt]{article}
\usepackage[top=2.5cm,bottom=2.5cm,left=2.5cm,right=2.5cm]{geometry}
%% \usepackage[brazil,brazilian]{babel}
\usepackage{amsmath, nccmath}
\usepackage{amsfonts}
\usepackage{amssymb}
\usepackage{bm}
\usepackage{multirow}
\usepackage{natbib}
\usepackage[colorlinks,citecolor=blue,urlcolor=blue]{hyperref}
\usepackage[utf8]{inputenc}
\usepackage{graphicx}  
\usepackage{float}
\usepackage{pdflscape} % horizontal page
\usepackage{tabularx}
\usepackage[english,vlined,ruled]{algorithm2e}
\usepackage{booktabs}
\usepackage{tikz}
\usetikzlibrary{positioning,shapes,arrows}
\usepackage{adjustbox}
\usepackage{blkarray}
\usepackage{tikz}
\usetikzlibrary{backgrounds}
\usetikzlibrary{arrows}
\usetikzlibrary{shapes,shapes.geometric,shapes.misc}

% this style is applied by default to any tikzpicture included via \tikzfig
\tikzstyle{tikzfig}=[baseline=-0.25em,scale=0.5]

% these are dummy properties used by TikZiT, but ignored by LaTex
\pgfkeys{/tikz/tikzit fill/.initial=0}
\pgfkeys{/tikz/tikzit draw/.initial=0}
\pgfkeys{/tikz/tikzit shape/.initial=0}
\pgfkeys{/tikz/tikzit category/.initial=0}

% standard layers used in .tikz files
\pgfdeclarelayer{edgelayer}
\pgfdeclarelayer{nodelayer}
\pgfsetlayers{background,edgelayer,nodelayer,main}

% style for blank nodes
\tikzstyle{none}=[inner sep=0mm]

% include a .tikz file
\newcommand{\tikzfig}[1]{%
{\tikzstyle{every picture}=[tikzfig]
\IfFileExists{#1.tikz}
  {\input{#1.tikz}}
  {%
    \IfFileExists{./figures/#1.tikz}
      {\input{./figures/#1.tikz}}
      {\tikz[baseline=-0.5em]{\node[draw=red,font=\color{red},fill=red!10!white] {\textit{#1}};}}%
  }}%
}

% the same as \tikzfig, but in a {center} environment
\newcommand{\ctikzfig}[1]{%
\begin{center}\rm
  \tikzfig{#1}
\end{center}}

% fix strange self-loops, which are PGF/TikZ default
\tikzstyle{every loop}=[]

\usepackage{setspace}
\onehalfspacing

\title{

  List of responses to the comments for the author of:
  A multinomial generalized linear mixed model for clustered competing
  risks data

}

\begin{document}
\maketitle

\section*{Co-Editor}

Based on the advice received, I have decided that your manuscript can be
accepted for publication after you have carried out the corrections as
suggested by the reviewer(s).

\subsection*{Author's response}

We thanks the positive evaluation and we addressed in the paper the
corrections and suggestions of the reviewers.

\section*{Reviewer #1}

The authors have positively answered to all the issues arisen.

\section*{Reviewer #2}

1. Please incorporate the comparison with He et al. (2022) discussed in
the author's response into the paper (Introduction/Discussion).

\subsection*{Author's response}

We thanks for the literature recommendation. The robust approach
proposed by \cite{ahnetal22} has been incorporated into the paper.

\subsection*{Reviewer #2}

2. The authors mentioned that the Laplace-approximated MLE converges
faster than the EM (which has a linear convergence rate), do we know at
what rate it converges, e.g., approximately quadratic?

\subsection*{Author's response}

We thanks fot the insightful comment. The Laplace approximation for the
latent effects of a mixed model consists of two optimizations, an inner
and an outer optimization. The inner one is made through a
Newton-Raphson algorithm, Newton´s method with a quadratic convergence
rate. The external optimization is made through a Quasi-Newton Method,
the BFGS for instance, which in our class of models has a superlinear
convergence rate.

\bibliographystyle{dcu} \bibliography{references}

\end{document}
