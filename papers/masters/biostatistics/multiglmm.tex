\documentclass[oupdraft]{bio}
\usepackage[colorlinks,urlcolor=blue,citecolor=blue]{hyperref}
\usepackage{url}
\usepackage{float}
\usepackage{tikz}
\usetikzlibrary{positioning,shapes,arrows}
\usepackage{adjustbox}
\usepackage{blkarray}
\usepackage{tikz}
\usetikzlibrary{backgrounds}
\usetikzlibrary{arrows}
\usetikzlibrary{shapes,shapes.geometric,shapes.misc}

% this style is applied by default to any tikzpicture included via \tikzfig
\tikzstyle{tikzfig}=[baseline=-0.25em,scale=0.5]

% these are dummy properties used by TikZiT, but ignored by LaTex
\pgfkeys{/tikz/tikzit fill/.initial=0}
\pgfkeys{/tikz/tikzit draw/.initial=0}
\pgfkeys{/tikz/tikzit shape/.initial=0}
\pgfkeys{/tikz/tikzit category/.initial=0}

% standard layers used in .tikz files
\pgfdeclarelayer{edgelayer}
\pgfdeclarelayer{nodelayer}
\pgfsetlayers{background,edgelayer,nodelayer,main}

% style for blank nodes
\tikzstyle{none}=[inner sep=0mm]

% include a .tikz file
\newcommand{\tikzfig}[1]{%
{\tikzstyle{every picture}=[tikzfig]
\IfFileExists{#1.tikz}
  {\input{#1.tikz}}
  {%
    \IfFileExists{./figures/#1.tikz}
      {\input{./figures/#1.tikz}}
      {\tikz[baseline=-0.5em]{\node[draw=red,font=\color{red},fill=red!10!white] {\textit{#1}};}}%
  }}%
}

% the same as \tikzfig, but in a {center} environment
\newcommand{\ctikzfig}[1]{%
\begin{center}\rm
  \tikzfig{#1}
\end{center}}

% fix strange self-loops, which are PGF/TikZ default
\tikzstyle{every loop}=[]


\history{Received March 18, 2022;
  revised ;
  accepted for publication }

\begin{document}

\title{
  A multinomial generalized linear mixed model for clustered competing
  risks data
}

\author{HENRIQUE APARECIDO LAUREANO$^\ast$\\
        \textit{Instituto de Pesquisa Pel\'{e} Pequeno Pr\'{i}ncipe,
                Curitiba, Brasil}\\
        RICARDO HASMUSSEN PETTERLE\\
        \textit{Departamento de Medicina Integrada,
                Universidade Federal do Paran\'{a}, Curitiba, Brasil}\\
        GUILHERME PARREIRA DA SILVA, PAULO JUSTINIANO RIBEIRO JUNIOR,\\
        WAGNER HUGO BONAT\\
        \textit{Laborat\'{o}rio de Estat\'{i}stica e
                Geoinforma\c{c}\~{a}o,
                Departamento de Estat\'{i}stica,
                Universidade Federal do Paran\'{a}, Curitiba, Brasil}\\
{henriqueaparecidolaureano@gmail.com}}

\markboth%
% First field is the short list of authors
{H. A. Laureano and others}
% Second field is the short title of the paper
{A multinomial GLMM for clustered competing risks data}

\maketitle

% Add a footnote for the corresponding author if one has been
% identified in the author list
\footnotetext{To whom correspondence should be addressed.}

\begin{abstract}
  {Clustered competing risks data are a complex failure time data
    scheme. Its main characteristics are the cluster structure, which
    implies a latent within-cluster dependence between its elements, and
    its multiple variables competing to be the one responsible for the
    occurrence of an event, the failure. To handle this kind of data, we
    propose a full likelihood approach, based on generalized linear
    mixed models instead the usual complex frailty model. We model the
    competing causes in the probability scale, in terms of the
    cumulative incidence function (CIF). A multinomial distribution is
    assumed for the competing causes and censorship, conditioned on the
    latent effects that are accommodated by a multivariate Gaussian
    distribution. The CIF is specified as the product of an
    instantaneous risk level function with a failure time trajectory
    level function. The estimation procedure is performed through the R
    package TMB (Template Model Builder), an \texttt{C++} based
    framework with efficient Laplace approximation and automatic
    differentiation routines. A large simulation study was performed,
    based on different latent structure formulations. The model fitting
    was challenging and our results indicated that a latent structure
    where both risk and failure time trajectory levels are correlated is
    required to reach reasonable estimation.}  {Cause-specific
    cumulative incidence function; Within-cluster dependence; Template
    Model Builder; Laplace approximation; Automatic differentiation.}
\end{abstract}

\section{Introduction}
\label{intro}

Competing risks data, and more generally failure time data, can be
modeled in two possible scales: the hazard and the probability scale,
with the former being the most popular. A competing risks process can be
seen as the multivariate extension of a failure time process, having
multiple causes competing to be the one responsible for the desired
event occurrence, properly, a failure. In \autoref{fig:crp} a visual aid
is provided considering \(m\) competing causes, where zero represents
the initial state.

\begin{figure}[H]
 \centering
 \scalebox{0.85}{
  \begin{tikzpicture}
   \begin{pgfonlayer}{nodelayer}
    \node [style=circle,draw=black,fill=white] (7) at (-7.5, 5.5) {1};
    \node [style=circle,draw=black,fill=white] (9) at (-7.5, 4.5) {2};
    \node [style=none] (11) at (-7.5, 3.5) {$\vdots$};
    \node [style=circle,draw=black,fill=white] (12) at (-7.5, 2.5) {$m$};
    \node [style=circle,draw=black,fill=white] (14) at (-12.25, 4) {0};
    \node [style=none] (29) at (-11.75, 4.25) {};
    \node [style=none] (30) at (-8, 5.5) {};
    \node [style=none] (31) at (-8, 4.5) {};
    \node [style=none] (32) at (-8, 2.5) {};
    \node [style=none] (36) at (-11.75, 4) {};
    \node [style=none] (37) at (-11.75, 3.75) {};
   \end{pgfonlayer}
   \begin{pgfonlayer}{edgelayer}
	\draw [style=1side] (37.center) to (32.center);
	\draw [style=1side] (36.center) to (31.center);
	\draw [style=1side] (29.center) to (30.center);
   \end{pgfonlayer}
  \end{tikzpicture}
 }
 \caption{Illustration of competing risks process.}
 \label{fig:crp}
\end{figure}

Failure time data is the branch of Statistics responsible to handle
random variables describing the time until the occurrence of an event, a
failure \citep{kalb&prentice,hougaard00}. The time until a failure is
called survival experience, and is the modeling object. To accommodate
the number of possible causes for a failure there is the competing risks
data scheme. More specifically, its clustered version with groups of
elements sharing some non-observed latent dependence structure.

When this framework is applied in real-world situations, we have to be
able to handle with the nonoccurrence of the desired event, by any of
the competing causes, for, let us say, \textit{logistic reasons}
(short-time study and outside scope causes are some examples). This,
generally noninformative, nonoccurrence of the event is called
censorship.

When the elements under study are organized in clusters (a family,
e.g.), it opens space to what is called \textit{family studies}. In
family studies, the goal is to accommodate the non-observed latent
dependence and try to understand the relationship between the family
elements. In other words, how the occurrence of an event in a subject
affects the survival experience for the same or similar event.

The survival experiences is usually modeled in the hazard (failure rate)
scale, and with the latent within-cluster dependence accommodation we
have what is called a frailty model
\citep{frailty78,frailty79,liang95,petersen98}. The use of frailty
models implies in complicated likelihood functions and inference
routines done via elaborated and slow EM algorithms
\citep{nielsen92,klein92} or inefficient MCMC schemes
\citep{hougaard00}. With multiple survival experiences, the general idea
is the same but with even more elaborated likelihoods
\citep{prentice78,therneau00} or mixture model approaches
\citep{larson85,kuk92}.

When in the hazard scale, the interpretations are in terms of hazard
rates. A less usual scale but with a more appealing interpretation is
the probability scale. For competing risks data, the work on the
probability scale is done by means of the cumulative incidence function
(CIF) \citep{andersen12}, with the main modeling approach being the
subdistribution \citep{fine&gray}.

For clustered competing risks data there are some available options but
with a lack of predominance. The options vary in terms of likelihood
specification, with its majority being designed for bivariate CIFs,
where increasing the CIF's dimension is a limitation. Some of the
existing options are (i) nonparametric approaches
\citep{cheng07,cheng09}; (ii) linear transformation models
\citep{fine99,gerds12}; (iii) semiparametric approaches based on
composite likelihoods \citep{shih,SCHEIKE}, estimating equations
\citep{crossoddsratioSCHEIKE,cheng&fine}, copulas
\citep{semiparametricSCHEIKE}, or mixtures \citep{naskar05,shi13}.

Besides the interpretation, by modeling the CIF it is possible to
specify complex within-cluster dependence structures. We follow
\cite{SCHEIKE} and work with a CIF specification based on its
decomposition in instantaneous risk and failure time trajectory
functions, with both being cluster-specifics and possible correlated. As
a modeling framework, we use a generalized linear mixed model (GLMM)
specification. Through a GLMM we have a straightforward full likelihood
specification, easy to virtually extend to any number of competing
causes, and capable to allow for complex CIF structures. To make the
estimation and inferential process the most efficient as possible we
take advantage of state-of-art computational libraries and efficiently
implemented routines under the \texttt{TMB} \citep{TMB} package of the R
\citep{R21} statistical software.

The class of generalized linear models (GLMs) \citep{GLM72} is probably
the most popular statistical modelling framework. Despite its
flexibility, the GLMs are not suitable for dependent data. For the
analysis of such data, \cite{laird82} proposed the random effects
regression models for longitudinal/repeated-measures data, and
\cite{breslow93} presented the GLMMs for the analysis of non-Gaussian
outcomes. In this framework, we can accommodate all competing causes of
failure and censorship under a multinomial probability distribution. The
latent within-cluster dependence is accommodated via a multivariate
normal distribution, and the cause-specific CIFs via the model's link
function.

The main goal of this work is to propose a GLMM approach to handle
clustered competing risks data with a flexible within-cluster dependence
structure. The model specification and the inferential routine are much
simpler than the usually used approaches, increasing its practical
relevance. The latent effects, the key complicator factor, are handled
out by means of an efficient Laplace approximation and automatic
differentiation routines. The main contributions of this article are:
(i) introducing the modeling of cause/cluster-specific CIFs of clustered
competing risks data into an efficient implementation of the GLMMs
framework; (ii) performing a extensive simulation study to check the
properties of the maximum likelihood estimator to learn the
cause-specific CIF forms and the feasibility of the within-cluster
dependence structure.; (iii) providing R code and \texttt{C++}
implementation for the used GLMMs.

%% The work is organized as follows. Section \ref{model} presents the
%% CIF specification and the multinomial GLMM. Section \ref{inference}
%% presents the estimation and inferential routines. Section
%% \ref{simulation} presents the performed simulation studies to check
%% the model viability. Finally, the main contributions of the article
%% are discussed in Section \ref{discussion}.

\bibliographystyle{biorefs}
\bibliography{references}

\end{document}
