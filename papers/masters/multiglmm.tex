\documentclass[a4paper,12pt]{article}
%% \usepackage[brazil,brazilian]{babel}
\usepackage{amsmath, nccmath}
\usepackage{amsfonts}
\usepackage{amssymb}
\usepackage{multirow}
\usepackage{natbib}
\usepackage[colorlinks,citecolor=blue,urlcolor=blue]{hyperref}
\usepackage[utf8]{inputenc}
\usepackage{graphicx}  
\usepackage{float}
\usepackage{pdflscape} % horizontal page
\usepackage{tabularx}
\usepackage[english,vlined,ruled]{algorithm2e}
\usepackage{booktabs}

\title{
  
  Modeling the cumulative incidence function of clustered competing
  risks data: a multinomial GLMM approach

}
\author{
  Henrique Aparecido Laureano
  \thanks{Laboratory of Statistics and Goeinformation,
    Departament of Statistics,
    Paran\'{a} Federal University, Curitiba, Brazil.
    E-mail: laureano@ufpr.br
  }~
  Wagner Hugo Bonat$^\ast$}
 
\begin{document}
\maketitle
\medskip

\begin{abstract}

  Clustered competing risks data is a special case of failure time
  data. Besides the cluster structure which implies a latent
  within-cluster dependence between its elements, this kind of data is
  characterized by 1) multiple causes/variables competing to be the one
  responsible for the occurrence of an event, a failure; and 2)
  censorship, when the event of interest happens or not for none of the
  competing causes, in the study period. To handle this type of data, we
  propose a generalized linear mixed model (GLMM) i.e., a latent-effects
  framework, instead of a usual survival model. In survival analysis,
  the modeling is usually done by means of the hazard rate, and the
  within-cluster dependence accommodation ends by generating a
  complicated likelihood function, sometimes intractable. We, on the
  other hand, model the clustered competing causes in the probability
  scale, in terms of the cumulative incidence function (CIF) of each
  competing cause. In our framework, we suppose a multinomial
  probability distribution for the competing causes and censorship,
  conditioned on the latent effects. The latent effects are accommodated
  via a multivariate Gaussian distribution and are modeled by the
  parameters of its covariance matrix. The probability distributions are
  connected via CIF, modeled here following \citeonline{} specification,
  based on its decomposition as the product of an instantaneous risk
  level function with a trajectory time level function. The latent
  effects are inserted in those level functions. To make the model
  parameters estimation the most efficient as possible, we use the
  template model builder (TMB) \cite{}. With this \texttt{R} \cite{}
  package, we have 1) the log-likelihood function written in
  \texttt{C++}; 2) access to efficient linear algebra libraries; 3)
  efficient Laplace approximation implementation for the latent-effects;
  and 4) an automatic differentiation (AD) routine, the state-of-the-art
  in derivatives computation. To check the estimability of our model a
  large simulation study is performed, based on different latent
  structure formulations, with the aim to verify which one is most
  adequate to real scenarios. The model presents to be of difficult
  estimation, with our results converging to a latent structure where
  the risk and trajectory time levels are correlated. In scenarios with
  high CIF the model exhibits the better results, but still with an
  excessive variance, showing that improvements are necessary.

\end{abstract}

\begin{flushleft}
 \textbf{Keywords}: 
 Clustered competing risks;
 Within-cluster dependence;
 Multinomial generalized linear mixed model (GLMM);
 TMB: Template Model Builder;
 Laplace approximation;
 Automatic differentiation (AD).                    
\end{flushleft}

\section{Introduction}

Regression models are the main statistical tool for investigating the
relationship between a response variable and a set of explanatory
variables. The class of generalized linear models~(GLMs)~\citep{GLM72}
is probably the most popular statistical modelling framework to deal
with Gaussian and non-Gaussian outcomes. Despite its flexibility, the
GLMs are not suitable for response variables with support limited to the
interval \((0, 1)\). In general, continuous bounded variables appear in
the form of rates, proportions, indexes and percentages and they can be
used in many research areas.

The analysis of bounded variables is generally performed by the
beta~\citep{} and simplex~\citep{} regression models. Besides that,
other regression models were proposed to analyze continuous bounded
variables on the interval \((0, 1)\). Some examples are the
unit-Weibull~\citep{}, Johnson \(S_{B}\)~\citep{}, Kumaraswamy~\citep{}
and unit gamma~\citep{} regression models. Additionally, using
second-moment assumptions~\cite{} developed a flexible class of
regression models to deal with continuous bounded variables on the
interval \([0, 1]\).

Although these models are useful in many applications, they are usually
limited to analyze independent data. In the case of longitudinal data,
it is essential that the regression model take into account the
longitudinal and/or grouped data structure. According to \cite{}
longitudinal data are repeated measures evaluated on the same subjects
over time, that are potentially correlated. Dependent data can also
arises in studies with block designs, spatial and multilevel
data~\citep{}. For the analysis of such data several methods have been
proposed over the last four decades.

\cite{} proposed the random effects regression models for longitudinal
data analysis. \cite{} presented the generalized linear mixed
models~(GLMMs) for the analysis of non-Gaussian outcomes. \cite{} and
\cite{} extended the GLMs for the analysis of longitudinal data using a
generalized estimating equation~(GEE) approach. \cite{gcmr} developed a
class of marginal models for modelling dependence structures in the
analysis of longitudinal data, time series and spatial based on Gaussian
copula models.

Based on the aforementioned approaches, some regression models have been
proposed to deal with longitudinal continuous bounded outcomes. GLMMs
based on beta distribution were employed in medical research~\citep{},
social sciences~\citep{} and behavioral studies~\citep{}. Other
regression models based on the simplex distribution were proposed for
modelling longitudinal data~\citep{}. Under the likelihood paradigm, the
simplex mixed models with applications is discussed in~\cite{}.

The main goal of this study is to propose the unit gamma mixed model to
deal with longitudinal continuous bounded outcomes. The unit gamma
distribution is new in the literature and has been explored in other
contexts, like control charts~\citep{}, comparison between different
methods for parameter estimation~\citep{} and likelihood ratio
tests~\citep{}. In this paper, we will investigate the unit gamma
distribution as an alternative to beta distributions for the analysis of
dependent data bounded on the interval \((0, 1)\). We considered this
distribution into the GLMM framework in order to fit regression models
with random effects. We use automatic differentiation~\citep{} and
Laplace approximation~\citep{tierney} for efficient estimation of the
proposed model through the \texttt{R}~\citep{R21} package
\texttt{TMB}~\citep{TMb}.

The main contributions of this article are: (i) introducing the unit
gamma distribution into the GLMMs framework; (ii) performing a extensive
simulation study to check the properties of the the maximum likelihood
estimator to deal with longitudinal continuous bounded outcomes; (iii)
applying the proposed model in two data sets from different fields of
application; (iv) providing \texttt{R} code and \texttt{C++}
implementation for the unit gamma mixed models.

The work are organized as follows. Section~\ref{} presents the unit
gamma mixed models. Section~\ref{} describes the method proposed for
parameter estimation and inference. The results of simulation studies
are reported in Section~\ref{}. Section~\ref{} ilustrates the
application of the model in two data sets. Finally, the main
contributions of the article are discussed in Section~\ref{}.

\section{Unit gamma mixed models}
\label{mixedmodel}

\section{Estimation and inference}
\label{inference}

\section{Simulation studies}
\label{simulation}

\begin{figure}[H]
 \centering
 \includegraphics[width=.99\textwidth]{}
 \caption{Fitted values by quarters, locations, random intercept.}
 \label{fig:FittedIQA}
\end{figure}

\section{Discussion}
\label{discussion}

\section*{Supplementary material}

\bibliographystyle{dcu}
\bibliography{references}

\end{document}
