Consider a cluster of random variables. Each random variable represents
the time until some event occurs. The random variables that compose the
cluster are assumed to be correlated, i.e., the method for the analysis
is flexible enough to be able to verify if this happens to that data. In
this thesis, the cluster is a family - more precisely, a part of a
family, i.e., a pair of twins; the random variables are the time until
the occurrence (or not) of an event in each twin; and the event under
focus is the occurrence of cancer.

When we deal with random variables, in the context of a statistical
model - a response of interest; that represents the time until some
event occurs, such events are generically referred to as
\textit{failures}. From this, we get the name of the field of study:
failure time data~\cite{kalb&prentice}. These events may not necessarily
consist of a failure, however, the major areas of application of the
methods that will be discussed here are biomedical studies and
industrial life testing. Thus, this name sounds appropriate.

Independent of the area of application the methods are the same, but the
name of what you're doing is different. In industrial life testing
applications, you perform what is called a reliability analysis; in
biomedical studies, you perform what is called a survival analysis. In
this thesis, we'll maintain our focus on the latter.

Generally speaking, the term survival analysis is applied when we deal
with a univariate set, i.e., we have one response variable. As an
example,

When we want to study the possible relation between the components (seen
as random variables) of a dataset in an associative manner, the perhaps
most common scientific way of doing that is by fitting a linear model.
More generally, a generalized linear model (GLM). To fit a GLM to a
dataset we basically need to select a probability distribution for the
so-called response variable \(Y_{i}\) (or dependent variable); and how
we'll approach the possible relation between the response and the other
variables \(\mathbf{X}_{i}\) (independent variables or covariates).

Generalized linear models~\cite{GLM72} allow for response distribution
other than normal, which configures the so-called linear models. In a
GLM we model the mean, \(\mu_{i}\), of the response random variable, and
it has the basic structure
\[
 g(\mu_{i}) = \mathbf{X}_{i} \bm{\beta},
\]
where \(\mu_{i} \equiv \mathbb{E}(Y_{i})\), \(g\) is a smooth monotonic
``link function'', \(\mathbf{X}_{i}\) is the \(i^\text{th}\) row of a
model matrix \(\mathbf{X}\), and \(\bm{\beta}\) is a vector of unknown
parameters. In addition, a GLM usually makes the distributional
assumption that the \(Y_{i}\) are independent and
\[
  Y_{i} \sim \text{some exponential family distribution}.
\]

The \textit{exponential family} of distributions includes many
distributions that are useful for practical modelling, such as the
Poisson (for counting data), binomial (dichotomic data), gamma
(continuous but positive) and normal (continuous data) distributions.
The comprehensive reference for GLMs is~\citeonline{GLM89}.

However broad it may be the range of models that can be constructed
thanks to the generality of the GLM framework, is plausible that for
some applications a specific modeling framework come up. This is the
case when the response variable is the time until some event occurs.
Such events are generically referred to as \textit{failures}, and its
major areas of use are biomedical studies and industrial life testing.
In the latter the commonly used name is reliability and in the former is
survival. In this thesis, we focus on the survival side.

A survival model consists

\section{GOALS}

\subsection{General goals}

Propor um modelo de regressão para análise de variáveis respostas
limitadas multivariada.

\subsection{Specific goals}

\begin{enumerate}
\item Estudar o desempenho do algoritmo NORTA (\emph{NORmal To
    Anything}) para simular variáveis aleatórias beta correlacionadas.

\item Especificar o modelo usando suposições de primeiro e segundo
  momentos.

\item Usar as funções de estimação quase-score e Pearson para estimar os
  parâmetros de regressão e dispersão, respectivamente.

\item Delinear estudos de simulação para explorar a flexibilidade do
  modelo para lidar com dados limitados em estudos longitudinais, além
  de checar propriedades dos estimadores em estudos com múltiplas
  respostas correlacionadas.

\item Adaptar técnicas de diagnóstico para o modelo proposto, como
  DFFITS, DFBETAS, distância de Cook e o gráfico de probabilidade
  meio-normal com envelope simulado.

\item Aplicar o modelo proposto em dois conjuntos de dados.
\end{enumerate}

\section{JUSTIFICATION}

\section{LIMITATION}

Este trabalho se restringe a propor um novo modelo de regressão para
análise de variáveis respostas limitadas multivariada. Para motivar o
novo modelo, serão apresentadas aplicações em dois conjuntos de dados,
que não são facilmente manipulados pelos métodos estatísticos
existentes. Portanto, testes de hipóteses e de comparações múltiplas
multivariados não serão desenvolvidos no decorrer deste trabalho.

\section{THESIS ORGANIZATION}

Esta dissertação contém seis capítulos incluindo esta introdução.
O~\autoref{cap:aplicacoes} descreve os dois conjuntos de dados que serão
usados como exemplos de aplicação no novo modelo.
O~\autoref{cap:fundamentacaoteorica} apresenta a revisão bibliográfica
que motivou este trabalho, introduz o modelo de regressão beta
(univariado), apresenta o algoritmo NORTA (\textit{NORmal To Anything})
usado nos estudos de simulação e discute brevemente as medidas de
bondade de ajuste usadas no trabalho. O~\autoref{cap:multivariatemodel}
propõe o modelo de regressão quase-beta multivariado, apresenta o método
usado para estimação e inferência e adapta técnicas de diagnóstico.
No~\autoref{cap:resultados} são apresentados os resultados de três
estudos de simulação, além da análise dos dados apresentados
no~\autoref{cap:aplicacoes}. Finalmente, o~\autoref{cap:considefinais}
discute as principais contribuições desta dissertação, além de
apresentar as conclusões seguidas por sugestões para futuros trabalhos.

% END ==================================================================