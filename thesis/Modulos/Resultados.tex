% ======================================================================

Neste Capítulo são apresentados os resultados de três estudos de
simulação, além da análise dos dados apresentados no
\autoref{cap:aplicacoes}. O primeiro estudo de simulação foi conduzido
para investigar o comportamento do algoritmo NORTA (NOR\textit{mal To
  Anything}) na simulação de variáveis aleatórias beta correlacionadas
(\autoref{cap:simul1}). O segundo visou checar propriedades dos
estimadores para os parâmetros de dispersão, no contexto de análise de
dados longitudinais (\autoref{cap:simulLong}). E o terceiro foi
delineado para explorar a flexibilidade dos estimadores para lidar com
múltiplas respostas correlacionadas (\autoref{cap:simul2}). Por fim, a
\autoref{cap:resultIQA} apresenta os resultados da análise dos dados
referente ao índice de qualidade da água (IQA), enquanto a
\autoref{cap:resultcorporal} apresenta os resultados correpondentes ao
percentual de gordura corporal.

\section{ESTUDOS DE SIMULAÇÃO}

\subsection{Comportamento do algoritmo NORTA}
\label{cap:simul1}

\begin{figure}[H]
  \vspace{0.4cm}
  \caption{VALORES MÍNIMOS E MÁXIMOS PARA A CORRELAÇÃO ENTRE DUAS
    VARIÁVEIS ALEATÓRIAS BETA EM FUNÇÃO DAS MÉDIAS MARGINAIS E
    DIFERENTES VALORES DO PARÂMETRO $\phi$}
  \setlength{\abovecaptionskip}{.0001pt}
  \vspace{-0.38cm}
  \includegraphics[width=16.0cm,height=7.6cm]{Figure10.pdf}
  \vspace{-0.7cm}
  \begin{footnotesize}
    \centering
    FONTE: O autor~(2018).
    \vspace{0.15cm}
  \end{footnotesize}
  \label{fig:simulnorta}
\end{figure}

\begin{equation}
  \label{eq:linearPredIQA}
  g(\mu_{jki}) = \beta_0  + \beta_{1j}~\texttt{local}_{ji} + \beta_{2k}~\texttt{trimestre}_{ki},
\end{equation}
\noindent
Na sequência, ajustou-se o modelo de regressão quase-beta multivariado
aos dados do IQA, considerando as quatro estruturas acima mencionadas
além de especificar a função de ligação \textit{logit} para o preditor
linear~(\autoref{eq:linearPredIQA}).

% END ==================================================================