\begin{frame}{Roteiro}

\tableofcontents

\end{frame}

\section{Contextualizando}\label{contextualizando}

\subsection{Representações por
estados}\label{representacoes-por-estados}

\begin{frame}

\begin{itemize}
 \pause \item
  \begin{columns}
   \begin{column}{.4\textwidth}
    \begin{center}
     Representação usual de um dado de sobrevivência:
    \end{center}
   \end{column}
   \begin{column}{.6\textwidth}
    \begin{center}
     \includegraphics*[height = .75cm, width = 6cm]{surv.png}
    \end{center}
   \end{column}
  \end{columns}
\end{itemize}

\vspace{.5cm}

\pause \textbf{Abordagens multiestados:}

\begin{itemize}
 \pause \item
  \begin{columns}
   \begin{column}{.3\textwidth}
    \begin{center}
     Riscos competitivos:
    \end{center}
   \end{column}
   \begin{column}{.7\textwidth}
    \begin{center}
     \includegraphics*[height = 2cm, width = 7cm]{rc.png}
    \end{center}
   \end{column}
  \end{columns}
\vspace{.25cm}
 \pause \item
  \begin{columns}
   \begin{column}{.2\textwidth}
    \begin{center}
     Multiestados:
    \end{center}
   \end{column}
   \begin{column}{.8\textwidth}
    \begin{center}
     \includegraphics*[height = 2.25cm, width = 8cm]{multi.png}
    \end{center}
   \end{column}
  \end{columns}
\end{itemize}

\end{frame}

\section{Conceitos e definições}\label{conceitos-e-definicoes}

\subsection{Modelos multiestados de sobrevivência
markovianos}\label{modelos-multiestados-de-sobrevivencia-markovianos}

\begin{frame}

\begin{block}{Modelo usual}
 \pause \[q_{rs}(\mathbf{Z}) =
          q_{rs}^{0}{\rm exp}(\bm{\beta}^{\top}\mathbf{Z})\]
 \begin{itemize}
  \pause \item \(\mathbf{Z}\) é um vetor de covariáveis em que é assumido
               efeito comum a todas as transições
\vspace{.3cm}
  \pause \item \(q_{rs}^{0}\) é a intensidade de transição ou taxa de falha
               de base para a transição do estado \(r\) para o estado \(s\),
 \end{itemize}
 \vspace{.3cm} \pause
 \[q_{rs}(t) = \lim_{\delta t \to 0}
               \frac{P(X(t + \delta t) = s | X(t) = r)}{\delta t}\]
\end{block}

\end{frame}

\begin{frame}

\begin{block}{Um modelo multiestado pode ser:}
\vspace{-.15cm}
 \begin{multicols}{3}
 \small
  \begin{itemize}
   \pause \item Não markoviano
   \pause \item Semimarkoviano
   \pause \item \textbf{Markoviano}
  \end{itemize}
 \end{multicols}
\end{block}

\vspace{.3cm}

\pause

\begin{block}{Pressuposto markoviano:}
 \pause Uma futura transição depende apenas do estado atual
\end{block}

\vspace{.3cm}

\pause

\begin{block}{Um modelo multiestado markoviano pode ser de três tipos:}
\vspace{-.15cm}
 \begin{multicols}{3}
 \small
  \begin{itemize}
   \pause \item Paramétrico
   \pause \item Não paramétrico
   \pause \item Semiparamétrico
  \end{itemize}
 \end{multicols}
\end{block}

\end{frame}

\begin{frame}

\begin{block}{Modelo paramétrico (\textit{package} msm do R)}
 \begin{itemize}
  \pause \item Distribuição de probabilidade assumida para o tempo médio de
               permanência em cada estado transiente \(r\), comumente
               exponencial
 \end{itemize}
\pause \textbf{Dois tipos:} \\
\vspace{.25cm}
\pause \textit{Tempo homogêneo}:
       intensidades de transição constantes ao longo \\
       \hskip 3.3cm do tempo (independentes de \(t\)) \\
\vspace{.25cm}
\pause \textit{Tempo não homogêneo}:
       intensidades de transição variáveis ao longo \\
       \hskip 3.9cm do tempo, constantes sob segmentos
\end{block}

\end{frame}

\begin{frame}

\begin{block}{Modelo (não e) semiparamétrico (\textit{package} mstate do R)}
 \begin{itemize}
  \pause \item Modelo de Cox estratificado por transição
  \vspace{.15cm}
  \pause \item Na ausência de covariáveis temos um modelo não paramétrico
 \end{itemize}
\vspace{.2cm} \pause Além do modelo usual, permite a especificação de
                     modelos mais elaborados com:
 \begin{itemize}
  \vspace{.2cm}
  \pause \item Diferentes efeitos das covariáveis em cada transição
  \vspace{.15cm}
  \pause \item Intensidades de transição proporcionais
  \vspace{.15cm}
  \pause \item Covariáveis que aparecem apenas em algumas transições
 \end{itemize}
\end{block}

\end{frame}

\begin{frame}

\begin{block}{Inferências}
 \begin{itemize}
  \pause \item Probabilidades de transição e de sobrevivência
\vspace{.15cm}
  \pause \item Tempos médios esperados de permanência em estados e para
               transição entre estados
 \end{itemize}
\end{block}\vspace{.2cm}

\pause

\begin{block}{Qualidade do ajuste}
 \pause \textit{Modelo paramétrico}: métodos formais e informais \\
\vspace{.15cm}
 \pause \textit{Modelo (não e) semiparamétrico}: verificação usual
 \begin{itemize}
  \pause \item Análise gráfica de resíduos e verificação da suposição de
               taxas de falha proporcionais (na presença de covariáveis)
 \end{itemize}
\end{block}

\end{frame}

\section{Aplicações}\label{aplicacoes}

\subsection{MASS II}\label{mass-ii}

\begin{frame}

\scriptsize

\begin{block}{MASS II}
 Pacientes com doença arterial coronariana multiarterial, angina estável e 
 função ventricular preservada
\end{block}

\pause \vspace{-.65cm}

\begin{center}
 \includegraphics*[width = 11.25cm]{mass.png}
\end{center}

\end{frame}

\begin{frame}

\vspace{-.2cm}\begin{center}
 \includegraphics*[height = 7.5cm, width = 8cm]{mass_times.png}
\end{center}

\end{frame}

\begin{frame}

\begin{block}{Modelo multiestado markoviano paramétrico}
\small \pause As covariáveis grupo de risco e histórico de IAM não são
              significativas
\end{block}

\vspace{.5cm}

\pause

\begin{center}
 \includegraphics*[width = 10.75cm]{times_msm_mass.png}
\end{center}

\end{frame}

\begin{frame}

\begin{block}{Modelo multiestado markoviano paramétrico}
\small As covariáveis grupo de risco e histórico de IAM não são 
       significativas
\end{block}

\begin{center}
 \includegraphics*[width = 10.75cm]{probs_mass_msm.png}
\end{center}

\end{frame}

\begin{frame}

\vspace{-.2cm}\begin{center}
 \includegraphics*[height = 7cm]{mass_msm.png}
\end{center}

\vspace{-.6cm}

\tiny

\begin{multicols}{4}
 \begin{itemize}
  \item 1 - TM
  \item 2 - ICP
  \item 3 - CRM
  \item 4 - CRM
  \item 5 - ICP
  \item 6 - IAM
  \item 7 - AVC
 \end{itemize}
\end{multicols}

\end{frame}

\begin{frame}

\begin{block}{Modelo multiestado markoviano (não e) semiparamétrico}
\small As covariáveis grupo de risco e histórico de IAM não são 
       significativas
\end{block}

\vspace{.5cm}\begin{center}
 \includegraphics*[height=3.5cm, width=10.75cm]{times_mass_mstate.png}
\end{center}

\end{frame}

\begin{frame}

\begin{block}{Modelo multiestado markoviano (não e) semiparamétrico}
 \small As covariáveis grupo de risco e histórico de IAM não são 
        significativas
\end{block}

\vspace{-.9cm}\begin{columns}
 \begin{column}{.75\textwidth}
  \begin{center}
   \includegraphics*[height = 6.2cm]{mass_probtrans_mstate.png}
  \end{center}
 \end{column}
 \scriptsize
 \begin{column}{.25\textwidth}
  \begin{center}
   \begin{itemize}
    \item 1 - TM
    \item 2 - ICP
    \item 4 - CRM
    \item 5 - ICP
    \item 6 - IAM
    \item 7 - AVC
    \item 9 - MORTE
   \end{itemize}
  \end{center}
 \end{column}
\end{columns}

\end{frame}

\begin{frame}

\vspace{-.35cm}\begin{columns}
 \begin{column}{.9\textwidth}
  \begin{center}
   \includegraphics*[height = 7.5cm]{mass_mstate.png}
  \end{center}
 \end{column}
 \tiny
 \begin{column}{.25\textwidth}
  \begin{center}
   \begin{itemize}
    \item 1 - TM
    \item 2 - ICP
    \item 3 - CRM
    \item 4 - CRM
    \item 5 - ICP
    \item 6 - IAM
    \item 7 - AVC
    \item 8 - ANGINA
    \item 9 - MORTE
   \end{itemize}
  \end{center}
 \end{column}
\end{columns}

\end{frame}

\subsection{Inoculação em frutos}\label{inoculacao-em-frutos}

\begin{frame}

\pause

\begin{block}{Objetivo}
 Verificar possíveis diferenças entre gêneros de \textit{Colletotrichum}
 em relação ao tempo com que a lesão progride nos frutos de maçã, e se
 existe diferença entre frutos com e sem ferimento
\end{block}

\vspace{.5cm}

\pause \textit{Colletotrichum?}

\vspace{.5cm}

\pause

\begin{itemize}
 \item O fungo Colletotrichum é o principal causador da doença Mancha
       Foliar de Glomerella (MFG), muito severa em pomares de macieira
       do estado do Paraná
\end{itemize}

\end{frame}

\begin{frame}

Representação dos estados

\vspace{.3cm}\begin{center}
 \includegraphics*[width = 11cm]{ic.png}
\end{center}

\end{frame}

\begin{frame}

\begin{block}{Modelo multiestado markoviano paramétrico}
 \begin{itemize}
  \item Diferença significativa entre frutos com e sem ferimento
  \item Sem diferença significativa entre os gêneros de
        \textit{Colletotrichum}
 \end{itemize}
\end{block}

\vspace{-.7cm}

\pause

\begin{center}
 \includegraphics*[height = 5cm, width = 11cm]{ic_msm.png}
\end{center}

\end{frame}

\begin{frame}

\begin{block}{Modelo multiestado markoviano paramétrico}
 \begin{itemize}
  \item Diferença significativa entre frutos com e sem ferimento
  \item Sem diferença significativa entre os gêneros de
        \textit{Colletotrichum}
 \end{itemize}
\end{block}

\begin{center}
 \includegraphics*[width = 10.75cm]{ic-table_msm.png}
\end{center}

\end{frame}

\begin{frame}

\begin{block}{Modelo multiestado markoviano (não e) semiparamétrico}
 \begin{itemize}
  \item Diferença significativa entre frutos com e sem ferimento
  \item Sem diferença significativa entre os gêneros de
        \textit{Colletotrichum}
 \end{itemize}
\end{block}

\pause

\begin{center}
 \includegraphics*[height = 3.5cm, width = 10.75cm]{ic-table_mstate.png}
\end{center}

\end{frame}

\begin{frame}

\begin{block}{Modelo multiestado markoviano (não e) semiparamétrico}
 \begin{itemize}
  \item Diferença significativa entre frutos com e sem ferimento
  \item Sem diferença significativa entre os gêneros de
        \textit{Colletotrichum}
 \end{itemize}
\end{block}

\vspace{-.7cm}\begin{center}
 \includegraphics*[height = 5cm, width = 11cm]{ic_mstate.png}
\end{center}

\end{frame}

\section{Considerações finais}\label{consideracoes-finais}

\begin{frame}

\begin{block}{Considerações finais}
 \begin{itemize}
  \pause \item Ambos os modelos geraram inferências muito similares
  \vspace{.2cm}
  \pause \item Modelo (não e) semiparamétrico se mostrou mais robusto
  \vspace{.2cm}
  \pause \item Ambos os modelos se mostraram altamente dependentes do
               tamanho amostral \\
  \vspace{.3cm}
  \pause \hskip 1cm \(\bigcirc\) Grande amostra \vspace{.5cm} \\
         \hskip 1cm \(\bigotimes\) Grande amostra em cada transição
 \end{itemize}
\end{block}

\end{frame}

\begin{frame}

\LARGE \centering Obrigado por seu tempo!

\end{frame}
