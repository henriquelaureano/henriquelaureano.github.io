% Este Capítulo descreve os dois conjuntos de dados que serão usados como
% exemplos de aplicação no novo modelo de regressão, proposto no. O
% primeiro conjunto se refere ao índice de qualidade da água de
% reservatórios de usinas hidrelétricas operadas pela COPEL no Estado do
% Paraná. Já o segundo conjunto de dados corresponde ao percentual de
% gordura corporal de indivíduos avaliados no Hospital de Clínicas da
% Universidade Federal do Paraná.

% \section{CONJUNTO DE DADOS I: ÍNDICE DE QUALIDADE DA ÁGUA}
% \label{cap:IQA}

% \begin{table}[H]
%   \centering
%   \setlength{\abovecaptionskip}{.0001pt}
%   \caption{ANÁLISE DESCRITIVA PARA O IQA POR TRIMESTRE E LOCAL}
%   \label{tab:descIQA}
%   \begin{tabular}{cccc}
%     \hline
%     \multirow{2}{*}{Trimestre} & \multicolumn{3}{c}{Local} \\
%     \cline{2-4}  & Montante & Reservatório & Jusante \\
%     \cline{2-4} 1   & $0,75\pm 0,11$   &  $0,80\pm 0,10$  &  $0,78\pm 0,10$  \\
%     2  &  $0,79\pm 0,10$  &  $0,83\pm 0,06$   &  $0,83\pm 0,07$     \\
%     3   &  $0,81\pm 0,07$   & $0,85\pm 0,05$   &  $0,83\pm 0,06$    \\
%     4   & $0,76\pm 0,10$    &  $0,81\pm 0,08$    &  $0,79\pm 0,09$    \\
%     \hline
%   \end{tabular}
%   \begin{footnotesize}
%     \vspace{0.05cm}
%     FONTE: O autor~(2018). \hspace{6.2cm}
%     \vspace{-0.15cm}
%   \end{footnotesize}
% \end{table}

% END ==================================================================