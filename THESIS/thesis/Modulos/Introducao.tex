% ======================================================================

Em diversas áreas de pesquisa é comum investigar a relação entre uma
variável de interesse com outras variáveis que compõem o estudo. Para
tanto, faz-se uso da técnica estatística de modelos de regressão, uma
vez que se pode estudar o relacionamento entre uma variável
resposta~(variável dependente) com possíveis variáveis
explicativas~(covariáveis)~\cite{montgomery2012introduction}. A
aplicação desta técnica estatística é ampla, abrangendo diversas áreas
do conhecimento como medicina, engenharias, agronomia, ciências sociais
dentre outras. Nesse contexto, um dos principais modelos de regressão e
sem dúvida um dos mais usados por usuários de estatística aplicada é o
clássico modelo de regressão linear (Gaussiano). No entanto, para uso
desse modelo alguns pressupostos devem ser atendidos, tais como erros
independentes e identicamente distribuídos segundo a distribuição normal
com média zero e variância constante~\cite{draper2014applied}. Na
prática, isso nem sempre acontece e a má especificação desse modelo pode
gerar erros padrões inconsistentes, além de outros problemas que
invalidam todo o processo de
inferência~\cite{myersmontgomeryvining,montgomery2012introduction}.
Apesar de amplamente utilizado, o modelo de regressão linear não é
adequado para respostas binárias, politômicas, contagens ou limitadas.

% ----------------------------------------------------------------------
\section{OBJETIVOS}
% ----------------------------------------------------------------------

% ----------------------------------------------------------------------
\subsection{Objetivo geral}
% ----------------------------------------------------------------------

Propor um modelo de regressão para análise de variáveis respostas
limitadas multivariada.

% ----------------------------------------------------------------------
\subsection{Objetivos específicos}
% ----------------------------------------------------------------------

\begin{enumerate}
\item Estudar o desempenho do algoritmo NORTA (\emph{NORmal To
    Anything}) para simular variáveis aleatórias beta correlacionadas.

\item Especificar o modelo usando suposições de primeiro e segundo
  momentos.

\item Usar as funções de estimação quase-score e Pearson para estimar os
  parâmetros de regressão e dispersão, respectivamente.

\item Delinear estudos de simulação para explorar a flexibilidade do
  modelo para lidar com dados limitados em estudos longitudinais, além
  de checar propriedades dos estimadores em estudos com múltiplas
  respostas correlacionadas.

\item Adaptar técnicas de diagnóstico para o modelo proposto, como
  DFFITS, DFBETAS, distância de Cook e o gráfico de probabilidade
  meio-normal com envelope simulado.

\item Aplicar o modelo proposto em dois conjuntos de dados.
\end{enumerate}

% ----------------------------------------------------------------------
  \section{JUSTIFICATIVA}
% ----------------------------------------------------------------------

% ----------------------------------------------------------------------
\section{LIMITAÇÕES}
% ----------------------------------------------------------------------

Este trabalho se restringe a propor um novo modelo de regressão para
análise de variáveis respostas limitadas multivariada. Para motivar o
novo modelo, serão apresentadas aplicações em dois conjuntos de dados,
que não são facilmente manipulados pelos métodos estatísticos
existentes. Portanto, testes de hipóteses e de comparações múltiplas
multivariados não serão desenvolvidos no decorrer deste trabalho.

% ----------------------------------------------------------------------
\section{ORGANIZAÇÃO DO TRABALHO}
% ----------------------------------------------------------------------

Esta dissertação contém seis capítulos incluindo esta introdução.
O~\autoref{cap:aplicacoes} descreve os dois conjuntos de dados que serão
usados como exemplos de aplicação no novo modelo.
O~\autoref{cap:fundamentacaoteorica} apresenta a revisão bibliográfica
que motivou este trabalho, introduz o modelo de regressão beta
(univariado), apresenta o algoritmo NORTA (\textit{NORmal To Anything})
usado nos estudos de simulação e discute brevemente as medidas de
bondade de ajuste usadas no trabalho. O~\autoref{cap:multivariatemodel}
propõe o modelo de regressão quase-beta multivariado, apresenta o método
usado para estimação e inferência e adapta técnicas de diagnóstico.
No~\autoref{cap:resultados} são apresentados os resultados de três
estudos de simulação, além da análise dos dados apresentados
no~\autoref{cap:aplicacoes}. Finalmente, o~\autoref{cap:considefinais}
discute as principais contribuições desta dissertação, além de
apresentar as conclusões seguidas por sugestões para futuros trabalhos.

% END ==================================================================