\message{ !name(datasets.tex)}
\message{ !name(datasets.tex) !offset(-2) }
This chapter describes how to simulate from our proposed multiGLMM and,
a real-based dataset used as an application example. The simulation
procedure is not straightforward, since we need to simulate the failure
and censoring times, controlling for a pre-specified censorship level.
The application example is a simulated dataset based on the Nordic
Cancer Union (NCU) twins data.

\section{SIMULATING FROM THE MULTIGLMM}
\label{cap:simu}

Being able to simulate some data from a model is a key task, fundamental
to assess the finite-sample properties and the estimation procedure of a
given statistical model.

Simulate from our multiGLMM, as specified in \autoref{eq:model}, is
basically a three-step procedure. In the first step, we have to simulate
the failures and censorship probabilities. The second step consists in
simulate the outputs based on the probabilities, i.e. the subject fails
or censors. The final step is the failure/censorship time simulation.

\begin{algorithm}
\caption{Put your caption here}
\begin{algorithmic}[1]

\Procedure{Roy}{$a,b$}       \Comment{This is a test}
    \State System Initialization
    \State Read the value
    \If{$condition = True$}
        \State Do this
        \If{$Condition \geq 1$}
        \State Do that
        \ElsIf{$Condition \neq 5$}
        \State Do another
        \State Do that as well
        \Else
        \State Do otherwise
        \EndIf
    \EndIf

    \While{$something \not= 0$}  \Comment{put some comments here}
        \State $var1 \leftarrow var2$  \Comment{another comment}
        \State $var3 \leftarrow var4$
    \EndWhile  \label{roy's loop}
\EndProcedure

\end{algorithmic}
\end{algorithm}

\section{REAL-BASED DATASET}
\label{cap:data}

% \begin{table}[H]
%   \centering
%   \setlength{\abovecaptionskip}{.0001pt}
%   \caption{ANÁLISE DESCRITIVA PARA O IQA POR TRIMESTRE E LOCAL}
%   \label{tab:descIQA}
%   \begin{tabular}{cccc}
%     \hline
%     \multirow{2}{*}{Trimestre} & \multicolumn{3}{c}{Local} \\
%     \cline{2-4}  & Montante & Reservatório & Jusante \\
%     \cline{2-4} 1   & $0,75\pm 0,11$   &  $0,80\pm 0,10$  &  $0,78\pm 0,10$  \\
%     2  &  $0,79\pm 0,10$  &  $0,83\pm 0,06$   &  $0,83\pm 0,07$     \\
%     3   &  $0,81\pm 0,07$   & $0,85\pm 0,05$   &  $0,83\pm 0,06$    \\
%     4   & $0,76\pm 0,10$    &  $0,81\pm 0,08$    &  $0,79\pm 0,09$    \\
%     \hline
%   \end{tabular}
%   \begin{footnotesize}
%     \vspace{0.05cm}
%     FONTE: O autor~(2018). \hspace{6.2cm}
%     \vspace{-0.15cm}
%   \end{footnotesize}
% \end{table}

% END ==================================================================
\message{ !name(datasets.tex) !offset(-77) }
