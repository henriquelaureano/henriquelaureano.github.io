This chapter presents the simulation study results. We have seventy-two
simulation scenarios, as detailed in \autoref{cap:datasets}. For each
scenario we simulate 500 samples. In total, we fit 36000 models.

\section{SIMULATION STUDY}
\label{cap:simures}

Let us just recap the parameter values used
\begin{align*}
 \text{High CIF configuration}:~&\quad
 \{\beta_{1} = -2,~\beta_{2} = -1.5,~\gamma_{1} = 1,~\gamma_{2} = 1.5,~
   w_{1} = 3,~w_{2} = 4
 \};\\
 \text{Low CIF configuration}:~&\quad
 \{\beta_{1} = 3,~\beta_{2} = 2.5,~\gamma_{1} = 2.6,~\gamma_{2} = 4,~
   w_{1} = 5,~w_{2} = 10
 \}.
\end{align*}
\begin{minipage}{0.15\textwidth}
 \begin{align*}
  \sigma_{u_{1}}^{2}   &= 1\\
  \sigma_{u_{2}}^{2}   &= 0.7,\\
  \sigma_{\eta_{1}}^{2} &= 0.6\\
  \sigma_{\eta_{2}}^{2} &= 0.9
 \end{align*}
\end{minipage}%
\begin{minipage}{0.85\textwidth}
 \[
  \text{Correlation structure}~=~\begin{blockarray}{ccccc}
                                  u_{1} & u_{2} & \eta_{1} & \eta_{2}\\
                                  \begin{block}{(cccc)c}
                                   1 & 0.1 & -0.5 &  0.3 & u_{1}\\
                                     &   1 &  0.3 & -0.4 & u_{2}\\
                                     &     &    1 &  0.2 & \eta_{1}\\
                                     &     &      &    1 & \eta_{2}\\
                                  \end{block}
                                 \end{blockarray}.
 \]
\end{minipage}

\vspace{0.3cm}
\noindent
The parameter values per se are not important. What is important is to
keep in mind the behaviors implied by them, and see if the proposed
model is able to estimate the true values in several different scenarios
and measure the quality of the estimates.

The take-home message for the fixed-effect parameters, is to show that
we can construct different level CIF scenarios. The \(\bm{\beta}\)s are
responsible for the curve maximum point or plateau, being in the risk
level CIF component, the \(\bm{\gamma}\)s and \(\bm{w}\)s are
responsible for basically the curve shape, being in the failure time
trajectory level CIF component. Its interpretation is presented in
detail in \autoref{cap:model}. About the latent-effects, the chosen
covariance structure is considerably high but still acceptable. The
underlying idea was to try to build a realistic covariance scenario and
consequently be able to check how the model performs in such conditions.

In the following pages we have several graphs summarizing the estimates
bias. In each figure, we have the estimate bias and its uncertainty
described by a Wald-based confidence interval i.e., \(\pm\) 1.96 the
bias standard deviation. This is a good uncertainty representation
choice since it is symmetric. In the \autoref{cap:appendixD}, we have
the same estimates bias but with its uncertainty measure being the
corresponding 2.5 and 97.5\% bias quantiles. We chose to use these
uncertainty representations uniquely based on the point estimates
instead of the standard error computations. In several scenarios, the
model fails to compute all the standard errors, caused by Hessian
numerical instabilities.

In each of the following estimates bias graphs, the seventy-two
scenarios are accommodated. We have up to four blocks of bars, each
block representing a model. In each block we have eighteen bars, each
bar representing the 500 fits in each of the eighteen
scenarios, \(4 \times 18 \times 500 = 36000\).

Each scenario name consists of a combination of three strings
\begin{itemize}
 \item The cluster size (cs), 2, 5, and 10;
 \item The CIF configuration, high and low;
 \item The sample size, 5, 30, and 60 thousand.
\end{itemize}
We have tried to fit a total of 36000 models but not all converged. To
show these characteristic, we control the bar widths. Something specific
can be said about each parameter but let us keep the focus on the
general remarks. Starting from the fixed-effect parameters
in \autoref{fig:biassdbeta1}, \autoref{fig:biassdbeta2},
\autoref{fig:biassdgama1}, \autoref{fig:biassdgama2},
\autoref{fig:biassdw1}, and \autoref{fig:biassdw2}, we have very nice
results that already show a strong inclination towards the complete
model's choice.

With a latent structure only in the risk level or in the failure time
trajectory level, the low CIF scenarios are the ones with a much smaller
bias-variance. In general, the mean-bias is small but the variances are
high. When we have a latent structure on both levels but we still assume
the cross-correlations as zero (block-diag model), the results get a
little bit better. Nevertheless, when we assume a non-zero
cross-correlation structure (complete model), basically everything
changes for the better. The mean biases get even closer to zero, the
standard deviations decrease 50\% or more, and mainly, now the high CIF
scenarios are the ones with a much smaller bias-variance. All this is
accomplished through the consideration of the cross-correlations.

In the \textit{simpler} models, with a latent structure just in one
level, is hard to see some significant difference between the clusters
and sample sizes. With the complete model, in the other hand, the
difference is clear: as we increase the clusters and the sample sizes,
the bias-variance decreases. The mean-bias is basically always the
same. In the risk model is hard to point-out a scenario as the best or
worst. For the time model, with the scenarios \texttt{cs02-high-05k}
and \texttt{cs05-high-60k}, we get a much bigger standard deviation in
the \(\bm{\beta}\)s parameter estimates. For the block-diag model, with
the scenario \texttt{cs05-low-05k}, the standard deviations are huge for
the shape curve parameter estimates of the competing cause 1. In
the \autoref{cap:appendixD}, with the 2.5 and 97.5\% bias quantiles, the
most extreme values are removed from the uncertainty
representation. There, the main characteristic is the parameter
estimates asymmetry.

\begin{figure}[H]
 \setlength{\abovecaptionskip}{.0001pt}
 \caption{PARAMETER \(\beta_{1}\) BIAS WITH \(\pm\) 1.96 STANDARD
          DEVIATIONS}
 \vspace{0.2cm}\centering
 \includegraphics[width=\textwidth]{bias2plotsd-1.png}\\
 \begin{footnotesize}
  SOURCE: The author (2021).
 \end{footnotesize}
 \label{fig:biassdbeta1}
\end{figure}

\begin{figure}[H]
 \setlength{\abovecaptionskip}{.0001pt}
 \caption{PARAMETER \(\beta_{2}\) BIAS WITH \(\pm\) 1.96 STANDARD
         DEVIATIONS}
 \vspace{0.2cm}\centering
 \includegraphics[width=\textwidth]{bias2plotsd-2.png}\\
 \begin{footnotesize}
  SOURCE: The author (2021).
 \end{footnotesize}
 \label{fig:biassdbeta2}
\end{figure}

\begin{figure}[H]
 \setlength{\abovecaptionskip}{.0001pt}
 \caption{PARAMETER \(\gamma_{1}\) BIAS WITH \(\pm\) 1.96 STANDARD
          DEVIATIONS}
 \vspace{0.2cm}\centering
 \includegraphics[width=\textwidth]{bias2plotsd-3.png}\\
 \begin{footnotesize}
  SOURCE: The author (2021).
 \end{footnotesize}
 \label{fig:biassdgama1}
\end{figure}

\begin{figure}[H]
 \setlength{\abovecaptionskip}{.0001pt}
 \caption{PARAMETER \(\gamma_{2}\) BIAS WITH \(\pm\) 1.96 STANDARD
          DEVIATIONS}
 \vspace{0.2cm}\centering
 \includegraphics[width=\textwidth]{bias2plotsd-4.png}\\
 \begin{footnotesize}
  SOURCE: The author (2021).
 \end{footnotesize}
 \label{fig:biassdgama2}
\end{figure}

\begin{figure}[H]
 \setlength{\abovecaptionskip}{.0001pt}
 \caption{PARAMETER \(w_{1}\) BIAS WITH \(\pm\) 1.96 STANDARD DEVIATIONS}
 \vspace{0.2cm}\centering
 \includegraphics[width=\textwidth]{bias2plotsd-5.png}\\
 \begin{footnotesize}
  SOURCE: The author (2021).
 \end{footnotesize}
 \label{fig:biassdw1}
\end{figure}

\begin{figure}[H]
 \setlength{\abovecaptionskip}{.0001pt}
 \caption{PARAMETER \(w_{2}\) BIAS WITH \(\pm\) 1.96 STANDARD DEVIATIONS}
 \vspace{0.2cm}\centering
 \includegraphics[width=\textwidth]{bias2plotsd-6.png}\\
 \begin{footnotesize}
  SOURCE: The author (2021).
 \end{footnotesize}
 \label{fig:biassdw2}
\end{figure}

\begin{figure}[H]
 \setlength{\abovecaptionskip}{.0001pt}
 \caption{PARAMETER \(\log(\sigma_{1}^{2})\) BIAS WITH \(\pm\) 1.96
          STANDARD DEVIATIONS}
 \vspace{0.2cm}\centering
 \includegraphics[width=\textwidth]{bias2plotsd-7.png}\\
 \begin{footnotesize}
  SOURCE: The author (2021).
 \end{footnotesize}
 \label{fig:biassdlogs2_1}
\end{figure}

\begin{figure}[H]
 \setlength{\abovecaptionskip}{.0001pt}
 \caption{PARAMETER \(\log(\sigma_{2}^{2})\) BIAS WITH \(\pm\) 1.96
          STANDARD DEVIATIONS}
 \vspace{0.2cm}\centering
 \includegraphics[width=\textwidth]{bias2plotsd-8.png}\\
 \begin{footnotesize}
  SOURCE: The author (2021).
 \end{footnotesize}
 \label{fig:biassdlogs2_2}
\end{figure}

\begin{figure}[H]
 \setlength{\abovecaptionskip}{.0001pt}
 \caption{PARAMETER \(\log(\sigma_{3}^{2})\) BIAS WITH \(\pm\) 1.96
          STANDARD DEVIATIONS}
 \vspace{0.2cm}\centering
 \includegraphics[width=\textwidth]{bias2plotsd-9.png}\\
 \begin{footnotesize}
  SOURCE: The author (2021).
 \end{footnotesize}
 \label{fig:biassdlogs2_3}
\end{figure}

\begin{figure}[H]
 \setlength{\abovecaptionskip}{.0001pt}
 \caption{PARAMETER \(\log(\sigma_{4}^{2})\) BIAS WITH \(\pm\) 1.96
          STANDARD DEVIATIONS}
 \vspace{0.2cm}\centering
 \includegraphics[width=\textwidth]{bias2plotsd-10.png}\\
 \begin{footnotesize}
  SOURCE: The author (2021).
 \end{footnotesize}
 \label{fig:biassdlogs2_4}
\end{figure}

\begin{figure}[H]
 \setlength{\abovecaptionskip}{.0001pt}
 \caption{PARAMETER \(z(\rho_{12})\) BIAS WITH \(\pm\) 1.96 STANDARD
          DEVIATIONS}
 \vspace{0.2cm}\centering
 \includegraphics[width=\textwidth]{bias2plotsd-11.png}\\
 \begin{footnotesize}
  SOURCE: The author (2021).
 \end{footnotesize}
 \label{fig:biassdrhoz12}
\end{figure}

\begin{figure}[H]
 \setlength{\abovecaptionskip}{.0001pt}
 \caption{PARAMETER \(z(\rho_{34})\) BIAS WITH \(\pm\) 1.96 STANDARD
          DEVIATIONS}
 \vspace{0.2cm}\centering
 \includegraphics[width=\textwidth]{bias2plotsd-12.png}\\
 \begin{footnotesize}
  SOURCE: The author (2021).
 \end{footnotesize}
 \label{fig:biassdrhoz34}
\end{figure}

\begin{figure}[H]
 \setlength{\abovecaptionskip}{.0001pt}
 \caption{PARAMETERS
          \(\{z(\rho_{13}),~z(\rho_{24}),~z(\rho_{14}),~z(\rho_{23})\}\)
          BIAS WITH \(\pm\) 1.96 STANDARD DEVIATIONS}
 \vspace{0.2cm}\centering
 \includegraphics[width=\textwidth]{bias2plotsd-13.png}\\
 \begin{footnotesize}
  SOURCE: The author (2021).
 \end{footnotesize}
 \label{fig:biassdrhoz4}
\end{figure}

With the log-variances presented in \autoref{fig:biassdlogs2_1},
\autoref{fig:biassdlogs2_2}, \autoref{fig:biassdlogs2_3}, and
\autoref{fig:biassdlogs2_4}, we have instead a similar behavior through
the models. For all the models, the high CIF scenarios are the ones with
a smaller mean and bias-variances. From the risk/time model to the
block-diag model, we do not see a significant improvement in terms of
bias reduction. Such improvement, however, is clear when we look at the
complete model. Again, the magick of considering the cross-correlations.

The same said about the log-variances, can be applied to the risk
correlations in \autoref{fig:biassdrhoz12}, with one addendum: the bias
reduction is even bigger. With the time correlation in
\autoref{fig:biassdrhoz34}, at least with clusters of size 2 and 5,
we get the same behavior observed with the fixed-effect parameters i.e.,
with the simpler models, the smaller biases are observed in the low CIF
scenarios. However, with the complete model, we get the opposite. With
the cross-correlations in
\autoref{fig:biassdrhoz4}, the mean and bias-variances are much smaller
in the high CIF scenarios.

The biggest bias-variances are obtained in the log-variances. A final
remark to be made is about convergences. With the simpler models, not
all of them work, having in some scenarios (generally the ones with 60
thousand data points) a 50\(\sim\)60\% convergence rate. With the
complete model, basically, almost all fits reach convergence
(\(\sim\)95\% performance).

After looking at the parameter estimates biases, let us take a look at
the implied mean-CIF curves. To nicely accommodate all seventy-two
scenarios we split the curves by level-CIF. In \autoref{fig:cifshigh} we
have the high CIF scenario curves and in \autoref{fig:cifslow} the low
CIF scenario curves. Since for all the models we have a latent structure
for the within-cluster dependency, the inherent idea is that this also
affect the fixed-effect parameter estimates. By taking its average in
each of the seventy-two scenarios, we are able to construct the mean CIF
curves.

In \autoref{fig:cifshigh} we have all the thirty-six curves obtained in
the high CIF scenarios. It is clear that with the complete model we get
a perfect fit in all nine scenarios. The risk and time models estimate
well the curve shape parameters but they fail to learn the max
incidence. A compensation between curves is clear.

\begin{figure}[H]
 \setlength{\abovecaptionskip}{.0001pt}
 \caption{HIGH CUMULATIVE INCIDENCE FUNCTION (CIF) SCENARIO CURVES}
 \vspace{0.2cm}\centering
 \includegraphics[width=\textwidth]{cifs-1.png}\\
 \begin{footnotesize}
  SOURCE: The author (2021).
 \end{footnotesize}
 \label{fig:cifshigh}
\end{figure}

\begin{figure}[H]
 \setlength{\abovecaptionskip}{.0001pt}
 \caption{LOW CUMULATIVE INCIDENCE FUNCTION (CIF) SCENARIO CURVES}
 \vspace{0.2cm}\centering
 \includegraphics[width=\textwidth]{cifs-2.png}\\
 \begin{footnotesize}
  SOURCE: The author (2021).
 \end{footnotesize}
 \label{fig:cifslow}
\end{figure}

Still in \autoref{fig:cifshigh}, in the risk model, there is a super
estimation of \(\beta_{1}\) in all scenarios. For failure cause 2, there
is a sub estimation. With the time model, we observe the opposite
compensation but on a smaller scale. With the time model, we get much
better curves than with the risk model. The block-diag model results are
a middle term between them. For the time model, the scenario with
cluster size 10 and 60 thousand data points is a highlight. For the
block-diag model, the highlight is the scenario with cluster size 5 and
30 thousand data points.

In the low CIF scenarios in \autoref{fig:cifslow}, the estimation is
clearly more difficult. The overall fits are bad, being impossible to
select a scenario with overall good results. For one of the failure
causes, the estimation quality is not so bad. The problem is when we
look to the other. An interesting scenario is the one with cluster size
2 and 60 thousand data points. In this scenario we see the worst fits
for failure cause 1, with a negative highlight in the block-diag
configuration. However, with this same model, for failure cause 2, it is
the scenario were we better learn the true curve. An interesting
compensation phenomena. The best joint fit is still with the complete
model.

Now we look at how the latent-effect parameter estimates distribute
themselves. Given the huge number of scenarios and the fact that is
harder to estimate covariance parameters, we chose to plot the parameter
estimates just in the scenarios with better performances. By the metrics
of small bias and CIF shape learning, the scenarios with better results
are the ones with high CIF and bigger sample sizes. We have the
densities for the variance parameter estimates, in each of these
scenarios, presented
in \autoref{fig:histologs2}. In \autoref{fig:historhoz} we have the same
for the correlation parameter estimates.

An interesting result is the clear difference between risk and time
models' covariance parameter estimates. With the risk model, we have an
evident super estimation and bigger variances. With the time model we
get much better results, but still with high variances. The block-diag
model generally performs better than the risk model and worst than the
time model, showing again to be a compromise between them. Besides the
bias itself, we should also pay attention to the values. We model the
variances in the log-scale, so a value 5, in reality, implies a variance
of \(\exp(5) = 148\). Terrible. This kind of problem do not sound to
appear with the complete model.

All correlations are quite well estimated, in all three scenarios, with
the complete model. Not only the correlations but the variances
also. The lack of any considerable difference between the covariance
densities, indicates no quality divergences in the results for different
cluster sizes. The densities in \autoref{fig:histologs2} and
\autoref{fig:historhoz} are the final corroboration indicating the good
performance of the maximum likelihood method in the complete
model.

Between the four tested models, the complete model was the one with the
smallest biases, better CIF shape learning, and precisest covariance
parameter estimates. In \autoref{fig:cor2plot} we have a heat-map of the
correlations between parameter estimates for the complete model in the
scenario with clusters of size 10, high CIF, and 60 thousand data
points.

We have a little bit of everything in the parameter estimates
correlations' heat-map. Some correlations are very close to zero, but we
also have strong positive and negative correlations. We can mention some
curiosities, but nothing pathological appears to happen, at least
nothing clear.

\begin{figure}[H]
 \setlength{\abovecaptionskip}{.0001pt}
 \caption{VARIANCE PARAMETERS DENSITIES IN THE SCENARIOS OF HIGH CIF AND
          60 THOUSAND DATA POINTS}
 \vspace{0.2cm}\centering
 \includegraphics[width=\textwidth]{histologs2-1.png}\\
 \begin{footnotesize}
  SOURCE: The author (2021).
 \end{footnotesize}
 \label{fig:histologs2}
\end{figure}

\begin{figure}[H]
 \setlength{\abovecaptionskip}{.0001pt}
 \caption{CORRELATION PARAMETERS DENSITIES IN THE SCENARIOS OF HIGH CIF
          AND 60 THOUSAND DATA POINTS}
 \vspace{0.2cm}\centering
 \includegraphics[width=\textwidth]{historhoz-1.png}\\
 \begin{footnotesize}
  SOURCE: The author (2021).
 \end{footnotesize}
 \label{fig:historhoz}
\end{figure}

In \autoref{fig:cor2plot}, all fixed-effect parameters are positive
correlated, with an emphasis on the correlation between \(\beta_{1}\)
and \(\beta_{2}\), and the one of the \(\bm{\beta}\)s with
the \(\bm{w}\)s. Another interesting observation is the strong negative
correlation between the \(\bm{\beta}\)s and the risk level
log-variances, and also the (less strong) positive correlation between
the \(\bm{\beta}\)s and the failure time trajectory level
log-variances. The risk level log-variances are (strongly) positively
correlated. So do the failure time trajectory level ones, but again, not
so strong as in the risk level. The correlations between the
log-variances of different levels are negative.

\begin{figure}[H]
 \setlength{\abovecaptionskip}{.0001pt}
 \caption{COMPLETE MODEL'S PARAMETERS CORRELATION HEAT-MAP IN THE
          SCENARIO OF CLUSTER SIZE 10, HIGH CIF, AND SIXTY-THOUSAND DATA
          POINTS}
 \centering
 \includegraphics[width=\textwidth]{cor2plot-1.png}\\
 \vspace{-0.2cm}
 \begin{footnotesize}
  SOURCE: The author (2021).
 \end{footnotesize}
 \label{fig:cor2plot}
\end{figure}

% END ==================================================================
