Consider a cluster of random variables representing the time until the
occurrence of some event. These random variables are assumed to be
correlated, i.e. for some biological or environmental reason it is not
adequate to assume independence between them. Also, we may be interested
in the occurrence of not only one specific event, having in practice a
competition of events to see which one happens first, if it
happens. Such events may also be of low probability albeit severe
consequences, this is the moment when the cluster correlation makes its
difference: the occurrence of an event in a cluster member should affect
the probability of the same happening in the others.

A realistic context that fits perfectly with the framework described
above is the study of disease incidence in family members, where each
member is indexed by a random variable and each cluster consists of a
familiar structure. The inspiration to the study of these kinds of
problems came from the work developed in \citeonline{SCHEIKE}, where
they studied breast cancer incidence in mothers and daughters but using
a complicated modeling framework. Based on that, the aim of this thesis
is to propose a simpler framework taking advantage of
several \textit{state-of-art} computational libraries and see how far we
can go in several scenarios. Until now we have just contextualized, we
still need to introduce the methodology. To do this, some definitions
and theoretical contexts are welcome.

When the object under study is a random variable representing the time
until some event occurs, we are in the field of \textit{failure time
data} \cite{kalb&prentice}. The occurrence of an event is generally
denoted \textit{failure}, and major areas of application are biomedical
studies and industrial life testing. In this thesis, we maintain our
focus on the former. As common in science, same methodologies can
receive different names depending on the area. In industrial life
testing is performed what is called a \textit{reliability analysis}; in
biomedical studies is performed what is called
\textit{survival analysis}. Generally, the term \textit{survival} is
applied when we are interested in the occurrence of only one event, a
\textit{failure time process}. When we are interested in the occurrence
of more than one event we enter in the yard of \textit{competing risks}
and \textit{multistate} models. A visual aid is presented on
\autoref{fig:intro1} and a comprehensive reference is
\citeonline{kalb&prentice}.

Failure time and competing risks processes may be seen as particular
cases of a multistate model. Besides the number of events (states) of
interest, the main difference between a multistate model and its
particular cases is that only in the multistate scenario we may have
transient states, using a \textit{stochastic process} language. In the
particular cases, all states besides the initial state 0, are absorbents
- once you reached it you do not leave. The simplest multistate model
that exemplify this behavior is the illness-death model,
\autoref{fig:intro1}~C), where a patient (initially in state 0) can get
sick (state 1) or die (state 2); if sick it can recover (returns to
state 0) or die. We work in this thesis only with competing risks
processes, and for each patient we need the time (age) until the
occurrence, or not, of the event.

\begin{figure}[H]
 \setlength{\abovecaptionskip}{.0001pt}
 \caption{ILLUSTRATION OF MULTISTATE MODELS FOR A A) FAILURE TIME
          PROCESS; B) COMPETING RISKS PROCESS; AND C) ILNESS-DEATH
          MODEL, THE SIMPLEST MULTISTATE MODEL}
 \vspace{0.5cm}\centering
 \tikzfig{fig1}\\
 \vspace{0.5cm}
 \begin{footnotesize}
  SOURCE: The author~(2021).
 \end{footnotesize}
 \label{fig:intro1}
\end{figure}

When for some known or unknown reason we are not able to see the
occurrence of an event, we have what is denoted \textit{censorship}.
Still in the illness-death model, during the period of follow up the
patient may not get sick or die, staying at state 0. This is denoted
\textit{right-censorship}; if a patient is in state 1 at the end of the
study, we are \textit{censored} to see him reaching the state 2 or
returning to state 0. This is the inherent idea to censorship and must
be present in the modeling framework, thus arriving in the so-called
\textit{survival models} \cite{kalb&prentice}.

A survival model deals with the survival experience. Usually, the
survival experience is modeled in the \textit{hazard} (failure rate)
scale and it can be expressed for a subject \(i\) as
\begin{equation}
  \lambda(t \mid \bm{x}_{i}) =
  \lambda_{0}(t) \times c(\bm{x}_{i} \bm{\beta})
  \quad \text{at time}~t,
  \label{eq:intro1}
\end{equation}
i.e. as the product of an arbitrary baseline hazard function
\(\lambda_{0}(\cdot)\), with a specific function form \(c(\cdot)\), that
will depend on the probability distribution to be chosen for the failure
time and on predictors/covariates/explanatory/independent variables
\(\bm{x}_{i} = [x_{1}~\dots~x_{p}]\), where \(\bm{\beta}^{\top} =
[\beta_{1}~\dots~\beta_{p}]\) is the parameters vector.

This structure is specified for a failure time process, as in
\autoref{fig:intro1}~A). Nevertheless, the idea is easy to extend. We
basically have the \autoref{eq:intro1}'s model to each cause-specific
(in a competing risks process) or transition (in a multistate process).
A complete and extensive detailing can be, again, found in
\citeonline{kalb&prentice}.

In this work we approach the case of clustered competing risks. Besides
the cause-specific structure, we have to deal with the fact that the
events are happening in related individuals. This configures what is
denoted \textit{family studies}, i.e. we have a cluster/group/family
dependence that needs to be considered, accommodated, and modeled. This,
possible, dependence is something that we do not actually measure but
know (or just suppose) that exists. In the statistical modeling language
this characteristic receives the name of \textit{random} or
\textit{latent effect}. A survival model with a latent effect,
association, or unobserved heterogeneity, is denoted
\textit{frailty model} \cite{frailty78, frailty79}. In its simplest
form, a frailty is an unobserved random proportionality factor that
modifies the hazard function of an individual, or of related
individuals. Frailty models are extensions of \autoref{eq:intro1}'s
model.

In the competing risks setting, the hazard scale (focusing on the
cause-specific hazard) is not the only possible scale to work on. A more
attractive possibility is to work on the probability scale
\cite{andersen12}, focusing on the cause-specific cumulative incidence
function (CIF). Besides the within-family dependence, in family studies
there is often a strong interest in describing age at disease onset,
which is directly described by the cause-specific CIF. Therefore, making
the probability scale a more attractive and logical choice. Since the
CIF plays a central role in this master thesis, it will be formally
defined later in a place with greater emphasis. With the definitions and
the theoretical context being made, let us be more specific.

To work with competing risks data on the probability scale plus a latent
structure allowing for within-cluster dependence of both risk and
timing, \citeonline{SCHEIKE} proposed a pairwise composite likelihood
approach based on the factorization of the cause-specific CIF as the
product of a cluster-specific risk level function with a
cluster-specific failure time trajectory function. A composite approach
\cite{lindsay88, cox&reid04, varin11} is a valid alternative to a full
likelihood analysis in high-dimensional situations when a full approach
is too computational costly or even inviable. In failure time data
problems, the composite likelihood function is built from the product of
marginal densities. The marginal specification implies a pairwise
approach since we need to add model layers to be able to handle with the
dependence structure. A clear advantage of this approach is that we do
not need to care about a joint distribution specification, which
generally translates also into a computational advantage. A disadvantage
is the model specification, which becomes much more complicated, besides
the number of small details to workaround from the fact of being working
with not an exact likelihood function.

We do not have any guarantees that a full likelihood inference procedure
is not viable here, so we try to reach the same goal of
\citeonline{SCHEIKE} albeit with a simpler framework taking advantage of
\textit{state-of-art} software, something still not so common in the
statistical modeling community. This simpler framework is a generalized
linear mixed model (GLMM). Instead of concentrating on failure time data
and consequently having a survival/frailty model based on the hazard
scale, or using a composite approach, we just build the joint/full
likelihood function (a multinomial model with its link function based on
the cluster-specific CIF, accouting for an appropriate latent effects
structure), marginalize (integrate out the latent effects) and optimize
it. A Fisherian approach per se.

To a better contextualization of our GLMM approach \cite{GLMM}, consider
a random suject \(i\). In a standard linear model we assume that the
response variable \(Y_{i}\), conditioned on the covariates
\(\bm{x}_{i}\), follows a normal/Gaussian distribution and what we do is
to model its mean, \(\mu_{i} \equiv \mathbb{E}(Y_{i} \mid \bm{x}_{i})\),
via a linear combination. As much well explained in \citeonline{GLM72},
with the aid of a \textit{link function} \(g(\cdot)\), this idea is
generalized to distributions of the \textit{exponential family}. Many of
its members are useful for practical modelling, such as the Poisson (for
counting data), binomial (dichotomic data), gamma (continuous but
positive) and Gaussian (continuous data) distributions. This extended
framework received the name of generalized linear model (GLM) and a
comprehensive reference is \citeonline{GLM89}.

What makes a GLM into a GLMM \cite{GLMM} is the addition of a latent
effect \(\bm{u}\) (then, \textit{m}ixed) into the mean structure. The
mean structure of a standard GLMM is defined as
\[
  g(\mu_{i}) = \bm{x}_{i}\bm{\beta} + \bm{z}_{i}\bm{u},
  \quad \bm{u} \sim \text{Multivariate Normal}(\bm{0},\bm{\Sigma})
\]
where the latent effect is assumed to follow a multivariate Gaussian
distribution of zero mean and a parametrized variance-covariance matrix
\(\bm{\Sigma}\). Its correct linkage to the mean structure is made
through the \(i^\text{th}\) vector row of a design-matrix \(\bm{Z}\).
The covariates are into \(\bm{x}_{i}\), the \(i^\text{th}\) vector row
of a model-matrix \(\bm{X}\), with \(\bm{\beta}\) being a vector of
unknown parameters.

\section{GOALS}

\subsection{General goals}

Propose and study the estimability of a multinomial generalized linear
mixed model (multiGLMM) to the cluster and cause-specific cumulative
incidence function (CIF) of clustered competing risks data.

\subsection{Specific goals}

\begin{enumerate}
 \item Simulate from the model, i.e. generate synthetic data to study
       statistical properties.

 \item Write the model in the Template Model Builder (TMB) software,
       developed by \citeonline{TMB} and possibly the most efficient
       likelihood-based way of doing such task.

 \item Take advantage of TMB's functionalities with special attention to
       the computation of gradients and Hessians via a
       \textit{state-of-art} automatic differentiation (AD)
       implementation; and a joint likelihood marginalization via an
       efficient Laplace approximation routine.

 \item Study the model identifiability through the proposition of
       different complexity level models in terms of parametric space
       and latent effect structures.

 \item Make exact likelihood-based inference to the cluster and
       cause-specific CIF of clustered competing risks data.
\end{enumerate}

\section{JUSTIFICATION}

In the biomedical statistical modeling literature, the study of disease
occurrence in related individuals receives the name of family studies.
Key points of interest are the within-family dependence and determining
the role of different risk factors. The within-family dependence may
reflect both disease heritability and the impact of shared environmental
effects. The role of different risk factors arrives in the class of
multivariate models, which options are limited in the statistical
literature. Thus, the number of statistical models for competing risks
data that accommodate the within-cluster/family dependence is even more
limited. Some modeling options are briefly commented in
\citeonline{SCHEIKE}, with his pairwise composite approach being
proposed as a new and better option to model the cause-specific
cumulative incidence function (CIF), describing age at disease onset, of
clustered competing risks data on the probability scale. We propose to
model the cause-specific CIF and accommodate the within-family
dependence in the same fashion (via a latent structure that allows the
absolute risk and the failure time distribution to vary between
families) but with an easier framework, based on a multinomial
generalized linear mixed model approach.

\section{LIMITATION}

This work restraint to the proposition and model identifiability study
of a multinomial model for the cause-specific cumulative incidence
function (CIF) of competing risks data, with a latent effect structure
to accommodate within-family dependence with regard to both risk and
timing. Given its considerable model complexity, hypothesis tests;
residual analysis; and good-of-fit measures are not contemplated.

\section{THESIS ORGANIZATION}

This master thesis contains 6 chapters including this introduction.
\autoref{cap:methods} presents a systematic review of the main aspects
involved in the formulation, optimzation, and implementation of a
generalized linear mixed model (GLMM). Given the modeling framework
overview, \autoref{cap:model} presents our multinomial GLMM (multiGLMM)
to model the cause-specific cumulative incidence function (CIF) of
clustered competing risks data. In \autoref{cap:datasets} we describe
the simulation procedure to generate synthetic data and present some
model particularities. In \autoref{cap:results} the obtained results are
presented, and in \autoref{cap:finalc} we discuss the contributions of
this thesis and present some suggestions for future work.

% END ==================================================================
