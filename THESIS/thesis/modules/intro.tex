Consider a cluster of random variables representing the time until some
event occurs. The random variables that compose that cluster are assumed
to be correlated, i.e. for some biological or environmental reason it is
not adequate to assume independence between them, the random variables.
Also, we may be interested in the occurrence of not only one specific
event, having in practice a competition of events to see which one
happens first, if it happens. Such events may also be of low probability
but with severe consequences. That is the moment when the correlation
ingredient makes its difference since the occurrence of an event in a
subject should affect the probability of the same happening in the
others.

A realistic and practical context that fits perfectly with the
theoretical framework described above is the study of disease incidence
in family members, where each member is indexed by a random variable and
each cluster consists of a family. The inspiration to the study of these
types of problems came from the work developed in \citeonline{SCHEIKE},
where they studied breast cancer incidence in mothers and daughters.
Based on that, the aim of this thesis is to propose a simpler framework
to make inference on the gender-specific cancer incidence in twins. The
twins' case is just one more in the range of possible applications to
our model, but given its vast intrinsic particularities, becomes the
focus application. Until now we only contextualized the problem and the
focus, we still need to introduce the methodology. To this, some
definitions and theoretical contexts are welcome.

When the object under study is random variables representing the time
until some event occurs, we're in the field of \textit{failure time
  data}~\cite{kalb&prentice}. The occurrence of an event is generally
called as a \textit{failure}, and major areas of application are
biomedical studies and industrial life testing. In this thesis, we
maintain our focus on the former. In industrial life testing
applications, is performed what is called a \textit{reliability
  analysis}; in biomedical studies, is performed what is called
\textit{survival analysis}. Generally, the term survival analysis is
applied when we're interested in the occurrence of only one event, a
\textit{failure time process}. When we're interested in the occurrence
of more than one event, like now, we enter in the yard of
\textit{competing risks} and \textit{multistate} models. A visual aid is
presented on \autoref{fig:intro1} and a comprehensive reference
is~\citeonline{kalb&prentice}.

\begin{figure}[H]
  % \vspace{0.35cm}
  \setlength{\abovecaptionskip}{.0001pt}
  \caption{BEHAVIOR ILLUSTRATIONS OF MULTISTATE MODELS FOR A) FAILURE
    TIME PROCESS; B) COMPETING RISKS; AND C) ILNESS-DEATH MODEL, THE
    SIMPLEST MULTISTATE MODEL}
  \vspace{0.425cm} \centering
  \tikzfig{fig1}\\
  \vspace{0.45cm}
  \begin{footnotesize}
    SOURCE: The author~(2020).
  \end{footnotesize}
  \label{fig:intro1}
\end{figure}

Failure time and competing risk processes may be seen as particular
cases of a multistate model. Besides the number of events (states) of
interest, a big difference between a multistate model and its particular
cases is that only in the multistate scenario we may have transient
states, using a \textit{stochastic process} language. In the particular
cases, all the states, besides the initial state 0, are absorbents -
once you reached it you don't leave. The simplest multistate model that
exemplify this behavior is the so-called illness-death model,
\autoref{fig:intro1}~C). A patient enters the study (state 0) and it can
get sick (state 1) or die (state 2); if sick it can recover (returns to
state 0) or die. In this thesis, we'll work only with competing risk
processes. For each individual, we have the time, age, until the
occurrence (or not) of cancer.

When for some know or unknown reason we're able to see the occurrence of
the event, we have what is called \textit{censorship}. Still in the
illness-death model: during the period of follow up the patient may not
get sick or die, staying at state 0, this is called a
\textit{right-censorship}; The same for state 1. If a patient is in
state 1 at the end of the study, we're \textit{censored} to see him
reaching the state 2 or returning to state 0. This is the inherent idea
to censorship and must be present in the modeling framework.

In a survival model what we model is the survival experience. Usually,
there are covariates (explanatory/independent variables) upon which
failure time may depend. The model is defined by the \textit{hazard}
(failure rate), \(\lambda(\cdot)\), at time \(t\) for an individual with
covariate vector \(\mathbf{X}_{i}\) and can be written as
\begin{equation}
  \lambda(t~;~\mathbf{X}_{i}) =
  \lambda_{0}(t) \times c(\mathbf{X}_{i} \bm{\beta}),
  \label{eq:intro1}
\end{equation}
where \(\bm{\beta}^{\top} = [\beta_{1}~\dots~\beta_{p}]\) is a vector
of regression parameters; \(\lambda_{0}(\cdot)\) is an arbitrary
baseline hazard function; and \(c\) is a specific function form, that
will depend on the chosen probability distribution for the failure time.
The structure of equation~\ref{eq:intro1} is made thinking in a simple
failure time process, as in~\autoref{fig:intro1}~A). However, is easy to
extend its idea. We basically have the equation~\ref{eq:intro1} model
for each cause-specific, in a competing risks process; or transition, in
a multistate process. A complete and extensive detailing can be, again,
found in~\citeonline{kalb&prentice}.

In this thesis, we approach the case of the clustered competing risks.
Besides the cause-specific structure, we have the fact that the disease
occurrences are happening in related individuals (twins). This
configures what is called \textit{family studies}. We have a cluster (a
family/pair of twins) dependence that needs to be considered. This,
possible, dependence is something that we don't actually measure, but
know (or just suppose) that exists. In the statistical modeling
language, this type of characteristic receives the name of
\textit{random} or \textit{latent} effect. A survival model with a
latent effect, association, or unobserved heterogeneity, is called a
\textit{frailty model}~\cite{frailty78, frailty79}. In its simplest
form, a frailty is an unobserved random proportionality factor that
modifies the hazard function of an individual, or of related
individuals. Frailty models are extensions of the equation
\ref{eq:intro1} model.

Specifically in the competing risks setting, we have to choose which of
two scales we want to work on. The hazard scale focusing on the
cause-specific hazard or on the probability scale focusing on the
cause-specific cumulative incidence. Both may complement each
other~\cite{andersen12}. However, in family studies, there is often a
strong interest in describing age at disease onset including
within-family dependence. The distribution of age at disease onset is
directly described by the cause-specific cumulative incidence.
Therefore, the probability scale is the chosen one here.

Frailty models are generally defined based on the hazard scale and with
a latent effect that modifies the hazard function via a proportionality
factor. To work on the probability scale and with a latent effect that
allows for within-cluster dependence of both risk and
timing,~\citeonline{SCHEIKE} proposed a pairwise composite likelihood
approach based based on a linear model with multinomial response
distribution and multivariate normal latent effects (in a frailty model
the common choice for the latent effects is the gamma distribution). The
idea in this thesis is to try to do that in a simpler manner via a
generalized linear mixed model (GLMM). Instead of concentrating on
failure time data and consequently having a survival/frailty model on
the hazard scale, or using a composite approach, we just build the joint
likelihood function (multinomial distrtibution with a link function
based on the cause-specific cumulative incidence function and
appropriate latent effects), integrate out the latent effects and
optimize the obtained distribution with respect to (wrt) its parameters.
With this approach, we easily work on the probability scale and are able
to specify the desired within-cluster dependence structure. To conclude,
a brief introduction of a GLMM.

In a standard linear model we assume that the response variable,
\(Y_{i}\), conditioned on the covariates follows a normal distribution,
and we model it's mean, \(\mu_{i} \equiv \mathbb{E}(Y_{i})\), via a
linear combination. Generalizing this idea with a smooth monotonic
``link function'' \(g\), we get a generalized linear model
(GLM)~\cite{GLM72} with the basic structure
\[
 g(\mu_{i}) = \mathbf{X}_{i} \bm{\beta},
\]
where \(\mathbf{X}_{i}\) is the \(i^\text{th}\) row of a
model matrix \(\mathbf{X}\), and \(\bm{\beta}\) is a vector of unknown
parameters. In a GLM the \(Y_{i}\) are independent and
\[
  Y_{i} \sim \text{some exponential family distribution}.
\]

The \textit{exponential family} of distributions includes many
distributions that are useful for practical modelling, such as the
Poisson (for counting data), binomial (dichotomic data), gamma
(continuous but positive) and normal (continuous data) distributions.
The comprehensive reference for GLMs is~\citeonline{GLM89}.

The insertion of a latent effect, \(\mathbf{b}\), in that structure
makes a mixed model. A linear model becomes a linear mixed model and a
GLM becomes a generalized linear mixed model (GLMM). We now have
\[
  g(\mu_{i}) = \mathbf{X}_{i} \bm{\beta} + \mathbf{Z}_{i}\mathbf{b},
  \quad \mathbf{b} \sim \mathcal{N}(\mathbf{0}, \bm{\psi})
\]
where the latent effect is assumed to follow a multivariate normal
distribution of mean zero and a given variance-covariance structure.

\section{GOALS}

\subsection{General goals}

Propose a multinomial GLMM to the cause-specific cumulative incidence
function of clustered competing risks data.

\subsection{Specific goals}

\begin{enumerate}
\item Simulate from the model following and adapting the guidelines
  from~\citeonline{SCHEIKE}.

\item Write the model via Template Model Builder (TMB)~\cite{TMB},
  possibly the most efficient way of doing so, and take advantage of its
  functionalities: compute all necessary gradients and Hessians via
  Automatic Differentiation and integrate out the latent effects of the
  joint likelihood via Laplace approximation.

\item Try different complexity levels of the proposed modeling framework
  to see how identifiable it is.

\item Apply the model to the Nordic Cancer Union (NCU) twins data.

\item Compare the results of the multinomial GLMM approach to the
  pairwise composite likelihood approach of~\citeonline{SCHEIKE}.
\end{enumerate}

\section{JUSTIFICATION}

In family studies examining disease occurrence in related individuals,
key points of interest are the within-family dependence and determining
the role of different risk factors. The within-family dependence may
reflect both disease heritability and the impact of shared environmental
effects. The number of statistical models for competing risks data that
accommodate the within-cluster (family) dependence is limited. We didn't
find, e.g., any GLMM approach to do that and didn't find any
justification to not do that. Therefore, we propose a multinomial GLMM
apporach that accommodates these key points by modeling the
cause-specific cumulative incidence function (that describes age at
disease onset) of the competing risks, using a latent structure that
allows the absolute risk and the failure time distribution to vary
between clusters (here, families).

\section{LIMITATION}

This work restraint to the proposition and application of a multinomial
model for competing risks data with a latent effect structure to
accommodate within-cluster dependence with regard to both risk and
timing. Given the elevated model complexity, hypothesis tests; residual
analysis; and good-of-fit measures will not be approached.

\section{THESIS ORGANIZATION}

This thesis contains 6 chapters including this introduction.
The~\autoref{cap:methods} presents a review of the main aspects of a
general GLMM and its respective inference procedures.
The~\autoref{cap:model} presents the multinomial GLMM with its
particular characteristics and in~\autoref{cap:datasets} we describes
how to simulate from the proposed model, and presents a dataset for a
real application. In~\autoref{cap:results} the obtained results are
presented, and in~\autoref{cap:finalc} we discuss the contributions of
this thesis and present some suggestions for future work.

% END ==================================================================