%%----------------------------------------------------------------------
%%                                                     Henrique Laureano
%%                                            henriquelaureano.github.io
%%                                      2021-mar-05 · Curitiba/PR/Brazil
%%----------------------------------------------------------------------
\documentclass[12pt, % tamanho da fonte
               openright, % capítulos começam em pág ímpar
                          % (insere página vazia caso preciso)
               oneside, % para impressão apenas em um lado do papel
               a4paper, % tamanho do papel.
               chapter=TITLE, % títulos de capítulos convertidos em
                              % letras maiúsculas
               section=TITLE, % títulos de seções convertidos em letras
                              % maiúsculas
               % subsection=TITLE, % títulos de subseções convertidos em
                                 % letras maiúsculas
               % subsubsection=TITLE, % títulos de subsubseções em
                                    % letras maiúsculas
               % sumario=tradicional,
               brazil,
               english % o último idioma é o principal do documento
]{abntex2}
% ----------------------------------------------------------------------
% \renewcommand{\ABNTEXsectionfont}{\normalfont}
\renewcommand{\cftsectionfont}{\normalsize}
% ----------------------------------------------------------------------
\renewcommand{\ABNTEXchapterfontsize}{\bfseries\large}
\renewcommand{\ABNTEXsectionfontsize}{\large}
\renewcommand{\ABNTEXsubsectionfontsize}{\large}
% ----------------------------------------------------------------------
\usepackage{lastpage}
% \usepackage{times}
\usepackage{etoolbox}
\usepackage{lmodern} % usa a fonte Latin Modern
% \usepackage[T1]{fontenc} % selecao de codigos de fonte
\usepackage[utf8]{inputenc} % sodificacao do documento (conversão
                            % automática dos acentos)
% \usepackage{lastpage} % ssado pela ficha catalográfica
\usepackage{indentfirst} % indenta o primeiro parágrafo de cada seção
\setlength{\parindent}{1.5cm} % espaçamento de 1.5cm do parágrafo
\usepackage{color} %red, green, blue, yellow, cyan, magenta, black, white
\usepackage{graphicx} % inclusão de gráficos
\usepackage{microtype} % para melhorias de justificação
\usepackage{setspace}
\usepackage{pdflscape} % pagina na horizontal
% \usepackage{lscape} % pagina na horizontal
\usepackage{wrapfig}
\usepackage{booktabs}
\usepackage{makecell}
\usepackage{pgf} % para geração de dummy text
\usepackage[english, hyperpageref]{backref} % paginas com as citações
% ----------------------------------------------------------------------
% quadro cinza =========================================================
% \usepackage{caption}
\usepackage[small]{caption}
\usepackage{tikz} % permite desenhos vetoriais
% define o ambiente para colocar o código R
% (draw=gray!50 fill=gray!20 originais)
\tikzstyle{mybox} = [draw=gray!95, fill=white, very thick, rectangle,
                     inner sep=7pt, inner ysep=7pt]
\tikzstyle{mybox2} = [draw=gray!50, fill=gray!20,
                      very thick, % para códigos em R (Apêndices)
                      rectangle, inner sep=7pt, inner ysep=7pt]
% \usepackage[scaled=0.8]{beramono} % usa esta nos verbatins
\usepackage{float} % controla e define objetos flutuantes
\usepackage{tocloft} % controla lista de objetos flutuantes
% ambiente flutuante para código R
\newcommand{\listofprogramname}{CODE} % nome dessa lista
\newlistof{program}{lol}{\listofprogramname} % configura os arquivos
                                             % auxiliares para fazer a
                                             % lista
\makeatletter % allows use of "@" before \begin{document}
% this creates a custom and simpler ruled box style
\newcommand\floatc@simplerule[2]{{\@fs@cfont #1 #2}\par}
\newcommand\fs@simplerule{\def\@fs@cfont{ % aqui o estilo de fonte, e.g.
                                          % \bfseries
  }\let\@fs@capt\floatc@simplerule
  \def\@fs@pre{} % antes do caption, pode ser uma régua
  \def\@fs@post{} % depois do float, pode ser uma régua
  \def\@fs@mid{\kern3pt} % espaço entre caption e corpo
  \let\@fs@iftopcapt\iftrue}
\floatstyle{simplerule} % define o estilo do ambiente
\newfloat{program}{thp}{lol}[chapter] % ambiente contado por capítulo
\floatname{program}{Code} % rótulo que vai aparecer na legenda
\newcommand{\programautorefname}{Code} % altera o padrão do contador
                                       % para seguir capítulos
\renewcommand{\theprogram}{\thechapter.\arabic{program}} % capítulo.
                                                         % programa:
\sloppy
% FIM - TIKZ (quadro cinza) ============================================
% ----------------------------------------------------------------------
% CONFIGURAÇÕES DE PACOTES
% Configurações do pacote backref
% Usado sem a opção hyperpageref de backref
\renewcommand{\backrefpagesname}{Cited in the page(s):~}
% Texto padrão antes do número das páginas
\renewcommand{\backref}{}
% Define os textos da citação
\renewcommand*{\backrefalt}[4]{
  \ifcase #1 %
  None text citation.%
  \or
  Cited on page #2.%
  \else
  Cited #1 times on pages #2.%
  \fi}%
% ----------------------------------------------------------------------
% Citações padrão ABNT
\usepackage[alf, abnt-and-type=&, abnt-etal-list=0]{abntex2cite}
% ----------------------------------------------------------------------
% New comand for anexos
\newcommand{\refanexo}[1]{\hyperref[#1]{Annex~\ref{#1}}}
% \renewcommand{\cftsectionfont}{\sfseries}
% \refanexo{anexo_xxx} usar esse comando!
% ----------------------------------------------------------------------
% para titulo em destaque sem sequencia de numeração
\newcommand{\datatitle}[1]{
  \normalsize \textsc{#1}
}
\usepackage{pslatex}
% \usepackage{mathptmx} fonte - times new roman em tudo
% ----------------------------------------------------------------------
% adequando o uppercase titulo dos elementos nas suas respectivas LISTAS
\renewcommand{\cftfigurename}{FIGURE\enspace}
\renewcommand{\cfttablename}{TABLE\enspace}
% ----------------------------------------------------------------------
% fontes matemáticas
\usepackage{mathpazo}
\usepackage{inconsolata}
\usepackage{verbatim}
\usepackage{bm}
% ----------------------------------------------------------------------
% \usepackage[table, xcdraw]{xcolor} % cédula colorida em tabelas
% \usepackage{pdflscape} % rotaciona página
\usepackage{Capa} % capa e folha de rosto com modificações
\usepackage{float} % melhor posicionamento de figuras
\usepackage{gensymb} % símbolo º
\usepackage[justification=justified, singlelinecheck=false]{caption}
% \usepackage{etoolbox} % configurações adicionais de macros
\usepackage{xparse}
\usepackage{algorithm} % pacote usado para gerar pseudocódigo
% \usepackage{algorithmic} % pacote usado para gerar pseudocódigo
\definecolor{silverSAS}{RGB}{138,141,143}
\usepackage{listings}
\lstset{ 
 language=R, % the language of the code
 numbers=left, % where to put the line-numbers
 numbersep=6pt, % how far the line-numbers are from the code
 backgroundcolor=\color{white}, % choose the background color.
                                % You must add \usepackage{color}
 showspaces=false, % show spaces adding particular underscores
 showstringspaces=false, % underline spaces within strings
 showtabs=false, % show tabs within strings adding particular underscores
 frame=single, % adds a frame around the code
 rulecolor=\color{black}, % if not set, the frame-color may be changed on
                          % line-breaks within not-black text
 tabsize=2, % sets default tabsize to 2 spaces
 captionpos=b, % sets the caption-position to bottom
 breaklines=true, % sets automatic line breaking
 breakatwhitespace=false, % sets if automatic breaks should only happen
                          % at whitespace
 keywordstyle=\color{black}, % keyword style
 commentstyle=\color{silverSAS}, % comment style
 stringstyle=\color{black}, % string literal style
 basicstyle=\footnotesize\ttfamily,
}
\usepackage{multirow}
\usepackage{multicol}
%% nice table ----------------------------------------------------------
\usepackage{bbding} % checkmarks
\usepackage{color, colortbl} % coloring rows or columns in tables
\definecolor{Gray}{gray}{.9} % coloring rows or columns in tables
%% ---------------------------------------------------------------------
\usepackage[noend]{algpseudocode}
\makeatletter
\renewcommand{\ALG@name}{ALGORITHM}
\makeatother
% ----------------------------------------------------------------------
\usepackage{xpatch}
\makeatletter

\xpatchcmd{\algorithmic}{\itemsep\z@}{\itemsep=0.5ex plus2pt}{}{}
\makeatother
% ----------------------------------------------------------------------
\usepackage{tikzit}
%%----------------------------------------------------------------------
%%                                                     Henrique Laureano
%%                                            henriquelaureano.github.io
%%                                      2021-mar-05 · Curitiba/PR/Brazil
%%----------------------------------------------------------------------
\documentclass[12pt, % tamanho da fonte
               openright, % capítulos começam em pág ímpar
                          % (insere página vazia caso preciso)
               oneside, % para impressão apenas em um lado do papel
               a4paper, % tamanho do papel.
               chapter=TITLE, % títulos de capítulos convertidos em
                              % letras maiúsculas
               section=TITLE, % títulos de seções convertidos em letras
                              % maiúsculas
               % subsection=TITLE, % títulos de subseções convertidos em
                                 % letras maiúsculas
               % subsubsection=TITLE, % títulos de subsubseções em
                                    % letras maiúsculas
               % sumario=tradicional,
               brazil,
               english % o último idioma é o principal do documento
]{abntex2}
% ----------------------------------------------------------------------
% \renewcommand{\ABNTEXsectionfont}{\normalfont}
\renewcommand{\cftsectionfont}{\normalsize}
% ----------------------------------------------------------------------
\renewcommand{\ABNTEXchapterfontsize}{\bfseries\large}
\renewcommand{\ABNTEXsectionfontsize}{\large}
\renewcommand{\ABNTEXsubsectionfontsize}{\large}
% ----------------------------------------------------------------------
\usepackage{lastpage}
% \usepackage{times}
\usepackage{etoolbox}
\usepackage{lmodern} % usa a fonte Latin Modern
% \usepackage[T1]{fontenc} % selecao de codigos de fonte
\usepackage[utf8]{inputenc} % sodificacao do documento (conversão
                            % automática dos acentos)
% \usepackage{lastpage} % ssado pela ficha catalográfica
\usepackage{indentfirst} % indenta o primeiro parágrafo de cada seção
\setlength{\parindent}{1.5cm} % espaçamento de 1.5cm do parágrafo
\usepackage{color} %red, green, blue, yellow, cyan, magenta, black, white
\usepackage{graphicx} % inclusão de gráficos
\usepackage{microtype} % para melhorias de justificação
\usepackage{setspace}
\usepackage{pdflscape} % pagina na horizontal
% \usepackage{lscape} % pagina na horizontal
\usepackage{wrapfig}
\usepackage{booktabs}
\usepackage{makecell}
\usepackage{pgf} % para geração de dummy text
\usepackage[english, hyperpageref]{backref} % paginas com as citações
% ----------------------------------------------------------------------
% quadro cinza =========================================================
% \usepackage{caption}
\usepackage[small]{caption}
\usepackage{tikz} % permite desenhos vetoriais
% define o ambiente para colocar o código R
% (draw=gray!50 fill=gray!20 originais)
\tikzstyle{mybox} = [draw=gray!95, fill=white, very thick, rectangle,
                     inner sep=7pt, inner ysep=7pt]
\tikzstyle{mybox2} = [draw=gray!50, fill=gray!20,
                      very thick, % para códigos em R (Apêndices)
                      rectangle, inner sep=7pt, inner ysep=7pt]
% \usepackage[scaled=0.8]{beramono} % usa esta nos verbatins
\usepackage{float} % controla e define objetos flutuantes
\usepackage{tocloft} % controla lista de objetos flutuantes
% ambiente flutuante para código R
\newcommand{\listofprogramname}{CODE} % nome dessa lista
\newlistof{program}{lol}{\listofprogramname} % configura os arquivos
                                             % auxiliares para fazer a
                                             % lista
\makeatletter % allows use of "@" before \begin{document}
% this creates a custom and simpler ruled box style
\newcommand\floatc@simplerule[2]{{\@fs@cfont #1 #2}\par}
\newcommand\fs@simplerule{\def\@fs@cfont{ % aqui o estilo de fonte, e.g.
                                          % \bfseries
  }\let\@fs@capt\floatc@simplerule
  \def\@fs@pre{} % antes do caption, pode ser uma régua
  \def\@fs@post{} % depois do float, pode ser uma régua
  \def\@fs@mid{\kern3pt} % espaço entre caption e corpo
  \let\@fs@iftopcapt\iftrue}
\floatstyle{simplerule} % define o estilo do ambiente
\newfloat{program}{thp}{lol}[chapter] % ambiente contado por capítulo
\floatname{program}{Code} % rótulo que vai aparecer na legenda
\newcommand{\programautorefname}{Code} % altera o padrão do contador
                                       % para seguir capítulos
\renewcommand{\theprogram}{\thechapter.\arabic{program}} % capítulo.
                                                         % programa:
\sloppy
% FIM - TIKZ (quadro cinza) ============================================
% ----------------------------------------------------------------------
% CONFIGURAÇÕES DE PACOTES
% Configurações do pacote backref
% Usado sem a opção hyperpageref de backref
\renewcommand{\backrefpagesname}{Cited in the page(s):~}
% Texto padrão antes do número das páginas
\renewcommand{\backref}{}
% Define os textos da citação
\renewcommand*{\backrefalt}[4]{
  \ifcase #1 %
  None text citation.%
  \or
  Cited on page #2.%
  \else
  Cited #1 times on pages #2.%
  \fi}%
% ----------------------------------------------------------------------
% Citações padrão ABNT
\usepackage[alf, abnt-and-type=&, abnt-etal-list=0]{abntex2cite}
% ----------------------------------------------------------------------
% New comand for anexos
\newcommand{\refanexo}[1]{\hyperref[#1]{Annex~\ref{#1}}}
% \renewcommand{\cftsectionfont}{\sfseries}
% \refanexo{anexo_xxx} usar esse comando!
% ----------------------------------------------------------------------
% para titulo em destaque sem sequencia de numeração
\newcommand{\datatitle}[1]{
  \normalsize \textsc{#1}
}
\usepackage{pslatex}
% \usepackage{mathptmx} fonte - times new roman em tudo
% ----------------------------------------------------------------------
% adequando o uppercase titulo dos elementos nas suas respectivas LISTAS
\renewcommand{\cftfigurename}{FIGURE\enspace}
\renewcommand{\cfttablename}{TABLE\enspace}
% ----------------------------------------------------------------------
% fontes matemáticas
\usepackage{mathpazo}
\usepackage{inconsolata}
\usepackage{verbatim}
\usepackage{bm}
% ----------------------------------------------------------------------
% \usepackage[table, xcdraw]{xcolor} % cédula colorida em tabelas
% \usepackage{pdflscape} % rotaciona página
\usepackage{Capa} % capa e folha de rosto com modificações
\usepackage{float} % melhor posicionamento de figuras
\usepackage{gensymb} % símbolo º
\usepackage[justification=justified, singlelinecheck=false]{caption}
% \usepackage{etoolbox} % configurações adicionais de macros
\usepackage{xparse}
\usepackage{algorithm} % pacote usado para gerar pseudocódigo
% \usepackage{algorithmic} % pacote usado para gerar pseudocódigo
\definecolor{silverSAS}{RGB}{138,141,143}
\usepackage{listings}
\lstset{ 
 language=R, % the language of the code
 numbers=left, % where to put the line-numbers
 numbersep=6pt, % how far the line-numbers are from the code
 backgroundcolor=\color{white}, % choose the background color.
                                % You must add \usepackage{color}
 showspaces=false, % show spaces adding particular underscores
 showstringspaces=false, % underline spaces within strings
 showtabs=false, % show tabs within strings adding particular underscores
 frame=single, % adds a frame around the code
 rulecolor=\color{black}, % if not set, the frame-color may be changed on
                          % line-breaks within not-black text
 tabsize=2, % sets default tabsize to 2 spaces
 captionpos=b, % sets the caption-position to bottom
 breaklines=true, % sets automatic line breaking
 breakatwhitespace=false, % sets if automatic breaks should only happen
                          % at whitespace
 keywordstyle=\color{black}, % keyword style
 commentstyle=\color{silverSAS}, % comment style
 stringstyle=\color{black}, % string literal style
 basicstyle=\footnotesize\ttfamily,
}
\usepackage{multirow}
\usepackage{multicol}
%% nice table ----------------------------------------------------------
\usepackage{bbding} % checkmarks
\usepackage{color, colortbl} % coloring rows or columns in tables
\definecolor{Gray}{gray}{.9} % coloring rows or columns in tables
%% ---------------------------------------------------------------------
\usepackage[noend]{algpseudocode}
\makeatletter
\renewcommand{\ALG@name}{ALGORITHM}
\makeatother
% ----------------------------------------------------------------------
\usepackage{xpatch}
\makeatletter

\xpatchcmd{\algorithmic}{\itemsep\z@}{\itemsep=0.5ex plus2pt}{}{}
\makeatother
% ----------------------------------------------------------------------
\usepackage{tikzit}
%%----------------------------------------------------------------------
%%                                                     Henrique Laureano
%%                                            henriquelaureano.github.io
%%                                      2021-mar-05 · Curitiba/PR/Brazil
%%----------------------------------------------------------------------
\documentclass[12pt, % tamanho da fonte
               openright, % capítulos começam em pág ímpar
                          % (insere página vazia caso preciso)
               oneside, % para impressão apenas em um lado do papel
               a4paper, % tamanho do papel.
               chapter=TITLE, % títulos de capítulos convertidos em
                              % letras maiúsculas
               section=TITLE, % títulos de seções convertidos em letras
                              % maiúsculas
               % subsection=TITLE, % títulos de subseções convertidos em
                                 % letras maiúsculas
               % subsubsection=TITLE, % títulos de subsubseções em
                                    % letras maiúsculas
               % sumario=tradicional,
               brazil,
               english % o último idioma é o principal do documento
]{abntex2}
% ----------------------------------------------------------------------
% \renewcommand{\ABNTEXsectionfont}{\normalfont}
\renewcommand{\cftsectionfont}{\normalsize}
% ----------------------------------------------------------------------
\renewcommand{\ABNTEXchapterfontsize}{\bfseries\large}
\renewcommand{\ABNTEXsectionfontsize}{\large}
\renewcommand{\ABNTEXsubsectionfontsize}{\large}
% ----------------------------------------------------------------------
\usepackage{lastpage}
% \usepackage{times}
\usepackage{etoolbox}
\usepackage{lmodern} % usa a fonte Latin Modern
% \usepackage[T1]{fontenc} % selecao de codigos de fonte
\usepackage[utf8]{inputenc} % sodificacao do documento (conversão
                            % automática dos acentos)
% \usepackage{lastpage} % ssado pela ficha catalográfica
\usepackage{indentfirst} % indenta o primeiro parágrafo de cada seção
\setlength{\parindent}{1.5cm} % espaçamento de 1.5cm do parágrafo
\usepackage{color} %red, green, blue, yellow, cyan, magenta, black, white
\usepackage{graphicx} % inclusão de gráficos
\usepackage{microtype} % para melhorias de justificação
\usepackage{setspace}
\usepackage{pdflscape} % pagina na horizontal
% \usepackage{lscape} % pagina na horizontal
\usepackage{wrapfig}
\usepackage{booktabs}
\usepackage{makecell}
\usepackage{pgf} % para geração de dummy text
\usepackage[english, hyperpageref]{backref} % paginas com as citações
% ----------------------------------------------------------------------
% quadro cinza =========================================================
% \usepackage{caption}
\usepackage[small]{caption}
\usepackage{tikz} % permite desenhos vetoriais
% define o ambiente para colocar o código R
% (draw=gray!50 fill=gray!20 originais)
\tikzstyle{mybox} = [draw=gray!95, fill=white, very thick, rectangle,
                     inner sep=7pt, inner ysep=7pt]
\tikzstyle{mybox2} = [draw=gray!50, fill=gray!20,
                      very thick, % para códigos em R (Apêndices)
                      rectangle, inner sep=7pt, inner ysep=7pt]
% \usepackage[scaled=0.8]{beramono} % usa esta nos verbatins
\usepackage{float} % controla e define objetos flutuantes
\usepackage{tocloft} % controla lista de objetos flutuantes
% ambiente flutuante para código R
\newcommand{\listofprogramname}{CODE} % nome dessa lista
\newlistof{program}{lol}{\listofprogramname} % configura os arquivos
                                             % auxiliares para fazer a
                                             % lista
\makeatletter % allows use of "@" before \begin{document}
% this creates a custom and simpler ruled box style
\newcommand\floatc@simplerule[2]{{\@fs@cfont #1 #2}\par}
\newcommand\fs@simplerule{\def\@fs@cfont{ % aqui o estilo de fonte, e.g.
                                          % \bfseries
  }\let\@fs@capt\floatc@simplerule
  \def\@fs@pre{} % antes do caption, pode ser uma régua
  \def\@fs@post{} % depois do float, pode ser uma régua
  \def\@fs@mid{\kern3pt} % espaço entre caption e corpo
  \let\@fs@iftopcapt\iftrue}
\floatstyle{simplerule} % define o estilo do ambiente
\newfloat{program}{thp}{lol}[chapter] % ambiente contado por capítulo
\floatname{program}{Code} % rótulo que vai aparecer na legenda
\newcommand{\programautorefname}{Code} % altera o padrão do contador
                                       % para seguir capítulos
\renewcommand{\theprogram}{\thechapter.\arabic{program}} % capítulo.
                                                         % programa:
\sloppy
% FIM - TIKZ (quadro cinza) ============================================
% ----------------------------------------------------------------------
% CONFIGURAÇÕES DE PACOTES
% Configurações do pacote backref
% Usado sem a opção hyperpageref de backref
\renewcommand{\backrefpagesname}{Cited in the page(s):~}
% Texto padrão antes do número das páginas
\renewcommand{\backref}{}
% Define os textos da citação
\renewcommand*{\backrefalt}[4]{
  \ifcase #1 %
  None text citation.%
  \or
  Cited on page #2.%
  \else
  Cited #1 times on pages #2.%
  \fi}%
% ----------------------------------------------------------------------
% Citações padrão ABNT
\usepackage[alf, abnt-and-type=&, abnt-etal-list=0]{abntex2cite}
% ----------------------------------------------------------------------
% New comand for anexos
\newcommand{\refanexo}[1]{\hyperref[#1]{Annex~\ref{#1}}}
% \renewcommand{\cftsectionfont}{\sfseries}
% \refanexo{anexo_xxx} usar esse comando!
% ----------------------------------------------------------------------
% para titulo em destaque sem sequencia de numeração
\newcommand{\datatitle}[1]{
  \normalsize \textsc{#1}
}
\usepackage{pslatex}
% \usepackage{mathptmx} fonte - times new roman em tudo
% ----------------------------------------------------------------------
% adequando o uppercase titulo dos elementos nas suas respectivas LISTAS
\renewcommand{\cftfigurename}{FIGURE\enspace}
\renewcommand{\cfttablename}{TABLE\enspace}
% ----------------------------------------------------------------------
% fontes matemáticas
\usepackage{mathpazo}
\usepackage{inconsolata}
\usepackage{verbatim}
\usepackage{bm}
% ----------------------------------------------------------------------
% \usepackage[table, xcdraw]{xcolor} % cédula colorida em tabelas
% \usepackage{pdflscape} % rotaciona página
\usepackage{Capa} % capa e folha de rosto com modificações
\usepackage{float} % melhor posicionamento de figuras
\usepackage{gensymb} % símbolo º
\usepackage[justification=justified, singlelinecheck=false]{caption}
% \usepackage{etoolbox} % configurações adicionais de macros
\usepackage{xparse}
\usepackage{algorithm} % pacote usado para gerar pseudocódigo
% \usepackage{algorithmic} % pacote usado para gerar pseudocódigo
\definecolor{silverSAS}{RGB}{138,141,143}
\usepackage{listings}
\lstset{ 
 language=R, % the language of the code
 numbers=left, % where to put the line-numbers
 numbersep=6pt, % how far the line-numbers are from the code
 backgroundcolor=\color{white}, % choose the background color.
                                % You must add \usepackage{color}
 showspaces=false, % show spaces adding particular underscores
 showstringspaces=false, % underline spaces within strings
 showtabs=false, % show tabs within strings adding particular underscores
 frame=single, % adds a frame around the code
 rulecolor=\color{black}, % if not set, the frame-color may be changed on
                          % line-breaks within not-black text
 tabsize=2, % sets default tabsize to 2 spaces
 captionpos=b, % sets the caption-position to bottom
 breaklines=true, % sets automatic line breaking
 breakatwhitespace=false, % sets if automatic breaks should only happen
                          % at whitespace
 keywordstyle=\color{black}, % keyword style
 commentstyle=\color{silverSAS}, % comment style
 stringstyle=\color{black}, % string literal style
 basicstyle=\footnotesize\ttfamily,
}
\usepackage{multirow}
\usepackage{multicol}
%% nice table ----------------------------------------------------------
\usepackage{bbding} % checkmarks
\usepackage{color, colortbl} % coloring rows or columns in tables
\definecolor{Gray}{gray}{.9} % coloring rows or columns in tables
%% ---------------------------------------------------------------------
\usepackage[noend]{algpseudocode}
\makeatletter
\renewcommand{\ALG@name}{ALGORITHM}
\makeatother
% ----------------------------------------------------------------------
\usepackage{xpatch}
\makeatletter

\xpatchcmd{\algorithmic}{\itemsep\z@}{\itemsep=0.5ex plus2pt}{}{}
\makeatother
% ----------------------------------------------------------------------
\usepackage{tikzit}
%%----------------------------------------------------------------------
%%                                                     Henrique Laureano
%%                                            henriquelaureano.github.io
%%                                      2021-mar-05 · Curitiba/PR/Brazil
%%----------------------------------------------------------------------
\documentclass[12pt, % tamanho da fonte
               openright, % capítulos começam em pág ímpar
                          % (insere página vazia caso preciso)
               oneside, % para impressão apenas em um lado do papel
               a4paper, % tamanho do papel.
               chapter=TITLE, % títulos de capítulos convertidos em
                              % letras maiúsculas
               section=TITLE, % títulos de seções convertidos em letras
                              % maiúsculas
               % subsection=TITLE, % títulos de subseções convertidos em
                                 % letras maiúsculas
               % subsubsection=TITLE, % títulos de subsubseções em
                                    % letras maiúsculas
               % sumario=tradicional,
               brazil,
               english % o último idioma é o principal do documento
]{abntex2}
% ----------------------------------------------------------------------
% \renewcommand{\ABNTEXsectionfont}{\normalfont}
\renewcommand{\cftsectionfont}{\normalsize}
% ----------------------------------------------------------------------
\renewcommand{\ABNTEXchapterfontsize}{\bfseries\large}
\renewcommand{\ABNTEXsectionfontsize}{\large}
\renewcommand{\ABNTEXsubsectionfontsize}{\large}
% ----------------------------------------------------------------------
\usepackage{lastpage}
% \usepackage{times}
\usepackage{etoolbox}
\usepackage{lmodern} % usa a fonte Latin Modern
% \usepackage[T1]{fontenc} % selecao de codigos de fonte
\usepackage[utf8]{inputenc} % sodificacao do documento (conversão
                            % automática dos acentos)
% \usepackage{lastpage} % ssado pela ficha catalográfica
\usepackage{indentfirst} % indenta o primeiro parágrafo de cada seção
\setlength{\parindent}{1.5cm} % espaçamento de 1.5cm do parágrafo
\usepackage{color} %red, green, blue, yellow, cyan, magenta, black, white
\usepackage{graphicx} % inclusão de gráficos
\usepackage{microtype} % para melhorias de justificação
\usepackage{setspace}
\usepackage{pdflscape} % pagina na horizontal
% \usepackage{lscape} % pagina na horizontal
\usepackage{wrapfig}
\usepackage{booktabs}
\usepackage{makecell}
\usepackage{pgf} % para geração de dummy text
\usepackage[english, hyperpageref]{backref} % paginas com as citações
% ----------------------------------------------------------------------
% quadro cinza =========================================================
% \usepackage{caption}
\usepackage[small]{caption}
\usepackage{tikz} % permite desenhos vetoriais
% define o ambiente para colocar o código R
% (draw=gray!50 fill=gray!20 originais)
\tikzstyle{mybox} = [draw=gray!95, fill=white, very thick, rectangle,
                     inner sep=7pt, inner ysep=7pt]
\tikzstyle{mybox2} = [draw=gray!50, fill=gray!20,
                      very thick, % para códigos em R (Apêndices)
                      rectangle, inner sep=7pt, inner ysep=7pt]
% \usepackage[scaled=0.8]{beramono} % usa esta nos verbatins
\usepackage{float} % controla e define objetos flutuantes
\usepackage{tocloft} % controla lista de objetos flutuantes
% ambiente flutuante para código R
\newcommand{\listofprogramname}{CODE} % nome dessa lista
\newlistof{program}{lol}{\listofprogramname} % configura os arquivos
                                             % auxiliares para fazer a
                                             % lista
\makeatletter % allows use of "@" before \begin{document}
% this creates a custom and simpler ruled box style
\newcommand\floatc@simplerule[2]{{\@fs@cfont #1 #2}\par}
\newcommand\fs@simplerule{\def\@fs@cfont{ % aqui o estilo de fonte, e.g.
                                          % \bfseries
  }\let\@fs@capt\floatc@simplerule
  \def\@fs@pre{} % antes do caption, pode ser uma régua
  \def\@fs@post{} % depois do float, pode ser uma régua
  \def\@fs@mid{\kern3pt} % espaço entre caption e corpo
  \let\@fs@iftopcapt\iftrue}
\floatstyle{simplerule} % define o estilo do ambiente
\newfloat{program}{thp}{lol}[chapter] % ambiente contado por capítulo
\floatname{program}{Code} % rótulo que vai aparecer na legenda
\newcommand{\programautorefname}{Code} % altera o padrão do contador
                                       % para seguir capítulos
\renewcommand{\theprogram}{\thechapter.\arabic{program}} % capítulo.
                                                         % programa:
\sloppy
% FIM - TIKZ (quadro cinza) ============================================
% ----------------------------------------------------------------------
% CONFIGURAÇÕES DE PACOTES
% Configurações do pacote backref
% Usado sem a opção hyperpageref de backref
\renewcommand{\backrefpagesname}{Cited in the page(s):~}
% Texto padrão antes do número das páginas
\renewcommand{\backref}{}
% Define os textos da citação
\renewcommand*{\backrefalt}[4]{
  \ifcase #1 %
  None text citation.%
  \or
  Cited on page #2.%
  \else
  Cited #1 times on pages #2.%
  \fi}%
% ----------------------------------------------------------------------
% Citações padrão ABNT
\usepackage[alf, abnt-and-type=&, abnt-etal-list=0]{abntex2cite}
% ----------------------------------------------------------------------
% New comand for anexos
\newcommand{\refanexo}[1]{\hyperref[#1]{Annex~\ref{#1}}}
% \renewcommand{\cftsectionfont}{\sfseries}
% \refanexo{anexo_xxx} usar esse comando!
% ----------------------------------------------------------------------
% para titulo em destaque sem sequencia de numeração
\newcommand{\datatitle}[1]{
  \normalsize \textsc{#1}
}
\usepackage{pslatex}
% \usepackage{mathptmx} fonte - times new roman em tudo
% ----------------------------------------------------------------------
% adequando o uppercase titulo dos elementos nas suas respectivas LISTAS
\renewcommand{\cftfigurename}{FIGURE\enspace}
\renewcommand{\cfttablename}{TABLE\enspace}
% ----------------------------------------------------------------------
% fontes matemáticas
\usepackage{mathpazo}
\usepackage{inconsolata}
\usepackage{verbatim}
\usepackage{bm}
% ----------------------------------------------------------------------
% \usepackage[table, xcdraw]{xcolor} % cédula colorida em tabelas
% \usepackage{pdflscape} % rotaciona página
\usepackage{Capa} % capa e folha de rosto com modificações
\usepackage{float} % melhor posicionamento de figuras
\usepackage{gensymb} % símbolo º
\usepackage[justification=justified, singlelinecheck=false]{caption}
% \usepackage{etoolbox} % configurações adicionais de macros
\usepackage{xparse}
\usepackage{algorithm} % pacote usado para gerar pseudocódigo
% \usepackage{algorithmic} % pacote usado para gerar pseudocódigo
\definecolor{silverSAS}{RGB}{138,141,143}
\usepackage{listings}
\lstset{ 
 language=R, % the language of the code
 numbers=left, % where to put the line-numbers
 numbersep=6pt, % how far the line-numbers are from the code
 backgroundcolor=\color{white}, % choose the background color.
                                % You must add \usepackage{color}
 showspaces=false, % show spaces adding particular underscores
 showstringspaces=false, % underline spaces within strings
 showtabs=false, % show tabs within strings adding particular underscores
 frame=single, % adds a frame around the code
 rulecolor=\color{black}, % if not set, the frame-color may be changed on
                          % line-breaks within not-black text
 tabsize=2, % sets default tabsize to 2 spaces
 captionpos=b, % sets the caption-position to bottom
 breaklines=true, % sets automatic line breaking
 breakatwhitespace=false, % sets if automatic breaks should only happen
                          % at whitespace
 keywordstyle=\color{black}, % keyword style
 commentstyle=\color{silverSAS}, % comment style
 stringstyle=\color{black}, % string literal style
 basicstyle=\footnotesize\ttfamily,
}
\usepackage{multirow}
\usepackage{multicol}
%% nice table ----------------------------------------------------------
\usepackage{bbding} % checkmarks
\usepackage{color, colortbl} % coloring rows or columns in tables
\definecolor{Gray}{gray}{.9} % coloring rows or columns in tables
%% ---------------------------------------------------------------------
\usepackage[noend]{algpseudocode}
\makeatletter
\renewcommand{\ALG@name}{ALGORITHM}
\makeatother
% ----------------------------------------------------------------------
\usepackage{xpatch}
\makeatletter

\xpatchcmd{\algorithmic}{\itemsep\z@}{\itemsep=0.5ex plus2pt}{}{}
\makeatother
% ----------------------------------------------------------------------
\usepackage{tikzit}
\input{thesis.tikzstyles}
% ----------------------------------------------------------------------
% pacote para fazer o checkmark
\usepackage{pifont} % http://ctan.org/pkg/pifont
\newcommand{\cmark}{\ding{51}}%
\newcommand{\xmark}{\ding{55}}%
% ----------------------------------------------------------------------
\usepackage{amsmath}
\usepackage{amsfonts}
\usepackage{amssymb}
\usepackage{pdfpages}
% \usepackage{times}
% \usepackage{helvet}
% \renewcommand{\familydefault}{\sfdefault}
% ----------------------------------------------------------------------
\NewDocumentCommand\cc{+u{\cc}}{\ignorespaces}
% ----------------------------------------------------------------------
% controle do espaçamento entre um parágrafo e outro:
\setlength{\parskip}{0.2cm} % tente também \onelineskip
% ----------------------------------------------------------------------
\titulo{MODELING THE CUMULATIVE INCIDENCE FUNCTION OF CLUSTERED
  COMPETING RISKS DATA: A MULTINOMIAL GLMM APPROACH}
\autor{HENRIQUE APARECIDO LAUREANO}
\data{2021}
\instituicao{FEDERAL UNIVERSITY OF PARANÁ}
\orientador{Prof. PhD Wagner Hugo Bonat}
\coorientador{Prof. PhD Paulo Justiniano Ribeiro Jr}
\tipotrabalho{Dissertação (mestrado)}
\preambulo{\small{Thesis presented to the Graduate Program of Numerical
    Methods in Engineering, Concentration Area in Mathematical
    Programming: Statistical Methods Applied in Engineering, Federal
    University of Paran\'{a}, as part of the requirements to the
    obtention of the Master's Degree in Sciences.}}
% ----------------------------------------------------------------------
% informações do PDF
\makeatletter
\hypersetup{
  % pagebackref=true,
  pdftitle={\@title},
  pdfauthor={\@author},
  pdfsubject={\imprimirpreambulo},
  % pdfkeywords = {}{}{}{},
  colorlinks=true, % false: boxed links; true: colored links
  linkcolor=blue, % color of internal links
  citecolor=blue, % color of links to bibliography
  filecolor=magenta, % color of file links
  urlcolor=blue,
  bookmarksdepth=4
}
\addto\captionsenglish{
  % adjusts names from abnTeX2
  \renewcommand{\folhaderostoname}{Title page}
  \renewcommand{\epigraphname}{Epigraph}
  \renewcommand{\dedicatorianame}{Dedication}
  \renewcommand{\errataname}{Errata sheet}
  \renewcommand{\agradecimentosname}{Acknowledgements}
  \renewcommand{\anexoname}{ANNEX}
  \renewcommand{\anexosname}{Annex}
  \renewcommand{\apendicename}{APPENDIX}
  \renewcommand{\apendicesname}{Appendix}
  \renewcommand{\orientadorname}{Supervisor:}
  \renewcommand{\coorientadorname}{Co-supervisor:}
  \renewcommand{\folhadeaprovacaoname}{Approval}
  \renewcommand{\resumoname}{Abstract}
  \renewcommand{\listadesiglasname}{List of abbreviations and acronyms}
  \renewcommand{\listadesimbolosname}{List of symbols}
  \renewcommand{\fontename}{Source}
  \renewcommand{\notaname}{Note}
  % adjusts names used by \autoref
  \renewcommand{\pageautorefname}{page}
  \renewcommand{\chapterautorefname}{Chapter}
  \renewcommand{\sectionautorefname}{Section}
  \renewcommand{\subsectionautorefname}{subsection}
  \renewcommand{\subsubsectionautorefname}{subsubsection}
  \renewcommand{\paragraphautorefname}{subsubsubsection}
}
\makeatother
% ----------------------------------------------------------------------
\graphicspath{{figures/}}
% ----------------------------------------------------------------------
\begin{document}
\selectlanguage{english}
% adequando o uppercase titulo dos elementos nas suas respectivas
% legendas
\renewcommand{\tablename}{TABLE }
\renewcommand{\figurename}{FIGURE }
% ----------------------------------------------------------------------
\frenchspacing % retira espaço extra obsoleto entre as frases
% ----------------------------------------------------------------------
% capa
\tikz[remember picture,overlay] \node[opacity=1,inner sep=0pt] at
(current page.center){
  \includegraphics[width=\paperwidth,
  height=\paperheight]{Figuras/ufpr_bg}};
% ----------------------------------------------------------------------
\imprimircapa
% ----------------------------------------------------------------------
% folha de rosto
\imprimirfolhaderosto
% ----------------------------------------------------------------------
% \begin{dedicatoria}
%   \vspace*{\fill}
%   ...
%   \vspace*{\fill}
% \end{dedicatoria}
% ----------------------------------------------------------------------
% ficha catalográfica

% \begin{fichacatalografica}
%   \includepdf{ficha.pdf}
% \end{fichacatalografica}
% ----------------------------------------------------------------------
% inserir folha de aprovação
% \begin{folhadeaprovacao}
%   \includepdf{termo.pdf}
% \end{folhadeaprovacao}
\begin{folhadeaprovacao}
 \begin{center}
   {\ABNTEXchapterfont\large\imprimirautor}

   \vspace*{\fill}\vspace*{\fill}
   \begin{center}
     \ABNTEXchapterfont\bfseries\large\imprimirtitulo
   \end{center}
   \vspace*{\fill}

    \hspace{.45\textwidth}
    \begin{minipage}{.5\textwidth}
       \imprimirpreambulo
    \end{minipage}
   \vspace*{\fill}
 \end{center}

 Master thesis approved. XXX XX, 2021.

  \assinatura{\textbf{\imprimirorientador}\\ Supervisor}
  \assinatura{\textbf{Prof. PhD Paulo Justiniano Ribeiro Jr}\\
    Co-supervisor}
  \assinatura{\textbf{Prof. PhD \(\dots\)}\\
    Internal Examinator - PPGMNE}
  \assinatura{\textbf{Prof. PhD \(\dots\)}\\
    Internal Examinator - PPGMNE}
  \assinatura{\textbf{Prof. PhD \(\dots\)}\\External Examiner - }

  \begin{center}
   \vspace*{0.5cm}
   {\large CURITIBA}
   \par
   {\large\imprimirdata}
   \vspace*{1cm}
 \end{center}

\end{folhadeaprovacao}
% ----------------------------------------------------------------------
\begin{dedicatoria}
  \vspace*{22.7cm}
  \begin{flushright}
    \begin{minipage}[H]{4.5cm}
      {To Celita and Olivio}
    \end{minipage}
  \end{flushright}
\end{dedicatoria}
% ----------------------------------------------------------------------
\begin{agradecimentos}
  As Moro said once, I'm thankful for everything and everyone.
\end{agradecimentos}

\begin{epigrafe}
  \vspace*{\fill}
  \begin{flushright}
    \textit{"It's not supposed to be easy."\\
             (Gregg Popovich)}
              % on Sao Antonio Spurs \(\times\) Oklahoma City Thunder,
              % first game of the 2012 Western Conference Finais
  \end{flushright}
\end{epigrafe}
% ----------------------------------------------------------------------
\newpage
\setlength{\absparsep}{18pt} % ajusta o espaçamento dos parágrafos do
                             % resumo
\setlength{\abstitleskip}{1cm} % adiciona mais um cm após o 'titulo' do
                               % resumo para ficar com 2cm
\begin{resumo}[]
  \vspace{-2cm}
  \begin{center}
    \bfseries{\large{\textsf{ABSTRACT}}}
  \end{center}
  \vspace{0.3cm}
  Failure time data \(\dots\)\\

  \textbf{Keywords}: Competing risks.
\end{resumo}
% ----------------------------------------------------------------------
\newpage
\setlength{\absparsep}{18pt} % ajusta o espaçamento dos parágrafos do
                             % resumo
\setlength{\abstitleskip}{1cm} % adiciona mais um cm após o 'titulo' do
                               % resumo para ficar com 2cm
\begin{resumo}[]
  \begin{otherlanguage*}{brazil}
    \vspace{-2cm}
    \begin{center}
      \bfseries{\large{\textsf{RESUMO}}}
    \end{center}
    \vspace{0.3cm}
    Dados de tempos de falha \(\dots\)\\

    \textbf{Palavras-chave}: Riscos competitivos.
  \end{otherlanguage*}
\end{resumo}
% ----------------------------------------------------------------------
\pdfbookmark[0]{\listfigurename}{lof}
\listoffigures*
\cleardoublepage
% ----------------------------------------------------------------------
\pdfbookmark[0]{\listtablename}{lot}
\listoftables*
\cleardoublepage
% ----------------------------------------------------------------------
\makeatletter
\renewcommand\numberline[1]{
	\leftskip -0.7em
	\rightskip 1.6em
	\parfillskip -\rightskip
	\parindent 0em
	\@tempdima 2.0em
	\vspace{0em}
  \advance\leftskip \@tempdima \null\nobreak\hskip -\leftskip
	ALGORITHM \normalfont #1 ~~ }
\makeatother
% ----------------------------------------------------------------------
\pdfbookmark[0]{\listalgorithmname}{loa}
\listofalgorithms
\cleardoublepage
% ----------------------------------------------------------------------
\makeatletter
\def\numberline#1{\hb@xt@\@tempdima{#1\hfil}}
\makeatother
% ----------------------------------------------------------------------
% \begin{siglas}
% \item[Fig.] Area of the $i^{th}$ component
% \item[456] Isto é um número
% \item[123] Isto é outro número
% \item[lauro cesar] este é o meu nome
% \end{siglas}
% ----------------------------------------------------------------------
% \begin{simbolos}
% \item[\(\mathbb{E}(\cdot)\)] The mathematical expectation of a random
%   variable \(\cdot\)
% \end{simbolos}
% ----------------------------------------------------------------------
\pdfbookmark[0]{\contentsname}{toc}
\tableofcontents*
\cleardoublepage
% ----------------------------------------------------------------------
\makepagestyle{abntheadings}
\makeevenhead{abntheadings}{\ABNTEXfontereduzida\thepage}{}{}
\makeoddhead{abntheadings}{}{}{\ABNTEXfontereduzida\thepage}
\makeheadrule{abntheadings}{\textwidth}{0in}
% ----------------------------------------------------------------------
\textual
% ----------------------------------------------------------------------
\chapter{Introduction}
\label{cap:intro}
\input{./modules/intro}
% ----------------------------------------------------------------------
\chapter{Generalized linear mixed models: formulation, optimization, and
  implementation}
\label{cap:methods}
\input{./modules/methods}
% ----------------------------------------------------------------------
\chapter{\(\text{multiGLMM}\): a multinomial GLMM for clustered
  competing risks data}
\label{cap:model}
\input{./modules/model}
% ----------------------------------------------------------------------
\chapter{simulation study datasets}
\label{cap:datasets}
\input{./modules/datasets}
% ----------------------------------------------------------------------
\chapter{Results}
\label{cap:results}
\input{./modules/results}
% ----------------------------------------------------------------------
\chapter{Final considerations}
\label{cap:finalc}
\input{./modules/finalc}
% ----------------------------------------------------------------------
\setlength{\afterchapskip}{\baselineskip}
% ----------------------------------------------------------------------
\bibliography{references}
% ----------------------------------------------------------------------
\postextual
% ----------------------------------------------------------------------
\begin{apendicesenv}
\partapendices
\addcontentsline{toc}{chapter}{\hspace{2.105cm}APPENDIX}
\renewcommand{\ABNTEXchapterfontsize}{\ABNTEXsectionfont}

\chapter{ANALYTIC GRADIENT OF THE LATENT EFFECTS FOR THE JOINT
         LOG-LIKELIHOOD FUNCTION OF THE MULTINOMIAL GLMM FOR CLUSTERED
         COMPETING RISKS DATA}
\label{cap:appendixA}

The following gradient components are computed by cluster, to be used
e.g., in a Newton optimization. Subject \(i\) at cluster \(j\) and for
competing cause \(k\)

\begin{align*}
  &\frac{\partial}{\partial u_{kj}}
    \log L(\bm{\theta}\mid\bm{y}_{j}, \bm{r}_{j}) =\\
  &y_{kij}\frac{1 +
    \sum_{m \neq k}^{K-1}\exp\{\bm{x}_{mij}\bm{\beta}_{mj} + u_{mj}\}
    }{1 +
    \sum_{n = 1}^{K-1}\exp\{\bm{x}_{nij}\bm{\beta}_{nj} + u_{nj}\}} -
    \left(\sum_{m \neq k}^{K-1} y_{mij}\right)
    \frac{\exp\{\bm{x}_{kij}\bm{\beta}_{kj} + u_{kj}\}
    }{1 +
    \sum_{n = 1}^{K-1}\exp\{\bm{x}_{nij}\bm{\beta}_{nj} + u_{nj}\}}-\\
  &y_{Kij}\frac{1}{1 +
    \sum_{n = 1}^{K-1}\exp\{\bm{x}_{nij}\bm{\beta}_{nj} + u_{nj}\}
    }\Bigg(\\
  &\frac{\exp\{\bm{x}_{kij}\bm{\beta}_{kj} + u_{kj}\}
    \left(1 +
    \sum_{m \neq k}^{K-1}\exp\{\bm{x}_{mij}\bm{\beta}_{mj} + u_{mj}\}
    \right)}{
    1 + \sum_{n = 1}^{K-1}\exp\{\bm{x}_{nij}\bm{\beta}_{nj} + u_{nj}\}
    }\times\\
  &\frac{w_{k}\frac{\delta}{2\delta t - 2t^{2}}
    \phi[w_{k}\text{arctanh}\left(\frac{t-\delta/2}{\delta/2}\right)
    - \bm{x}_{kij}\bm{\gamma}_{kj} - \eta_{kj}
    ]}{1 - w_{n}\frac{\delta}{2\delta t - 2t^{2}}
    \phi[w_{n}\text{arctanh}\left(\frac{t-\delta/2}{\delta/2}\right)
    - \bm{x}_{nij}\bm{\gamma}_{nj} - \eta_{nj}]} -
    \frac{\exp\{\bm{x}_{kij}\bm{\beta}_{kj} + u_{kj}\}}{
    1 + \sum_{n = 1}^{K-1}\exp\{\bm{x}_{nij}\bm{\beta}_{nj} + u_{nj}\}}
    \times\\
  &\frac{
    \sum_{m \neq k}^{K-1}
    w_{m}\frac{\delta}{2\delta t - 2t^{2}}
    \phi[w_{m}\text{arctanh}\left(\frac{t-\delta/2}{\delta/2}\right)
    - \bm{x}_{mij}\bm{\gamma}_{mj} - \eta_{mj}]
    \exp\{\bm{x}_{mij}\bm{\beta}_{mj} + u_{mj}\}}{
    1 - w_{n}\frac{\delta}{2\delta t - 2t^{2}}
    \phi[w_{n}\text{arctanh}\left(\frac{t-\delta/2}{\delta/2}\right)
    - \bm{x}_{nij}\bm{\gamma}_{nj} - \eta_{nj}]}\Bigg) -\\
  &\bm{e_{k}^{\top}Qr_{j}},\\
  %% -------------------------------------------------------------------
  \\
  %% -------------------------------------------------------------------
  &\frac{\partial}{\partial \eta_{kj}}
  \log L(\bm{\theta}\mid\bm{y}_{j}, \bm{r}_{j}) =\\
  &y_{kij} (w_{k}\text{arctanh}\left(\frac{t-\delta/2}{\delta/2}\right)
    - \bm{x}_{kij}\bm{\gamma}_{kj} - \eta_{kj}) -\\
  &y_{Kij}\frac{\exp\{\bm{x}_{kij}\bm{\beta}_{kj} + u_{kj}\}
    }{1 + \sum_{n = 1}^{K-1}\exp\{\bm{x}_{nij}\bm{\beta}_{nj} + u_{nj}\}}
    \times\\
  &\frac{
    w_{k}\frac{\delta}{2\delta t - 2t^{2}}
    (w_{k}\text{arctanh}\left(\frac{t-\delta/2}{\delta/2}\right)
    - \bm{x}_{kij}\bm{\gamma}_{kj} - \eta_{kj})
    \phi[w_{k} \text{arctanh}\left(\frac{t-\delta/2}{\delta/2}\right)
    - \bm{x}_{kij}\bm{\gamma}_{kj} - \eta_{kj}
    ]}{1 -
    \sum_{n = 1}^{K-1}
    \frac{\exp\{\bm{x}_{nij}\bm{\beta}_{nj} + u_{nj}\}}{1 +
    \sum_{n = 1}^{K-1}\exp\{\bm{x}_{nij}\bm{\beta}_{nj} + u_{nj}\}}
    w_{n}\frac{\delta}{2\delta t - 2t^{2}}
    \phi[w_{n}\text{arctanh}\left(\frac{t-\delta/2}{\delta/2}\right)
    - \bm{x}_{nij}\bm{\gamma}_{nj} - \eta_{nj}]} -\\
  &\bm{e_{k}^{\top}Qr_{j}},
\end{align*}
with \(\bm{e_{k}^{\top}}\) begin a vector with \(1\) at the \(k\)-th
position and zero elsewhere.

\chapter{ANALYTIC HESSIAN OF THE LATENT EFFECTS FOR THE JOINT
         LOG-LIKELIHOOD FUNCTION OF THE MULTINOMIAL GLMM FOR CLUSTERED
         COMPETING RISKS DATA}
\label{cap:appendixB}

The following hessian components are computed by cluster, to be used
e.g., in a Newton optimization. Subject \(i\) at cluster \(j\) and for
competing cause \(k\)
\begin{align*}
  &\frac{\partial^{2}}{\partial u_{kj}^{2}}
    \log L(\bm{\theta}\mid\bm{y}_{j}, \bm{r}_{j}) =\\
  &-\frac{\left(\sum_{k = 1}^{K-1} y_{kij}\right)
    \exp\{\bm{x}_{kij}\bm{\beta}_{kj} + u_{kj}\}
    \left(1 +
    \sum_{m \neq k}^{K-1}\exp\{\bm{x}_{mij}\bm{\beta}_{mj} + u_{mj}\}
    \right)}{\left(1 +
    \sum_{n = 1}^{K-1}\exp\{\bm{x}_{nij}\bm{\beta}_{nj} + u_{nj}\}
    \right)^{2}} +\\
  &\frac{y_{Kij}
    \exp\{\bm{x}_{kij} \bm{\beta}_{kj} + u_{kj}\}}{
    1 + \sum_{n = 1}^{K-1}\exp\{\bm{x}_{nij} \bm{\beta}_{nj} + u_{nj}\}
    }\times\\
  &\frac{
    \sum_{m \neq k}^{K-1}w_{m}\frac{\delta}{2\delta t - 2t^{2}}
    \phi[w_{m}\text{arctanh}\left(\frac{t-\delta/2}{\delta/2}\right)
    - \bm{x}_{mij}\bm{\gamma}_{mj} - \eta_{mj}]
    \exp\{\bm{x}_{mij}\bm{\beta}_{mj} + u_{mj}\}}{1 +
    \sum_{n = 1}^{K-1}\exp\{\bm{x}_{nij}\bm{\beta}_{nj} + u_{nj}\}
    (1 - w_{n}\frac{\delta}{2\delta t - 2t^{2}}
    \phi[w_{n}\text{arctanh}\left(\frac{t-\delta/2}{\delta/2}\right)
    - \bm{x}_{nij}\bm{\gamma}_{nj} - \eta_{nj}])} -\\
  &\frac{
    y_{Kij}
    w_{k}\frac{\delta}{2\delta t - 2t^{2}}
    \phi[w_{k}\text{arctanh}\left(\frac{t-\delta/2}{\delta/2}\right)
    - \bm{x}_{kij}\bm{\gamma}_{kj} - \eta_{kj}] }{1 +
    \sum_{n = 1}^{K-1}\exp\{\bm{x}_{nij}\bm{\beta}_{nj} + u_{nj}\}
    }\times\\
  &\frac{\exp\{\bm{x}_{kij}\bm{\beta}_{kj} + u_{kj}\}
    \left(1 +
    \sum_{m \neq k}^{K-1}\exp\{\bm{x}_{mij}\bm{\beta}_{mj} + u_{mj}\}
    \right)}{1 +
    \sum_{n = 1}^{K-1}\exp\{\bm{x}_{nij}\bm{\beta}_{nj} + u_{nj}\}
    (1 - w_{n}\frac{\delta}{2\delta t - 2t^{2}}
    \phi[w_{n}\text{arctanh}\left(\frac{t-\delta/2}{\delta/2}\right)
    - \bm{x}_{nij}\bm{\gamma}_{nj} - \eta_{nj}])} -\\
  &\frac{y_{Kij}\exp\{\bm{x}_{kij}\bm{\beta}_{kj} + u_{kj}\}}{\left(1 +
    \sum_{n = 1}^{K-1}\exp\{\bm{x}_{nij}\bm{\beta}_{nj} + u_{nj}\}
    \right)^{2}}\Bigg(\\
  &\frac{\sum_{m \neq k}^{K-1}
    w_{m}\frac{\delta}{2\delta t - 2t^{2}}
    \phi[w_{m}\text{arctanh}\left(\frac{t-\delta/2}{\delta/2}\right)
    - \bm{x}_{mij}\bm{\gamma}_{mj} - \eta_{mj}]
    \exp\{\bm{x}_{mij}\bm{\beta}_{mj} + u_{mj}\}}{\left(1 +
    \sum_{n = 1}^{K-1}\exp\{\bm{x}_{nij}\bm{\beta}_{nj} + u_{nj}\}
    (1 - w_{n}\frac{\delta}{2\delta t - 2t^{2}}
    \phi[w_{n}\text{arctanh}\left(\frac{t-\delta/2}{\delta/2}\right)
    - \bm{x}_{nij}\bm{\gamma}_{nj} - \eta_{nj}])\right)^{2}}-\\
  &\frac{w_{k}\frac{\delta}{2\delta t - 2t^{2}}
    \phi[w_{k}\text{arctanh}\left(\frac{t-\delta/2}{\delta/2}\right)
    - \bm{x}_{kij}\bm{\gamma}_{kj} - \eta_{kj}]\left(1 +
    \sum_{m \neq k}^{K-1}\exp\{\bm{x}_{mij}\bm{\beta}_{mj} + u_{mj}\}
    \right)}{\left(1 +
    \sum_{n = 1}^{K-1}\exp\{\bm{x}_{nij}\bm{\beta}_{nj} + u_{nj}\}
    (1 - w_{n}\frac{\delta}{2\delta t - 2t^{2}}
    \phi[w_{n}\text{arctanh}\left(\frac{t-\delta/2}{\delta/2}\right)
    - \bm{x}_{nij}\bm{\gamma}_{nj} - \eta_{nj}])\right)^{2}}\Bigg)\\
  &\times\Bigg(\Big(1 +\\
  &\sum_{n = 1}^{K-1}\exp\{\bm{x}_{nij}\bm{\beta}_{nj} + u_{nj}\}
    (1 - w_{n}\frac{\delta}{2\delta t - 2t^{2}}
    \phi[w_{n}\text{arctanh}\left(\frac{t-\delta/2}{\delta/2}\right)
    - \bm{x}_{nij}\bm{\gamma}_{nj} - \eta_{nj}])\Big) +\\
  &\Big(1 +
    \sum_{n = 1}^{K-1}\exp\{\bm{x}_{nij} \bm{\beta}_{nj} + u_{nj}\}
    \Big)\times\\
  &(1 - w_{k}\frac{\delta}{2\delta t - 2t^{2}}
    \phi[w_{k}\text{arctanh}\left(\frac{t-\delta/2}{\delta/2}\right)
    - \bm{x}_{kij}\bm{\gamma}_{kj} - \eta_{kj}])\Bigg)
    - \bm{e_{k}^{\top}Q},
\end{align*}

\begin{align*}
  &\frac{\partial^{2}}{\partial \eta_{kj}^{2}}
    \log L(\bm{\theta}\mid\bm{y}_{j}, \bm{r}_{j}) =\\
  &- y_{kij} - y_{Kij}
    \frac{\exp\{\bm{x}_{kij} \bm{\beta}_{kj} + u_{kj}\}}{1 +
    \sum_{n = 1}^{K-1}\exp\{\bm{x}_{nij} \bm{\beta}_{nj} + u_{nj}\}}\Bigg(\\
  &w_{k}\frac{\delta}{2\delta t - 2t^{2}}
    \phi[w_{k}\text{arctanh}\left(\frac{t-\delta/2}{\delta/2}\right)
    - \bm{x}_{kij}\bm{\gamma}_{kj} - \eta_{kj}]\times\\
  &\frac{\left(
    w_{k} \text{arctanh}\left(\frac{t-\delta/2}{\delta/2}\right)
    - \bm{x}_{kij}\bm{\gamma}_{kj} - \eta_{kj}
    \right)^{2} - 1}{
    1 - \sum_{n = 1}^{K-1}
    \frac{\exp\{\bm{x}_{nij}\bm{\beta}_{nj} + u_{nj}\}}{1 +
    \sum_{n = 1}^{K-1}\exp\{\bm{x}_{nij} \bm{\beta}_{nj} + u_{nj}\}}
    w_{n}\frac{\delta}{2\delta t - 2t^{2}}
    \phi[w_{n}\text{arctanh}\left(\frac{t-\delta/2}{\delta/2}\right)
    - \bm{x}_{nij}\bm{\gamma}_{nj} - \eta_{nj}]} -\\
  &\frac{\left(
    w_{k}\frac{\delta}{2\delta t - 2t^{2}}
    (w_{k}\text{arctanh}\left(\frac{t-\delta/2}{\delta/2}\right)
    - \bm{x}_{kij}\bm{\gamma}_{kj} - \eta_{kj})
    \phi[w_{k}\text{arctanh}\left(\frac{t-\delta/2}{\delta/2}\right)
    - \bm{x}_{kij}\bm{\gamma}_{kj} - \eta_{kj}]\right)^{2}}{\left(1 -
    \sum_{n = 1}^{K-1}
    \frac{\exp\{\bm{x}_{nij} \bm{\beta}_{nj} + u_{nj}\}}{1 +
    \sum_{n = 1}^{K-1}\exp\{\bm{x}_{nij} \bm{\beta}_{nj} + u_{nj}\}}
    w_{n}\frac{\delta}{2\delta t - 2t^{2}}
    \phi[w_{n}\text{arctanh}\left(\frac{t-\delta/2}{\delta/2}\right)
    - \bm{x}_{nij}\bm{\gamma}_{nj} - \eta_{nj}]\right)^{2}}\\
  &\Bigg) - \bm{e_{k}^{\top}Q},
\end{align*}

\begin{align*}
  &\frac{\partial^{2}}{\partial u_{kj} u_{mj}}
    \log L(\bm{\theta}\mid\bm{y}_{j}, \bm{r}_{j}) =\\
  &\left(\sum_{k = 1}^{K-1} y_{kij}\right)
    \frac{
    \exp\{\bm{x}_{kij}\bm{\beta}_{kj} + u_{kj}\}
    \exp\{\bm{x}_{mij}\bm{\beta}_{mj} + u_{mj}\}}{
    \left(1 +
    \sum_{n = 1}^{K-1}\exp\{\bm{x}_{nij} \bm{\beta}_{nj} + u_{nj}\}
    \right)^{2}} +\\
  &\frac{
    y_{Kij}
    \exp\{\bm{x}_{kij}\bm{\beta}_{kj} + u_{kj}\}
    \exp\{\bm{x}_{mij} \bm{\beta}_{mj} + u_{mj}\}}{1 +
    \sum_{n = 1}^{K-1}\exp\{\bm{x}_{nij} \bm{\beta}_{nj} + u_{nj}\}}\Bigg(\\
  &\frac{
    w_{m}\frac{\delta}{2\delta t - 2t^{2}}
    \phi[w_{m}\text{arctanh}\left(\frac{t-\delta/2}{\delta/2}\right)
    - \bm{x}_{mij}\bm{\gamma}_{mj} - \eta_{mj}]}{1 +
    \sum_{n = 1}^{K-1}\exp\{\bm{x}_{nij} \bm{\beta}_{nj} + u_{nj}\}
    (1 - w_{n}\frac{\delta}{2\delta t - 2t^{2}}
    \phi[w_{n}\text{arctanh}\left(\frac{t-\delta/2}{\delta/2}\right)
    - \bm{x}_{nij}\bm{\gamma}_{nj} - \eta_{nj}])} -\\
  &\frac{
    w_{k}\frac{\delta}{2\delta t - 2t^{2}}
    \phi[w_{k}\text{arctanh}\left(\frac{t-\delta/2}{\delta/2}\right)
    - \bm{x}_{kij}\bm{\gamma}_{kj} - \eta_{kj}]}{1 +
    \sum_{n = 1}^{K-1}\exp\{\bm{x}_{nij}\bm{\beta}_{nj} + u_{nj}\}
    (1 - w_{n}\frac{\delta}{2\delta t - 2t^{2}}
    \phi[w_{n}\text{arctanh}\left(\frac{t-\delta/2}{\delta/2}\right)
    - \bm{x}_{nij}\bm{\gamma}_{nj} - \eta_{nj}])}\Bigg) -\\
  &\frac{y_{Kij}}{
    \left(1 +
    \sum_{n = 1}^{K-1}\exp\{\bm{x}_{nij}\bm{\beta}_{nj} + u_{nj}\}
    \right)^{2}}\Bigg(\exp\{\bm{x}_{kij}\bm{\beta}_{kj} + u_{kj}\}\Bigg(\\
  &\frac{
    \sum_{m \neq k}^{K-1}
    w_{m}\frac{\delta}{2\delta t - 2t^{2}}
    \phi[w_{m}\text{arctanh}\left(\frac{t-\delta/2}{\delta/2}\right)
    - \bm{x}_{mij}\bm{\gamma}_{mj} - \eta_{mj}]
    \exp\{\bm{x}_{mij} \bm{\beta}_{mj} + u_{mj}\}}{
    \left(1 + \sum_{n = 1}^{K-1}\exp\{\bm{x}_{nij}\bm{\beta}_{nj} + u_{nj}\}
    (1 - w_{n}\frac{\delta}{2\delta t - 2t^{2}}
    \phi[w_{n}\text{arctanh}\left(\frac{t-\delta/2}{\delta/2}\right)
    - \bm{x}_{nij}\bm{\gamma}_{nj} - \eta_{nj}])\right)^{2}} -\\
  &\frac{
    w_{k}\frac{\delta}{2\delta t - 2t^{2}}
    \phi[w_{k}\text{arctanh}\left(\frac{t-\delta/2}{\delta/2}\right)
    - \bm{x}_{kij}\bm{\gamma}_{kj} - \eta_{kj}]
    \left(1 +
    \sum_{m \neq k}^{K-1}\exp\{\bm{x}_{mij}\bm{\beta}_{mj} + u_{mj}\}
    \right)}{\left(1 +
    \sum_{n = 1}^{K-1}\exp\{\bm{x}_{nij} \bm{\beta}_{nj} + u_{nj}\}
    (1 - w_{n}\frac{\delta}{2\delta t - 2t^{2}}
    \phi[w_{n}\text{arctanh}\left(\frac{t-\delta/2}{\delta/2}\right)
    - \bm{x}_{nij}\bm{\gamma}_{nj} - \eta_{nj}])\right)^{2}}\Bigg)
\end{align*}
\begin{align*}
  &\Bigg)\times\Bigg(\exp\{\bm{x}_{mij}\bm{\beta}_{mj} + u_{mj}\}
    \Big(1 +\\
  &\sum_{n = 1}^{K-1}\exp\{\bm{x}_{nij}\bm{\beta}_{nj} + u_{nj}\}
    (1 - w_{n}\frac{\delta}{2\delta t - 2t^{2}}
    \phi[w_{n}\text{arctanh}\left(\frac{t-\delta/2}{\delta/2}\right)
    - \bm{x}_{nij}\bm{\gamma}_{nj} - \eta_{nj}])\Big) +\\
  &\exp\{\bm{x}_{mij}\bm{\beta}_{mj} + u_{mj}\}
    (1 - w_{m}\frac{\delta}{2\delta t - 2t^{2}}
    \phi[w_{m}\text{arctanh}\left(\frac{t-\delta/2}{\delta/2}\right)
    - \bm{x}_{mij}\bm{\gamma}_{mj} - \eta_{mj}])\Big(1 +\\
  &\sum_{n = 1}^{K-1}\exp\{\bm{x}_{nij}\bm{\beta}_{nj} + u_{nj}\}\Big)
    \Bigg) - \bm{e_{k}^{\top}Q},
\end{align*}

\begin{align*}
  &\frac{\partial^{2}}{\partial \eta_{kj} \eta_{mj}}
    \log L(\bm{\theta}\mid\bm{y}_{j}, \bm{r}_{j}) =\\
  &- y_{Kij}\frac{
    \exp\{\bm{x}_{kij}\bm{\beta}_{kj} + u_{kj}\}}{1 +
    \sum_{n = 1}^{K-1}\exp\{\bm{x}_{nij}\bm{\beta}_{nj} + u_{nj}\}}\times\\
  &\frac{w_{k}\frac{\delta}{2\delta t - 2t^{2}}
    (w_{k}\text{arctanh}\left(\frac{t-\delta/2}{\delta/2}\right)
    - \bm{x}_{kij}\bm{\gamma}_{kj} - \eta_{kj})
    \phi[w_{k}\text{arctanh}\left(\frac{t-\delta/2}{\delta/2}\right)
    - \bm{x}_{kij}\bm{\gamma}_{kj} - \eta_{kj}]}{\left(1 -
    \sum_{n = 1}^{K-1}\frac{\exp\{\bm{x}_{nij}\bm{\beta}_{nj} + u_{nj}\}
    }{1 +
    \sum_{n = 1}^{K-1}\exp\{\bm{x}_{nij}\bm{\beta}_{nj} + u_{nj}\}}
    w_{n}\frac{\delta}{2\delta t - 2t^{2}}
    \phi[w_{n}\text{arctanh}\left(\frac{t-\delta/2}{\delta/2}\right)
    - \bm{x}_{nij}\bm{\gamma}_{nj} - \eta_{nj}]\right)^{2}}\times\\
  &\frac{\exp\{\bm{x}_{mij}\bm{\beta}_{mj} + u_{mj}\}}{1 +
    \sum_{n = 1}^{K-1}\exp\{\bm{x}_{nij}\bm{\beta}_{nj} + u_{nj}\}}
    w_{m}\frac{\delta}{2\delta t - 2t^{2}}
    (w_{m}\text{arctanh}\left(\frac{t-\delta/2}{\delta/2}\right)
    - \bm{x}_{mij}\bm{\gamma}_{mj} - \eta_{mj})\times\\
  &\phi[w_{m}\text{arctanh}\left(\frac{t-\delta/2}{\delta/2}\right)
    - \bm{x}_{mij}\bm{\gamma}_{mj} - \eta_{mj}] - \bm{e_{k}^{\top}Q},
\end{align*}

\begin{align*}
  &\frac{\partial^{2}}{\partial \eta_{kj} u_{kj}}
    \log L(\bm{\theta}\mid\bm{y}_{j}, \bm{r}_{j}) =\\
  &y_{Kij}
    \frac{\exp\{\bm{x}_{kij}\bm{\beta}_{kj} + u_{kj}\}}{1 +
    \sum_{n = 1}^{K-1}\exp\{\bm{x}_{nij}\bm{\beta}_{nj} + u_{nj}\}}\times\\
  &\frac{
    w_{k}\frac{\delta}{2\delta t - 2t^{2}}
    (w_{k}\text{arctanh}\left(\frac{t-\delta/2}{\delta/2}\right)
    - \bm{x}_{kij}\bm{\gamma}_{kj} - \eta_{kj})
    \phi[w_{k}\text{arctanh}\left(\frac{t-\delta/2}{\delta/2}\right)
    - \bm{x}_{kij}\bm{\gamma}_{kj} - \eta_{kj})]}{
    \left(1 -
    \sum_{n = 1}^{K-1}\frac{\exp\{\bm{x}_{nij}\bm{\beta}_{nj} + u_{nj}\}}{
    1 +
    \sum_{n = 1}^{K-1}\exp\{\bm{x}_{nij}\bm{\beta}_{nj} + u_{nj}\}}
    w_{n}\frac{\delta}{2\delta t - 2t^{2}}
    \phi[w_{n}\text{arctanh}\left(\frac{t-\delta/2}{\delta/2}\right)
    - \bm{x}_{nij}\bm{\gamma}_{nj} - \eta_{nj}]\right)^{2}}\times\\
  &\Bigg(
    \sum_{n \neq k}^{K-1}
    \frac{
    \exp\{\bm{x}_{nij}\bm{\beta}_{nj} + u_{nj}\}
    \exp\{\bm{x}_{kij}\bm{\beta}_{kj} + u_{kj}\}}{
    \left(1 +
    \sum_{n = 1}^{K-1}\exp\{\bm{x}_{nij}\bm{\beta}_{nj} + u_{nj}\}
    \right)^{2}}\times\\
  &w_{n}\frac{\delta}{2\delta t - 2t^{2}}
    \phi[w_{n}\text{arctanh}\left(\frac{t-\delta/2}{\delta/2}\right)
    - \bm{x}_{nij}\bm{\gamma}_{nj} - \eta_{nj}] -\\
  &\frac{\exp\{\bm{x}_{kij}\bm{\beta}_{kj} + u_{kj}\}
    \left(
    \left(1 +
    \sum_{n = 1}^{K-1}\exp\{\bm{x}_{nij}\bm{\beta}_{nj} + u_{nj}\}
    \right) - \exp\{\bm{x}_{kij} \bm{\beta}_{kj} + u_{kj}\}
    \right)}{
    \left(1 +
    \sum_{n = 1}^{K-1}\exp\{\bm{x}_{nij}\bm{\beta}_{nj} + u_{nj}\}
    \right)^{2}}\times\\
  &w_{k}\frac{\delta}{2\delta t - 2t^{2}}
    \phi[w_{k}\text{arctanh}\left(\frac{t-\delta/2}{\delta/2}\right)
    - \bm{x}_{kij}\bm{\gamma}_{kj} - \eta_{kj}]\Bigg) -
\end{align*}
\begin{align*}
  &y_{Kij}
    \frac{
    \frac{\exp\{\bm{x}_{kij}\bm{\beta}_{kj} + u_{kj}\}
    \left(
    \left(1 +
    \sum_{n = 1}^{K-1}\exp\{\bm{x}_{nij}\bm{\beta}_{ni} + u_{nj}\}
    \right) - \exp\{\bm{x}_{kij} \bm{\beta}_{kj} + u_{kj}\}
    \right)}{
    \left(1 +
    \sum_{n = 1}^{K-1}\exp\{\bm{x}_{nij}\bm{\beta}_{nj} + u_{nj}\}
    \right)^{2}}}{1 -
    \sum_{n = 1}^{K-1}\frac{\exp\{\bm{x}_{nij}\bm{\beta}_{nj} + u_{nj}\}}{
    1 + \sum_{n = 1}^{K-1}\exp\{\bm{x}_{nij}\bm{\beta}_{nj} + u_{nj}\}}
    w_{n}\frac{\delta}{2\delta t - 2t^{2}}
    \phi[w_{n}\text{arctanh}\left(\frac{t-\delta/2}{\delta/2}\right)
    - \bm{x}_{nij}\bm{\gamma}_{nj} - \eta_{nj}]}\times\\
  &w_{k}\frac{\delta}{2\delta t - 2t^{2}}
    (w_{k}\text{arctanh}\left(\frac{t-\delta/2}{\delta/2}\right)
    - \bm{x}_{kij}\bm{\gamma}_{kj} - \eta_{kj})\times\\
  &\phi[w_{k}\text{arctanh}\left(\frac{t-\delta/2}{\delta/2}\right)
    - \bm{x}_{kij}\bm{\gamma}_{kj} - \eta_{kj}] - \bm{e_{k}^{\top}Q},
\end{align*}

\begin{align*}
  &\frac{\partial^{2}}{\partial \eta_{kj} u_{mj}}
    \log L(\bm{\theta}\mid\bm{y}_{j}, \bm{r}_{j}) =\\
  &y_{Kij}
    \frac{\exp\{\bm{x}_{kij}\bm{\beta}_{kj} + u_{kj}\}
    \exp\{\bm{x}_{mij}\bm{\beta}_{mj} + u_{mj}\}}{
    \left(1 +
    \sum_{n = 1}^{K-1}\exp\{\bm{x}_{nij}\bm{\beta}_{nj} + u_{nj}\}
    \right)^{2}}\times\\
  &\frac{
    w_{k}\frac{\delta}{2\delta t - 2t^{2}}
    (w_{k} \text{arctanh}\left(\frac{t-\delta/2}{\delta/2}\right)
    - \bm{x}_{kij}\bm{\gamma}_{kj} - \eta_{kj})
    \phi[w_{k}\text{arctanh}\left(\frac{t-\delta/2}{\delta/2}\right)
    - \bm{x}_{kij}\bm{\gamma}_{kj} - \eta_{kj})]}{1 - \sum_{n = 1}^{K-1}
    \frac{\exp\{\bm{x}_{nij}\bm{\beta}_{nj} + u_{nj}\}}{1 +
    \sum_{n = 1}^{K-1}\exp\{\bm{x}_{nij}\bm{\beta}_{nj} + u_{nj}\}}
    w_{n}\frac{\delta}{2\delta t - 2t^{2}}
    \phi[w_{n}\text{arctanh}\left(\frac{t-\delta/2}{\delta/2}\right)
    - \bm{x}_{nij}\bm{\gamma}_{nj} - \eta_{nj}]} +\\
  &y_{Kij}
    \frac{\exp\{\bm{x}_{kij}\bm{\beta}_{kj} + u_{kj}\}}{1 +
    \sum_{n = 1}^{K-1}\exp\{\bm{x}_{nij}\bm{\beta}_{nj} + u_{nj}\}}\times\\
  &\frac{
    w_{k}\frac{\delta}{2\delta t - 2t^{2}}
    (w_{k}\text{arctanh}\left(\frac{t-\delta/2}{\delta/2}\right)
    - \bm{x}_{kij}\bm{\gamma}_{kj} - \eta_{kj})
    \phi[w_{k}\text{arctanh}\left(\frac{t-\delta/2}{\delta/2}\right)
    - \bm{x}_{kij}\bm{\gamma}_{kj} - \eta_{kj})]}{
    \left(1 - \sum_{n = 1}^{K-1}
    \frac{\exp\{\bm{x}_{nij}\bm{\beta}_{nj} + u_{nj}\}}{1 +
    \sum_{n = 1}^{K-1}\exp\{\bm{x}_{nij}\bm{\beta}_{nj} + u_{nj}\}}
    w_{n}\frac{\delta}{2\delta t - 2t^{2}}
    \phi[w_{n}\text{arctanh}\left(\frac{t-\delta/2}{\delta/2}\right)
    - \bm{x}_{nij}\bm{\gamma}_{nj} - \eta_{nj}]\right)^{2}}\times\\
  &\Bigg(
    \sum_{n \neq m}^{K-1}\frac{
    \exp\{\bm{x}_{nij}\bm{\beta}_{nj} + u_{nj}\}
    \exp\{\bm{x}_{mij}\bm{\beta}_{mj} + u_{mj}\}}{
    \left(1 +
    \sum_{n = 1}^{K-1}\exp\{\bm{x}_{nij}\bm{\beta}_{nj} + u_{nj}\}
    \right)^{2}}\times\\
  &w_{n}\frac{\delta}{2\delta t - 2t^{2}}
    \phi[w_{n}\text{arctanh}\left(\frac{t-\delta/2}{\delta/2}\right)
    - \bm{x}_{nij}\bm{\gamma}_{nj} - \eta_{nj}] -\\
  &\frac{\exp\{\bm{x}_{mij}\bm{\beta}_{mj} + u_{mj}\}
    \left(
    \left(1 +
    \sum_{n = 1}^{K-1}\exp\{\bm{x}_{nij}\bm{\beta}_{nj} + u_{nj}\}
    \right) - \exp\{\bm{x}_{mij} \bm{\beta}_{mj} + u_{mj}\}
    \right)}{
    \left(1 + \sum_{n = 1}^{K-1}\exp\{\bm{x}_{nij}\bm{\beta}_{nj} + u_{nj}\}
    \right)^{2}}\times\\
  &w_{m}\frac{\delta}{2\delta t - 2t^{2}}
    \phi[w_{m}\text{arctanh}\left(\frac{t-\delta/2}{\delta/2}\right)
    - \bm{x}_{mij}\bm{\gamma}_{mj} - \eta_{mj}]\Bigg) - \bm{e_{k}^{\top}Q},
\end{align*}
with \(\bm{e_{k}^{\top}}\) begin a vector with \(1\) at the \(k\)-th
position and zero elsewhere.

\chapter{\texttt{R} CODE TO SIMULATE FROM A \(\text{multiGLMM}\) WITH
         TWO COMPETING CAUSES AND CLUSTERS OF SIZE TWO. FOR MORE
         INFORMATION CHECK SECTION \ref{cap:simu}}
\label{cap:appendixC}

\lstinputlisting[firstline=97,lastline=153]{datasets.Rmd}
\vspace{-0.5cm}
\begin{center}
 \begin{footnotesize}
  SOURCE: The author (2021).
 \end{footnotesize}
\end{center}

\end{apendicesenv}
% ----------------------------------------------------------------------
% \begin{anexosenv}
% \partanexos
% \addcontentsline{toc}{chapter}{\hspace{2.105cm}ANNEX}
% \renewcommand{\ABNTEXchapterfontsize}{\ABNTEXsectionfont}
% \end{anexosenv}
%-----------------------------------------------------------------------
\phantompart
\printindex
%-----------------------------------------------------------------------
\end{document}
% END ==================================================================

% ----------------------------------------------------------------------
% pacote para fazer o checkmark
\usepackage{pifont} % http://ctan.org/pkg/pifont
\newcommand{\cmark}{\ding{51}}%
\newcommand{\xmark}{\ding{55}}%
% ----------------------------------------------------------------------
\usepackage{amsmath}
\usepackage{amsfonts}
\usepackage{amssymb}
\usepackage{pdfpages}
% \usepackage{times}
% \usepackage{helvet}
% \renewcommand{\familydefault}{\sfdefault}
% ----------------------------------------------------------------------
\NewDocumentCommand\cc{+u{\cc}}{\ignorespaces}
% ----------------------------------------------------------------------
% controle do espaçamento entre um parágrafo e outro:
\setlength{\parskip}{0.2cm} % tente também \onelineskip
% ----------------------------------------------------------------------
\titulo{MODELING THE CUMULATIVE INCIDENCE FUNCTION OF CLUSTERED
  COMPETING RISKS DATA: A MULTINOMIAL GLMM APPROACH}
\autor{HENRIQUE APARECIDO LAUREANO}
\data{2021}
\instituicao{FEDERAL UNIVERSITY OF PARANÁ}
\orientador{Prof. PhD Wagner Hugo Bonat}
\coorientador{Prof. PhD Paulo Justiniano Ribeiro Jr}
\tipotrabalho{Dissertação (mestrado)}
\preambulo{\small{Thesis presented to the Graduate Program of Numerical
    Methods in Engineering, Concentration Area in Mathematical
    Programming: Statistical Methods Applied in Engineering, Federal
    University of Paran\'{a}, as part of the requirements to the
    obtention of the Master's Degree in Sciences.}}
% ----------------------------------------------------------------------
% informações do PDF
\makeatletter
\hypersetup{
  % pagebackref=true,
  pdftitle={\@title},
  pdfauthor={\@author},
  pdfsubject={\imprimirpreambulo},
  % pdfkeywords = {}{}{}{},
  colorlinks=true, % false: boxed links; true: colored links
  linkcolor=blue, % color of internal links
  citecolor=blue, % color of links to bibliography
  filecolor=magenta, % color of file links
  urlcolor=blue,
  bookmarksdepth=4
}
\addto\captionsenglish{
  % adjusts names from abnTeX2
  \renewcommand{\folhaderostoname}{Title page}
  \renewcommand{\epigraphname}{Epigraph}
  \renewcommand{\dedicatorianame}{Dedication}
  \renewcommand{\errataname}{Errata sheet}
  \renewcommand{\agradecimentosname}{Acknowledgements}
  \renewcommand{\anexoname}{ANNEX}
  \renewcommand{\anexosname}{Annex}
  \renewcommand{\apendicename}{APPENDIX}
  \renewcommand{\apendicesname}{Appendix}
  \renewcommand{\orientadorname}{Supervisor:}
  \renewcommand{\coorientadorname}{Co-supervisor:}
  \renewcommand{\folhadeaprovacaoname}{Approval}
  \renewcommand{\resumoname}{Abstract}
  \renewcommand{\listadesiglasname}{List of abbreviations and acronyms}
  \renewcommand{\listadesimbolosname}{List of symbols}
  \renewcommand{\fontename}{Source}
  \renewcommand{\notaname}{Note}
  % adjusts names used by \autoref
  \renewcommand{\pageautorefname}{page}
  \renewcommand{\chapterautorefname}{Chapter}
  \renewcommand{\sectionautorefname}{Section}
  \renewcommand{\subsectionautorefname}{subsection}
  \renewcommand{\subsubsectionautorefname}{subsubsection}
  \renewcommand{\paragraphautorefname}{subsubsubsection}
}
\makeatother
% ----------------------------------------------------------------------
\graphicspath{{figures/}}
% ----------------------------------------------------------------------
\begin{document}
\selectlanguage{english}
% adequando o uppercase titulo dos elementos nas suas respectivas
% legendas
\renewcommand{\tablename}{TABLE }
\renewcommand{\figurename}{FIGURE }
% ----------------------------------------------------------------------
\frenchspacing % retira espaço extra obsoleto entre as frases
% ----------------------------------------------------------------------
% capa
\tikz[remember picture,overlay] \node[opacity=1,inner sep=0pt] at
(current page.center){
  \includegraphics[width=\paperwidth,
  height=\paperheight]{Figuras/ufpr_bg}};
% ----------------------------------------------------------------------
\imprimircapa
% ----------------------------------------------------------------------
% folha de rosto
\imprimirfolhaderosto
% ----------------------------------------------------------------------
% \begin{dedicatoria}
%   \vspace*{\fill}
%   ...
%   \vspace*{\fill}
% \end{dedicatoria}
% ----------------------------------------------------------------------
% ficha catalográfica

% \begin{fichacatalografica}
%   \includepdf{ficha.pdf}
% \end{fichacatalografica}
% ----------------------------------------------------------------------
% inserir folha de aprovação
% \begin{folhadeaprovacao}
%   \includepdf{termo.pdf}
% \end{folhadeaprovacao}
\begin{folhadeaprovacao}
 \begin{center}
   {\ABNTEXchapterfont\large\imprimirautor}

   \vspace*{\fill}\vspace*{\fill}
   \begin{center}
     \ABNTEXchapterfont\bfseries\large\imprimirtitulo
   \end{center}
   \vspace*{\fill}

    \hspace{.45\textwidth}
    \begin{minipage}{.5\textwidth}
       \imprimirpreambulo
    \end{minipage}
   \vspace*{\fill}
 \end{center}

 Master thesis approved. XXX XX, 2021.

  \assinatura{\textbf{\imprimirorientador}\\ Supervisor}
  \assinatura{\textbf{Prof. PhD Paulo Justiniano Ribeiro Jr}\\
    Co-supervisor}
  \assinatura{\textbf{Prof. PhD \(\dots\)}\\
    Internal Examinator - PPGMNE}
  \assinatura{\textbf{Prof. PhD \(\dots\)}\\
    Internal Examinator - PPGMNE}
  \assinatura{\textbf{Prof. PhD \(\dots\)}\\External Examiner - }

  \begin{center}
   \vspace*{0.5cm}
   {\large CURITIBA}
   \par
   {\large\imprimirdata}
   \vspace*{1cm}
 \end{center}

\end{folhadeaprovacao}
% ----------------------------------------------------------------------
\begin{dedicatoria}
  \vspace*{22.7cm}
  \begin{flushright}
    \begin{minipage}[H]{4.5cm}
      {To Celita and Olivio}
    \end{minipage}
  \end{flushright}
\end{dedicatoria}
% ----------------------------------------------------------------------
\begin{agradecimentos}
  As Moro said once, I'm thankful for everything and everyone.
\end{agradecimentos}

\begin{epigrafe}
  \vspace*{\fill}
  \begin{flushright}
    \textit{"It's not supposed to be easy."\\
             (Gregg Popovich)}
              % on Sao Antonio Spurs \(\times\) Oklahoma City Thunder,
              % first game of the 2012 Western Conference Finais
  \end{flushright}
\end{epigrafe}
% ----------------------------------------------------------------------
\newpage
\setlength{\absparsep}{18pt} % ajusta o espaçamento dos parágrafos do
                             % resumo
\setlength{\abstitleskip}{1cm} % adiciona mais um cm após o 'titulo' do
                               % resumo para ficar com 2cm
\begin{resumo}[]
  \vspace{-2cm}
  \begin{center}
    \bfseries{\large{\textsf{ABSTRACT}}}
  \end{center}
  \vspace{0.3cm}
  Failure time data \(\dots\)\\

  \textbf{Keywords}: Competing risks.
\end{resumo}
% ----------------------------------------------------------------------
\newpage
\setlength{\absparsep}{18pt} % ajusta o espaçamento dos parágrafos do
                             % resumo
\setlength{\abstitleskip}{1cm} % adiciona mais um cm após o 'titulo' do
                               % resumo para ficar com 2cm
\begin{resumo}[]
  \begin{otherlanguage*}{brazil}
    \vspace{-2cm}
    \begin{center}
      \bfseries{\large{\textsf{RESUMO}}}
    \end{center}
    \vspace{0.3cm}
    Dados de tempos de falha \(\dots\)\\

    \textbf{Palavras-chave}: Riscos competitivos.
  \end{otherlanguage*}
\end{resumo}
% ----------------------------------------------------------------------
\pdfbookmark[0]{\listfigurename}{lof}
\listoffigures*
\cleardoublepage
% ----------------------------------------------------------------------
\pdfbookmark[0]{\listtablename}{lot}
\listoftables*
\cleardoublepage
% ----------------------------------------------------------------------
\makeatletter
\renewcommand\numberline[1]{
	\leftskip -0.7em
	\rightskip 1.6em
	\parfillskip -\rightskip
	\parindent 0em
	\@tempdima 2.0em
	\vspace{0em}
  \advance\leftskip \@tempdima \null\nobreak\hskip -\leftskip
	ALGORITHM \normalfont #1 ~~ }
\makeatother
% ----------------------------------------------------------------------
\pdfbookmark[0]{\listalgorithmname}{loa}
\listofalgorithms
\cleardoublepage
% ----------------------------------------------------------------------
\makeatletter
\def\numberline#1{\hb@xt@\@tempdima{#1\hfil}}
\makeatother
% ----------------------------------------------------------------------
% \begin{siglas}
% \item[Fig.] Area of the $i^{th}$ component
% \item[456] Isto é um número
% \item[123] Isto é outro número
% \item[lauro cesar] este é o meu nome
% \end{siglas}
% ----------------------------------------------------------------------
% \begin{simbolos}
% \item[\(\mathbb{E}(\cdot)\)] The mathematical expectation of a random
%   variable \(\cdot\)
% \end{simbolos}
% ----------------------------------------------------------------------
\pdfbookmark[0]{\contentsname}{toc}
\tableofcontents*
\cleardoublepage
% ----------------------------------------------------------------------
\makepagestyle{abntheadings}
\makeevenhead{abntheadings}{\ABNTEXfontereduzida\thepage}{}{}
\makeoddhead{abntheadings}{}{}{\ABNTEXfontereduzida\thepage}
\makeheadrule{abntheadings}{\textwidth}{0in}
% ----------------------------------------------------------------------
\textual
% ----------------------------------------------------------------------
\chapter{Introduction}
\label{cap:intro}
Consider a cluster of random variables. Each random variable represents
the time until some event occurs. The random variables that compose the
cluster are assumed to be correlated, i.e., the method for the analysis
is flexible enough to be able to verify if this happens to that data. In
this thesis, the cluster is a family - more precisely, a part of a
family, i.e., a pair of twins; the random variables are the time until
the occurrence (or not) of an event in each twin; and the event under
focus is the occurrence of cancer.

When we deal with random variables, in the context of a statistical
model - a response of interest; that represents the time until some
event occurs, such events are generically referred to as
\textit{failures}. From this, we get the name of the field of study:
failure time data~\cite{kalb&prentice}. These events may not necessarily
consist of a failure, however, the major areas of application of the
methods that will be discussed here are biomedical studies and
industrial life testing. Thus, this name sounds appropriate.

Independent of the area of application the methods are the same, but the
name of what you're doing is different. In industrial life testing
applications, you perform what is called a reliability analysis; in
biomedical studies, you perform what is called a survival analysis. In
this thesis, we'll maintain our focus on the latter.

Generally speaking, the term survival analysis is applied when we deal
with a univariate set, i.e., we have one response variable. As an
example,

When we want to study the possible relation between the components (seen
as random variables) of a dataset in an associative manner, the perhaps
most common scientific way of doing that is by fitting a linear model.
More generally, a generalized linear model (GLM). To fit a GLM to a
dataset we basically need to select a probability distribution for the
so-called response variable \(Y_{i}\) (or dependent variable); and how
we'll approach the possible relation between the response and the other
variables \(\mathbf{X}_{i}\) (independent variables or covariates).

Generalized linear models~\cite{GLM72} allow for response distribution
other than normal, which configures the so-called linear models. In a
GLM we model the mean, \(\mu_{i}\), of the response random variable, and
it has the basic structure
\[
 g(\mu_{i}) = \mathbf{X}_{i} \bm{\beta},
\]
where \(\mu_{i} \equiv \mathbb{E}(Y_{i})\), \(g\) is a smooth monotonic
``link function'', \(\mathbf{X}_{i}\) is the \(i^\text{th}\) row of a
model matrix \(\mathbf{X}\), and \(\bm{\beta}\) is a vector of unknown
parameters. In addition, a GLM usually makes the distributional
assumption that the \(Y_{i}\) are independent and
\[
  Y_{i} \sim \text{some exponential family distribution}.
\]

The \textit{exponential family} of distributions includes many
distributions that are useful for practical modelling, such as the
Poisson (for counting data), binomial (dichotomic data), gamma
(continuous but positive) and normal (continuous data) distributions.
The comprehensive reference for GLMs is~\citeonline{GLM89}.

However broad it may be the range of models that can be constructed
thanks to the generality of the GLM framework, is plausible that for
some applications a specific modeling framework come up. This is the
case when the response variable is the time until some event occurs.
Such events are generically referred to as \textit{failures}, and its
major areas of use are biomedical studies and industrial life testing.
In the latter the commonly used name is reliability and in the former is
survival. In this thesis, we focus on the survival side.

A survival model consists

\section{GOALS}

\subsection{General goals}

Propor um modelo de regressão para análise de variáveis respostas
limitadas multivariada.

\subsection{Specific goals}

\begin{enumerate}
\item Estudar o desempenho do algoritmo NORTA (\emph{NORmal To
    Anything}) para simular variáveis aleatórias beta correlacionadas.

\item Especificar o modelo usando suposições de primeiro e segundo
  momentos.

\item Usar as funções de estimação quase-score e Pearson para estimar os
  parâmetros de regressão e dispersão, respectivamente.

\item Delinear estudos de simulação para explorar a flexibilidade do
  modelo para lidar com dados limitados em estudos longitudinais, além
  de checar propriedades dos estimadores em estudos com múltiplas
  respostas correlacionadas.

\item Adaptar técnicas de diagnóstico para o modelo proposto, como
  DFFITS, DFBETAS, distância de Cook e o gráfico de probabilidade
  meio-normal com envelope simulado.

\item Aplicar o modelo proposto em dois conjuntos de dados.
\end{enumerate}

\section{JUSTIFICATION}

\section{LIMITATION}

Este trabalho se restringe a propor um novo modelo de regressão para
análise de variáveis respostas limitadas multivariada. Para motivar o
novo modelo, serão apresentadas aplicações em dois conjuntos de dados,
que não são facilmente manipulados pelos métodos estatísticos
existentes. Portanto, testes de hipóteses e de comparações múltiplas
multivariados não serão desenvolvidos no decorrer deste trabalho.

\section{THESIS ORGANIZATION}

Esta dissertação contém seis capítulos incluindo esta introdução.
O~\autoref{cap:aplicacoes} descreve os dois conjuntos de dados que serão
usados como exemplos de aplicação no novo modelo.
O~\autoref{cap:fundamentacaoteorica} apresenta a revisão bibliográfica
que motivou este trabalho, introduz o modelo de regressão beta
(univariado), apresenta o algoritmo NORTA (\textit{NORmal To Anything})
usado nos estudos de simulação e discute brevemente as medidas de
bondade de ajuste usadas no trabalho. O~\autoref{cap:multivariatemodel}
propõe o modelo de regressão quase-beta multivariado, apresenta o método
usado para estimação e inferência e adapta técnicas de diagnóstico.
No~\autoref{cap:resultados} são apresentados os resultados de três
estudos de simulação, além da análise dos dados apresentados
no~\autoref{cap:aplicacoes}. Finalmente, o~\autoref{cap:considefinais}
discute as principais contribuições desta dissertação, além de
apresentar as conclusões seguidas por sugestões para futuros trabalhos.

% END ==================================================================
% ----------------------------------------------------------------------
\chapter{Generalized linear mixed models: formulation, optimization, and
  implementation}
\label{cap:methods}
This chapter presents a systematic review of the main theoretical
aspects involved in the construction, estimation and implementation of a
generalized linear mixed model (GLMM). We start in \autoref{cap:joint}
with the model construction framework, concluding with the so-called
joint likelihood function. \autoref{cap:laplace} address the integration
of that joint likelihood, a necessary and fundamental step in our
modeling approach, resulting in a marginal likelihood function.
\autoref{cap:opt} discusses available alternatives for the optimization
of the marginal distribution obtained through that integration.
\autoref{cap:ad} talks about automatic differentiation, the most
efficent manner of computing derivatives, and a key point for us. Last
but not least, in \autoref{cap:tmb} we present the computational tool
used to peform all the discussed procedure, the TMB: Template Model
Builder. A very exciting \texttt{R} \cite{R18} package developed
by~\citeonline{TMB}.

\section{JOINT LIKELIHOOD}
\label{cap:joint}

A standard, univariate, GLMM models an \(n\)-vector of exponential
family random variables, \(\mathbf{Y}\), with conditional expected
value, \(\bm{\mu} \equiv \mathbb{E}(\mathbf{Y} \mid \mathbf{X},
\mathbf{u})\), via a linear predictor equation expressed by
\begin{equation}
  g(\bm{\mu}) = \mathbf{X} \bm{\beta} + \mathbf{Zu}, \quad
  \mathbf{u} \sim \mathcal{N}(\mathbf{0}, \bm{\Sigma}).
  \label{eq:gmu}
\end{equation}

That is, a GLMM is a generalized linear model (GLM) in which the linear
predictor depends on some Gaussian latent effects, \(\mathbf{u}\), times
a latent effects model matrix \(\mathbf{Z}\). The idea embedded in that
matrix is exemplified in~\autoref{eq:Zu}. Suppose, e.g., three
individuals and each has two measures. This configures a simple repeated
measures context, the focus of this work. It is reasonable to admit that
each individual has a particular latent effect value. Consequently,
\begin{equation}
  \mathbf{Zu} = \begin{bmatrix}
                 1 & 0 & 0\\
                 1 & 0 & 0\\
                 0 & 1 & 0\\
                 0 & 1 & 0\\
                 0 & 0 & 1\\
                 0 & 0 & 1\\
                \end{bmatrix} \begin{bmatrix}
                               u_{1}\\
                               u_{2}\\
                               u_{3}\\
                              \end{bmatrix} = \begin{bmatrix}
                                               u_{1}\\
                                               u_{1}\\
                                               u_{2}\\
                                               u_{2}\\
                                               u_{3}\\
                                               u_{3}\\
                                              \end{bmatrix},
  \label{eq:Zu}
\end{equation}
where \(\mathbf{u}^{\top} = [u_{1}~u_{2}~u_{3}]\) and \(\mathbf{Z}\) has
the role of projecting the values of \(\mathbf{u}\) to match the number
of measures.

We model this mean structure into a combination of probability
distributions. It's a combination since we have to assume probabilistic
structures for the observed and non-observed, latent, data. For each
observed variable \(y_{ij}\), we have a probability distribution of the
exponential family, denoted by \(f(y_{ij} \mid \mathbf{u}_{i},
\bm{\theta})\). For the non-observed latent effect we have, generally, a
(multivariate) Gaussian distribution, denoted by \(f(\mathbf{u}_{i} \mid
\bm{\Sigma})\). For each individual or unity under study, \(i\), and to
each measure, \(j\), we have the product of these probability densities,
a likelihood contribution.

We want to estimate the parameter vector \(\bm{\theta} =
[\bm{\beta}~\bm{\Sigma}]^{\top}\) of~\autoref{eq:gmu}. Besides the role
of emphasizing the fact that \(\bm{\mu}\) is a function of
\(\bm{\theta}\) and that we want to estimate \(\bm{\theta}\), the
likelihood function ties the probability densities. i.e., the likelihood
is the product of the probability densities product for each subject.
Since the \(Y_{i}\) are mutually independent, the likelihood of
\(\bm{\theta}\) is
\begin{equation}
  L(\bm{\theta} \mid \mathbf{y}, \mathbf{u}) =
  \prod_{i=1}^{n}~\prod_{j=1}^{n_{i}}~
  f(y_{ij} \mid \mathbf{u}_{i}, \bm{\beta}, \bm{\Sigma})~
  f(\mathbf{u}_{i} \mid \bm{\Sigma}).
  \label{eq:joint}
\end{equation}

From standard probability theory is easy to see that in the right-hand
side (r.h.s.) we have a joint density, consequently,~\autoref{eq:joint}
represents what is called a joint likelihood function. What makes
working with this joint likelihood problematic is that we don't have all
the information necessary to just optimize it and get the desired
parameter estimates. The latent effect \(\mathbf{u}\) is
\textit{latent}, we don't observe it. To handle with this we basically
have two available paths.

\section{LAPLACE APPROXIMATION}
\label{cap:laplace}

To deal with the joint likelihood in~\autoref{eq:joint}~we have a choice
to make. Be or not to be Bayesian. Each choice has its own difficulties,
advantages, and characteristics.

The Bayesian path assumes that all \(\bm{\theta}\) components are random
variables. With all parameters being treated as random variables, and
since we don't observe them, what the Bayesian framework does is try to
compute the mode of each ``parameter'' marginal distribution via a
sampling algorithm, called MCMC: Markov Chain Monte Carlo~\cite{MCMC,
  Diaconis}. The advantage is that we can reach an MCMC algorithm to
basically any statistical model, the disadvantages are that this
approach is very time consuming and we have to propose prior
distributions to each ``parameter''. These prior proposals are not
always easy to make, and the resulting marginal distributions can be
very depending on it.

A Bayesian approach can be applied in basically any context. However, in
complex scenarios they can be the only available method to maximize the
likelihood. This isn't the case here. We have a joint density where one
of the random variables isn't observed, but we're not interested in it,
only in the variance parameters inherent in it. Again, from standard
probability theory, if we have a joint density we can just integrate out
the undesired variable. This results in
\begin{equation}
  \begin{aligned}
    L(\bm{\theta} \mid \mathbf{y}) &=
    \prod_{i=1}^{n}~\int_{\mathcal{R}^{\mathbf{u}_{i}}}
    \left\{
      \prod_{j=1}^{n_{i}}~
      f(y_{ij} \mid \mathbf{u}_{i}, \bm{\beta}, \bm{\Sigma})~
      f(\mathbf{u}_{i} \mid \bm{\Sigma})
    \right\} \text{d} \mathbf{u}_{i}\\
    &= \prod_{i=1}^{n}~\int_{\mathcal{R}^{\mathbf{u}_{i}}}~
    f(\mathbf{y}_{i}, \mathbf{u}_{i} \mid \bm{\theta})~
    \text{d} \mathbf{u}_{i},
    \label{eq:generalmarginal}
  \end{aligned}
\end{equation}
a marginal density that keeps the parameters of the integrated variable.

If the response distribution in our mixed model is Gaussian, is
analytically tractable to integrate \(\mathbf{u}\) out of the joint
density. Consequently, is possible to evaluate the likelihood exactly.
This is the case of linear mixed models and the main difference to the
GLMMs. When the response distribution isn't Gaussian, generally, isn't
anymore analytically tractable to integrate out the latent effect. So
what do we do? Well, basically we have two options.

We can avoid the integrals in~\autoref{eq:generalmarginal}, replacing it
by integrals that are sometimes more analytically tractable. This can be
performed via an algorithm called EM:
Expectation-Maximization~\cite{EM77}. This method is considered a little
bit naive and generally isn't recommended if you have a better option.
Our better option consists in take advantage of the exponential family
structure and the fact that we're dealing with Gaussian latent effects.
These ideas converge to what is called, \textit{Laplace
  approximation}~\cite{molenberghs&verbeke, LA4H, tierney, corestats}.

If the integral is analytically intractable, we can approximate it to
obtain a tractable closed-form expression, allowing the numerical
maximization of the marginal likelihood~\cite{patrao}. The Laplace
approximation has been designed to approximate integrals in the form
\begin{equation}
  \int_{\mathcal{R}^{\mathbf{u}_{i}}}
  \exp\{Q(\mathbf{u}_{i})\} \text{d} \mathbf{u}_{i}
  \approx (2\pi)^{n_{\mathbf{u}}/2}~
  |{Q}''(\mathbf{\hat{u}}_{i})|^{-1/2}~\exp\{Q(\mathbf{\hat{u}}_{i})\},
  \label{eq:laplace}
\end{equation}
where \(Q(\mathbf{u}_{i})\) is a known, unimodal bounded function and
\(\mathbf{\hat{u}}_{i}\) is the value for which \(Q(\mathbf{u}_{i})\) is
maximized. As~\citeonline{corestats}~shows, a Laplace approximation
consists of a second order Taylor expansion of \(\log f(\mathbf{y}_{i},
\mathbf{u}_{i} \mid \bm{\theta})\), about \(\mathbf{\hat{u}}_{i}\), that
gives
\[
  \log f(\mathbf{y}_{i}, \mathbf{u}_{i} \mid \bm{\theta}) \approx
  \log f(\mathbf{y}_{i}, \mathbf{\hat{u}}_{i} \mid \bm{\theta}) -
  \frac{1}{2}
  (\mathbf{u}_{i} - \mathbf{\hat{u}}_{i})^{\top}\mathbf{H}~
  (\mathbf{u}_{i} - \mathbf{\hat{u}}_{i}),
\]
where \(\mathbf{H} = - \nabla_{u}^{2} \log f(\mathbf{y}_{i},
\mathbf{\hat{u}}_{i} \mid \bm{\theta})\). Hence, we can approximate the
joint by
\begin{equation}
  f(\mathbf{y}_{i}, \mathbf{u}_{i} \mid \bm{\theta}) \approx
  f(\mathbf{y}_{i}, \mathbf{\hat{u}}_{i} \mid \bm{\theta})~\exp
  \left\{- \frac{1}{2}
    (\mathbf{u}_{i} - \mathbf{\hat{u}}_{i})^{\top}\mathbf{H}~
    (\mathbf{u}_{i} - \mathbf{\hat{u}}_{i})
  \right\}.
  \label{eq:taylor}
\end{equation}

From here we start to take advantage of the points mentioned above.
First, the fact that we're working with Gaussian distributed latent
effects. In~\autoref{eq:taylor}~we see the core of a Gaussian density,
that complete is
\[
  \int_{\mathcal{R}^{\mathbf{u}_{i}}}
  \frac{1}{(2 \pi)^{n_{\mathbf{u}}/2}~|\mathbf{H}^{-1}|^{1/2}}~\exp
  \left\{- \frac{1}{2}
    (\mathbf{u}_{i} - \mathbf{\hat{u}}_{i})^{\top}\mathbf{H}~
    (\mathbf{u}_{i} - \mathbf{\hat{u}}_{i})
  \right\} \text{d} \mathbf{u}_{i} = 1,
\]
and integrates to 1. Integrating~\autoref{eq:taylor}, it follows that
\begin{align*}
  \int_{\mathcal{R}^{\mathbf{u}_{i}}}
  f(\mathbf{y}_{i}, \mathbf{u}_{i} \mid \bm{\theta})
  \text{d} \mathbf{u}_{i}
  &\approx f(\mathbf{y}_{i}, \mathbf{\hat{u}}_{i} \mid \bm{\theta})
    \int_{\mathcal{R}^{\mathbf{u}_{i}}} \exp
    \left\{- \frac{1}{2}
    (\mathbf{u}_{i} - \mathbf{\hat{u}}_{i})^{\top}\mathbf{H}~
    (\mathbf{u}_{i} - \mathbf{\hat{u}}_{i})
    \right\} \text{d} \mathbf{u}_{i}\\
  &= (2 \pi)^{n_{\mathbf{u}}/2}~|\mathbf{H}|^{-1/2}~
    f(\mathbf{y}_{i}, \mathbf{\hat{u}}_{i} \mid \bm{\theta}),
\end{align*}
i.e., we get~\autoref{eq:laplace}, a first order Laplace approximation
to the integral. Careful accounting of the approximation error shows it
to generally be \(\mathcal{O}(n^{-1})\) where \(n\) is the sample size,
assuming a fixed length for \(\mathbf{u}_{i}\)~\cite{corestats}.

The second advantage of a Laplace approximation approach in a GLMM is
the exponential family structure. In a usual GLMM, the response follows
a one-parameter exponential family distribution, that can be written as
\[
  f(\mathbf{y}_{i} \mid \mathbf{u}_{i}, \bm{\theta}) = \exp
  \left\{\mathbf{y}_{i}^{\top}
    (\mathbf{X}_{i}\bm{\beta} + \mathbf{Z}_{i}\mathbf{u}_{i}) -
    \mathbf{1}_{i}^{\top}
    b(\mathbf{X}_{i}\bm{\beta} + \mathbf{Z}_{i}\mathbf{u}_{i}) +
    \mathbf{1}_{i}^{\top} c(\mathbf{y}_{i})
  \right\},
\]
where \(b(\cdot)\) and \(c(\cdot)\) are known functions. This general
and easy to compute expression, together with a (multivariate) Gaussian
distribution, highlights the convenience of the Laplace method. The
\(Q(\mathbf{u}_{i})\) function to be maximized can then be expressed as
\begin{equation}
  \begin{aligned}
    Q(\mathbf{u}_{i}) &= \mathbf{y}_{i}^{\top}
    (\mathbf{X}_{i}\bm{\beta} + \mathbf{Z}_{i}\mathbf{u}_{i}) -
    \mathbf{1}_{i}^{\top}
    b(\mathbf{X}_{i}\bm{\beta} + \mathbf{Z}_{i}\mathbf{u}_{i}) +
    \mathbf{1}_{i}^{\top} c(\mathbf{y}_{i})\\
    &- \frac{n_{\mathbf{u}}}{2} \log (2 \pi) -
    \frac{1}{2} \log |\bm{\Sigma}| -
    \frac{1}{2} \mathbf{u}_{i}^{\top} \bm{\Sigma}^{-1}~\mathbf{u}_{i}.
  \end{aligned}
\end{equation}

The approximation in~\autoref{eq:laplace} requires the maximum
\(\mathbf{\hat{u}}_{i}\) of the function \(Q(\mathbf{u}_{i})\). Since we
assume a Gaussian distribution with a known mean for the latent effects,
we have the perfect initial guess for a gradient-based method as the
Newton-Raphson (NR) algorithm. The NR method consists of an iterative
scheme as follows:
\[
  \mathbf{u}_{i}^{(k+1)} = \mathbf{u}_{i}^{(k)} -
  {Q}''(\mathbf{u}_{i}^{(k)})^{-1}~{Q}'(\mathbf{u}_{i}^{(k)}),
\]
until convergence, which gives \(\mathbf{\hat{u}}_{i}\). At this stage,
all parameters are considered known.~\citeonline{patrao}~presents the
generic expressions for the derivatives required by the NR method, given
by the following:
\begin{equation}
  \begin{aligned}
    {Q}'(\mathbf{u}_{i}^{(k)}) &= \{\mathbf{y}_{i} -
    {b}'(\mathbf{X}_{i}\bm{\beta} +
    \mathbf{Z}_{i}\mathbf{u}_{i}^{(k)})\}^{\top} -
    {\mathbf{u}_{i}^{(k)}}^{\top} \bm{\Sigma}^{-1},\\
    {Q}''(\mathbf{u}_{i}^{(k)}) &=
    - \text{diag}\{{b}''(\mathbf{X}_{i}\bm{\beta} +
    \mathbf{Z}_{i}\mathbf{u}_{i}^{(k)})\} - \bm{\Sigma}^{-1}.
  \end{aligned}
  \nonumber
\end{equation}

Finally, the marginal log-likelihood returned by the Laplace
approximation for each invividual or unit under study, is as follows:
\begin{equation}
  \begin{aligned}
    l(\bm{\theta} \mid \mathbf{y}_{i}) =
    \log L(\bm{\theta} \mid \mathbf{y}_{i}) &=
    \frac{n}{2} \log (2 \pi) - \frac{1}{2} \log
    \left|
      \text{diag}\{{b}''(\mathbf{X}_{i}\bm{\beta} +
      \mathbf{Z}_{i}\mathbf{\hat{u}}_{i})\} + \bm{\Sigma}^{-1}
    \right|\\
    &+ \mathbf{y}_{i}^{\top}
    (\mathbf{X}_{i} \bm{\beta} + \mathbf{Z}_{i} \mathbf{\hat{u}}_{i}) -
    \mathbf{1}_{i}^{\top}
    b(\mathbf{X}_{i}\bm{\beta} + \mathbf{Z}_{i} \mathbf{\hat{u}}_{i}) +
    \mathbf{1}_{i}^{\top} c(\mathbf{y}_{i})\\
    &- \frac{n_{\mathbf{u}}}{2} \log (2 \pi) -
    \frac{1}{2} \log |\bm{\Sigma}| - \frac{1}{2}
    \mathbf{\hat{u}}_{i}^{\top}\bm{\Sigma}^{-1}~\mathbf{\hat{u}}_{i},
  \end{aligned}
  \nonumber
\end{equation}
that can now be numerically maximized over the model parameters.

\section{MARGINAL LIKELIHOOD OPTIMIZATION}
\label{cap:opt}

At this point is clear that we have two optimizations to be made. An
``inside'' and an ``outside'' optimization. The inside optimization is
performed into the Laplace approximation layer via a Newton-Raphson
algorithm, a Newton's method. The outside optimization is made with the
Laplace approximation outputs, i.e., the maximization step
of~\autoref{eq:generalmarginal}'s marginal log-likelihood over its
parameters. This task is usually performed via a quasi-Newton method. We
focus here on two of the most traditional ones, the
Broyden-Fletcher-Goldfarb-Shanno (BFGS) algorithm and the PORT routines.

The inside optimization is the joint log-likelihood numerical
maximization w.r.t. its latent effects. This is kind of a simple task
since all model parameters are considered as fixed and we ``know'' that
the latent effects are distributed with zero mean, i.e., we have the
perfect initial guess. In this context, the use of a Newton's method is
straightforward. When we talk about the outside optimization it is a
completely different scenario. It is not straightforward to find a good
initial guess or reach convergence, so more robust methods are a good
choice.

In optimization, Newton methods are algorithms for finding local maxima
and minima of functions, i.e., the search for the zeroes of the gradient
of that function. Newton methods are characterized by the use of a
symmetric matrix of function's second derivatives, the Hessian matrix.
Quasi-Newton methods are based on Newton's method and are seen as an
alternative to it. They can be used if the Hessian is unavailable or is
too expensive to compute at every iteration.

As shown in~\citeonline{nocedal&wright}, chief advantages of
quasi-Newton methods over Newton's method are that the Hessian matrix
doesn't need to be computed, its approximated; and it also doesn't need
to be inverted. Newton's method requires the Hessian to be inverted,
typically by solving a system of linear equations - often quite costly.
In contrast, quasi-Newton methods usually generate an estimate of it
directly. As in Newton's method, they use a second-order approximation
to find the minimum of a function \(f(\mathbf{x})\). The Taylor series
of \(f(\mathbf{x})\) around an iterate is
\[
  f(\mathbf{x}_{k} + \Delta\mathbf{x}) \approx
  f(\mathbf{x}_{k}) + \nabla f(\mathbf{x}_{k})^{\top} \Delta\mathbf{x} +
  \frac{1}{2} \Delta\mathbf{x}^{\top} \mathbf{B}~\Delta\mathbf{x},
\]
where \(\nabla f(\cdot)\) is the gradient, and \(\mathbf{B}\) an
approximation to the Hessian matrix. The gradient of this approximation
w.r.t. \(\Delta\mathbf{x}\) is
\[
  \nabla f(\mathbf{x}_{k} + \Delta\mathbf{x}) \approx
  \nabla f(\mathbf{x}_{k}) + \mathbf{B}~\Delta\mathbf{x},
\]
setting this gradient to zero provides the Newton step:
\[
  \Delta\mathbf{x} = - \mathbf{B}^{-1} \nabla f(\mathbf{x}_{k}).
\]

The Hessian approximation \(\mathbf{B}\) is chosen to satisfy
\[
  \nabla f(\mathbf{x}_{k} + \Delta\mathbf{x}) =
  \nabla f(\mathbf{x}_{k}) + \mathbf{B}~\Delta\mathbf{x},
\]
which is called the \textit{secant} equation, the Taylor series of the
gradient itself. Solving for \(\mathbf{B}\) and applying the Newton's
step with the updated value is equivalent to the \textit{secant} method.
Quasi-Newton methods are a generalization of the secant method to find
the root of the first derivative for multidimensional problems. The
various quasi-Newton methods differ in their choice of the solution to
the secant equation.

In a general quasi-Newton method, the unknown \(\mathbf{x}_{k}\) is
updated applying the Newton's step calculated using the current
approximate Hessian matrix \(\mathbf{B}_{k}\) in the following fashion:
\begin{itemize}
\item \(\Delta \mathbf{x}_{k} = -\alpha_{k}\mathbf{B}_{k}^{-1}\nabla
  f(\mathbf{x}_{k})\), with \(\alpha\) chosen to satisfy the so called
  Wolfe conditions~\cite[p.~34]{nocedal&wright};

\item \(\mathbf{x}_{k+1} = \mathbf{x}_{k} + \Delta\mathbf{x}_{k}\);

\item The gradient computed at the new point \(\nabla
  f(\mathbf{x}_{k+1})\), and \(\mathbf{y}_{k} = \nabla
  f(\mathbf{x}_{k+1}) - \nabla f(\mathbf{x}_{k})\) is used to update the
  approximate Hessian \(\mathbf{B}_{k+1}\), or directly its inverse
  \(\mathbf{H}_{k+1} = \mathbf{B}_{k+1}^{-1}\).
\end{itemize}

The most popular quasi-Newton method is the BFGS algorithm, named for
its discoverers Broyden, Fletcher, Goldfarb, and Shanno. It has the
following update formula
\begin{align*}
  \mathbf{B}_{k+1} &= \mathbf{B}_{k} +
                     \frac{\mathbf{y}_{k}\mathbf{y}_{k}^{\top}}{
                     \mathbf{y}_{k}^{\top}\Delta\mathbf{x}_{k}} -
                     \frac{\mathbf{B}_{k}\Delta\mathbf{x}_{k}
                     (\mathbf{B}_{k}\Delta\mathbf{x}_{k})^{\top}}{
                     \Delta\mathbf{x}_{k}^{\top}\mathbf{B}_{k}
                     \Delta\mathbf{x}_{k}},\\
  \mathbf{H}_{k+1} = \mathbf{B}_{k+1}^{-1}
                   &= \left(
                     \mathbf{I} -
                     \frac{\Delta\mathbf{x}_{k}\mathbf{y}_{k}^{\top}}{
                     \mathbf{y}_{k}^{\top}\Delta\mathbf{x}_{k}}
                     \right) \mathbf{H}_{k}
                     \left(
                     \mathbf{I} -
                     \frac{\mathbf{y}_{k}\Delta\mathbf{x}_{k}^{\top}}{
                     \mathbf{y}_{k}^{\top}\Delta\mathbf{x}_{k}}
                     \right) +
                     \frac{\Delta\mathbf{x}_{k}
                     \Delta\mathbf{x}_{k}^{\top}}{
                     \mathbf{y}_{k}^{\top}\Delta\mathbf{x}_{k}}.
\end{align*}

Another quasi-Newton method, popular in statistical data analysis, is
the one based on PORT routines~\url{http://www.netlib.org/port/}. A
Fortran mathematical subroutine library designed to be \textit{portable}
over different types of computers, and developed by David Gay in the
Bell Labs~\cite{PORTreport}. It is a quasi-Newton adaptive nonlinear
least-squares algorithm~\cite{PORTpaper} with the following update
formula
\begin{align*}
  \mathbf{B}_{k+1} &= \mathbf{B}_{k}\\
                   &+ \frac{
                     \left(\mathbf{y}_{k} -
                     \mathbf{B}_{k}\Delta\mathbf{x}_{k}\right)
                     \Delta\mathbf{x}_{k}^{\top}\mathbf{B}_{k} +
                     \mathbf{B}_{k}\Delta\mathbf{x}_{k}
                     \left(\mathbf{y}_{k} -
                     \mathbf{B}_{k}\Delta\mathbf{x}_{k}\right)^{\top}}{
                     \Delta\mathbf{x}_{k}^{\top}\mathbf{B}_{k}
                     \Delta\mathbf{x}_{k}}\\
                   &- \frac{\Delta\mathbf{x}_{k}^{\top}
                     \left(\mathbf{y}_{k} -
                     \mathbf{B}_{k}\Delta\mathbf{x}_{k}\right)
                     \mathbf{B}_{k}\Delta\mathbf{x}_{k}
                     \Delta\mathbf{x}_{k}^{\top}\mathbf{B}_{k}}{
                     \left(\Delta\mathbf{x}_{k}^{\top}\mathbf{B}_{k}
                     \Delta\mathbf{x}_{k}\right)^{\top}
                     \Delta\mathbf{x}_{k}^{\top}\mathbf{B}_{k}
                     \Delta\mathbf{x}_{k}}.
\end{align*}

As~\citeonline{nocedal&wright} points out, each quasi-Newton method
iteration can be performed at a cost of \(\mathcal{O}(n^{2})\)
arithmetic operations (plus the cost of function and gradient
evaluations); there are no \(\mathcal{O}(n^{3})\) operations such as
linear system solves or matrix-matrix operations. For the BFGS algorithm
is known that the rate of convergence is superlinear, but this is a
valid assumption to any quasi-Newton method, which is fast enough for
most practical purposes. Even though Newton's method converges more
rapidly, quadratically, its cost per iteration usually is higher,
because of its need for second derivatives and solution of a linear
system.

In this thesis, the used BFGS implementation is the one in the
\texttt{R}~\cite{R18}~function \texttt{optim()}, and the PORT routine
used is the one implemented in the \texttt{R} function
\texttt{nlminb()}.

\section{AUTOMATIC DIFFERENTIATION}
\label{cap:ad}

Computing gradients, \(\nabla f(\mathbf{x})\), are a fundamental and
crucial task, but also the main computational bottleneck for any Newton
and quasi-Newton method. Consequently, these computations are
fundamental to the development of this thesis. We choose to use the most
efficient manner of computing gradients, and one of the best scientific
computing techniques, the \textit{automatic differentiation} (AD)
procedure. AD has two modes, the so-called forward and reverse mode. We
will talk a bit about both, but we will use only the reverse mode. The
reason can be illustraded by a simple example, given later.

Automatic differentiation, also called algorithmic differentiation or
computational differentiation, is a set of techniques to numerically and
recursively evaluate the derivative of a function specified by a
computer program. AD techniques are based on the observation that any
function, no matter how complicated, is evaluated by performing a
sequence of simple elementary operations involving just one or two
arguments at a time. Derivatives of arbitrary order can be computed
automatically, automatized and accurately to working precision. Most of
the information in this section was taken of \citeonline{peyre}, but
\citeonline[p.~120]{corestats} and \citeonline[p.~204]{nocedal&wright}
are also very good references.

The most common differentiation approaches are finite differences (FD)
and symbolic calculus. Considering a function \(f: \mathbb{R}^{p}
\rightarrow \mathbb{R}\) and the goal of deriving a method to evaluate
\(\nabla f: \mathbb{R}^{p} \rightarrow \mathbb{R}^{p}\), the
approximation of this vector field via FD would require \(p + 1\)
evaluations of \(f\). The same task via reverse mode AD has in most
cases a cost proportional to a single evaluation of \(f\). AD is similar
to symbolic calculus in the sense that it provides an exact gradient
computation, up to machine precision. However, symbolic calculus does
not takes into account the underlying algorithm which compute the
function, while AD factorizes the computation of the derivative
according to an efficient algorithm.

The use of AD is inherent to the use of a computational graph,
\autoref{fig:compgraph}. Assuming that \(f\) is implemented in an
algorithm, the goal is to compute the derivatives
\begin{align*}
  &\frac{\partial f(\mathbf{x})}{\partial\mathbf{x}_{k}} \in
  \mathbb{R}^{n_{t} \times n_{k}},\\
  &\text{for a numerical algorithm
         (succession of functions) of the form}\\
  &\forall~k = s + 1, \dots, t, \quad
    \mathbf{x}_{k} = f_{k}(\mathbf{x}_{1}, \dots, \mathbf{x}_{k-1}),
\end{align*}
where \(f_{k}\) is a function which only depends on the previous
variables.

\begin{figure}[H]
  % \vspace{0.35cm}
  \setlength{\abovecaptionskip}{.0001pt}
  \caption{A COMPUTATIONAL GRAPH}
  \vspace{0.425cm} \centering
  \includegraphics[width=.8\textwidth]{computational_graph.png}
  \\
  \vspace{0.45cm}
  \begin{footnotesize}
    SOURCE:~\citeonline[p.~31]{peyre}.
  \end{footnotesize}
  \label{fig:compgraph}
\end{figure}

The computational graph, \autoref{fig:compgraph}, has the role of
represent the linking of the variables involved in \(f_{k}\) to
\(\mathbf{x}_{k}\). The evaluation of \(f(\mathbf{x})\) corresponds to a
forward traversal of this graph. Now, how exactly we evaluate \(f\)
through the graph? Via one of the AD modes.

\subsection{Forward Mode}

The forward mode correspond to the usual way of computing differentials.
The method initialize with the derivative of the input nodes
\[
  \frac{\partial \mathbf{x}_{1}}{\partial \mathbf{x}_{1}} =
  \text{Id}_{n_{1} \times n_{1}}, \quad
  \frac{\partial \mathbf{x}_{2}}{\partial \mathbf{x}_{1}} =
  \mathbf{0}_{n_{2} \times n_{1}}, \quad
  \frac{\partial \mathbf{x}_{s}}{\partial \mathbf{x}_{1}} =
  \mathbf{0}_{n_{s} \times n_{1}},
\]
and then iteratively make use of the following recursion formula
\begin{align*}
  &\forall~k = s + 1, \dots, t,\\
  &\frac{\partial\mathbf{x}_{k}}{\partial\mathbf{x}_{1}} =
    \sum_{l~\in~\text{father}(k)}
    \frac{\partial\mathbf{x}_{k}}{\partial\mathbf{x}_{l}} \times
    \frac{\partial\mathbf{x}_{l}}{\partial\mathbf{x}_{1}} =
    \sum_{l~\in~\text{father}(k)}
    \frac{\partial}{\partial\mathbf{x}_{l}}
    f_{k}(\mathbf{x}_{1}, \dots, \mathbf{x}_{k-1}) \times
    \frac{\partial\mathbf{x}_{l}}{\partial\mathbf{x}_{1}}.
\end{align*}

The notation ``father(\(k\))'' denotes the nodes \(l < k\) of the graph
that are connected to \(k\). We make use of \citeonline[p.~32]{peyre}'s
simple example.

\noindent\textbf{Example.}\hspace{.5cm}
Consider the function
\[
  f(x, y) = y\log(x) + \sqrt{y\log(x)}
\]
with the corresponding computational graph being displayed in
\autoref{fig:excompgraph}.

\begin{figure}[H]
  % \vspace{0.35cm}
  \setlength{\abovecaptionskip}{.0001pt}
  \caption{EXAMPLE OF A SIMPLE COMPUTATIONAL GRAPH}
  \vspace{0.425cm} \centering
  \includegraphics[width=.8\textwidth]{ex-computational_graph.png}
  \\
  \vspace{0.45cm}
  \begin{footnotesize}
    SOURCE:~\citeonline[p.~33]{peyre}.
  \end{footnotesize}
  \label{fig:excompgraph}
\end{figure}

The forward mode iterations to compute the derivative w.r.t. \(x\),
following the computational graph, is given by
\begin{align*}
  \frac{\partial x}{\partial x} &= 1, \quad
  \frac{\partial y}{\partial x} = 0\\
  \frac{\partial a}{\partial x} &=
  \frac{\partial a}{\partial x} \frac{\partial x}{\partial x} =
  \frac{1}{x} \frac{\partial x}{\partial x} \qquad
  &\{x \mapsto a = \log(x)\}\\
  \frac{\partial b}{\partial x} &=
  \frac{\partial b}{\partial a} \frac{\partial a}{\partial x} +
  \frac{\partial b}{\partial y} \frac{\partial y}{\partial x} =
  y \frac{\partial a}{\partial x} + 0 \qquad
  &\{(y, a) \mapsto b = ya\}\\
  \frac{\partial c}{\partial x} &=
  \frac{\partial c}{\partial b} \frac{\partial b}{\partial x} =
  \frac{1}{2\sqrt{b}} \frac{\partial b}{\partial x} \qquad
  &\{b \mapsto c = \sqrt{b}\}\\
  \frac{\partial f}{\partial x} &=
  \frac{\partial f}{\partial b} \frac{\partial b}{\partial x} +
  \frac{\partial f}{\partial c} \frac{\partial c}{\partial x} =
  1 \frac{\partial b}{\partial x} + 1 \frac{\partial c}{\partial x}
  \qquad &\{(b, c) \mapsto f = b + c\}
\end{align*}

To compute the derivative w.r.t. \(y\) we run another forward process
\begin{align*}
  \frac{\partial x}{\partial y} &= 0, \quad
  \frac{\partial y}{\partial y} = 1\\
  \frac{\partial a}{\partial y} &=
  \frac{\partial a}{\partial x} \frac{\partial x}{\partial y} = 0 \qquad
  &\{x \mapsto a = \log(x)\}\\
  \frac{\partial b}{\partial y} &=
  \frac{\partial b}{\partial a} \frac{\partial a}{\partial y} +
  \frac{\partial b}{\partial y} \frac{\partial y}{\partial y} =
  0 + a \frac{\partial y}{\partial y}\qquad
  &\{(y, a) \mapsto b = ya\}\\
  \frac{\partial c}{\partial y} &=
  \frac{\partial c}{\partial b} \frac{\partial b}{\partial y} =
  \frac{1}{2\sqrt{b}} \frac{\partial b}{\partial y} \qquad
  &\{b \mapsto c = \sqrt{b}\}\\
  \frac{\partial f}{\partial y} &=
  \frac{\partial f}{\partial b} \frac{\partial b}{\partial y} +
  \frac{\partial f}{\partial c} \frac{\partial c}{\partial y} =
  1 \frac{\partial b}{\partial y} + 1 \frac{\partial c}{\partial y}
  \qquad &\{(b, c) \mapsto f = b + c\}
\end{align*}

\subsection{Reverse Mode}

Instead of evaluating the differentials for all the input nodes, which
is problematic for a large number of nodes, the reverse mode evaluates
the differentials of the output node w.r.t. all the inner nodes.

The method initialize with the derivative of the final node
\[
  \frac{\partial \mathbf{x}_{t}}{\partial \mathbf{x}_{t}} =
  \text{Id}_{n_{y} \times n_{y}},
\]
and then, from the last to the first node, iteratively make use of the
following recursion formula
\begin{align*}
  &\forall~k = t - 1, t - 2, \dots, 1,\\
  &\frac{\partial\mathbf{x}_{t}}{\partial\mathbf{x}_{k}} =
    \sum_{m~\in~\text{son}(k)}
    \frac{\partial\mathbf{x}_{t}}{\partial\mathbf{x}_{m}} \times
    \frac{\partial\mathbf{x}_{m}}{\partial\mathbf{x}_{k}} =
    \sum_{m~\in~\text{son}(k)}
    \frac{\partial\mathbf{x}_{t}}{\partial\mathbf{x}_{m}} \times
    \frac{\partial}{\partial\mathbf{x}_{k}}
    f_{m}(\mathbf{x}_{1}, \dots, \mathbf{x}_{m}).
\end{align*}

The notation ``son(\(k\))'' denotes the nodes \(m < k\) of the graph
that are connected to \(k\). To be clear, the same simple example.

\noindent\textbf{Example.}\hspace{.5cm}
Consider, again, the function
\[
  f(x, y) = y\log(x) + \sqrt{y\log(x)}.
\]

The iterations of the reverse mode is given by
\begin{align*}
  \frac{\partial f}{\partial f} &= 1\\
  \frac{\partial f}{\partial c} &=
  \frac{\partial f}{\partial f} \frac{\partial f}{\partial c} =
  \frac{\partial f}{\partial f} 1\qquad &\{c \mapsto f = b + c\}\\
  \frac{\partial f}{\partial b} &=
  \frac{\partial f}{\partial c} \frac{\partial c}{\partial b} +
  \frac{\partial f}{\partial f} \frac{\partial f}{\partial b} =
  \frac{\partial f}{\partial c} \frac{1}{2\sqrt{b}} +
  \frac{\partial f}{\partial f} 1\qquad
  &\{b \mapsto c = \sqrt{b},~b \mapsto f = b + c\}\\
  \frac{\partial f}{\partial a} &=
  \frac{\partial f}{\partial b} \frac{\partial b}{\partial a} =
  \frac{\partial f}{\partial b} y\qquad &\{a \mapsto b = ya\}\\
  \frac{\partial f}{\partial y} &=
  \frac{\partial f}{\partial b} \frac{\partial b}{\partial y} =
  \frac{\partial f}{\partial b} a \qquad &\{y \mapsto b = ya\}\\
  \frac{\partial f}{\partial x} &=
  \frac{\partial f}{\partial a} \frac{\partial a}{\partial x} =
  \frac{\partial f}{\partial a} \frac{1}{x} \qquad
  &\{x \mapsto a = \log(x)\}
\end{align*}

This is the advantage of reverse mode over the forward mode. A single
traversal over the computational graph allows to compute both
derivatives w.r.t. \(x, y\), while the forward mode necessities two
processes.

An drawback of the reverse mode is the need to store the entire
computational graph, which is needed for the reverse sweep. In
principle, storage of this graph is not too difficult to implement.
However, the main benefit of AD is higher accuracy, and in many
applications the cost is not critical.


\section{TMB: TEMPLATE MODEL BUILDER}
\label{cap:tmb}

Note that the goal of AD is not to define an efficient computational
graph, it is up to the user to provide it. However, computing an
efficient graph associated to a mathematical formula is a complicated
combinatorial problem. Thus, since our goal is to be able to fit our
desired statistical models, a computational tool able to efficiently
define and implement this computational graph is make necessary. To
solve this and many other tasks, we have the Template Model Builder
(TMB) \cite{TMB}.

TMB \url{ http://tmb-project.org} is an \texttt{R} \cite{R18} package
for fitting statistical latent variable models to data, inpired by AD
Model Builder (ADMB) \cite{ADMB}. ADMB is a statistical application for
fitting nonlinear statistical models and solve optimization problems,
that implements AD using \texttt{C++} classes and a native template
language. Unlike most \texttt{R} packages, in TMB the model is
formulated in \texttt{C++}. This characteristic provides great
flexibility, but requires some familiarity with the
\texttt{C}/\texttt{C++} programming language. With TMB a user should be
able to quickly implement complex latent effect models through simple
\texttt{C++} templates.

In this chapter we describe step-by-step all the processes involved in
the creation and parameter estimation of a GLMM. With the TMB, all this
is put in practice in an efficient and robust fashion.

A user needs to provide just the joint likelihood function writing in a
\texttt{C++} template. If the model presents latent effects, during the
compilation the latent effects will be integrated out via an efficient
Laplace approximation routine, with a Newton algorithm inside, and the
marginal log-likelihood gradient will be also computed. These marginal
log-likelihood will be returned into an \texttt{R} object, that can then
be optimized using the user's favorite quasi-Newton routine, available
in \texttt{R}. To do all that, TMB combines some state of the art
software

\begin{itemize}
\item \texttt{CppAD}, a \texttt{C++} AD package
  \url{https://coin-or.github.io/CppAD/};
\item \texttt{Eigen} \cite{eigen}, a \texttt{C++} templated
  matrix-vector library;
\item \texttt{CHOLMOD}, sparse matrix routines available from
  \texttt{R}, used to obtain an efficient implementation of the Laplace
  approximation with exact derivatives
  \url{https://developer.nvidia.com/cholmod};
\item Parallelism through \texttt{BLAS}: Basic Linear Algebra
  Subprograms \url{http://www.netlib.org/blas/}.
\end{itemize}

Also, some of its key characteristics are

\begin{itemize}
\item TMB employs AD to calculate first and second order derivatives of
  the likelihood function or any objective function in \texttt{C++};
\item The objective function, and its derivatives, can be called from
  \texttt{R}. Hence, parameter estimation via \texttt{optim()} or
  \texttt{nlminb()} is easy to be performed;
\item Standard deviations of any parameter, or derived parameter, can be
  obtained via the \textit{delta method}.
\end{itemize}

Here we focus on GLMMs, but basically any statistical model with a
latent structure (or not), linear (or not), can be fitted with TMB. In
times of \textit{big data}, and with the TMB's authors having a
professional preference for state-of-space and spatial models, TMB has
also automatic sparseness detection.and some other nice built tools. Pre
and post-processing of data should be done in \texttt{R}.

A TMB Users' mailing list exists, and it is extremely helpful for taking
doubts and questions \url{https://groups.google.com/g/tmb-users}. Also,
a very didactic and comprehensive documentation with several examples is
available online
\url{https://kaskr.github.io/adcomp/_book/Tutorial.html}.

% END ==================================================================
% ----------------------------------------------------------------------
\chapter{\(\text{multiGLMM}\): a multinomial GLMM for clustered
  competing risks data}
\label{cap:model}
We are handling with a complex survival data structure, the clustered
competing risks setting. But we are using a general statistical modeling
framework, the generalized linear mixed models (GLMMs), that was not
made for this purpose.

To model competing risks data, one has to choose in which scale to work.
We can work on the hazard scale dealing with the cause-specific hazard
or on the probability scale dealing with the cause-specific cumulative
incidence function (CIF). With the correct link function, we can make an
appropriate GLMM to work on that probability scale.

Our focus in this thesis is to be able to deal with complex family
studies, where there is generally a strong interest in describing age at
disease onset in the scenarios of within-cluster dependence. The
distribution of age at disease onset is directly described by the
cause-specific CIF. To make a GLMM work for this type of data we need to
accommodate the cause-specific CIFs and the censorings. Assuming the
conditional distribution for our model response as multinomial already
deals with both left-truncation and right-censoring, avoiding the
specification of a censoring distribution. The cause-specific CIFs can
be modeled via the link function of our, then, multinomial GLMM
(multiGLMM). The multinomial distribution also guarantees that the CIFs
of all causes are modeled.

Our choice of a general framework tries to make the inference of this
complex model, easier. Besides, taking advantage of all the procedures
mentioned in the previous chapter.

\section{CLUSTER-SPECIFIC CUMULATIVE INCIDENCE FUNCTION
  (CIF)}
\label{cap:cif}

Consider that the observed follow-up time of an individual is given by
\(T = \min(T^{\ast},~C)\), where \(T^{\ast}\) denote the failure time
and \(C\) denote the censoring time. Given the possible covariates \(X\)
(that can be time-dependent), for a cause-specific of failure \(k\) the
CIF is defined as
\begin{align*}
  F_{k}(t \mid X) &= \mathbb{P}[T \leq t, K = k \mid X]\\
                  &= \int_{0}^{t} f_{k}(z \mid X)~\text{d}z\\
                  &= \int_{0}^{t} \lambda_{k}(z \mid X)~S(z \mid X)
                    ~\text{d}z, \quad t > 0, \quad k = 1, \dots, K.
\end{align*}
where \(f_{k}(t \mid X)\) is the (sub)density for the time to a type
\(k\) failure. This is the general definition of a CIF, and to define it
we need to define the functions that compose the subdensity.

The first is the cause-specific hazard function or process
\[
  \lambda_{k}(t \mid X) =
  \lim_{h \rightarrow 0}~h^{-1}
  \mathbb{P}[t \leq T < t + h, K = k \mid T \geq t, X],
  \quad t > 0, \quad k = 1, \dots, K.
\]

In words, the cause-specific hazard function, \(\lambda_{k}(t \mid X)\),
represents the instantaneous rate for failures of type \(k\) at time
\(t\) given \(X\) and all other failure types (competing causes). If
we sum up all cause-specific hazard function we get the overall hazard
function,
\[
  \lambda(t \mid X) = \sum_{k=1}^{K}\lambda_{k}(t \mid X).
\]

From the overall hazard function we arrive in the overall survival function,
\[
  S(t \mid X) =
  \mathbb{P}[T > t \mid X] =
  \exp\left\{-\int_{0}^{t} \lambda(z \mid X)~\text{d}z\right\},
\]
the second function that compose the subdensity \(f_{k}(t \mid X)\). A
comprehensive reference for all these definitions is the book of
\citeonline{kalb&prentice}.

Until this point, we were talking about a general CIF's definition. We
need now a precise framework telling how to take into consideration our
clustered/family structure. We use the same CIF specification of
\citeonline{SCHEIKE}, i.e. the approach that motivated this thesis.

For two competing causes of failure, the cause-specific CIFs are
specified in the following manner,
\begin{equation}
  F_{k} (t \mid X, u_{1}, u_{2}, \eta_{k}) =
  \underbrace{\pi_{k}(X, u_{1}, u_{2})}_{
    \substack{\text{cluster-specific}\\\text{risk level}}}\times
  \underbrace{\Phi[w_{k} g(t) - X^{\top}\gamma_{k} - \eta_{k}]}_{
    \substack{\text{cluster-specific}\\\text{failure time trajectory}}
  }, \quad t > 0, \quad k = 1,~2.
  \label{eq:cif}
\end{equation}
i.e. as a product of a cluster-specific risk level and a
cluster-specific failure time trajectory, resulting in a
cluster-specific CIF.

What makes the components in \autoref{eq:cif} cluster-specific are
\(\bm{u} = \{u_{1}, u_{2}\}\) and \(\bm{\eta} = \{\eta_{1},
\eta_{2}\}\), Gaussian distributed latent effects with zero mean and
potentially correlated, i.e.
\[
  \begin{bmatrix} u_{1}\\u_{2}\\\eta_{1}\\\eta_{2} \end{bmatrix} \sim
  \mathcal{N} \left(\begin{bmatrix} 0\\0\\0\\0\end{bmatrix},
    \begin{bmatrix}
      \sigma_{u_{1}}^{2}&
      \text{cov}(u_{1},~u_{2})&
      \text{cov}(u_{1},~\eta_{1})&\text{cov}(u_{1},~\eta_{2})\\
      &\sigma_{u_{2}}^{2}&
      \text{cov}(u_{2},~\eta_{1})&\text{cov}(u_{2},~\eta_{2})\\
      &&\sigma_{\eta_{1}}^{2}&\text{cov}(\eta_{1},~\eta_{2})\\
      &&&\sigma_{\eta_{2}}^{2}
    \end{bmatrix}\right).
\]

The cluster-specific survival function is given as \(S(t \mid X, \bm{u},
\bm{\eta}) = 1 - F_{1} (t \mid X, \bm{u}, \eta_{1}) - F_{2} (t \mid X,
\bm{u}, \eta_{2})\).

Since we use the same CIF specification of \citeonline{SCHEIKE}, the
following descriptions and details are essentially the same encountered
in the paper.

Focusing first on the second component of \autoref{eq:cif}. The
cluster-specific failure time trajectory
\[
  \Phi[w_{k} g(t) - X^{\top}\gamma_{k} - \eta_{k}],
  \quad t > 0, \quad k = 1, ~2,
\]
where \(\Phi(\cdot)\) is the cumulative distribution function of a
standard Gaussian distribution.

Instead of \(w_{k} g(t)\), in \citeonline{SCHEIKE} is specified
\(\alpha_{k}(g(t))\), where \(\alpha_{k}(\cdot)\) are monotonically
increasing functions known up to a finite-dimensional parameter vector,
\(w_{k}\). Examples are monotonically increasing B-spline or piecewise
lienar functions. However, to try to simplify the model structure we
consider just the finite-dimensional parameter vector. The bottom line
is that the authors do the same approach in their applications.

With regard to the function \(g(t)\), it plays a crucial role since the
separation of the CIF in \autoref{eq:cif} is only possible with it. A
time \(t\) transformation given by
\[
  g(t) = \text{arctanh}\left(\frac{t - \delta/2}{\delta/2}\right),
  \quad t\in~]0,~\delta[, \quad g(t)\in~]-\infty,~\infty[,
\]
where \(\delta\) depends on the data and cannot exceed the maximum
observed follow-up time \(\tau\), i.e. \(\delta \leq \tau\). With this
transformation, based on a Fisher transformation, the value of the
cluster-specific failure time trajectory is equal 1, at time \(\delta\).
Consequently, \(F_{k} (\delta \mid X, \bm{u}, \eta_{k}) = \pi_{k}(X \mid
\bm{u})\) and, we can interpret \(\pi_{1}(X \mid \bm{u})\) and
\(\pi_{2}(X \mid \bm{u})\) as the cause-specific cluster-specific risk
levels, at time \(\delta\).

The cluster-specific risk levels are modeled by a multinomial logistic
regression model with latent effects, i.e.
\begin{equation}
  \pi_{k}(X, \bm{u}) =
  \frac{\exp\{X^{\top}\beta_{k} + u_{k}\}}{1 +
    \exp\{X^{\top}\beta_{1} + u_{1}\} +
    \exp\{X^{\top}\beta_{2} + u_{2}\}}, \quad k = 1,~2,
  \label{eq:risklevel}
\end{equation}
where \(\beta_{k}\)'s are the parameters responsible for quantifying the
impact of the covariates in the cause-specific risk levels. For
individuals from the same chuster/family, at the same time point, the
\(\beta_{k}\)s have the well-known odds ratio interpretation.

The \(\gamma_{k}\)'s are the parameters responsible for quantifying the
impact of the covariates in the cause-specific failure time
trajectories, i.e. the shape of the cumulative incidence, and
consequently how quickly the cluster-specific risk levels observed at
time \(\delta\) are reached. The fact that \(\gamma_{k}\) enters
negatively in the cluster-specific failure time trajectory makes that a
negative value causes an advance towards the cluster-specific risk
level, whereas a covariate with a positive effect causes a delay.

Within-cluster dependence is induced by the latent effects in \(\bm{u}\)
and \(\bm{\eta}\), but they don't have an easy interpretation. To help
in the discussion, \autoref{fig:cif} illustrates the cluster-specific
CIF for a given failure cause, let's call it failure cause 1 (in total
we have two).

\begin{figure}[H]
  % \vspace{0.35cm}
  \setlength{\abovecaptionskip}{.0001pt}
  \caption{ILLUSTRATION OF A CLUSTER-SPECIFIC CUMULATIVE INCIDENCE
    FUNCTION (CIF), PROPOSED BY \citeonline{SCHEIKE}, FOR A GIVEN
    FAILURE CAUSE 1. FROM A CONFIGURATION WITH \(X = 1\) FOR ALL
    SUBJECTS AND WITH \(\beta_{1} = -1.9\), \(\beta_{2} = -0.2\),
    \(\gamma_{1} = 1\), \(w_{1} = 3\) AND \(u_{2} = 0\). THE VARIATION
    BETWEEN FRAMES IS GIVEN BY THE LATENT EFFECTS \(u_{1}\) AND
    \(\eta_{1}\)}
  \vspace{0.425cm} \centering
  \includegraphics[width=\textwidth]{cif-1.png}
  \\
  \vspace{0.45cm}
  \begin{footnotesize}
    SOURCE: The author (2020).
  \end{footnotesize}
  \label{fig:cif}
\end{figure}

The latent effects \(u_{1}\) and \(u_{2}\) always appear together in the
cluster-specific risk level, as consequency they have a joint effect on
the cumulative incidence of both causes. Nevertheless, as we can see in
\autoref{fig:cif}, an increase in \(u_{k}\) will increase the risk of
failure from cause \(k\) and vice versa. The interpretation of
\(\text{cov}(\eta_{1},~\eta_{2})\) and \(\text{cov}(u_{1},~u_{2})\) is
more or less straightforward. With regard to
\(\text{cov}(u_{k},~\eta_{k})\), a negative correlation between
\(\eta_{k}\) and \(u_{k}\) imply that when \(\eta_{k}\) decreases,
\(u_{k}\) increases and conversely when \(\eta_{K}\) increases,
\(u_{k}\) decreases. In other words, an increased risk level is reached
quickly and a decreased risk level is reached later, respectively.

Practical situations with a positive within-cause correlation are hard
to find, i.e. where an increased risk level is associated with a late
onset and vice versa. However, a positive cross-cause correlation
between \(\eta\) and \(u\) sounds more realistic. i.e. where late onset
of one failure cause is associated with a high absolute risk of another
failure cause.

The latent effects are assumed independent across clusters and shared by
individuals within the same cluster/family.

\section{MODEL SPECIFICATION}
\label{cap:modelitself}

Our generalized linear mixed model (GLMM) is specified in the following
fashion. For two competing causes of failure, a subject \(i\), with
cluster \(j\), in the time \(t\), we have
\begin{align}
  y_{i j t} \mid \{u_{1j},~u_{2j},~\eta_{1j},~\eta_{2j}\}&\sim
  \text{Multinomial}(p_{1ijt},~p_{2ijt},~p_{3ijt})\nonumber\\
  \nonumber\\
  \begin{bmatrix} u_{1}\\u_{2}\\\eta_{1}\\\eta_{2} \end{bmatrix}&\sim
  \mathcal{N} \left(\begin{bmatrix} 0\\0\\0\\0\end{bmatrix},
  \begin{bmatrix}
    \sigma_{u_{1}}^{2}&
    \text{cov}(u_{1},~u_{2})&
    \text{cov}(u_{1},~\eta_{1})&\text{cov}(u_{1},~\eta_{2})\\
    &\sigma_{u_{2}}^{2}&
    \text{cov}(u_{2},~\eta_{1})&\text{cov}(u_{2},~\eta_{2})\\
    &&\sigma_{\eta_{1}}^{2}&\text{cov}(\eta_{1},~\eta_{2})\\
    &&&\sigma_{\eta_{2}}^{2}
  \end{bmatrix}\right)\nonumber\\
  \nonumber\\
  p_{kijt} &=
  \frac{\partial}{\partial t}F_{k} (t \mid X, u_{1}, u_{2}, \eta_{k})
  \nonumber\\
  &= \frac{\exp\{\bm{x}_{kij}\bm{\beta}_{ki} + u_{kj}\}}{
    1 + \sum_{m=1}^{K-1}\exp\{\bm{x}_{mij}\bm{\beta}_{mi} + u_{mj}\}}
  \label{eq:model}\\
  &\times w_{k}\frac{\delta}{2\delta t - 2t^{2}}~
  \phi\left(
    w_{k}
    \text{arctanh}\left(\frac{t-\delta/2}{\delta/2}\right)
    - \bm{x}_{kij}\bm{\gamma}_{ki} - \eta_{kj}
    \right),\nonumber\\ k = 1,~2.\nonumber
\end{align}

The chosen link function to represent the probabilities is given by the
derivative w.r.t. time \(t\) of the cluster-specific CIF. The choice of
a multinomial logistic regression model ensures that the sum of the
predicted cause-specific CIFs does not exceed 1.

Considering two competing causes of failure, we have a multinomial with
three classes. The third class exists to handle the censorship and its
probability is given by the complementary to reach 1. This framework in
\autoref{eq:model} results in what we call multiGLMM, a multinomial
GLMM.

For a random sample, the corresponding marginal likelihood functions in
given by
\begin{align}
  L(\bm{\theta}~;~y)
  &= \prod_{j=1}^{J}~\int_{\Re^{4}}
    \pi(y_{j} \mid \bm{r}_{j})\times\pi(\bm{r}_{j})~\text{d}\bm{r}_{j}
    \nonumber\\
  &= \prod_{j=1}^{J}~\int_{\Re^{4}}
    \Bigg\{
    \underbrace{\prod_{i=1}^{n_{j}}~\prod_{t=1}^{n_{ij}}
    \Bigg(
    \frac{(\sum_{k=1}^{K}y_{kijt})!}{y_{1ijt}!~y_{2ijt}!~y_{3ijt}!}~
    \prod_{k=1}^{K} p_{kijt}^{y_{kijt}}
    \Bigg)}_{\substack{\text{fixed effect component}}}
  \Bigg\}\times\nonumber\\
  &\hspace{2cm}\underbrace{
    (2\pi)^{-2} |\Sigma|^{-1/2} \exp
    \left\{-\frac{1}{2}\bm{r}_{j}^{\top} \Sigma^{-1} \bm{r}_{j}\right\}
    }_{\substack{\text{latent effect component}}}
    \text{d}\bm{r}_{j}\nonumber\\
  &= \prod_{j=1}^{J}~\int_{\Re^{4}}
    \Bigg\{
    \underbrace{\prod_{i=1}^{n_{j}}~\prod_{t=1}^{n_{ij}}
    \prod_{k=1}^{K} p_{kijt}^{y_{kijt}}
    }_{\substack{\text{fixed effect}}}
  \Bigg\}\underbrace{
  (2\pi)^{-2} |\Sigma|^{-1/2} \exp
  \left\{-\frac{1}{2}\bm{r}_{j}^{\top} \Sigma^{-1} \bm{r}_{j}\right\}
  }_{\substack{\text{latent effect component}}}
  \text{d}\bm{r}_{j}\label{eq:loglik},
\end{align}
where \(\bm{\theta} = [\bm{\beta}~\bm{\gamma}~\bm{w}~\bm{\sigma^{2}}~
\bm{\varrho}]^{\top}\) is the parameters vector to be maximized. In our
framework, a subject can fail from just one competing cause or get
censor, at a given time. Thus, the fraction of factorials in the fixed
effect component is made only by 0's and 1's. Finally, returning the
value 1 .The matrix \(\Sigma\) is the variance-covariance matrix, which
components are given by \(\bm{\sigma}^{2}\) and \(\bm{\varrho}\).

Now, \autoref{eq:loglik} in words. To each cluster (family) \(j\) we
have a product of two components. The fixed effect component, given by a
multinomial distribution with its probabilities specified through the
cluster-specific CIF (\autoref{eq:cif}) and, the latent effect
component, given by a multivariate Gaussian distribution.

To each subject \(i\) that composes a cluster \(j\) we have its specific
fixed effects contribution. The likelihood in \autoref{eq:loglik} is the
most general as possible, allowing for repeated measures to each
subject. Since all subjects of a given cluster shares the same latent
effect, we have just one latent effect contribution multiplying the
product of fixed effects contribution. As we don't observe the latent
effect variables, \(\bm{r}_{j}\), we integrate out in it. With two
competing causes of failure, we have four latent effects (a multivariate
Gaussian distribution in four dimensions). As consequence, for each
cluster, we approximate an integral in four dimensions. The product of
these approximated integrals results in the called marginal likelihood,
to be maximized in \(\bm{\theta}\).

% END ==================================================================
% ----------------------------------------------------------------------
\chapter{simulation study datasets}
\label{cap:datasets}
This chapter describes how to simulate from our multiGLMM, and describes
a real-based dataset used as an application example. The simulation
procedure is addressed in \autoref{cap:simu}. In \autoref{cap:data} a
simulated dataset based on the Nordic Cancer Union (NCU) twins data is
presented as an application example.

\section{SIMULATING FROM THE MODEL}
\label{cap:simu}

Being able to simulate data from a model is a key task, fundamental to
assess the finite-sample properties and the estimation procedure
liability of a given statistical model. The step-by-step describing the
simulation procedure of our multiGLMM is presented on Algorithm
\autoref{alg:algo}, following the model hierarchical structure
stipulated in Formula \autoref{eq:model}.

\begin{algorithm}[H]
 \caption{SIMULATING FROM A \(\text{multiGLMM}\) FOR CLUSTERED COMPETING
          RISKS DATA}
 \label{alg:algo}
 \begin{algorithmic}[1]
  \State
   Set \(J\), the number of clusters
  \State
   Set \(n_{j}\), the number of cluster elements
   \Comment{can be of different sizes}
  \State
   Set \(K-1\), the number of competing causes of failure
  \State
   Set the model parameter values \(\bm{\theta} =
   [\bm{\beta}~\bm{\gamma}~\bm{w}~\bm{\sigma^{2}}~\bm{\varrho}]^{\top}\)
  \State
   Sample \(J\) latent effect vectors from a
   \(\mathcal{N}_{(K-1)\times(K-1)}(\bm{0},~\Sigma(\bm{\sigma^{2}},
     \bm{\varrho}))\)
  \State
   Set \(\delta\)
   \Comment{maximum follow-up time}
  \State
   Set the failure times \(t_{ij}\)
  \State
   Compute the competing risks probabilities
   \begin{align*}
      p_{kijt}
      &= \frac{\exp\{\bm{x}_{kij}\bm{\beta}_{ki} + u_{kj}\}}{
        1 +
        \sum_{m=1}^{K-1}\exp\{\bm{x}_{mij}\bm{\beta}_{mi} + u_{mj}\}}\\
      &\times
        w_{k}\frac{\delta}{2\delta t - 2t^{2}}~
        \phi\left(
        w_{k}
        \text{arctanh}\left(\frac{t-\delta/2}{\delta/2}\right)
        - \bm{x}_{kij}\bm{\gamma}_{ki} - \eta_{kj}
        \right),\\
      p_{Kijt}
      &= 1 - \sum_{k = 1}^{K - 1} p_{kijt}, \quad k = 1,~2,~\dots,~K -1
    \end{align*}
    \State
    Sample \(J\times n_{j}\) vectors from a
    \(\text{Multinomial}(p_{1ijt},~p_{2ijt},~\dots,~p_{Kijt})\)
    \State
    If \(t_{kij} = \delta\), subject moves to class K
    \Comment{any failure at time \(\delta\) is censored}
    \State
    \Return
    To each individual, its failure/censoring time and from which
    cause-specific it is
  \end{algorithmic}
\end{algorithm}
\vspace{-1cm}
\begin{footnotesize}
  \begin{center}
    SOURCE: The author (2021).
  \end{center}
\end{footnotesize}

Sample \(\varsigma\sim\text{U}(0,~1)\)

Compute the cause-specific failure times by solving
\[
 \varsigma = \Phi[w_{k} g(t_{k}) - X^{\top}\gamma_{k} - \eta_{k}]
 \quad\text{for } t_{k}, \quad k = 1,~2,~\dots,~K - 1
\]

The model described in \autoref{eq:model} is in its most general form,
i.e. allowing for multiple measures at each subject and varying
coefficients. However, we focus on a simpler structure without
covariates, a single measure per subject, and common coefficients.
Putting in practice Algorithm \autoref{alg:algo}, we use the following
model configuration

\begin{align}
  p_{kijt}
  &= \frac{\exp\{\beta_{ki} + u_{kj}\}}{
    1 + \sum_{m=1}^{K-1}\exp\{\beta_{mi} + u_{mj}\}}\nonumber\\
  &\times w_{k}\frac{\delta}{2\delta t - 2t^{2}}~
    \phi\left(
    w_{k}
    \text{arctanh}\left(\frac{t-\delta/2}{\delta/2}\right)
    - \gamma_{ki} - \eta_{kj}
    \right),\quad k = 1,~2,\nonumber\\
  \text{with }\quad
  \bm{\beta}_{i} &= [-2~~~1.5]^{\top}\nonumber\\
  \bm{\gamma}_{i} &= [1.2~~~1]^{\top}\label{eq:modelconfig}\\
  \bm{w} &= [3~~~5]^{\top}\nonumber\\
  \bm{u}_{j} &= [0~~~0]^{\top}\quad
               \bm{\eta}_{j} = [0~~~0]^{\top}\nonumber.
\end{align}
Based on that we get the cluster-specific CIF's and failure
probabilities, its CIF derivatives (dCIF) w.r.t. time \(t\), presented
respectively in \autoref{fig:datasimucif}.

\begin{figure}[H]
  \setlength{\abovecaptionskip}{.0001pt}
  \caption{CLUSTER-SPECIFIC CUMULATIVE INCIDENCE FUNCTIONS (CIF) AND
    RESPECTIVE DERIVATIVES W.R.T. TIME (\(\text{dCIF}\)) FOR A MODEL
    WITH TWO COMPETING CAUSES OF FAILURE, WITHOUT COVARIATES AND THE
    FOLLOWING CONFIGURATION: \(\beta_{1} = -2\), \(\beta_{2} = -1.5\),
    \(\gamma_{1} = 1.2\), \(\gamma = 1\), \(w_{1} = 3\), \(w_{2} = 5\)
    AND LATENT EFFECTS FIXED AT ZERO}
  \vspace{0.2cm} \centering
  \includegraphics[width=\textwidth]{datasimucif-1.png}
  \\
  \begin{footnotesize}
    SOURCE: The author (2020).
  \end{footnotesize}
  \label{fig:datasimucif}
\end{figure}

By adding the latent structure
\[
  \begin{bmatrix} u_{1}\\u_{2}\\\eta_{1}\\\eta_{2} \end{bmatrix}
  \sim\mathcal{N} \left(
    \begin{bmatrix} 0\\0\\0\\0 \end{bmatrix},
    \begin{bmatrix}
      1&0.4&0.5&0.4\\
      &1&0.4&0.3\\
      &&1&0.4\\
      &&&1
    \end{bmatrix}\right),
\]
in \autoref{eq:modelconfig}, we generate a complete model sample with
500 clusters/pairs of twins, summarized in \autoref{fig:datasimu}.

\begin{figure}[H]
  % \vspace{0.35cm}
  \setlength{\abovecaptionskip}{.0001pt}
  \caption{SUMMARY OF A SIMULATED DATASET WITH 500 PAIRS OF TWINS. A)
    TIME BY TWIN; B) TIMES BOXPLOT; C) PROBABILITIES SCATTERPLOT D)
    \(y_{3}\)'S \%}
  \vspace{0.2cm} \centering
  \includegraphics[width=\textwidth]{datasimu-1.png}
  \\
  \vspace{0.2cm}
  \begin{footnotesize}
    SOURCE: The author (2020).
  \end{footnotesize}
  \label{fig:datasimu}
\end{figure}

\section{REAL-BASED DATASET}
\label{cap:data}

% END ==================================================================

% ----------------------------------------------------------------------
\chapter{Results}
\label{cap:results}
This chapter presents the simulation study results. We have seventy-two
simulation scenarios, as detailed in \autoref{cap:datasets}. For each
scenario we simulate 500 samples. In total, we fit 36000 models.

\section{SIMULATION STUDY}
\label{cap:simures}

Let us just recap the parameter values used
\begin{align*}
 \text{High CIF configuration}:~&\quad
 \{\beta_{1} = -2,~\beta_{2} = -1.5,~\gamma_{1} = 1,~\gamma_{2} = 1.5,~
   w_{1} = 3,~w_{2} = 4
 \};\\
 \text{Low CIF configuration}:~&\quad
 \{\beta_{1} = 3,~\beta_{2} = 2.5,~\gamma_{1} = 2.6,~\gamma_{2} = 4,~
   w_{1} = 5,~w_{2} = 10
 \}.
\end{align*}
\begin{minipage}{0.15\textwidth}
 \begin{align*}
  \sigma_{u_{1}}^{2}   &= 1\\
  \sigma_{u_{2}}^{2}   &= 0.7,\\
  \sigma_{\eta_{1}}^{2} &= 0.6\\
  \sigma_{\eta_{2}}^{2} &= 0.9
 \end{align*}
\end{minipage}%
\begin{minipage}{0.85\textwidth}
 \[
  \text{Correlation structure}~=~\begin{blockarray}{ccccc}
                                  u_{1} & u_{2} & \eta_{1} & \eta_{2}\\
                                  \begin{block}{(cccc)c}
                                   1 & 0.1 & -0.5 &  0.3 & u_{1}\\
                                     &   1 &  0.3 & -0.4 & u_{2}\\
                                     &     &    1 &  0.2 & \eta_{1}\\
                                     &     &      &    1 & \eta_{2}\\
                                  \end{block}
                                 \end{blockarray}.
 \]
\end{minipage}

\vspace{0.3cm}
\noindent
The parameter values per se are not important. What is important is to
keep in mind the behaviors implied by them, and see if the proposed
model is able to estimate the true values in several different scenarios
and measure the quality of the estimates.

The take-home message for the fixed-effect parameters, is to show that
we can construct different level CIF scenarios. The \(\bm{\beta}\)s are
responsible for the curve maximum point or plateau, being in the risk
level CIF component, the \(\bm{\gamma}\)s and \(\bm{w}\)s are
responsible for basically the curve shape, being in the failure time
trajectory level CIF component. Its interpretation is presented in
detail in \autoref{cap:model}. About the latent-effects, the chosen
covariance structure is considerably high but still acceptable. The
underlying idea was to try to build a realistic covariance scenario and
consequently be able to check how the model performs in such conditions.

In the following pages we have several graphs summarizing the estimates
bias. In each figure, we have the estimate bias and its uncertainty
described by a Wald-based confidence interval i.e., \(\pm\) 1.96 the
bias standard deviation. This is a good uncertainty representation
choice since it is symmetric. In the \autoref{cap:appendixD}, we have
the same estimates bias but with its uncertainty measure being the
corresponding 2.5 and 97.5\% bias quantiles. We chose to use these
uncertainty representations uniquely based on the point estimates
instead of the standard error computations. In several scenarios, the
model fails to compute all the standard errors, caused by Hessian
numerical instabilities.

In each of the following estimates bias graphs, the seventy-two
scenarios are accommodated. We have up to four blocks of bars, each
block representing a model. In each block we have eighteen bars, each
bar representing the 500 fits in each of the eighteen
scenarios, \(4 \times 18 \times 500 = 36000\).

Each scenario name consists of a combination of three strings
\begin{itemize}
 \item The cluster size (cs), 2, 5, and 10;
 \item The CIF configuration, high and low;
 \item The sample size, 5, 30, and 60 thousand.
\end{itemize}
We have tried to fit a total of 36000 models but not all converged. To
show these characteristic, we control the bar widths. Something specific
can be said about each parameter but let us keep the focus on the
general remarks. Starting from the fixed-effect parameters
in \autoref{fig:biassdbeta1}, \autoref{fig:biassdbeta2},
\autoref{fig:biassdgama1}, \autoref{fig:biassdgama2},
\autoref{fig:biassdw1}, and \autoref{fig:biassdw2}, we have very nice
results that already show a strong inclination towards the complete
model's choice.

With a latent structure only in the risk level or in the failure time
trajectory level, the low CIF scenarios are the ones with a much smaller
bias-variance. In general, the mean-bias is small but the variances are
high. When we have a latent structure on both levels but we still assume
the cross-correlations as zero (block-diag model), the results get a
little bit better. Nevertheless, when we assume a non-zero
cross-correlation structure (complete model), basically everything
changes for the better. The mean biases get even closer to zero, the
standard deviations decrease 50\% or more, and mainly, now the high CIF
scenarios are the ones with a much smaller bias-variance. All this is
accomplished through the consideration of the cross-correlations.

In the \textit{simpler} models, with a latent structure just in one
level, is hard to see some significant difference between the clusters
and sample sizes. With the complete model, in the other hand, the
difference is clear: as we increase the clusters and the sample sizes,
the bias-variance decreases. The mean-bias is basically always the
same. In the risk model is hard to point-out a scenario as the best or
worst. For the time model, with the scenarios \texttt{cs02-high-05k}
and \texttt{cs05-high-60k}, we get a much bigger standard deviation in
the \(\bm{\beta}\)s parameter estimates. For the block-diag model, with
the scenario \texttt{cs05-low-05k}, the standard deviations are huge for
the shape curve parameter estimates of the competing cause 1. In
the \autoref{cap:appendixD}, with the 2.5 and 97.5\% bias quantiles, the
most extreme values are removed from the uncertainty
representation. There, the main characteristic is the parameter
estimates asymmetry.

\begin{figure}[H]
 \setlength{\abovecaptionskip}{.0001pt}
 \caption{PARAMETER \(\beta_{1}\) BIAS WITH \(\pm\) 1.96 STANDARD
          DEVIATIONS}
 \vspace{0.2cm}\centering
 \includegraphics[width=\textwidth]{bias2plotsd-1.png}\\
 \begin{footnotesize}
  SOURCE: The author (2021).
 \end{footnotesize}
 \label{fig:biassdbeta1}
\end{figure}

\begin{figure}[H]
 \setlength{\abovecaptionskip}{.0001pt}
 \caption{PARAMETER \(\beta_{2}\) BIAS WITH \(\pm\) 1.96 STANDARD
         DEVIATIONS}
 \vspace{0.2cm}\centering
 \includegraphics[width=\textwidth]{bias2plotsd-2.png}\\
 \begin{footnotesize}
  SOURCE: The author (2021).
 \end{footnotesize}
 \label{fig:biassdbeta2}
\end{figure}

\begin{figure}[H]
 \setlength{\abovecaptionskip}{.0001pt}
 \caption{PARAMETER \(\gamma_{1}\) BIAS WITH \(\pm\) 1.96 STANDARD
          DEVIATIONS}
 \vspace{0.2cm}\centering
 \includegraphics[width=\textwidth]{bias2plotsd-3.png}\\
 \begin{footnotesize}
  SOURCE: The author (2021).
 \end{footnotesize}
 \label{fig:biassdgama1}
\end{figure}

\begin{figure}[H]
 \setlength{\abovecaptionskip}{.0001pt}
 \caption{PARAMETER \(\gamma_{2}\) BIAS WITH \(\pm\) 1.96 STANDARD
          DEVIATIONS}
 \vspace{0.2cm}\centering
 \includegraphics[width=\textwidth]{bias2plotsd-4.png}\\
 \begin{footnotesize}
  SOURCE: The author (2021).
 \end{footnotesize}
 \label{fig:biassdgama2}
\end{figure}

\begin{figure}[H]
 \setlength{\abovecaptionskip}{.0001pt}
 \caption{PARAMETER \(w_{1}\) BIAS WITH \(\pm\) 1.96 STANDARD DEVIATIONS}
 \vspace{0.2cm}\centering
 \includegraphics[width=\textwidth]{bias2plotsd-5.png}\\
 \begin{footnotesize}
  SOURCE: The author (2021).
 \end{footnotesize}
 \label{fig:biassdw1}
\end{figure}

\begin{figure}[H]
 \setlength{\abovecaptionskip}{.0001pt}
 \caption{PARAMETER \(w_{2}\) BIAS WITH \(\pm\) 1.96 STANDARD DEVIATIONS}
 \vspace{0.2cm}\centering
 \includegraphics[width=\textwidth]{bias2plotsd-6.png}\\
 \begin{footnotesize}
  SOURCE: The author (2021).
 \end{footnotesize}
 \label{fig:biassdw2}
\end{figure}

\begin{figure}[H]
 \setlength{\abovecaptionskip}{.0001pt}
 \caption{PARAMETER \(\log(\sigma_{1}^{2})\) BIAS WITH \(\pm\) 1.96
          STANDARD DEVIATIONS}
 \vspace{0.2cm}\centering
 \includegraphics[width=\textwidth]{bias2plotsd-7.png}\\
 \begin{footnotesize}
  SOURCE: The author (2021).
 \end{footnotesize}
 \label{fig:biassdlogs2_1}
\end{figure}

\begin{figure}[H]
 \setlength{\abovecaptionskip}{.0001pt}
 \caption{PARAMETER \(\log(\sigma_{2}^{2})\) BIAS WITH \(\pm\) 1.96
          STANDARD DEVIATIONS}
 \vspace{0.2cm}\centering
 \includegraphics[width=\textwidth]{bias2plotsd-8.png}\\
 \begin{footnotesize}
  SOURCE: The author (2021).
 \end{footnotesize}
 \label{fig:biassdlogs2_2}
\end{figure}

\begin{figure}[H]
 \setlength{\abovecaptionskip}{.0001pt}
 \caption{PARAMETER \(\log(\sigma_{3}^{2})\) BIAS WITH \(\pm\) 1.96
          STANDARD DEVIATIONS}
 \vspace{0.2cm}\centering
 \includegraphics[width=\textwidth]{bias2plotsd-9.png}\\
 \begin{footnotesize}
  SOURCE: The author (2021).
 \end{footnotesize}
 \label{fig:biassdlogs2_3}
\end{figure}

\begin{figure}[H]
 \setlength{\abovecaptionskip}{.0001pt}
 \caption{PARAMETER \(\log(\sigma_{4}^{2})\) BIAS WITH \(\pm\) 1.96
          STANDARD DEVIATIONS}
 \vspace{0.2cm}\centering
 \includegraphics[width=\textwidth]{bias2plotsd-10.png}\\
 \begin{footnotesize}
  SOURCE: The author (2021).
 \end{footnotesize}
 \label{fig:biassdlogs2_4}
\end{figure}

\begin{figure}[H]
 \setlength{\abovecaptionskip}{.0001pt}
 \caption{PARAMETER \(z(\rho_{12})\) BIAS WITH \(\pm\) 1.96 STANDARD
          DEVIATIONS}
 \vspace{0.2cm}\centering
 \includegraphics[width=\textwidth]{bias2plotsd-11.png}\\
 \begin{footnotesize}
  SOURCE: The author (2021).
 \end{footnotesize}
 \label{fig:biassdrhoz12}
\end{figure}

\begin{figure}[H]
 \setlength{\abovecaptionskip}{.0001pt}
 \caption{PARAMETER \(z(\rho_{34})\) BIAS WITH \(\pm\) 1.96 STANDARD
          DEVIATIONS}
 \vspace{0.2cm}\centering
 \includegraphics[width=\textwidth]{bias2plotsd-12.png}\\
 \begin{footnotesize}
  SOURCE: The author (2021).
 \end{footnotesize}
 \label{fig:biassdrhoz34}
\end{figure}

\begin{figure}[H]
 \setlength{\abovecaptionskip}{.0001pt}
 \caption{PARAMETERS
          \(\{z(\rho_{13}),~z(\rho_{24}),~z(\rho_{14}),~z(\rho_{23})\}\)
          BIAS WITH \(\pm\) 1.96 STANDARD DEVIATIONS}
 \vspace{0.2cm}\centering
 \includegraphics[width=\textwidth]{bias2plotsd-13.png}\\
 \begin{footnotesize}
  SOURCE: The author (2021).
 \end{footnotesize}
 \label{fig:biassdrhoz4}
\end{figure}

With the log-variances presented in \autoref{fig:biassdlogs2_1},
\autoref{fig:biassdlogs2_2}, \autoref{fig:biassdlogs2_3}, and
\autoref{fig:biassdlogs2_4}, we have instead a similar behavior through
the models. For all the models, the high CIF scenarios are the ones with
a smaller mean and bias-variances. From the risk/time model to the
block-diag model, we do not see a significant improvement in terms of
bias reduction. Such improvement, however, is clear when we look at the
complete model. Again, the magick of considering the cross-correlations.

The same said about the log-variances, can be applied to the risk
correlations in \autoref{fig:biassdrhoz12}, with one addendum: the bias
reduction is even bigger. With the time correlation in
\autoref{fig:biassdrhoz34}, at least with clusters of size 2 and 5,
we get the same behavior observed with the fixed-effect parameters i.e.,
with the simpler models, the smaller biases are observed in the low CIF
scenarios. However, with the complete model, we get the opposite. With
the cross-correlations in
\autoref{fig:biassdrhoz4}, the mean and bias-variances are much smaller
in the high CIF scenarios.

The biggest bias-variances are obtained in the log-variances. A final
remark to be made is about convergences. With the simpler models, not
all of them work, having in some scenarios (generally the ones with 60
thousand data points) a 50\(\sim\)60\% convergence rate. With the
complete model, basically, almost all fits reach convergence
(\(\sim\)95\% performance).

After looking at the parameter estimates biases, let us take a look at
the implied mean-CIF curves. To nicely accommodate all seventy-two
scenarios we split the curves by level-CIF. In \autoref{fig:cifshigh} we
have the high CIF scenario curves and in \autoref{fig:cifslow} the low
CIF scenario curves. Since for all the models we have a latent structure
for the within-cluster dependency, the inherent idea is that this also
affect the fixed-effect parameter estimates. By taking its average in
each of the seventy-two scenarios, we are able to construct the mean CIF
curves.

In \autoref{fig:cifshigh} we have all the thirty-six curves obtained in
the high CIF scenarios. It is clear that with the complete model we get
a perfect fit in all nine scenarios. The risk and time models estimate
well the curve shape parameters but they fail to learn the max
incidence. A compensation between curves is clear.

\begin{figure}[H]
 \setlength{\abovecaptionskip}{.0001pt}
 \caption{HIGH CUMULATIVE INCIDENCE FUNCTION (CIF) SCENARIO CURVES}
 \vspace{0.2cm}\centering
 \includegraphics[width=\textwidth]{cifs-1.png}\\
 \begin{footnotesize}
  SOURCE: The author (2021).
 \end{footnotesize}
 \label{fig:cifshigh}
\end{figure}

\begin{figure}[H]
 \setlength{\abovecaptionskip}{.0001pt}
 \caption{LOW CUMULATIVE INCIDENCE FUNCTION (CIF) SCENARIO CURVES}
 \vspace{0.2cm}\centering
 \includegraphics[width=\textwidth]{cifs-2.png}\\
 \begin{footnotesize}
  SOURCE: The author (2021).
 \end{footnotesize}
 \label{fig:cifslow}
\end{figure}

Still in \autoref{fig:cifshigh}, in the risk model, there is a super
estimation of \(\beta_{1}\) in all scenarios. For failure cause 2, there
is a sub estimation. With the time model, we observe the opposite
compensation but on a smaller scale. With the time model, we get much
better curves than with the risk model. The block-diag model results are
a middle term between them. For the time model, the scenario with
cluster size 10 and 60 thousand data points is a highlight. For the
block-diag model, the highlight is the scenario with cluster size 5 and
30 thousand data points.

In the low CIF scenarios in \autoref{fig:cifslow}, the estimation is
clearly more difficult. The overall fits are bad, being impossible to
select a scenario with overall good results. For one of the failure
causes, the estimation quality is not so bad. The problem is when we
look to the other. An interesting scenario is the one with cluster size
2 and 60 thousand data points. In this scenario we see the worst fits
for failure cause 1, with a negative highlight in the block-diag
configuration. However, with this same model, for failure cause 2, it is
the scenario were we better learn the true curve. An interesting
compensation phenomena. The best joint fit is still with the complete
model.

Now we look at how the latent-effect parameter estimates distribute
themselves. Given the huge number of scenarios and the fact that is
harder to estimate covariance parameters, we chose to plot the parameter
estimates just in the scenarios with better performances. By the metrics
of small bias and CIF shape learning, the scenarios with better results
are the ones with high CIF and bigger sample sizes. We have the
densities for the variance parameter estimates, in each of these
scenarios, presented
in \autoref{fig:histologs2}. In \autoref{fig:historhoz} we have the same
for the correlation parameter estimates.

An interesting result is the clear difference between risk and time
models' covariance parameter estimates. With the risk model, we have an
evident super estimation and bigger variances. With the time model we
get much better results, but still with high variances. The block-diag
model generally performs better than the risk model and worst than the
time model, showing again to be a compromise between them. Besides the
bias itself, we should also pay attention to the values. We model the
variances in the log-scale, so a value 5, in reality, implies a variance
of \(\exp(5) = 148\). Terrible. This kind of problem do not sound to
appear with the complete model.

All correlations are quite well estimated, in all three scenarios, with
the complete model. Not only the correlations but the variances
also. The lack of any considerable difference between the covariance
densities, indicates no quality divergences in the results for different
cluster sizes. The densities in \autoref{fig:histologs2} and
\autoref{fig:historhoz} are the final corroboration indicating the good
performance of the maximum likelihood method in the complete
model.

Between the four tested models, the complete model was the one with the
smallest biases, better CIF shape learning, and precisest covariance
parameter estimates. In \autoref{fig:cor2plot} we have a heat-map of the
correlations between parameter estimates for the complete model in the
scenario with clusters of size 10, high CIF, and 60 thousand data
points.

We have a little bit of everything in the parameter estimates
correlations' heat-map. Some correlations are very close to zero, but we
also have strong positive and negative correlations. We can mention some
curiosities, but nothing pathological appears to happen, at least
nothing clear.

\begin{figure}[H]
 \setlength{\abovecaptionskip}{.0001pt}
 \caption{VARIANCE PARAMETERS DENSITIES IN THE SCENARIOS OF HIGH CIF AND
          60 THOUSAND DATA POINTS}
 \vspace{0.2cm}\centering
 \includegraphics[width=\textwidth]{histologs2-1.png}\\
 \begin{footnotesize}
  SOURCE: The author (2021).
 \end{footnotesize}
 \label{fig:histologs2}
\end{figure}

\begin{figure}[H]
 \setlength{\abovecaptionskip}{.0001pt}
 \caption{CORRELATION PARAMETERS DENSITIES IN THE SCENARIOS OF HIGH CIF
          AND 60 THOUSAND DATA POINTS}
 \vspace{0.2cm}\centering
 \includegraphics[width=\textwidth]{historhoz-1.png}\\
 \begin{footnotesize}
  SOURCE: The author (2021).
 \end{footnotesize}
 \label{fig:historhoz}
\end{figure}

In \autoref{fig:cor2plot}, all fixed-effect parameters are positive
correlated, with an emphasis on the correlation between \(\beta_{1}\)
and \(\beta_{2}\), and the one of the \(\bm{\beta}\)s with
the \(\bm{w}\)s. Another interesting observation is the strong negative
correlation between the \(\bm{\beta}\)s and the risk level
log-variances, and also the (less strong) positive correlation between
the \(\bm{\beta}\)s and the failure time trajectory level
log-variances. The risk level log-variances are (strongly) positively
correlated. So do the failure time trajectory level ones, but again, not
so strong as in the risk level. The correlations between the
log-variances of different levels are negative.

\begin{figure}[H]
 \setlength{\abovecaptionskip}{.0001pt}
 \caption{COMPLETE MODEL'S PARAMETERS CORRELATION HEAT-MAP IN THE
          SCENARIO OF CLUSTER SIZE 10, HIGH CIF, AND SIXTY-THOUSAND DATA
          POINTS}
 \centering
 \includegraphics[width=\textwidth]{cor2plot-1.png}\\
 \vspace{-0.2cm}
 \begin{footnotesize}
  SOURCE: The author (2021).
 \end{footnotesize}
 \label{fig:cor2plot}
\end{figure}

% END ==================================================================

% ----------------------------------------------------------------------
\chapter{Final considerations}
\label{cap:finalc}
The general goal of this master thesis was the proposition and study of
a maximum likelihood estimation approach for the analysis of clustered
competing risks data. Focused on the probability scale, by means of the
cumulative incidence function (CIF), instead of the hazard scale usual
in the survival modeling literature \cite{kalb&prentice}. We model the
clustered competing risks on a latent-effects framework, a generalized
linear mixed model (GLMM) \cite{GLMM}, with a multinomial distribution
for the competing risks and censorship, conditioned on the
latent-effects. The within-cluster latent dependency is accommodated by
a multivariate Gaussian distribution and is modeled via its covariance
matrix parameters.

The failures by the competing causes and their respective censorships
are modeled in the probability scale, by means of the CIF
\cite{kalb&prentice, andersen12}. The CIF is accommodated in our GLMM
framework in terms of the link function \cite{GLM89}, as the product of
two functions, one responsible to model the instantaneous risk and the
other the failure time trajectory, both in a cluster-specific
fashion. The shape of these functions is described in detail
in \autoref{cap:model}. This particular GLMM formulation is what makes
our model, particular. Thus, we have what we call a multiGLMM: a
multinomial GLMM for clustered competing risks data.

The two-function product CIF formulation was taken from
\citeonline{SCHEIKE} but there they use a different framework, a
composite likelihood framework \cite{lindsay88, cox&reid04, varin11}.
Here we do a full likelihood analysis instead. A composite approach is
generally used when a full likelihood approach is impossible or
computationally impracticable. Our goal here was to assess a full
likelihood framework taking advantage of state-of-the-art computational
libraries and very efficient algorithm implementations. We have all this
with the \texttt{R} \cite{R21} package TMB \cite{TMB}.

The applications in focus here were family studies. This kind of study
is characterized by involving big samples, generally, populations. Also,
generally having a high number of small clusters, families. A maximum
likelihood approach with the use of efficiently implemented Laplace
approximations \cite{tierney,patrao} together with an automatic
differentiation (AD) \cite{corestats,nocedal&wright} routine, all via
TMB, is able to handle a considerably high number of clusters,
independent of its size. The multinomial distribution assumption, on its
own, is an excellent probabilistic choice since it can accommodate
virtually any number of competing causes of failure and its
censorship. The presence of those two characteristics in our multiGLMM
makes it an efficient and scalable modeling framework for clustered
competing risks data.

Even with our modeling framework being virtually able to handle any
number of competing causes of failure, we restrained ourselves to work
here with only two of them. With two competing causes, we have
a \(4\times4\) covariance matrix for the latent effects, which implies
ten covariance parameters, which is already a lot of parameters to be
estimated in a latent structure. Since our goal was to study the
viability of the maximum likelihood estimation method, we kept it with
two causes.

All models from the simulation study were run, in a parallelized
fashion, in one of the two following Linux systems:
\begin{description}
 \item[System 1]
  12 Intel (R) Core (TM) i7-8750H CPU @ 2.20GHz processors
  with 16GB RAM;
 \item[System 2]
  30 Intel (R) Xeon (R) CPU E5-2690 v2 @ 3.00GHz processors
  and 206GB RAM.
\end{description}

Each risk and time model run is not so time-consuming, generally never
taking more than 5 minutes. The inherent idea is that for each cluster
we are always performing two-dimension integral approximations and we
have \textit{just} three covariance parameters. With the block-diag
model, we are theoretically in four dimensions. However, since the
covariance matrix is, block-diagonal, we experienced several numerical
instability problems. The solution, as can be seen in the
\autoref{cap:blockdiagModel} (\autoref{cap:appendixD}) code, was to
split it into two two-dimension matrices, since the \(4\times4\)
covariance matrix is block-diagonal.  This simple solution solved all
numerical instability problems. The computational time was only a little
bit bigger than with the risk and time models.

Finally, the complete model. In the biggest scenario, with 60 thousand
data points and clusters of size 2 i.e., with 30 thousand four-dimension
integral approximations (ten parameters in the covariance matrix), the
model fitting takes 30 minutes, in parallel, with TMB. Before doing the
TMB implementation, to really understand what we were doing, we did a
complete \texttt{R} implementation. We wrote the marginal log-likelihood
in \texttt{R}, based on our own Laplace approximation \cite{patrao} and
Newton-Raphson implementation (the gradients, \autoref{cap:appendixA},
and Hessian, \autoref{cap:appendixB}, were computed by hand and
implemented). Running this complete \texttt{R} implementation in a
scenario with 20 thousand data points and clusters of size 2, took
around 30 hours, parallelizing it between all threads of system 1. In
summary, by using TMB we were able to increase the model size 3 times
and to decrease the computational time 60 times. An incredible
performance gain.

Still, with the complete model, we performed a Bayesian analysis via
\texttt{tmbstan} \cite{tmbstan}. \texttt{tmbstan} enables MCMC sampling
\cite{MCMC, Diaconis} from a TMB model object using Stan \cite{Stan,
RStan}. Sampling can be performed with or without a Laplace
approximation for the random effects. We performed just one Bayesian
model fitting in a modest scenario with 5 thousand data points and
clusters of size 2. It took around 1 whole week of parallelized
processing in system 1. The results were basically the same as the ones
obtained with TMB but this high computational time just reinforces the
MCMC framework limitation.

An important point to be made here is about TMB's memory consumption. As
the sample size increases, the dimension of the model matrices also
increases. This, summed to a high number of clusters (Laplace
approximations to be performed), turns out to be a computational
nightmare. For several models, even the 16GB RAM of system 1 was not
enough. The bottleneck appears to be in the AD tape, which is made in
parallel, by default, if the model fitting is in parallel. By turning
this option off (line 11 of \autoref{cap:rscript}
(\autoref{cap:appendixD}) code), we were able to save a lot of memory,
making several models practicable.

Model the CIF of clustered competing risks data is far from being
trivial or straightforward. The formulation in \autoref{eq:cif} implies
the desired curve behavior, \autoref{fig:datasimucif}. However, in
counterpart, its derivatives w.r.t. time, generates very small
probabilities for the failure competing causes, ending by concentrating
almost everything on censorship, \autoref{fig:datasimu}. For each
competing cause with poor data representativity, we have three curve
shape parameters to estimate, implying the necessity of having a lot of
data to then have enough information about the causes.

We proposed for our multiGLMM an ideally complete latent-effects
formulation i.e., correlated latent effects on both levels,
instantaneous risk and failure time trajectory. The main underlying idea
of the \autoref{cap:results} simulation study was to see in which
scenarios we would be able to learn all the involved mean and covariance
parameters. As part of that, simpler formulations were proposed i.e.,
latent-effects in only one level, or in both but without
cross-correlations. As result, we got that latent effects only in the
risk level did not work. The optimization appears to get lost as if
something is missing. Inserting latent effects only in the failure time
trajectory level returned better results, but still not satisfactorily
good. In most of the evaluated scenarios, the block-diagonal model
appeared to be in the middle of them, as a compromise. The best results
were obtained with the complete model i.e. when we consider the
cross-correlations between levels. In general, we still observe some
high variances between the parameter estimates, but given all the
problem characteristics mentioned earlier, sounds to be reasonable. On
average, the complete model works fine, mainly in the scenarios of high
CIF configuration, and also as expected, as the sample size
increases. We can also say that as the cluster size increases, the
estimates get better but we did not have very strong results supporting
that.

\section{FUTURE WORKS}
\label{cap:future}

As show in \autoref{cap:results} results, even with the complete model
specification, the parameter estimates present an excessive variance.
In terms of a traditional GLMM specification \cite{GLMM}, we do not have
a lot more to do. We are already using a smart quasi-Newton algorithm
\cite{PORTpaper}, the most efficient derivatives computation technique
(AD) \cite{peyre}, and an also efficient Laplace approximation routine
\cite{corestats, patrao}, via TMB \cite{TMB}. We could change the
Laplace approximation for an adaptative Gaussian quadrature
\cite{quadrature}, but we do not see any good reason to do that.

There are two possible paths here. We could instead of a conditional
modeling framework (GLMM/latent-effects model), employ a marginal
modeling framework. In this framework, instead of caring about the
specification of a probability distribution to the competing causes
conditioned on the latent effects, we just care about the specification
of a mean and a variance structure. This approach does not have a
likelihood function per se, but the estimation procedure tends to be
easier than with the GLMM one. A marginal modeling framework that can be
used here is the multivariate covariance generalized linear model
(McGLM) \cite{mcglm, rmcglm}. How to exactly model the CIF of clustered
competing risks data in this framework, is something to still be figured
out.

The other path is by the use of a different way of modeling the
dependence structure. Instead of a latent-effects approach, we could use
copulas \cite{copulas,semiparametricSCHEIKE,gcmr,factorcopulas}. How to
do that is something to still be figured out by us, in terms of which
kind (conditional or marginal) and version (Archimedean-, Gauss-,
Maltesian-, \(t\)-, hyperbolic-, zebra-, and elliptical-) of copula to
use, besides the estimability issue.

% END ==================================================================

% ----------------------------------------------------------------------
\setlength{\afterchapskip}{\baselineskip}
% ----------------------------------------------------------------------
\bibliography{references}
% ----------------------------------------------------------------------
\postextual
% ----------------------------------------------------------------------
\begin{apendicesenv}
\partapendices
\addcontentsline{toc}{chapter}{\hspace{2.105cm}APPENDIX}
\renewcommand{\ABNTEXchapterfontsize}{\ABNTEXsectionfont}

\chapter{ANALYTIC GRADIENT OF THE LATENT EFFECTS FOR THE JOINT
         LOG-LIKELIHOOD FUNCTION OF THE MULTINOMIAL GLMM FOR CLUSTERED
         COMPETING RISKS DATA}
\label{cap:appendixA}

The following gradient components are computed by cluster, to be used
e.g., in a Newton optimization. Subject \(i\) at cluster \(j\) and for
competing cause \(k\)

\begin{align*}
  &\frac{\partial}{\partial u_{kj}}
    \log L(\bm{\theta}\mid\bm{y}_{j}, \bm{r}_{j}) =\\
  &y_{kij}\frac{1 +
    \sum_{m \neq k}^{K-1}\exp\{\bm{x}_{mij}\bm{\beta}_{mj} + u_{mj}\}
    }{1 +
    \sum_{n = 1}^{K-1}\exp\{\bm{x}_{nij}\bm{\beta}_{nj} + u_{nj}\}} -
    \left(\sum_{m \neq k}^{K-1} y_{mij}\right)
    \frac{\exp\{\bm{x}_{kij}\bm{\beta}_{kj} + u_{kj}\}
    }{1 +
    \sum_{n = 1}^{K-1}\exp\{\bm{x}_{nij}\bm{\beta}_{nj} + u_{nj}\}}-\\
  &y_{Kij}\frac{1}{1 +
    \sum_{n = 1}^{K-1}\exp\{\bm{x}_{nij}\bm{\beta}_{nj} + u_{nj}\}
    }\Bigg(\\
  &\frac{\exp\{\bm{x}_{kij}\bm{\beta}_{kj} + u_{kj}\}
    \left(1 +
    \sum_{m \neq k}^{K-1}\exp\{\bm{x}_{mij}\bm{\beta}_{mj} + u_{mj}\}
    \right)}{
    1 + \sum_{n = 1}^{K-1}\exp\{\bm{x}_{nij}\bm{\beta}_{nj} + u_{nj}\}
    }\times\\
  &\frac{w_{k}\frac{\delta}{2\delta t - 2t^{2}}
    \phi[w_{k}\text{arctanh}\left(\frac{t-\delta/2}{\delta/2}\right)
    - \bm{x}_{kij}\bm{\gamma}_{kj} - \eta_{kj}
    ]}{1 - w_{n}\frac{\delta}{2\delta t - 2t^{2}}
    \phi[w_{n}\text{arctanh}\left(\frac{t-\delta/2}{\delta/2}\right)
    - \bm{x}_{nij}\bm{\gamma}_{nj} - \eta_{nj}]} -
    \frac{\exp\{\bm{x}_{kij}\bm{\beta}_{kj} + u_{kj}\}}{
    1 + \sum_{n = 1}^{K-1}\exp\{\bm{x}_{nij}\bm{\beta}_{nj} + u_{nj}\}}
    \times\\
  &\frac{
    \sum_{m \neq k}^{K-1}
    w_{m}\frac{\delta}{2\delta t - 2t^{2}}
    \phi[w_{m}\text{arctanh}\left(\frac{t-\delta/2}{\delta/2}\right)
    - \bm{x}_{mij}\bm{\gamma}_{mj} - \eta_{mj}]
    \exp\{\bm{x}_{mij}\bm{\beta}_{mj} + u_{mj}\}}{
    1 - w_{n}\frac{\delta}{2\delta t - 2t^{2}}
    \phi[w_{n}\text{arctanh}\left(\frac{t-\delta/2}{\delta/2}\right)
    - \bm{x}_{nij}\bm{\gamma}_{nj} - \eta_{nj}]}\Bigg) -\\
  &\bm{e_{k}^{\top}Qr_{j}},\\
  %% -------------------------------------------------------------------
  \\
  %% -------------------------------------------------------------------
  &\frac{\partial}{\partial \eta_{kj}}
  \log L(\bm{\theta}\mid\bm{y}_{j}, \bm{r}_{j}) =\\
  &y_{kij} (w_{k}\text{arctanh}\left(\frac{t-\delta/2}{\delta/2}\right)
    - \bm{x}_{kij}\bm{\gamma}_{kj} - \eta_{kj}) -\\
  &y_{Kij}\frac{\exp\{\bm{x}_{kij}\bm{\beta}_{kj} + u_{kj}\}
    }{1 + \sum_{n = 1}^{K-1}\exp\{\bm{x}_{nij}\bm{\beta}_{nj} + u_{nj}\}}
    \times\\
  &\frac{
    w_{k}\frac{\delta}{2\delta t - 2t^{2}}
    (w_{k}\text{arctanh}\left(\frac{t-\delta/2}{\delta/2}\right)
    - \bm{x}_{kij}\bm{\gamma}_{kj} - \eta_{kj})
    \phi[w_{k} \text{arctanh}\left(\frac{t-\delta/2}{\delta/2}\right)
    - \bm{x}_{kij}\bm{\gamma}_{kj} - \eta_{kj}
    ]}{1 -
    \sum_{n = 1}^{K-1}
    \frac{\exp\{\bm{x}_{nij}\bm{\beta}_{nj} + u_{nj}\}}{1 +
    \sum_{n = 1}^{K-1}\exp\{\bm{x}_{nij}\bm{\beta}_{nj} + u_{nj}\}}
    w_{n}\frac{\delta}{2\delta t - 2t^{2}}
    \phi[w_{n}\text{arctanh}\left(\frac{t-\delta/2}{\delta/2}\right)
    - \bm{x}_{nij}\bm{\gamma}_{nj} - \eta_{nj}]} -\\
  &\bm{e_{k}^{\top}Qr_{j}},
\end{align*}
with \(\bm{e_{k}^{\top}}\) begin a vector with \(1\) at the \(k\)-th
position and zero elsewhere.

\chapter{ANALYTIC HESSIAN OF THE LATENT EFFECTS FOR THE JOINT
         LOG-LIKELIHOOD FUNCTION OF THE MULTINOMIAL GLMM FOR CLUSTERED
         COMPETING RISKS DATA}
\label{cap:appendixB}

The following hessian components are computed by cluster, to be used
e.g., in a Newton optimization. Subject \(i\) at cluster \(j\) and for
competing cause \(k\)
\begin{align*}
  &\frac{\partial^{2}}{\partial u_{kj}^{2}}
    \log L(\bm{\theta}\mid\bm{y}_{j}, \bm{r}_{j}) =\\
  &-\frac{\left(\sum_{k = 1}^{K-1} y_{kij}\right)
    \exp\{\bm{x}_{kij}\bm{\beta}_{kj} + u_{kj}\}
    \left(1 +
    \sum_{m \neq k}^{K-1}\exp\{\bm{x}_{mij}\bm{\beta}_{mj} + u_{mj}\}
    \right)}{\left(1 +
    \sum_{n = 1}^{K-1}\exp\{\bm{x}_{nij}\bm{\beta}_{nj} + u_{nj}\}
    \right)^{2}} +\\
  &\frac{y_{Kij}
    \exp\{\bm{x}_{kij} \bm{\beta}_{kj} + u_{kj}\}}{
    1 + \sum_{n = 1}^{K-1}\exp\{\bm{x}_{nij} \bm{\beta}_{nj} + u_{nj}\}
    }\times\\
  &\frac{
    \sum_{m \neq k}^{K-1}w_{m}\frac{\delta}{2\delta t - 2t^{2}}
    \phi[w_{m}\text{arctanh}\left(\frac{t-\delta/2}{\delta/2}\right)
    - \bm{x}_{mij}\bm{\gamma}_{mj} - \eta_{mj}]
    \exp\{\bm{x}_{mij}\bm{\beta}_{mj} + u_{mj}\}}{1 +
    \sum_{n = 1}^{K-1}\exp\{\bm{x}_{nij}\bm{\beta}_{nj} + u_{nj}\}
    (1 - w_{n}\frac{\delta}{2\delta t - 2t^{2}}
    \phi[w_{n}\text{arctanh}\left(\frac{t-\delta/2}{\delta/2}\right)
    - \bm{x}_{nij}\bm{\gamma}_{nj} - \eta_{nj}])} -\\
  &\frac{
    y_{Kij}
    w_{k}\frac{\delta}{2\delta t - 2t^{2}}
    \phi[w_{k}\text{arctanh}\left(\frac{t-\delta/2}{\delta/2}\right)
    - \bm{x}_{kij}\bm{\gamma}_{kj} - \eta_{kj}] }{1 +
    \sum_{n = 1}^{K-1}\exp\{\bm{x}_{nij}\bm{\beta}_{nj} + u_{nj}\}
    }\times\\
  &\frac{\exp\{\bm{x}_{kij}\bm{\beta}_{kj} + u_{kj}\}
    \left(1 +
    \sum_{m \neq k}^{K-1}\exp\{\bm{x}_{mij}\bm{\beta}_{mj} + u_{mj}\}
    \right)}{1 +
    \sum_{n = 1}^{K-1}\exp\{\bm{x}_{nij}\bm{\beta}_{nj} + u_{nj}\}
    (1 - w_{n}\frac{\delta}{2\delta t - 2t^{2}}
    \phi[w_{n}\text{arctanh}\left(\frac{t-\delta/2}{\delta/2}\right)
    - \bm{x}_{nij}\bm{\gamma}_{nj} - \eta_{nj}])} -\\
  &\frac{y_{Kij}\exp\{\bm{x}_{kij}\bm{\beta}_{kj} + u_{kj}\}}{\left(1 +
    \sum_{n = 1}^{K-1}\exp\{\bm{x}_{nij}\bm{\beta}_{nj} + u_{nj}\}
    \right)^{2}}\Bigg(\\
  &\frac{\sum_{m \neq k}^{K-1}
    w_{m}\frac{\delta}{2\delta t - 2t^{2}}
    \phi[w_{m}\text{arctanh}\left(\frac{t-\delta/2}{\delta/2}\right)
    - \bm{x}_{mij}\bm{\gamma}_{mj} - \eta_{mj}]
    \exp\{\bm{x}_{mij}\bm{\beta}_{mj} + u_{mj}\}}{\left(1 +
    \sum_{n = 1}^{K-1}\exp\{\bm{x}_{nij}\bm{\beta}_{nj} + u_{nj}\}
    (1 - w_{n}\frac{\delta}{2\delta t - 2t^{2}}
    \phi[w_{n}\text{arctanh}\left(\frac{t-\delta/2}{\delta/2}\right)
    - \bm{x}_{nij}\bm{\gamma}_{nj} - \eta_{nj}])\right)^{2}}-\\
  &\frac{w_{k}\frac{\delta}{2\delta t - 2t^{2}}
    \phi[w_{k}\text{arctanh}\left(\frac{t-\delta/2}{\delta/2}\right)
    - \bm{x}_{kij}\bm{\gamma}_{kj} - \eta_{kj}]\left(1 +
    \sum_{m \neq k}^{K-1}\exp\{\bm{x}_{mij}\bm{\beta}_{mj} + u_{mj}\}
    \right)}{\left(1 +
    \sum_{n = 1}^{K-1}\exp\{\bm{x}_{nij}\bm{\beta}_{nj} + u_{nj}\}
    (1 - w_{n}\frac{\delta}{2\delta t - 2t^{2}}
    \phi[w_{n}\text{arctanh}\left(\frac{t-\delta/2}{\delta/2}\right)
    - \bm{x}_{nij}\bm{\gamma}_{nj} - \eta_{nj}])\right)^{2}}\Bigg)\\
  &\times\Bigg(\Big(1 +\\
  &\sum_{n = 1}^{K-1}\exp\{\bm{x}_{nij}\bm{\beta}_{nj} + u_{nj}\}
    (1 - w_{n}\frac{\delta}{2\delta t - 2t^{2}}
    \phi[w_{n}\text{arctanh}\left(\frac{t-\delta/2}{\delta/2}\right)
    - \bm{x}_{nij}\bm{\gamma}_{nj} - \eta_{nj}])\Big) +\\
  &\Big(1 +
    \sum_{n = 1}^{K-1}\exp\{\bm{x}_{nij} \bm{\beta}_{nj} + u_{nj}\}
    \Big)\times\\
  &(1 - w_{k}\frac{\delta}{2\delta t - 2t^{2}}
    \phi[w_{k}\text{arctanh}\left(\frac{t-\delta/2}{\delta/2}\right)
    - \bm{x}_{kij}\bm{\gamma}_{kj} - \eta_{kj}])\Bigg)
    - \bm{e_{k}^{\top}Q},
\end{align*}

\begin{align*}
  &\frac{\partial^{2}}{\partial \eta_{kj}^{2}}
    \log L(\bm{\theta}\mid\bm{y}_{j}, \bm{r}_{j}) =\\
  &- y_{kij} - y_{Kij}
    \frac{\exp\{\bm{x}_{kij} \bm{\beta}_{kj} + u_{kj}\}}{1 +
    \sum_{n = 1}^{K-1}\exp\{\bm{x}_{nij} \bm{\beta}_{nj} + u_{nj}\}}\Bigg(\\
  &w_{k}\frac{\delta}{2\delta t - 2t^{2}}
    \phi[w_{k}\text{arctanh}\left(\frac{t-\delta/2}{\delta/2}\right)
    - \bm{x}_{kij}\bm{\gamma}_{kj} - \eta_{kj}]\times\\
  &\frac{\left(
    w_{k} \text{arctanh}\left(\frac{t-\delta/2}{\delta/2}\right)
    - \bm{x}_{kij}\bm{\gamma}_{kj} - \eta_{kj}
    \right)^{2} - 1}{
    1 - \sum_{n = 1}^{K-1}
    \frac{\exp\{\bm{x}_{nij}\bm{\beta}_{nj} + u_{nj}\}}{1 +
    \sum_{n = 1}^{K-1}\exp\{\bm{x}_{nij} \bm{\beta}_{nj} + u_{nj}\}}
    w_{n}\frac{\delta}{2\delta t - 2t^{2}}
    \phi[w_{n}\text{arctanh}\left(\frac{t-\delta/2}{\delta/2}\right)
    - \bm{x}_{nij}\bm{\gamma}_{nj} - \eta_{nj}]} -\\
  &\frac{\left(
    w_{k}\frac{\delta}{2\delta t - 2t^{2}}
    (w_{k}\text{arctanh}\left(\frac{t-\delta/2}{\delta/2}\right)
    - \bm{x}_{kij}\bm{\gamma}_{kj} - \eta_{kj})
    \phi[w_{k}\text{arctanh}\left(\frac{t-\delta/2}{\delta/2}\right)
    - \bm{x}_{kij}\bm{\gamma}_{kj} - \eta_{kj}]\right)^{2}}{\left(1 -
    \sum_{n = 1}^{K-1}
    \frac{\exp\{\bm{x}_{nij} \bm{\beta}_{nj} + u_{nj}\}}{1 +
    \sum_{n = 1}^{K-1}\exp\{\bm{x}_{nij} \bm{\beta}_{nj} + u_{nj}\}}
    w_{n}\frac{\delta}{2\delta t - 2t^{2}}
    \phi[w_{n}\text{arctanh}\left(\frac{t-\delta/2}{\delta/2}\right)
    - \bm{x}_{nij}\bm{\gamma}_{nj} - \eta_{nj}]\right)^{2}}\\
  &\Bigg) - \bm{e_{k}^{\top}Q},
\end{align*}

\begin{align*}
  &\frac{\partial^{2}}{\partial u_{kj} u_{mj}}
    \log L(\bm{\theta}\mid\bm{y}_{j}, \bm{r}_{j}) =\\
  &\left(\sum_{k = 1}^{K-1} y_{kij}\right)
    \frac{
    \exp\{\bm{x}_{kij}\bm{\beta}_{kj} + u_{kj}\}
    \exp\{\bm{x}_{mij}\bm{\beta}_{mj} + u_{mj}\}}{
    \left(1 +
    \sum_{n = 1}^{K-1}\exp\{\bm{x}_{nij} \bm{\beta}_{nj} + u_{nj}\}
    \right)^{2}} +\\
  &\frac{
    y_{Kij}
    \exp\{\bm{x}_{kij}\bm{\beta}_{kj} + u_{kj}\}
    \exp\{\bm{x}_{mij} \bm{\beta}_{mj} + u_{mj}\}}{1 +
    \sum_{n = 1}^{K-1}\exp\{\bm{x}_{nij} \bm{\beta}_{nj} + u_{nj}\}}\Bigg(\\
  &\frac{
    w_{m}\frac{\delta}{2\delta t - 2t^{2}}
    \phi[w_{m}\text{arctanh}\left(\frac{t-\delta/2}{\delta/2}\right)
    - \bm{x}_{mij}\bm{\gamma}_{mj} - \eta_{mj}]}{1 +
    \sum_{n = 1}^{K-1}\exp\{\bm{x}_{nij} \bm{\beta}_{nj} + u_{nj}\}
    (1 - w_{n}\frac{\delta}{2\delta t - 2t^{2}}
    \phi[w_{n}\text{arctanh}\left(\frac{t-\delta/2}{\delta/2}\right)
    - \bm{x}_{nij}\bm{\gamma}_{nj} - \eta_{nj}])} -\\
  &\frac{
    w_{k}\frac{\delta}{2\delta t - 2t^{2}}
    \phi[w_{k}\text{arctanh}\left(\frac{t-\delta/2}{\delta/2}\right)
    - \bm{x}_{kij}\bm{\gamma}_{kj} - \eta_{kj}]}{1 +
    \sum_{n = 1}^{K-1}\exp\{\bm{x}_{nij}\bm{\beta}_{nj} + u_{nj}\}
    (1 - w_{n}\frac{\delta}{2\delta t - 2t^{2}}
    \phi[w_{n}\text{arctanh}\left(\frac{t-\delta/2}{\delta/2}\right)
    - \bm{x}_{nij}\bm{\gamma}_{nj} - \eta_{nj}])}\Bigg) -\\
  &\frac{y_{Kij}}{
    \left(1 +
    \sum_{n = 1}^{K-1}\exp\{\bm{x}_{nij}\bm{\beta}_{nj} + u_{nj}\}
    \right)^{2}}\Bigg(\exp\{\bm{x}_{kij}\bm{\beta}_{kj} + u_{kj}\}\Bigg(\\
  &\frac{
    \sum_{m \neq k}^{K-1}
    w_{m}\frac{\delta}{2\delta t - 2t^{2}}
    \phi[w_{m}\text{arctanh}\left(\frac{t-\delta/2}{\delta/2}\right)
    - \bm{x}_{mij}\bm{\gamma}_{mj} - \eta_{mj}]
    \exp\{\bm{x}_{mij} \bm{\beta}_{mj} + u_{mj}\}}{
    \left(1 + \sum_{n = 1}^{K-1}\exp\{\bm{x}_{nij}\bm{\beta}_{nj} + u_{nj}\}
    (1 - w_{n}\frac{\delta}{2\delta t - 2t^{2}}
    \phi[w_{n}\text{arctanh}\left(\frac{t-\delta/2}{\delta/2}\right)
    - \bm{x}_{nij}\bm{\gamma}_{nj} - \eta_{nj}])\right)^{2}} -\\
  &\frac{
    w_{k}\frac{\delta}{2\delta t - 2t^{2}}
    \phi[w_{k}\text{arctanh}\left(\frac{t-\delta/2}{\delta/2}\right)
    - \bm{x}_{kij}\bm{\gamma}_{kj} - \eta_{kj}]
    \left(1 +
    \sum_{m \neq k}^{K-1}\exp\{\bm{x}_{mij}\bm{\beta}_{mj} + u_{mj}\}
    \right)}{\left(1 +
    \sum_{n = 1}^{K-1}\exp\{\bm{x}_{nij} \bm{\beta}_{nj} + u_{nj}\}
    (1 - w_{n}\frac{\delta}{2\delta t - 2t^{2}}
    \phi[w_{n}\text{arctanh}\left(\frac{t-\delta/2}{\delta/2}\right)
    - \bm{x}_{nij}\bm{\gamma}_{nj} - \eta_{nj}])\right)^{2}}\Bigg)
\end{align*}
\begin{align*}
  &\Bigg)\times\Bigg(\exp\{\bm{x}_{mij}\bm{\beta}_{mj} + u_{mj}\}
    \Big(1 +\\
  &\sum_{n = 1}^{K-1}\exp\{\bm{x}_{nij}\bm{\beta}_{nj} + u_{nj}\}
    (1 - w_{n}\frac{\delta}{2\delta t - 2t^{2}}
    \phi[w_{n}\text{arctanh}\left(\frac{t-\delta/2}{\delta/2}\right)
    - \bm{x}_{nij}\bm{\gamma}_{nj} - \eta_{nj}])\Big) +\\
  &\exp\{\bm{x}_{mij}\bm{\beta}_{mj} + u_{mj}\}
    (1 - w_{m}\frac{\delta}{2\delta t - 2t^{2}}
    \phi[w_{m}\text{arctanh}\left(\frac{t-\delta/2}{\delta/2}\right)
    - \bm{x}_{mij}\bm{\gamma}_{mj} - \eta_{mj}])\Big(1 +\\
  &\sum_{n = 1}^{K-1}\exp\{\bm{x}_{nij}\bm{\beta}_{nj} + u_{nj}\}\Big)
    \Bigg) - \bm{e_{k}^{\top}Q},
\end{align*}

\begin{align*}
  &\frac{\partial^{2}}{\partial \eta_{kj} \eta_{mj}}
    \log L(\bm{\theta}\mid\bm{y}_{j}, \bm{r}_{j}) =\\
  &- y_{Kij}\frac{
    \exp\{\bm{x}_{kij}\bm{\beta}_{kj} + u_{kj}\}}{1 +
    \sum_{n = 1}^{K-1}\exp\{\bm{x}_{nij}\bm{\beta}_{nj} + u_{nj}\}}\times\\
  &\frac{w_{k}\frac{\delta}{2\delta t - 2t^{2}}
    (w_{k}\text{arctanh}\left(\frac{t-\delta/2}{\delta/2}\right)
    - \bm{x}_{kij}\bm{\gamma}_{kj} - \eta_{kj})
    \phi[w_{k}\text{arctanh}\left(\frac{t-\delta/2}{\delta/2}\right)
    - \bm{x}_{kij}\bm{\gamma}_{kj} - \eta_{kj}]}{\left(1 -
    \sum_{n = 1}^{K-1}\frac{\exp\{\bm{x}_{nij}\bm{\beta}_{nj} + u_{nj}\}
    }{1 +
    \sum_{n = 1}^{K-1}\exp\{\bm{x}_{nij}\bm{\beta}_{nj} + u_{nj}\}}
    w_{n}\frac{\delta}{2\delta t - 2t^{2}}
    \phi[w_{n}\text{arctanh}\left(\frac{t-\delta/2}{\delta/2}\right)
    - \bm{x}_{nij}\bm{\gamma}_{nj} - \eta_{nj}]\right)^{2}}\times\\
  &\frac{\exp\{\bm{x}_{mij}\bm{\beta}_{mj} + u_{mj}\}}{1 +
    \sum_{n = 1}^{K-1}\exp\{\bm{x}_{nij}\bm{\beta}_{nj} + u_{nj}\}}
    w_{m}\frac{\delta}{2\delta t - 2t^{2}}
    (w_{m}\text{arctanh}\left(\frac{t-\delta/2}{\delta/2}\right)
    - \bm{x}_{mij}\bm{\gamma}_{mj} - \eta_{mj})\times\\
  &\phi[w_{m}\text{arctanh}\left(\frac{t-\delta/2}{\delta/2}\right)
    - \bm{x}_{mij}\bm{\gamma}_{mj} - \eta_{mj}] - \bm{e_{k}^{\top}Q},
\end{align*}

\begin{align*}
  &\frac{\partial^{2}}{\partial \eta_{kj} u_{kj}}
    \log L(\bm{\theta}\mid\bm{y}_{j}, \bm{r}_{j}) =\\
  &y_{Kij}
    \frac{\exp\{\bm{x}_{kij}\bm{\beta}_{kj} + u_{kj}\}}{1 +
    \sum_{n = 1}^{K-1}\exp\{\bm{x}_{nij}\bm{\beta}_{nj} + u_{nj}\}}\times\\
  &\frac{
    w_{k}\frac{\delta}{2\delta t - 2t^{2}}
    (w_{k}\text{arctanh}\left(\frac{t-\delta/2}{\delta/2}\right)
    - \bm{x}_{kij}\bm{\gamma}_{kj} - \eta_{kj})
    \phi[w_{k}\text{arctanh}\left(\frac{t-\delta/2}{\delta/2}\right)
    - \bm{x}_{kij}\bm{\gamma}_{kj} - \eta_{kj})]}{
    \left(1 -
    \sum_{n = 1}^{K-1}\frac{\exp\{\bm{x}_{nij}\bm{\beta}_{nj} + u_{nj}\}}{
    1 +
    \sum_{n = 1}^{K-1}\exp\{\bm{x}_{nij}\bm{\beta}_{nj} + u_{nj}\}}
    w_{n}\frac{\delta}{2\delta t - 2t^{2}}
    \phi[w_{n}\text{arctanh}\left(\frac{t-\delta/2}{\delta/2}\right)
    - \bm{x}_{nij}\bm{\gamma}_{nj} - \eta_{nj}]\right)^{2}}\times\\
  &\Bigg(
    \sum_{n \neq k}^{K-1}
    \frac{
    \exp\{\bm{x}_{nij}\bm{\beta}_{nj} + u_{nj}\}
    \exp\{\bm{x}_{kij}\bm{\beta}_{kj} + u_{kj}\}}{
    \left(1 +
    \sum_{n = 1}^{K-1}\exp\{\bm{x}_{nij}\bm{\beta}_{nj} + u_{nj}\}
    \right)^{2}}\times\\
  &w_{n}\frac{\delta}{2\delta t - 2t^{2}}
    \phi[w_{n}\text{arctanh}\left(\frac{t-\delta/2}{\delta/2}\right)
    - \bm{x}_{nij}\bm{\gamma}_{nj} - \eta_{nj}] -\\
  &\frac{\exp\{\bm{x}_{kij}\bm{\beta}_{kj} + u_{kj}\}
    \left(
    \left(1 +
    \sum_{n = 1}^{K-1}\exp\{\bm{x}_{nij}\bm{\beta}_{nj} + u_{nj}\}
    \right) - \exp\{\bm{x}_{kij} \bm{\beta}_{kj} + u_{kj}\}
    \right)}{
    \left(1 +
    \sum_{n = 1}^{K-1}\exp\{\bm{x}_{nij}\bm{\beta}_{nj} + u_{nj}\}
    \right)^{2}}\times\\
  &w_{k}\frac{\delta}{2\delta t - 2t^{2}}
    \phi[w_{k}\text{arctanh}\left(\frac{t-\delta/2}{\delta/2}\right)
    - \bm{x}_{kij}\bm{\gamma}_{kj} - \eta_{kj}]\Bigg) -
\end{align*}
\begin{align*}
  &y_{Kij}
    \frac{
    \frac{\exp\{\bm{x}_{kij}\bm{\beta}_{kj} + u_{kj}\}
    \left(
    \left(1 +
    \sum_{n = 1}^{K-1}\exp\{\bm{x}_{nij}\bm{\beta}_{ni} + u_{nj}\}
    \right) - \exp\{\bm{x}_{kij} \bm{\beta}_{kj} + u_{kj}\}
    \right)}{
    \left(1 +
    \sum_{n = 1}^{K-1}\exp\{\bm{x}_{nij}\bm{\beta}_{nj} + u_{nj}\}
    \right)^{2}}}{1 -
    \sum_{n = 1}^{K-1}\frac{\exp\{\bm{x}_{nij}\bm{\beta}_{nj} + u_{nj}\}}{
    1 + \sum_{n = 1}^{K-1}\exp\{\bm{x}_{nij}\bm{\beta}_{nj} + u_{nj}\}}
    w_{n}\frac{\delta}{2\delta t - 2t^{2}}
    \phi[w_{n}\text{arctanh}\left(\frac{t-\delta/2}{\delta/2}\right)
    - \bm{x}_{nij}\bm{\gamma}_{nj} - \eta_{nj}]}\times\\
  &w_{k}\frac{\delta}{2\delta t - 2t^{2}}
    (w_{k}\text{arctanh}\left(\frac{t-\delta/2}{\delta/2}\right)
    - \bm{x}_{kij}\bm{\gamma}_{kj} - \eta_{kj})\times\\
  &\phi[w_{k}\text{arctanh}\left(\frac{t-\delta/2}{\delta/2}\right)
    - \bm{x}_{kij}\bm{\gamma}_{kj} - \eta_{kj}] - \bm{e_{k}^{\top}Q},
\end{align*}

\begin{align*}
  &\frac{\partial^{2}}{\partial \eta_{kj} u_{mj}}
    \log L(\bm{\theta}\mid\bm{y}_{j}, \bm{r}_{j}) =\\
  &y_{Kij}
    \frac{\exp\{\bm{x}_{kij}\bm{\beta}_{kj} + u_{kj}\}
    \exp\{\bm{x}_{mij}\bm{\beta}_{mj} + u_{mj}\}}{
    \left(1 +
    \sum_{n = 1}^{K-1}\exp\{\bm{x}_{nij}\bm{\beta}_{nj} + u_{nj}\}
    \right)^{2}}\times\\
  &\frac{
    w_{k}\frac{\delta}{2\delta t - 2t^{2}}
    (w_{k} \text{arctanh}\left(\frac{t-\delta/2}{\delta/2}\right)
    - \bm{x}_{kij}\bm{\gamma}_{kj} - \eta_{kj})
    \phi[w_{k}\text{arctanh}\left(\frac{t-\delta/2}{\delta/2}\right)
    - \bm{x}_{kij}\bm{\gamma}_{kj} - \eta_{kj})]}{1 - \sum_{n = 1}^{K-1}
    \frac{\exp\{\bm{x}_{nij}\bm{\beta}_{nj} + u_{nj}\}}{1 +
    \sum_{n = 1}^{K-1}\exp\{\bm{x}_{nij}\bm{\beta}_{nj} + u_{nj}\}}
    w_{n}\frac{\delta}{2\delta t - 2t^{2}}
    \phi[w_{n}\text{arctanh}\left(\frac{t-\delta/2}{\delta/2}\right)
    - \bm{x}_{nij}\bm{\gamma}_{nj} - \eta_{nj}]} +\\
  &y_{Kij}
    \frac{\exp\{\bm{x}_{kij}\bm{\beta}_{kj} + u_{kj}\}}{1 +
    \sum_{n = 1}^{K-1}\exp\{\bm{x}_{nij}\bm{\beta}_{nj} + u_{nj}\}}\times\\
  &\frac{
    w_{k}\frac{\delta}{2\delta t - 2t^{2}}
    (w_{k}\text{arctanh}\left(\frac{t-\delta/2}{\delta/2}\right)
    - \bm{x}_{kij}\bm{\gamma}_{kj} - \eta_{kj})
    \phi[w_{k}\text{arctanh}\left(\frac{t-\delta/2}{\delta/2}\right)
    - \bm{x}_{kij}\bm{\gamma}_{kj} - \eta_{kj})]}{
    \left(1 - \sum_{n = 1}^{K-1}
    \frac{\exp\{\bm{x}_{nij}\bm{\beta}_{nj} + u_{nj}\}}{1 +
    \sum_{n = 1}^{K-1}\exp\{\bm{x}_{nij}\bm{\beta}_{nj} + u_{nj}\}}
    w_{n}\frac{\delta}{2\delta t - 2t^{2}}
    \phi[w_{n}\text{arctanh}\left(\frac{t-\delta/2}{\delta/2}\right)
    - \bm{x}_{nij}\bm{\gamma}_{nj} - \eta_{nj}]\right)^{2}}\times\\
  &\Bigg(
    \sum_{n \neq m}^{K-1}\frac{
    \exp\{\bm{x}_{nij}\bm{\beta}_{nj} + u_{nj}\}
    \exp\{\bm{x}_{mij}\bm{\beta}_{mj} + u_{mj}\}}{
    \left(1 +
    \sum_{n = 1}^{K-1}\exp\{\bm{x}_{nij}\bm{\beta}_{nj} + u_{nj}\}
    \right)^{2}}\times\\
  &w_{n}\frac{\delta}{2\delta t - 2t^{2}}
    \phi[w_{n}\text{arctanh}\left(\frac{t-\delta/2}{\delta/2}\right)
    - \bm{x}_{nij}\bm{\gamma}_{nj} - \eta_{nj}] -\\
  &\frac{\exp\{\bm{x}_{mij}\bm{\beta}_{mj} + u_{mj}\}
    \left(
    \left(1 +
    \sum_{n = 1}^{K-1}\exp\{\bm{x}_{nij}\bm{\beta}_{nj} + u_{nj}\}
    \right) - \exp\{\bm{x}_{mij} \bm{\beta}_{mj} + u_{mj}\}
    \right)}{
    \left(1 + \sum_{n = 1}^{K-1}\exp\{\bm{x}_{nij}\bm{\beta}_{nj} + u_{nj}\}
    \right)^{2}}\times\\
  &w_{m}\frac{\delta}{2\delta t - 2t^{2}}
    \phi[w_{m}\text{arctanh}\left(\frac{t-\delta/2}{\delta/2}\right)
    - \bm{x}_{mij}\bm{\gamma}_{mj} - \eta_{mj}]\Bigg) - \bm{e_{k}^{\top}Q},
\end{align*}
with \(\bm{e_{k}^{\top}}\) begin a vector with \(1\) at the \(k\)-th
position and zero elsewhere.

\chapter{\texttt{R} CODE TO SIMULATE FROM A \(\text{multiGLMM}\) WITH
         TWO COMPETING CAUSES AND CLUSTERS OF SIZE TWO. FOR MORE
         INFORMATION CHECK SECTION \ref{cap:simu}}
\label{cap:appendixC}

\lstinputlisting[firstline=97,lastline=153]{datasets.Rmd}
\vspace{-0.5cm}
\begin{center}
 \begin{footnotesize}
  SOURCE: The author (2021).
 \end{footnotesize}
\end{center}

\end{apendicesenv}
% ----------------------------------------------------------------------
% \begin{anexosenv}
% \partanexos
% \addcontentsline{toc}{chapter}{\hspace{2.105cm}ANNEX}
% \renewcommand{\ABNTEXchapterfontsize}{\ABNTEXsectionfont}
% \end{anexosenv}
%-----------------------------------------------------------------------
\phantompart
\printindex
%-----------------------------------------------------------------------
\end{document}
% END ==================================================================

% ----------------------------------------------------------------------
% pacote para fazer o checkmark
\usepackage{pifont} % http://ctan.org/pkg/pifont
\newcommand{\cmark}{\ding{51}}%
\newcommand{\xmark}{\ding{55}}%
% ----------------------------------------------------------------------
\usepackage{amsmath}
\usepackage{amsfonts}
\usepackage{amssymb}
\usepackage{pdfpages}
% \usepackage{times}
% \usepackage{helvet}
% \renewcommand{\familydefault}{\sfdefault}
% ----------------------------------------------------------------------
\NewDocumentCommand\cc{+u{\cc}}{\ignorespaces}
% ----------------------------------------------------------------------
% controle do espaçamento entre um parágrafo e outro:
\setlength{\parskip}{0.2cm} % tente também \onelineskip
% ----------------------------------------------------------------------
\titulo{MODELING THE CUMULATIVE INCIDENCE FUNCTION OF CLUSTERED
  COMPETING RISKS DATA: A MULTINOMIAL GLMM APPROACH}
\autor{HENRIQUE APARECIDO LAUREANO}
\data{2021}
\instituicao{FEDERAL UNIVERSITY OF PARANÁ}
\orientador{Prof. PhD Wagner Hugo Bonat}
\coorientador{Prof. PhD Paulo Justiniano Ribeiro Jr}
\tipotrabalho{Dissertação (mestrado)}
\preambulo{\small{Thesis presented to the Graduate Program of Numerical
    Methods in Engineering, Concentration Area in Mathematical
    Programming: Statistical Methods Applied in Engineering, Federal
    University of Paran\'{a}, as part of the requirements to the
    obtention of the Master's Degree in Sciences.}}
% ----------------------------------------------------------------------
% informações do PDF
\makeatletter
\hypersetup{
  % pagebackref=true,
  pdftitle={\@title},
  pdfauthor={\@author},
  pdfsubject={\imprimirpreambulo},
  % pdfkeywords = {}{}{}{},
  colorlinks=true, % false: boxed links; true: colored links
  linkcolor=blue, % color of internal links
  citecolor=blue, % color of links to bibliography
  filecolor=magenta, % color of file links
  urlcolor=blue,
  bookmarksdepth=4
}
\addto\captionsenglish{
  % adjusts names from abnTeX2
  \renewcommand{\folhaderostoname}{Title page}
  \renewcommand{\epigraphname}{Epigraph}
  \renewcommand{\dedicatorianame}{Dedication}
  \renewcommand{\errataname}{Errata sheet}
  \renewcommand{\agradecimentosname}{Acknowledgements}
  \renewcommand{\anexoname}{ANNEX}
  \renewcommand{\anexosname}{Annex}
  \renewcommand{\apendicename}{APPENDIX}
  \renewcommand{\apendicesname}{Appendix}
  \renewcommand{\orientadorname}{Supervisor:}
  \renewcommand{\coorientadorname}{Co-supervisor:}
  \renewcommand{\folhadeaprovacaoname}{Approval}
  \renewcommand{\resumoname}{Abstract}
  \renewcommand{\listadesiglasname}{List of abbreviations and acronyms}
  \renewcommand{\listadesimbolosname}{List of symbols}
  \renewcommand{\fontename}{Source}
  \renewcommand{\notaname}{Note}
  % adjusts names used by \autoref
  \renewcommand{\pageautorefname}{page}
  \renewcommand{\chapterautorefname}{Chapter}
  \renewcommand{\sectionautorefname}{Section}
  \renewcommand{\subsectionautorefname}{subsection}
  \renewcommand{\subsubsectionautorefname}{subsubsection}
  \renewcommand{\paragraphautorefname}{subsubsubsection}
}
\makeatother
% ----------------------------------------------------------------------
\graphicspath{{figures/}}
% ----------------------------------------------------------------------
\begin{document}
\selectlanguage{english}
% adequando o uppercase titulo dos elementos nas suas respectivas
% legendas
\renewcommand{\tablename}{TABLE }
\renewcommand{\figurename}{FIGURE }
% ----------------------------------------------------------------------
\frenchspacing % retira espaço extra obsoleto entre as frases
% ----------------------------------------------------------------------
% capa
\tikz[remember picture,overlay] \node[opacity=1,inner sep=0pt] at
(current page.center){
  \includegraphics[width=\paperwidth,
  height=\paperheight]{Figuras/ufpr_bg}};
% ----------------------------------------------------------------------
\imprimircapa
% ----------------------------------------------------------------------
% folha de rosto
\imprimirfolhaderosto
% ----------------------------------------------------------------------
% \begin{dedicatoria}
%   \vspace*{\fill}
%   ...
%   \vspace*{\fill}
% \end{dedicatoria}
% ----------------------------------------------------------------------
% ficha catalográfica

% \begin{fichacatalografica}
%   \includepdf{ficha.pdf}
% \end{fichacatalografica}
% ----------------------------------------------------------------------
% inserir folha de aprovação
% \begin{folhadeaprovacao}
%   \includepdf{termo.pdf}
% \end{folhadeaprovacao}
\begin{folhadeaprovacao}
 \begin{center}
   {\ABNTEXchapterfont\large\imprimirautor}

   \vspace*{\fill}\vspace*{\fill}
   \begin{center}
     \ABNTEXchapterfont\bfseries\large\imprimirtitulo
   \end{center}
   \vspace*{\fill}

    \hspace{.45\textwidth}
    \begin{minipage}{.5\textwidth}
       \imprimirpreambulo
    \end{minipage}
   \vspace*{\fill}
 \end{center}

 Master thesis approved. XXX XX, 2021.

  \assinatura{\textbf{\imprimirorientador}\\ Supervisor}
  \assinatura{\textbf{Prof. PhD Paulo Justiniano Ribeiro Jr}\\
    Co-supervisor}
  \assinatura{\textbf{Prof. PhD \(\dots\)}\\
    Internal Examinator - PPGMNE}
  \assinatura{\textbf{Prof. PhD \(\dots\)}\\
    Internal Examinator - PPGMNE}
  \assinatura{\textbf{Prof. PhD \(\dots\)}\\External Examiner - }

  \begin{center}
   \vspace*{0.5cm}
   {\large CURITIBA}
   \par
   {\large\imprimirdata}
   \vspace*{1cm}
 \end{center}

\end{folhadeaprovacao}
% ----------------------------------------------------------------------
\begin{dedicatoria}
  \vspace*{22.7cm}
  \begin{flushright}
    \begin{minipage}[H]{4.5cm}
      {To Celita and Olivio}
    \end{minipage}
  \end{flushright}
\end{dedicatoria}
% ----------------------------------------------------------------------
\begin{agradecimentos}
  As Moro said once, I'm thankful for everything and everyone.
\end{agradecimentos}

\begin{epigrafe}
  \vspace*{\fill}
  \begin{flushright}
    \textit{"It's not supposed to be easy."\\
             (Gregg Popovich)}
              % on Sao Antonio Spurs \(\times\) Oklahoma City Thunder,
              % first game of the 2012 Western Conference Finais
  \end{flushright}
\end{epigrafe}
% ----------------------------------------------------------------------
\newpage
\setlength{\absparsep}{18pt} % ajusta o espaçamento dos parágrafos do
                             % resumo
\setlength{\abstitleskip}{1cm} % adiciona mais um cm após o 'titulo' do
                               % resumo para ficar com 2cm
\begin{resumo}[]
  \vspace{-2cm}
  \begin{center}
    \bfseries{\large{\textsf{ABSTRACT}}}
  \end{center}
  \vspace{0.3cm}
  Failure time data \(\dots\)\\

  \textbf{Keywords}: Competing risks.
\end{resumo}
% ----------------------------------------------------------------------
\newpage
\setlength{\absparsep}{18pt} % ajusta o espaçamento dos parágrafos do
                             % resumo
\setlength{\abstitleskip}{1cm} % adiciona mais um cm após o 'titulo' do
                               % resumo para ficar com 2cm
\begin{resumo}[]
  \begin{otherlanguage*}{brazil}
    \vspace{-2cm}
    \begin{center}
      \bfseries{\large{\textsf{RESUMO}}}
    \end{center}
    \vspace{0.3cm}
    Dados de tempos de falha \(\dots\)\\

    \textbf{Palavras-chave}: Riscos competitivos.
  \end{otherlanguage*}
\end{resumo}
% ----------------------------------------------------------------------
\pdfbookmark[0]{\listfigurename}{lof}
\listoffigures*
\cleardoublepage
% ----------------------------------------------------------------------
\pdfbookmark[0]{\listtablename}{lot}
\listoftables*
\cleardoublepage
% ----------------------------------------------------------------------
\makeatletter
\renewcommand\numberline[1]{
	\leftskip -0.7em
	\rightskip 1.6em
	\parfillskip -\rightskip
	\parindent 0em
	\@tempdima 2.0em
	\vspace{0em}
  \advance\leftskip \@tempdima \null\nobreak\hskip -\leftskip
	ALGORITHM \normalfont #1 ~~ }
\makeatother
% ----------------------------------------------------------------------
\pdfbookmark[0]{\listalgorithmname}{loa}
\listofalgorithms
\cleardoublepage
% ----------------------------------------------------------------------
\makeatletter
\def\numberline#1{\hb@xt@\@tempdima{#1\hfil}}
\makeatother
% ----------------------------------------------------------------------
% \begin{siglas}
% \item[Fig.] Area of the $i^{th}$ component
% \item[456] Isto é um número
% \item[123] Isto é outro número
% \item[lauro cesar] este é o meu nome
% \end{siglas}
% ----------------------------------------------------------------------
% \begin{simbolos}
% \item[\(\mathbb{E}(\cdot)\)] The mathematical expectation of a random
%   variable \(\cdot\)
% \end{simbolos}
% ----------------------------------------------------------------------
\pdfbookmark[0]{\contentsname}{toc}
\tableofcontents*
\cleardoublepage
% ----------------------------------------------------------------------
\makepagestyle{abntheadings}
\makeevenhead{abntheadings}{\ABNTEXfontereduzida\thepage}{}{}
\makeoddhead{abntheadings}{}{}{\ABNTEXfontereduzida\thepage}
\makeheadrule{abntheadings}{\textwidth}{0in}
% ----------------------------------------------------------------------
\textual
% ----------------------------------------------------------------------
\chapter{Introduction}
\label{cap:intro}
Consider a cluster of random variables. Each random variable represents
the time until some event occurs. The random variables that compose the
cluster are assumed to be correlated, i.e., the method for the analysis
is flexible enough to be able to verify if this happens to that data. In
this thesis, the cluster is a family - more precisely, a part of a
family, i.e., a pair of twins; the random variables are the time until
the occurrence (or not) of an event in each twin; and the event under
focus is the occurrence of cancer.

When we deal with random variables, in the context of a statistical
model - a response of interest; that represents the time until some
event occurs, such events are generically referred to as
\textit{failures}. From this, we get the name of the field of study:
failure time data~\cite{kalb&prentice}. These events may not necessarily
consist of a failure, however, the major areas of application of the
methods that will be discussed here are biomedical studies and
industrial life testing. Thus, this name sounds appropriate.

Independent of the area of application the methods are the same, but the
name of what you're doing is different. In industrial life testing
applications, you perform what is called a reliability analysis; in
biomedical studies, you perform what is called a survival analysis. In
this thesis, we'll maintain our focus on the latter.

Generally speaking, the term survival analysis is applied when we deal
with a univariate set, i.e., we have one response variable. As an
example,

When we want to study the possible relation between the components (seen
as random variables) of a dataset in an associative manner, the perhaps
most common scientific way of doing that is by fitting a linear model.
More generally, a generalized linear model (GLM). To fit a GLM to a
dataset we basically need to select a probability distribution for the
so-called response variable \(Y_{i}\) (or dependent variable); and how
we'll approach the possible relation between the response and the other
variables \(\mathbf{X}_{i}\) (independent variables or covariates).

Generalized linear models~\cite{GLM72} allow for response distribution
other than normal, which configures the so-called linear models. In a
GLM we model the mean, \(\mu_{i}\), of the response random variable, and
it has the basic structure
\[
 g(\mu_{i}) = \mathbf{X}_{i} \bm{\beta},
\]
where \(\mu_{i} \equiv \mathbb{E}(Y_{i})\), \(g\) is a smooth monotonic
``link function'', \(\mathbf{X}_{i}\) is the \(i^\text{th}\) row of a
model matrix \(\mathbf{X}\), and \(\bm{\beta}\) is a vector of unknown
parameters. In addition, a GLM usually makes the distributional
assumption that the \(Y_{i}\) are independent and
\[
  Y_{i} \sim \text{some exponential family distribution}.
\]

The \textit{exponential family} of distributions includes many
distributions that are useful for practical modelling, such as the
Poisson (for counting data), binomial (dichotomic data), gamma
(continuous but positive) and normal (continuous data) distributions.
The comprehensive reference for GLMs is~\citeonline{GLM89}.

However broad it may be the range of models that can be constructed
thanks to the generality of the GLM framework, is plausible that for
some applications a specific modeling framework come up. This is the
case when the response variable is the time until some event occurs.
Such events are generically referred to as \textit{failures}, and its
major areas of use are biomedical studies and industrial life testing.
In the latter the commonly used name is reliability and in the former is
survival. In this thesis, we focus on the survival side.

A survival model consists

\section{GOALS}

\subsection{General goals}

Propor um modelo de regressão para análise de variáveis respostas
limitadas multivariada.

\subsection{Specific goals}

\begin{enumerate}
\item Estudar o desempenho do algoritmo NORTA (\emph{NORmal To
    Anything}) para simular variáveis aleatórias beta correlacionadas.

\item Especificar o modelo usando suposições de primeiro e segundo
  momentos.

\item Usar as funções de estimação quase-score e Pearson para estimar os
  parâmetros de regressão e dispersão, respectivamente.

\item Delinear estudos de simulação para explorar a flexibilidade do
  modelo para lidar com dados limitados em estudos longitudinais, além
  de checar propriedades dos estimadores em estudos com múltiplas
  respostas correlacionadas.

\item Adaptar técnicas de diagnóstico para o modelo proposto, como
  DFFITS, DFBETAS, distância de Cook e o gráfico de probabilidade
  meio-normal com envelope simulado.

\item Aplicar o modelo proposto em dois conjuntos de dados.
\end{enumerate}

\section{JUSTIFICATION}

\section{LIMITATION}

Este trabalho se restringe a propor um novo modelo de regressão para
análise de variáveis respostas limitadas multivariada. Para motivar o
novo modelo, serão apresentadas aplicações em dois conjuntos de dados,
que não são facilmente manipulados pelos métodos estatísticos
existentes. Portanto, testes de hipóteses e de comparações múltiplas
multivariados não serão desenvolvidos no decorrer deste trabalho.

\section{THESIS ORGANIZATION}

Esta dissertação contém seis capítulos incluindo esta introdução.
O~\autoref{cap:aplicacoes} descreve os dois conjuntos de dados que serão
usados como exemplos de aplicação no novo modelo.
O~\autoref{cap:fundamentacaoteorica} apresenta a revisão bibliográfica
que motivou este trabalho, introduz o modelo de regressão beta
(univariado), apresenta o algoritmo NORTA (\textit{NORmal To Anything})
usado nos estudos de simulação e discute brevemente as medidas de
bondade de ajuste usadas no trabalho. O~\autoref{cap:multivariatemodel}
propõe o modelo de regressão quase-beta multivariado, apresenta o método
usado para estimação e inferência e adapta técnicas de diagnóstico.
No~\autoref{cap:resultados} são apresentados os resultados de três
estudos de simulação, além da análise dos dados apresentados
no~\autoref{cap:aplicacoes}. Finalmente, o~\autoref{cap:considefinais}
discute as principais contribuições desta dissertação, além de
apresentar as conclusões seguidas por sugestões para futuros trabalhos.

% END ==================================================================
% ----------------------------------------------------------------------
\chapter{Generalized linear mixed models: formulation, optimization, and
  implementation}
\label{cap:methods}
This chapter presents a systematic review of the main theoretical
aspects involved in the construction, estimation and implementation of a
generalized linear mixed model (GLMM). We start in \autoref{cap:joint}
with the model construction framework, concluding with the so-called
joint likelihood function. \autoref{cap:laplace} address the integration
of that joint likelihood, a necessary and fundamental step in our
modeling approach, resulting in a marginal likelihood function.
\autoref{cap:opt} discusses available alternatives for the optimization
of the marginal distribution obtained through that integration.
\autoref{cap:ad} talks about automatic differentiation, the most
efficent manner of computing derivatives, and a key point for us. Last
but not least, in \autoref{cap:tmb} we present the computational tool
used to peform all the discussed procedure, the TMB: Template Model
Builder. A very exciting \texttt{R} \cite{R18} package developed
by~\citeonline{TMB}.

\section{JOINT LIKELIHOOD}
\label{cap:joint}

A standard, univariate, GLMM models an \(n\)-vector of exponential
family random variables, \(\mathbf{Y}\), with conditional expected
value, \(\bm{\mu} \equiv \mathbb{E}(\mathbf{Y} \mid \mathbf{X},
\mathbf{u})\), via a linear predictor equation expressed by
\begin{equation}
  g(\bm{\mu}) = \mathbf{X} \bm{\beta} + \mathbf{Zu}, \quad
  \mathbf{u} \sim \mathcal{N}(\mathbf{0}, \bm{\Sigma}).
  \label{eq:gmu}
\end{equation}

That is, a GLMM is a generalized linear model (GLM) in which the linear
predictor depends on some Gaussian latent effects, \(\mathbf{u}\), times
a latent effects model matrix \(\mathbf{Z}\). The idea embedded in that
matrix is exemplified in~\autoref{eq:Zu}. Suppose, e.g., three
individuals and each has two measures. This configures a simple repeated
measures context, the focus of this work. It is reasonable to admit that
each individual has a particular latent effect value. Consequently,
\begin{equation}
  \mathbf{Zu} = \begin{bmatrix}
                 1 & 0 & 0\\
                 1 & 0 & 0\\
                 0 & 1 & 0\\
                 0 & 1 & 0\\
                 0 & 0 & 1\\
                 0 & 0 & 1\\
                \end{bmatrix} \begin{bmatrix}
                               u_{1}\\
                               u_{2}\\
                               u_{3}\\
                              \end{bmatrix} = \begin{bmatrix}
                                               u_{1}\\
                                               u_{1}\\
                                               u_{2}\\
                                               u_{2}\\
                                               u_{3}\\
                                               u_{3}\\
                                              \end{bmatrix},
  \label{eq:Zu}
\end{equation}
where \(\mathbf{u}^{\top} = [u_{1}~u_{2}~u_{3}]\) and \(\mathbf{Z}\) has
the role of projecting the values of \(\mathbf{u}\) to match the number
of measures.

We model this mean structure into a combination of probability
distributions. It's a combination since we have to assume probabilistic
structures for the observed and non-observed, latent, data. For each
observed variable \(y_{ij}\), we have a probability distribution of the
exponential family, denoted by \(f(y_{ij} \mid \mathbf{u}_{i},
\bm{\theta})\). For the non-observed latent effect we have, generally, a
(multivariate) Gaussian distribution, denoted by \(f(\mathbf{u}_{i} \mid
\bm{\Sigma})\). For each individual or unity under study, \(i\), and to
each measure, \(j\), we have the product of these probability densities,
a likelihood contribution.

We want to estimate the parameter vector \(\bm{\theta} =
[\bm{\beta}~\bm{\Sigma}]^{\top}\) of~\autoref{eq:gmu}. Besides the role
of emphasizing the fact that \(\bm{\mu}\) is a function of
\(\bm{\theta}\) and that we want to estimate \(\bm{\theta}\), the
likelihood function ties the probability densities. i.e., the likelihood
is the product of the probability densities product for each subject.
Since the \(Y_{i}\) are mutually independent, the likelihood of
\(\bm{\theta}\) is
\begin{equation}
  L(\bm{\theta} \mid \mathbf{y}, \mathbf{u}) =
  \prod_{i=1}^{n}~\prod_{j=1}^{n_{i}}~
  f(y_{ij} \mid \mathbf{u}_{i}, \bm{\beta}, \bm{\Sigma})~
  f(\mathbf{u}_{i} \mid \bm{\Sigma}).
  \label{eq:joint}
\end{equation}

From standard probability theory is easy to see that in the right-hand
side (r.h.s.) we have a joint density, consequently,~\autoref{eq:joint}
represents what is called a joint likelihood function. What makes
working with this joint likelihood problematic is that we don't have all
the information necessary to just optimize it and get the desired
parameter estimates. The latent effect \(\mathbf{u}\) is
\textit{latent}, we don't observe it. To handle with this we basically
have two available paths.

\section{LAPLACE APPROXIMATION}
\label{cap:laplace}

To deal with the joint likelihood in~\autoref{eq:joint}~we have a choice
to make. Be or not to be Bayesian. Each choice has its own difficulties,
advantages, and characteristics.

The Bayesian path assumes that all \(\bm{\theta}\) components are random
variables. With all parameters being treated as random variables, and
since we don't observe them, what the Bayesian framework does is try to
compute the mode of each ``parameter'' marginal distribution via a
sampling algorithm, called MCMC: Markov Chain Monte Carlo~\cite{MCMC,
  Diaconis}. The advantage is that we can reach an MCMC algorithm to
basically any statistical model, the disadvantages are that this
approach is very time consuming and we have to propose prior
distributions to each ``parameter''. These prior proposals are not
always easy to make, and the resulting marginal distributions can be
very depending on it.

A Bayesian approach can be applied in basically any context. However, in
complex scenarios they can be the only available method to maximize the
likelihood. This isn't the case here. We have a joint density where one
of the random variables isn't observed, but we're not interested in it,
only in the variance parameters inherent in it. Again, from standard
probability theory, if we have a joint density we can just integrate out
the undesired variable. This results in
\begin{equation}
  \begin{aligned}
    L(\bm{\theta} \mid \mathbf{y}) &=
    \prod_{i=1}^{n}~\int_{\mathcal{R}^{\mathbf{u}_{i}}}
    \left\{
      \prod_{j=1}^{n_{i}}~
      f(y_{ij} \mid \mathbf{u}_{i}, \bm{\beta}, \bm{\Sigma})~
      f(\mathbf{u}_{i} \mid \bm{\Sigma})
    \right\} \text{d} \mathbf{u}_{i}\\
    &= \prod_{i=1}^{n}~\int_{\mathcal{R}^{\mathbf{u}_{i}}}~
    f(\mathbf{y}_{i}, \mathbf{u}_{i} \mid \bm{\theta})~
    \text{d} \mathbf{u}_{i},
    \label{eq:generalmarginal}
  \end{aligned}
\end{equation}
a marginal density that keeps the parameters of the integrated variable.

If the response distribution in our mixed model is Gaussian, is
analytically tractable to integrate \(\mathbf{u}\) out of the joint
density. Consequently, is possible to evaluate the likelihood exactly.
This is the case of linear mixed models and the main difference to the
GLMMs. When the response distribution isn't Gaussian, generally, isn't
anymore analytically tractable to integrate out the latent effect. So
what do we do? Well, basically we have two options.

We can avoid the integrals in~\autoref{eq:generalmarginal}, replacing it
by integrals that are sometimes more analytically tractable. This can be
performed via an algorithm called EM:
Expectation-Maximization~\cite{EM77}. This method is considered a little
bit naive and generally isn't recommended if you have a better option.
Our better option consists in take advantage of the exponential family
structure and the fact that we're dealing with Gaussian latent effects.
These ideas converge to what is called, \textit{Laplace
  approximation}~\cite{molenberghs&verbeke, LA4H, tierney, corestats}.

If the integral is analytically intractable, we can approximate it to
obtain a tractable closed-form expression, allowing the numerical
maximization of the marginal likelihood~\cite{patrao}. The Laplace
approximation has been designed to approximate integrals in the form
\begin{equation}
  \int_{\mathcal{R}^{\mathbf{u}_{i}}}
  \exp\{Q(\mathbf{u}_{i})\} \text{d} \mathbf{u}_{i}
  \approx (2\pi)^{n_{\mathbf{u}}/2}~
  |{Q}''(\mathbf{\hat{u}}_{i})|^{-1/2}~\exp\{Q(\mathbf{\hat{u}}_{i})\},
  \label{eq:laplace}
\end{equation}
where \(Q(\mathbf{u}_{i})\) is a known, unimodal bounded function and
\(\mathbf{\hat{u}}_{i}\) is the value for which \(Q(\mathbf{u}_{i})\) is
maximized. As~\citeonline{corestats}~shows, a Laplace approximation
consists of a second order Taylor expansion of \(\log f(\mathbf{y}_{i},
\mathbf{u}_{i} \mid \bm{\theta})\), about \(\mathbf{\hat{u}}_{i}\), that
gives
\[
  \log f(\mathbf{y}_{i}, \mathbf{u}_{i} \mid \bm{\theta}) \approx
  \log f(\mathbf{y}_{i}, \mathbf{\hat{u}}_{i} \mid \bm{\theta}) -
  \frac{1}{2}
  (\mathbf{u}_{i} - \mathbf{\hat{u}}_{i})^{\top}\mathbf{H}~
  (\mathbf{u}_{i} - \mathbf{\hat{u}}_{i}),
\]
where \(\mathbf{H} = - \nabla_{u}^{2} \log f(\mathbf{y}_{i},
\mathbf{\hat{u}}_{i} \mid \bm{\theta})\). Hence, we can approximate the
joint by
\begin{equation}
  f(\mathbf{y}_{i}, \mathbf{u}_{i} \mid \bm{\theta}) \approx
  f(\mathbf{y}_{i}, \mathbf{\hat{u}}_{i} \mid \bm{\theta})~\exp
  \left\{- \frac{1}{2}
    (\mathbf{u}_{i} - \mathbf{\hat{u}}_{i})^{\top}\mathbf{H}~
    (\mathbf{u}_{i} - \mathbf{\hat{u}}_{i})
  \right\}.
  \label{eq:taylor}
\end{equation}

From here we start to take advantage of the points mentioned above.
First, the fact that we're working with Gaussian distributed latent
effects. In~\autoref{eq:taylor}~we see the core of a Gaussian density,
that complete is
\[
  \int_{\mathcal{R}^{\mathbf{u}_{i}}}
  \frac{1}{(2 \pi)^{n_{\mathbf{u}}/2}~|\mathbf{H}^{-1}|^{1/2}}~\exp
  \left\{- \frac{1}{2}
    (\mathbf{u}_{i} - \mathbf{\hat{u}}_{i})^{\top}\mathbf{H}~
    (\mathbf{u}_{i} - \mathbf{\hat{u}}_{i})
  \right\} \text{d} \mathbf{u}_{i} = 1,
\]
and integrates to 1. Integrating~\autoref{eq:taylor}, it follows that
\begin{align*}
  \int_{\mathcal{R}^{\mathbf{u}_{i}}}
  f(\mathbf{y}_{i}, \mathbf{u}_{i} \mid \bm{\theta})
  \text{d} \mathbf{u}_{i}
  &\approx f(\mathbf{y}_{i}, \mathbf{\hat{u}}_{i} \mid \bm{\theta})
    \int_{\mathcal{R}^{\mathbf{u}_{i}}} \exp
    \left\{- \frac{1}{2}
    (\mathbf{u}_{i} - \mathbf{\hat{u}}_{i})^{\top}\mathbf{H}~
    (\mathbf{u}_{i} - \mathbf{\hat{u}}_{i})
    \right\} \text{d} \mathbf{u}_{i}\\
  &= (2 \pi)^{n_{\mathbf{u}}/2}~|\mathbf{H}|^{-1/2}~
    f(\mathbf{y}_{i}, \mathbf{\hat{u}}_{i} \mid \bm{\theta}),
\end{align*}
i.e., we get~\autoref{eq:laplace}, a first order Laplace approximation
to the integral. Careful accounting of the approximation error shows it
to generally be \(\mathcal{O}(n^{-1})\) where \(n\) is the sample size,
assuming a fixed length for \(\mathbf{u}_{i}\)~\cite{corestats}.

The second advantage of a Laplace approximation approach in a GLMM is
the exponential family structure. In a usual GLMM, the response follows
a one-parameter exponential family distribution, that can be written as
\[
  f(\mathbf{y}_{i} \mid \mathbf{u}_{i}, \bm{\theta}) = \exp
  \left\{\mathbf{y}_{i}^{\top}
    (\mathbf{X}_{i}\bm{\beta} + \mathbf{Z}_{i}\mathbf{u}_{i}) -
    \mathbf{1}_{i}^{\top}
    b(\mathbf{X}_{i}\bm{\beta} + \mathbf{Z}_{i}\mathbf{u}_{i}) +
    \mathbf{1}_{i}^{\top} c(\mathbf{y}_{i})
  \right\},
\]
where \(b(\cdot)\) and \(c(\cdot)\) are known functions. This general
and easy to compute expression, together with a (multivariate) Gaussian
distribution, highlights the convenience of the Laplace method. The
\(Q(\mathbf{u}_{i})\) function to be maximized can then be expressed as
\begin{equation}
  \begin{aligned}
    Q(\mathbf{u}_{i}) &= \mathbf{y}_{i}^{\top}
    (\mathbf{X}_{i}\bm{\beta} + \mathbf{Z}_{i}\mathbf{u}_{i}) -
    \mathbf{1}_{i}^{\top}
    b(\mathbf{X}_{i}\bm{\beta} + \mathbf{Z}_{i}\mathbf{u}_{i}) +
    \mathbf{1}_{i}^{\top} c(\mathbf{y}_{i})\\
    &- \frac{n_{\mathbf{u}}}{2} \log (2 \pi) -
    \frac{1}{2} \log |\bm{\Sigma}| -
    \frac{1}{2} \mathbf{u}_{i}^{\top} \bm{\Sigma}^{-1}~\mathbf{u}_{i}.
  \end{aligned}
\end{equation}

The approximation in~\autoref{eq:laplace} requires the maximum
\(\mathbf{\hat{u}}_{i}\) of the function \(Q(\mathbf{u}_{i})\). Since we
assume a Gaussian distribution with a known mean for the latent effects,
we have the perfect initial guess for a gradient-based method as the
Newton-Raphson (NR) algorithm. The NR method consists of an iterative
scheme as follows:
\[
  \mathbf{u}_{i}^{(k+1)} = \mathbf{u}_{i}^{(k)} -
  {Q}''(\mathbf{u}_{i}^{(k)})^{-1}~{Q}'(\mathbf{u}_{i}^{(k)}),
\]
until convergence, which gives \(\mathbf{\hat{u}}_{i}\). At this stage,
all parameters are considered known.~\citeonline{patrao}~presents the
generic expressions for the derivatives required by the NR method, given
by the following:
\begin{equation}
  \begin{aligned}
    {Q}'(\mathbf{u}_{i}^{(k)}) &= \{\mathbf{y}_{i} -
    {b}'(\mathbf{X}_{i}\bm{\beta} +
    \mathbf{Z}_{i}\mathbf{u}_{i}^{(k)})\}^{\top} -
    {\mathbf{u}_{i}^{(k)}}^{\top} \bm{\Sigma}^{-1},\\
    {Q}''(\mathbf{u}_{i}^{(k)}) &=
    - \text{diag}\{{b}''(\mathbf{X}_{i}\bm{\beta} +
    \mathbf{Z}_{i}\mathbf{u}_{i}^{(k)})\} - \bm{\Sigma}^{-1}.
  \end{aligned}
  \nonumber
\end{equation}

Finally, the marginal log-likelihood returned by the Laplace
approximation for each invividual or unit under study, is as follows:
\begin{equation}
  \begin{aligned}
    l(\bm{\theta} \mid \mathbf{y}_{i}) =
    \log L(\bm{\theta} \mid \mathbf{y}_{i}) &=
    \frac{n}{2} \log (2 \pi) - \frac{1}{2} \log
    \left|
      \text{diag}\{{b}''(\mathbf{X}_{i}\bm{\beta} +
      \mathbf{Z}_{i}\mathbf{\hat{u}}_{i})\} + \bm{\Sigma}^{-1}
    \right|\\
    &+ \mathbf{y}_{i}^{\top}
    (\mathbf{X}_{i} \bm{\beta} + \mathbf{Z}_{i} \mathbf{\hat{u}}_{i}) -
    \mathbf{1}_{i}^{\top}
    b(\mathbf{X}_{i}\bm{\beta} + \mathbf{Z}_{i} \mathbf{\hat{u}}_{i}) +
    \mathbf{1}_{i}^{\top} c(\mathbf{y}_{i})\\
    &- \frac{n_{\mathbf{u}}}{2} \log (2 \pi) -
    \frac{1}{2} \log |\bm{\Sigma}| - \frac{1}{2}
    \mathbf{\hat{u}}_{i}^{\top}\bm{\Sigma}^{-1}~\mathbf{\hat{u}}_{i},
  \end{aligned}
  \nonumber
\end{equation}
that can now be numerically maximized over the model parameters.

\section{MARGINAL LIKELIHOOD OPTIMIZATION}
\label{cap:opt}

At this point is clear that we have two optimizations to be made. An
``inside'' and an ``outside'' optimization. The inside optimization is
performed into the Laplace approximation layer via a Newton-Raphson
algorithm, a Newton's method. The outside optimization is made with the
Laplace approximation outputs, i.e., the maximization step
of~\autoref{eq:generalmarginal}'s marginal log-likelihood over its
parameters. This task is usually performed via a quasi-Newton method. We
focus here on two of the most traditional ones, the
Broyden-Fletcher-Goldfarb-Shanno (BFGS) algorithm and the PORT routines.

The inside optimization is the joint log-likelihood numerical
maximization w.r.t. its latent effects. This is kind of a simple task
since all model parameters are considered as fixed and we ``know'' that
the latent effects are distributed with zero mean, i.e., we have the
perfect initial guess. In this context, the use of a Newton's method is
straightforward. When we talk about the outside optimization it is a
completely different scenario. It is not straightforward to find a good
initial guess or reach convergence, so more robust methods are a good
choice.

In optimization, Newton methods are algorithms for finding local maxima
and minima of functions, i.e., the search for the zeroes of the gradient
of that function. Newton methods are characterized by the use of a
symmetric matrix of function's second derivatives, the Hessian matrix.
Quasi-Newton methods are based on Newton's method and are seen as an
alternative to it. They can be used if the Hessian is unavailable or is
too expensive to compute at every iteration.

As shown in~\citeonline{nocedal&wright}, chief advantages of
quasi-Newton methods over Newton's method are that the Hessian matrix
doesn't need to be computed, its approximated; and it also doesn't need
to be inverted. Newton's method requires the Hessian to be inverted,
typically by solving a system of linear equations - often quite costly.
In contrast, quasi-Newton methods usually generate an estimate of it
directly. As in Newton's method, they use a second-order approximation
to find the minimum of a function \(f(\mathbf{x})\). The Taylor series
of \(f(\mathbf{x})\) around an iterate is
\[
  f(\mathbf{x}_{k} + \Delta\mathbf{x}) \approx
  f(\mathbf{x}_{k}) + \nabla f(\mathbf{x}_{k})^{\top} \Delta\mathbf{x} +
  \frac{1}{2} \Delta\mathbf{x}^{\top} \mathbf{B}~\Delta\mathbf{x},
\]
where \(\nabla f(\cdot)\) is the gradient, and \(\mathbf{B}\) an
approximation to the Hessian matrix. The gradient of this approximation
w.r.t. \(\Delta\mathbf{x}\) is
\[
  \nabla f(\mathbf{x}_{k} + \Delta\mathbf{x}) \approx
  \nabla f(\mathbf{x}_{k}) + \mathbf{B}~\Delta\mathbf{x},
\]
setting this gradient to zero provides the Newton step:
\[
  \Delta\mathbf{x} = - \mathbf{B}^{-1} \nabla f(\mathbf{x}_{k}).
\]

The Hessian approximation \(\mathbf{B}\) is chosen to satisfy
\[
  \nabla f(\mathbf{x}_{k} + \Delta\mathbf{x}) =
  \nabla f(\mathbf{x}_{k}) + \mathbf{B}~\Delta\mathbf{x},
\]
which is called the \textit{secant} equation, the Taylor series of the
gradient itself. Solving for \(\mathbf{B}\) and applying the Newton's
step with the updated value is equivalent to the \textit{secant} method.
Quasi-Newton methods are a generalization of the secant method to find
the root of the first derivative for multidimensional problems. The
various quasi-Newton methods differ in their choice of the solution to
the secant equation.

In a general quasi-Newton method, the unknown \(\mathbf{x}_{k}\) is
updated applying the Newton's step calculated using the current
approximate Hessian matrix \(\mathbf{B}_{k}\) in the following fashion:
\begin{itemize}
\item \(\Delta \mathbf{x}_{k} = -\alpha_{k}\mathbf{B}_{k}^{-1}\nabla
  f(\mathbf{x}_{k})\), with \(\alpha\) chosen to satisfy the so called
  Wolfe conditions~\cite[p.~34]{nocedal&wright};

\item \(\mathbf{x}_{k+1} = \mathbf{x}_{k} + \Delta\mathbf{x}_{k}\);

\item The gradient computed at the new point \(\nabla
  f(\mathbf{x}_{k+1})\), and \(\mathbf{y}_{k} = \nabla
  f(\mathbf{x}_{k+1}) - \nabla f(\mathbf{x}_{k})\) is used to update the
  approximate Hessian \(\mathbf{B}_{k+1}\), or directly its inverse
  \(\mathbf{H}_{k+1} = \mathbf{B}_{k+1}^{-1}\).
\end{itemize}

The most popular quasi-Newton method is the BFGS algorithm, named for
its discoverers Broyden, Fletcher, Goldfarb, and Shanno. It has the
following update formula
\begin{align*}
  \mathbf{B}_{k+1} &= \mathbf{B}_{k} +
                     \frac{\mathbf{y}_{k}\mathbf{y}_{k}^{\top}}{
                     \mathbf{y}_{k}^{\top}\Delta\mathbf{x}_{k}} -
                     \frac{\mathbf{B}_{k}\Delta\mathbf{x}_{k}
                     (\mathbf{B}_{k}\Delta\mathbf{x}_{k})^{\top}}{
                     \Delta\mathbf{x}_{k}^{\top}\mathbf{B}_{k}
                     \Delta\mathbf{x}_{k}},\\
  \mathbf{H}_{k+1} = \mathbf{B}_{k+1}^{-1}
                   &= \left(
                     \mathbf{I} -
                     \frac{\Delta\mathbf{x}_{k}\mathbf{y}_{k}^{\top}}{
                     \mathbf{y}_{k}^{\top}\Delta\mathbf{x}_{k}}
                     \right) \mathbf{H}_{k}
                     \left(
                     \mathbf{I} -
                     \frac{\mathbf{y}_{k}\Delta\mathbf{x}_{k}^{\top}}{
                     \mathbf{y}_{k}^{\top}\Delta\mathbf{x}_{k}}
                     \right) +
                     \frac{\Delta\mathbf{x}_{k}
                     \Delta\mathbf{x}_{k}^{\top}}{
                     \mathbf{y}_{k}^{\top}\Delta\mathbf{x}_{k}}.
\end{align*}

Another quasi-Newton method, popular in statistical data analysis, is
the one based on PORT routines~\url{http://www.netlib.org/port/}. A
Fortran mathematical subroutine library designed to be \textit{portable}
over different types of computers, and developed by David Gay in the
Bell Labs~\cite{PORTreport}. It is a quasi-Newton adaptive nonlinear
least-squares algorithm~\cite{PORTpaper} with the following update
formula
\begin{align*}
  \mathbf{B}_{k+1} &= \mathbf{B}_{k}\\
                   &+ \frac{
                     \left(\mathbf{y}_{k} -
                     \mathbf{B}_{k}\Delta\mathbf{x}_{k}\right)
                     \Delta\mathbf{x}_{k}^{\top}\mathbf{B}_{k} +
                     \mathbf{B}_{k}\Delta\mathbf{x}_{k}
                     \left(\mathbf{y}_{k} -
                     \mathbf{B}_{k}\Delta\mathbf{x}_{k}\right)^{\top}}{
                     \Delta\mathbf{x}_{k}^{\top}\mathbf{B}_{k}
                     \Delta\mathbf{x}_{k}}\\
                   &- \frac{\Delta\mathbf{x}_{k}^{\top}
                     \left(\mathbf{y}_{k} -
                     \mathbf{B}_{k}\Delta\mathbf{x}_{k}\right)
                     \mathbf{B}_{k}\Delta\mathbf{x}_{k}
                     \Delta\mathbf{x}_{k}^{\top}\mathbf{B}_{k}}{
                     \left(\Delta\mathbf{x}_{k}^{\top}\mathbf{B}_{k}
                     \Delta\mathbf{x}_{k}\right)^{\top}
                     \Delta\mathbf{x}_{k}^{\top}\mathbf{B}_{k}
                     \Delta\mathbf{x}_{k}}.
\end{align*}

As~\citeonline{nocedal&wright} points out, each quasi-Newton method
iteration can be performed at a cost of \(\mathcal{O}(n^{2})\)
arithmetic operations (plus the cost of function and gradient
evaluations); there are no \(\mathcal{O}(n^{3})\) operations such as
linear system solves or matrix-matrix operations. For the BFGS algorithm
is known that the rate of convergence is superlinear, but this is a
valid assumption to any quasi-Newton method, which is fast enough for
most practical purposes. Even though Newton's method converges more
rapidly, quadratically, its cost per iteration usually is higher,
because of its need for second derivatives and solution of a linear
system.

In this thesis, the used BFGS implementation is the one in the
\texttt{R}~\cite{R18}~function \texttt{optim()}, and the PORT routine
used is the one implemented in the \texttt{R} function
\texttt{nlminb()}.

\section{AUTOMATIC DIFFERENTIATION}
\label{cap:ad}

Computing gradients, \(\nabla f(\mathbf{x})\), are a fundamental and
crucial task, but also the main computational bottleneck for any Newton
and quasi-Newton method. Consequently, these computations are
fundamental to the development of this thesis. We choose to use the most
efficient manner of computing gradients, and one of the best scientific
computing techniques, the \textit{automatic differentiation} (AD)
procedure. AD has two modes, the so-called forward and reverse mode. We
will talk a bit about both, but we will use only the reverse mode. The
reason can be illustraded by a simple example, given later.

Automatic differentiation, also called algorithmic differentiation or
computational differentiation, is a set of techniques to numerically and
recursively evaluate the derivative of a function specified by a
computer program. AD techniques are based on the observation that any
function, no matter how complicated, is evaluated by performing a
sequence of simple elementary operations involving just one or two
arguments at a time. Derivatives of arbitrary order can be computed
automatically, automatized and accurately to working precision. Most of
the information in this section was taken of \citeonline{peyre}, but
\citeonline[p.~120]{corestats} and \citeonline[p.~204]{nocedal&wright}
are also very good references.

The most common differentiation approaches are finite differences (FD)
and symbolic calculus. Considering a function \(f: \mathbb{R}^{p}
\rightarrow \mathbb{R}\) and the goal of deriving a method to evaluate
\(\nabla f: \mathbb{R}^{p} \rightarrow \mathbb{R}^{p}\), the
approximation of this vector field via FD would require \(p + 1\)
evaluations of \(f\). The same task via reverse mode AD has in most
cases a cost proportional to a single evaluation of \(f\). AD is similar
to symbolic calculus in the sense that it provides an exact gradient
computation, up to machine precision. However, symbolic calculus does
not takes into account the underlying algorithm which compute the
function, while AD factorizes the computation of the derivative
according to an efficient algorithm.

The use of AD is inherent to the use of a computational graph,
\autoref{fig:compgraph}. Assuming that \(f\) is implemented in an
algorithm, the goal is to compute the derivatives
\begin{align*}
  &\frac{\partial f(\mathbf{x})}{\partial\mathbf{x}_{k}} \in
  \mathbb{R}^{n_{t} \times n_{k}},\\
  &\text{for a numerical algorithm
         (succession of functions) of the form}\\
  &\forall~k = s + 1, \dots, t, \quad
    \mathbf{x}_{k} = f_{k}(\mathbf{x}_{1}, \dots, \mathbf{x}_{k-1}),
\end{align*}
where \(f_{k}\) is a function which only depends on the previous
variables.

\begin{figure}[H]
  % \vspace{0.35cm}
  \setlength{\abovecaptionskip}{.0001pt}
  \caption{A COMPUTATIONAL GRAPH}
  \vspace{0.425cm} \centering
  \includegraphics[width=.8\textwidth]{computational_graph.png}
  \\
  \vspace{0.45cm}
  \begin{footnotesize}
    SOURCE:~\citeonline[p.~31]{peyre}.
  \end{footnotesize}
  \label{fig:compgraph}
\end{figure}

The computational graph, \autoref{fig:compgraph}, has the role of
represent the linking of the variables involved in \(f_{k}\) to
\(\mathbf{x}_{k}\). The evaluation of \(f(\mathbf{x})\) corresponds to a
forward traversal of this graph. Now, how exactly we evaluate \(f\)
through the graph? Via one of the AD modes.

\subsection{Forward Mode}

The forward mode correspond to the usual way of computing differentials.
The method initialize with the derivative of the input nodes
\[
  \frac{\partial \mathbf{x}_{1}}{\partial \mathbf{x}_{1}} =
  \text{Id}_{n_{1} \times n_{1}}, \quad
  \frac{\partial \mathbf{x}_{2}}{\partial \mathbf{x}_{1}} =
  \mathbf{0}_{n_{2} \times n_{1}}, \quad
  \frac{\partial \mathbf{x}_{s}}{\partial \mathbf{x}_{1}} =
  \mathbf{0}_{n_{s} \times n_{1}},
\]
and then iteratively make use of the following recursion formula
\begin{align*}
  &\forall~k = s + 1, \dots, t,\\
  &\frac{\partial\mathbf{x}_{k}}{\partial\mathbf{x}_{1}} =
    \sum_{l~\in~\text{father}(k)}
    \frac{\partial\mathbf{x}_{k}}{\partial\mathbf{x}_{l}} \times
    \frac{\partial\mathbf{x}_{l}}{\partial\mathbf{x}_{1}} =
    \sum_{l~\in~\text{father}(k)}
    \frac{\partial}{\partial\mathbf{x}_{l}}
    f_{k}(\mathbf{x}_{1}, \dots, \mathbf{x}_{k-1}) \times
    \frac{\partial\mathbf{x}_{l}}{\partial\mathbf{x}_{1}}.
\end{align*}

The notation ``father(\(k\))'' denotes the nodes \(l < k\) of the graph
that are connected to \(k\). We make use of \citeonline[p.~32]{peyre}'s
simple example.

\noindent\textbf{Example.}\hspace{.5cm}
Consider the function
\[
  f(x, y) = y\log(x) + \sqrt{y\log(x)}
\]
with the corresponding computational graph being displayed in
\autoref{fig:excompgraph}.

\begin{figure}[H]
  % \vspace{0.35cm}
  \setlength{\abovecaptionskip}{.0001pt}
  \caption{EXAMPLE OF A SIMPLE COMPUTATIONAL GRAPH}
  \vspace{0.425cm} \centering
  \includegraphics[width=.8\textwidth]{ex-computational_graph.png}
  \\
  \vspace{0.45cm}
  \begin{footnotesize}
    SOURCE:~\citeonline[p.~33]{peyre}.
  \end{footnotesize}
  \label{fig:excompgraph}
\end{figure}

The forward mode iterations to compute the derivative w.r.t. \(x\),
following the computational graph, is given by
\begin{align*}
  \frac{\partial x}{\partial x} &= 1, \quad
  \frac{\partial y}{\partial x} = 0\\
  \frac{\partial a}{\partial x} &=
  \frac{\partial a}{\partial x} \frac{\partial x}{\partial x} =
  \frac{1}{x} \frac{\partial x}{\partial x} \qquad
  &\{x \mapsto a = \log(x)\}\\
  \frac{\partial b}{\partial x} &=
  \frac{\partial b}{\partial a} \frac{\partial a}{\partial x} +
  \frac{\partial b}{\partial y} \frac{\partial y}{\partial x} =
  y \frac{\partial a}{\partial x} + 0 \qquad
  &\{(y, a) \mapsto b = ya\}\\
  \frac{\partial c}{\partial x} &=
  \frac{\partial c}{\partial b} \frac{\partial b}{\partial x} =
  \frac{1}{2\sqrt{b}} \frac{\partial b}{\partial x} \qquad
  &\{b \mapsto c = \sqrt{b}\}\\
  \frac{\partial f}{\partial x} &=
  \frac{\partial f}{\partial b} \frac{\partial b}{\partial x} +
  \frac{\partial f}{\partial c} \frac{\partial c}{\partial x} =
  1 \frac{\partial b}{\partial x} + 1 \frac{\partial c}{\partial x}
  \qquad &\{(b, c) \mapsto f = b + c\}
\end{align*}

To compute the derivative w.r.t. \(y\) we run another forward process
\begin{align*}
  \frac{\partial x}{\partial y} &= 0, \quad
  \frac{\partial y}{\partial y} = 1\\
  \frac{\partial a}{\partial y} &=
  \frac{\partial a}{\partial x} \frac{\partial x}{\partial y} = 0 \qquad
  &\{x \mapsto a = \log(x)\}\\
  \frac{\partial b}{\partial y} &=
  \frac{\partial b}{\partial a} \frac{\partial a}{\partial y} +
  \frac{\partial b}{\partial y} \frac{\partial y}{\partial y} =
  0 + a \frac{\partial y}{\partial y}\qquad
  &\{(y, a) \mapsto b = ya\}\\
  \frac{\partial c}{\partial y} &=
  \frac{\partial c}{\partial b} \frac{\partial b}{\partial y} =
  \frac{1}{2\sqrt{b}} \frac{\partial b}{\partial y} \qquad
  &\{b \mapsto c = \sqrt{b}\}\\
  \frac{\partial f}{\partial y} &=
  \frac{\partial f}{\partial b} \frac{\partial b}{\partial y} +
  \frac{\partial f}{\partial c} \frac{\partial c}{\partial y} =
  1 \frac{\partial b}{\partial y} + 1 \frac{\partial c}{\partial y}
  \qquad &\{(b, c) \mapsto f = b + c\}
\end{align*}

\subsection{Reverse Mode}

Instead of evaluating the differentials for all the input nodes, which
is problematic for a large number of nodes, the reverse mode evaluates
the differentials of the output node w.r.t. all the inner nodes.

The method initialize with the derivative of the final node
\[
  \frac{\partial \mathbf{x}_{t}}{\partial \mathbf{x}_{t}} =
  \text{Id}_{n_{y} \times n_{y}},
\]
and then, from the last to the first node, iteratively make use of the
following recursion formula
\begin{align*}
  &\forall~k = t - 1, t - 2, \dots, 1,\\
  &\frac{\partial\mathbf{x}_{t}}{\partial\mathbf{x}_{k}} =
    \sum_{m~\in~\text{son}(k)}
    \frac{\partial\mathbf{x}_{t}}{\partial\mathbf{x}_{m}} \times
    \frac{\partial\mathbf{x}_{m}}{\partial\mathbf{x}_{k}} =
    \sum_{m~\in~\text{son}(k)}
    \frac{\partial\mathbf{x}_{t}}{\partial\mathbf{x}_{m}} \times
    \frac{\partial}{\partial\mathbf{x}_{k}}
    f_{m}(\mathbf{x}_{1}, \dots, \mathbf{x}_{m}).
\end{align*}

The notation ``son(\(k\))'' denotes the nodes \(m < k\) of the graph
that are connected to \(k\). To be clear, the same simple example.

\noindent\textbf{Example.}\hspace{.5cm}
Consider, again, the function
\[
  f(x, y) = y\log(x) + \sqrt{y\log(x)}.
\]

The iterations of the reverse mode is given by
\begin{align*}
  \frac{\partial f}{\partial f} &= 1\\
  \frac{\partial f}{\partial c} &=
  \frac{\partial f}{\partial f} \frac{\partial f}{\partial c} =
  \frac{\partial f}{\partial f} 1\qquad &\{c \mapsto f = b + c\}\\
  \frac{\partial f}{\partial b} &=
  \frac{\partial f}{\partial c} \frac{\partial c}{\partial b} +
  \frac{\partial f}{\partial f} \frac{\partial f}{\partial b} =
  \frac{\partial f}{\partial c} \frac{1}{2\sqrt{b}} +
  \frac{\partial f}{\partial f} 1\qquad
  &\{b \mapsto c = \sqrt{b},~b \mapsto f = b + c\}\\
  \frac{\partial f}{\partial a} &=
  \frac{\partial f}{\partial b} \frac{\partial b}{\partial a} =
  \frac{\partial f}{\partial b} y\qquad &\{a \mapsto b = ya\}\\
  \frac{\partial f}{\partial y} &=
  \frac{\partial f}{\partial b} \frac{\partial b}{\partial y} =
  \frac{\partial f}{\partial b} a \qquad &\{y \mapsto b = ya\}\\
  \frac{\partial f}{\partial x} &=
  \frac{\partial f}{\partial a} \frac{\partial a}{\partial x} =
  \frac{\partial f}{\partial a} \frac{1}{x} \qquad
  &\{x \mapsto a = \log(x)\}
\end{align*}

This is the advantage of reverse mode over the forward mode. A single
traversal over the computational graph allows to compute both
derivatives w.r.t. \(x, y\), while the forward mode necessities two
processes.

An drawback of the reverse mode is the need to store the entire
computational graph, which is needed for the reverse sweep. In
principle, storage of this graph is not too difficult to implement.
However, the main benefit of AD is higher accuracy, and in many
applications the cost is not critical.


\section{TMB: TEMPLATE MODEL BUILDER}
\label{cap:tmb}

Note that the goal of AD is not to define an efficient computational
graph, it is up to the user to provide it. However, computing an
efficient graph associated to a mathematical formula is a complicated
combinatorial problem. Thus, since our goal is to be able to fit our
desired statistical models, a computational tool able to efficiently
define and implement this computational graph is make necessary. To
solve this and many other tasks, we have the Template Model Builder
(TMB) \cite{TMB}.

TMB \url{ http://tmb-project.org} is an \texttt{R} \cite{R18} package
for fitting statistical latent variable models to data, inpired by AD
Model Builder (ADMB) \cite{ADMB}. ADMB is a statistical application for
fitting nonlinear statistical models and solve optimization problems,
that implements AD using \texttt{C++} classes and a native template
language. Unlike most \texttt{R} packages, in TMB the model is
formulated in \texttt{C++}. This characteristic provides great
flexibility, but requires some familiarity with the
\texttt{C}/\texttt{C++} programming language. With TMB a user should be
able to quickly implement complex latent effect models through simple
\texttt{C++} templates.

In this chapter we describe step-by-step all the processes involved in
the creation and parameter estimation of a GLMM. With the TMB, all this
is put in practice in an efficient and robust fashion.

A user needs to provide just the joint likelihood function writing in a
\texttt{C++} template. If the model presents latent effects, during the
compilation the latent effects will be integrated out via an efficient
Laplace approximation routine, with a Newton algorithm inside, and the
marginal log-likelihood gradient will be also computed. These marginal
log-likelihood will be returned into an \texttt{R} object, that can then
be optimized using the user's favorite quasi-Newton routine, available
in \texttt{R}. To do all that, TMB combines some state of the art
software

\begin{itemize}
\item \texttt{CppAD}, a \texttt{C++} AD package
  \url{https://coin-or.github.io/CppAD/};
\item \texttt{Eigen} \cite{eigen}, a \texttt{C++} templated
  matrix-vector library;
\item \texttt{CHOLMOD}, sparse matrix routines available from
  \texttt{R}, used to obtain an efficient implementation of the Laplace
  approximation with exact derivatives
  \url{https://developer.nvidia.com/cholmod};
\item Parallelism through \texttt{BLAS}: Basic Linear Algebra
  Subprograms \url{http://www.netlib.org/blas/}.
\end{itemize}

Also, some of its key characteristics are

\begin{itemize}
\item TMB employs AD to calculate first and second order derivatives of
  the likelihood function or any objective function in \texttt{C++};
\item The objective function, and its derivatives, can be called from
  \texttt{R}. Hence, parameter estimation via \texttt{optim()} or
  \texttt{nlminb()} is easy to be performed;
\item Standard deviations of any parameter, or derived parameter, can be
  obtained via the \textit{delta method}.
\end{itemize}

Here we focus on GLMMs, but basically any statistical model with a
latent structure (or not), linear (or not), can be fitted with TMB. In
times of \textit{big data}, and with the TMB's authors having a
professional preference for state-of-space and spatial models, TMB has
also automatic sparseness detection.and some other nice built tools. Pre
and post-processing of data should be done in \texttt{R}.

A TMB Users' mailing list exists, and it is extremely helpful for taking
doubts and questions \url{https://groups.google.com/g/tmb-users}. Also,
a very didactic and comprehensive documentation with several examples is
available online
\url{https://kaskr.github.io/adcomp/_book/Tutorial.html}.

% END ==================================================================
% ----------------------------------------------------------------------
\chapter{\(\text{multiGLMM}\): a multinomial GLMM for clustered
  competing risks data}
\label{cap:model}
We are handling with a complex survival data structure, the clustered
competing risks setting. But we are using a general statistical modeling
framework, the generalized linear mixed models (GLMMs), that was not
made for this purpose.

To model competing risks data, one has to choose in which scale to work.
We can work on the hazard scale dealing with the cause-specific hazard
or on the probability scale dealing with the cause-specific cumulative
incidence function (CIF). With the correct link function, we can make an
appropriate GLMM to work on that probability scale.

Our focus in this thesis is to be able to deal with complex family
studies, where there is generally a strong interest in describing age at
disease onset in the scenarios of within-cluster dependence. The
distribution of age at disease onset is directly described by the
cause-specific CIF. To make a GLMM work for this type of data we need to
accommodate the cause-specific CIFs and the censorings. Assuming the
conditional distribution for our model response as multinomial already
deals with both left-truncation and right-censoring, avoiding the
specification of a censoring distribution. The cause-specific CIFs can
be modeled via the link function of our, then, multinomial GLMM
(multiGLMM). The multinomial distribution also guarantees that the CIFs
of all causes are modeled.

Our choice of a general framework tries to make the inference of this
complex model, easier. Besides, taking advantage of all the procedures
mentioned in the previous chapter.

\section{CLUSTER-SPECIFIC CUMULATIVE INCIDENCE FUNCTION
  (CIF)}
\label{cap:cif}

Consider that the observed follow-up time of an individual is given by
\(T = \min(T^{\ast},~C)\), where \(T^{\ast}\) denote the failure time
and \(C\) denote the censoring time. Given the possible covariates \(X\)
(that can be time-dependent), for a cause-specific of failure \(k\) the
CIF is defined as
\begin{align*}
  F_{k}(t \mid X) &= \mathbb{P}[T \leq t, K = k \mid X]\\
                  &= \int_{0}^{t} f_{k}(z \mid X)~\text{d}z\\
                  &= \int_{0}^{t} \lambda_{k}(z \mid X)~S(z \mid X)
                    ~\text{d}z, \quad t > 0, \quad k = 1, \dots, K.
\end{align*}
where \(f_{k}(t \mid X)\) is the (sub)density for the time to a type
\(k\) failure. This is the general definition of a CIF, and to define it
we need to define the functions that compose the subdensity.

The first is the cause-specific hazard function or process
\[
  \lambda_{k}(t \mid X) =
  \lim_{h \rightarrow 0}~h^{-1}
  \mathbb{P}[t \leq T < t + h, K = k \mid T \geq t, X],
  \quad t > 0, \quad k = 1, \dots, K.
\]

In words, the cause-specific hazard function, \(\lambda_{k}(t \mid X)\),
represents the instantaneous rate for failures of type \(k\) at time
\(t\) given \(X\) and all other failure types (competing causes). If
we sum up all cause-specific hazard function we get the overall hazard
function,
\[
  \lambda(t \mid X) = \sum_{k=1}^{K}\lambda_{k}(t \mid X).
\]

From the overall hazard function we arrive in the overall survival function,
\[
  S(t \mid X) =
  \mathbb{P}[T > t \mid X] =
  \exp\left\{-\int_{0}^{t} \lambda(z \mid X)~\text{d}z\right\},
\]
the second function that compose the subdensity \(f_{k}(t \mid X)\). A
comprehensive reference for all these definitions is the book of
\citeonline{kalb&prentice}.

Until this point, we were talking about a general CIF's definition. We
need now a precise framework telling how to take into consideration our
clustered/family structure. We use the same CIF specification of
\citeonline{SCHEIKE}, i.e. the approach that motivated this thesis.

For two competing causes of failure, the cause-specific CIFs are
specified in the following manner,
\begin{equation}
  F_{k} (t \mid X, u_{1}, u_{2}, \eta_{k}) =
  \underbrace{\pi_{k}(X, u_{1}, u_{2})}_{
    \substack{\text{cluster-specific}\\\text{risk level}}}\times
  \underbrace{\Phi[w_{k} g(t) - X^{\top}\gamma_{k} - \eta_{k}]}_{
    \substack{\text{cluster-specific}\\\text{failure time trajectory}}
  }, \quad t > 0, \quad k = 1,~2.
  \label{eq:cif}
\end{equation}
i.e. as a product of a cluster-specific risk level and a
cluster-specific failure time trajectory, resulting in a
cluster-specific CIF.

What makes the components in \autoref{eq:cif} cluster-specific are
\(\bm{u} = \{u_{1}, u_{2}\}\) and \(\bm{\eta} = \{\eta_{1},
\eta_{2}\}\), Gaussian distributed latent effects with zero mean and
potentially correlated, i.e.
\[
  \begin{bmatrix} u_{1}\\u_{2}\\\eta_{1}\\\eta_{2} \end{bmatrix} \sim
  \mathcal{N} \left(\begin{bmatrix} 0\\0\\0\\0\end{bmatrix},
    \begin{bmatrix}
      \sigma_{u_{1}}^{2}&
      \text{cov}(u_{1},~u_{2})&
      \text{cov}(u_{1},~\eta_{1})&\text{cov}(u_{1},~\eta_{2})\\
      &\sigma_{u_{2}}^{2}&
      \text{cov}(u_{2},~\eta_{1})&\text{cov}(u_{2},~\eta_{2})\\
      &&\sigma_{\eta_{1}}^{2}&\text{cov}(\eta_{1},~\eta_{2})\\
      &&&\sigma_{\eta_{2}}^{2}
    \end{bmatrix}\right).
\]

The cluster-specific survival function is given as \(S(t \mid X, \bm{u},
\bm{\eta}) = 1 - F_{1} (t \mid X, \bm{u}, \eta_{1}) - F_{2} (t \mid X,
\bm{u}, \eta_{2})\).

Since we use the same CIF specification of \citeonline{SCHEIKE}, the
following descriptions and details are essentially the same encountered
in the paper.

Focusing first on the second component of \autoref{eq:cif}. The
cluster-specific failure time trajectory
\[
  \Phi[w_{k} g(t) - X^{\top}\gamma_{k} - \eta_{k}],
  \quad t > 0, \quad k = 1, ~2,
\]
where \(\Phi(\cdot)\) is the cumulative distribution function of a
standard Gaussian distribution.

Instead of \(w_{k} g(t)\), in \citeonline{SCHEIKE} is specified
\(\alpha_{k}(g(t))\), where \(\alpha_{k}(\cdot)\) are monotonically
increasing functions known up to a finite-dimensional parameter vector,
\(w_{k}\). Examples are monotonically increasing B-spline or piecewise
lienar functions. However, to try to simplify the model structure we
consider just the finite-dimensional parameter vector. The bottom line
is that the authors do the same approach in their applications.

With regard to the function \(g(t)\), it plays a crucial role since the
separation of the CIF in \autoref{eq:cif} is only possible with it. A
time \(t\) transformation given by
\[
  g(t) = \text{arctanh}\left(\frac{t - \delta/2}{\delta/2}\right),
  \quad t\in~]0,~\delta[, \quad g(t)\in~]-\infty,~\infty[,
\]
where \(\delta\) depends on the data and cannot exceed the maximum
observed follow-up time \(\tau\), i.e. \(\delta \leq \tau\). With this
transformation, based on a Fisher transformation, the value of the
cluster-specific failure time trajectory is equal 1, at time \(\delta\).
Consequently, \(F_{k} (\delta \mid X, \bm{u}, \eta_{k}) = \pi_{k}(X \mid
\bm{u})\) and, we can interpret \(\pi_{1}(X \mid \bm{u})\) and
\(\pi_{2}(X \mid \bm{u})\) as the cause-specific cluster-specific risk
levels, at time \(\delta\).

The cluster-specific risk levels are modeled by a multinomial logistic
regression model with latent effects, i.e.
\begin{equation}
  \pi_{k}(X, \bm{u}) =
  \frac{\exp\{X^{\top}\beta_{k} + u_{k}\}}{1 +
    \exp\{X^{\top}\beta_{1} + u_{1}\} +
    \exp\{X^{\top}\beta_{2} + u_{2}\}}, \quad k = 1,~2,
  \label{eq:risklevel}
\end{equation}
where \(\beta_{k}\)'s are the parameters responsible for quantifying the
impact of the covariates in the cause-specific risk levels. For
individuals from the same chuster/family, at the same time point, the
\(\beta_{k}\)s have the well-known odds ratio interpretation.

The \(\gamma_{k}\)'s are the parameters responsible for quantifying the
impact of the covariates in the cause-specific failure time
trajectories, i.e. the shape of the cumulative incidence, and
consequently how quickly the cluster-specific risk levels observed at
time \(\delta\) are reached. The fact that \(\gamma_{k}\) enters
negatively in the cluster-specific failure time trajectory makes that a
negative value causes an advance towards the cluster-specific risk
level, whereas a covariate with a positive effect causes a delay.

Within-cluster dependence is induced by the latent effects in \(\bm{u}\)
and \(\bm{\eta}\), but they don't have an easy interpretation. To help
in the discussion, \autoref{fig:cif} illustrates the cluster-specific
CIF for a given failure cause, let's call it failure cause 1 (in total
we have two).

\begin{figure}[H]
  % \vspace{0.35cm}
  \setlength{\abovecaptionskip}{.0001pt}
  \caption{ILLUSTRATION OF A CLUSTER-SPECIFIC CUMULATIVE INCIDENCE
    FUNCTION (CIF), PROPOSED BY \citeonline{SCHEIKE}, FOR A GIVEN
    FAILURE CAUSE 1. FROM A CONFIGURATION WITH \(X = 1\) FOR ALL
    SUBJECTS AND WITH \(\beta_{1} = -1.9\), \(\beta_{2} = -0.2\),
    \(\gamma_{1} = 1\), \(w_{1} = 3\) AND \(u_{2} = 0\). THE VARIATION
    BETWEEN FRAMES IS GIVEN BY THE LATENT EFFECTS \(u_{1}\) AND
    \(\eta_{1}\)}
  \vspace{0.425cm} \centering
  \includegraphics[width=\textwidth]{cif-1.png}
  \\
  \vspace{0.45cm}
  \begin{footnotesize}
    SOURCE: The author (2020).
  \end{footnotesize}
  \label{fig:cif}
\end{figure}

The latent effects \(u_{1}\) and \(u_{2}\) always appear together in the
cluster-specific risk level, as consequency they have a joint effect on
the cumulative incidence of both causes. Nevertheless, as we can see in
\autoref{fig:cif}, an increase in \(u_{k}\) will increase the risk of
failure from cause \(k\) and vice versa. The interpretation of
\(\text{cov}(\eta_{1},~\eta_{2})\) and \(\text{cov}(u_{1},~u_{2})\) is
more or less straightforward. With regard to
\(\text{cov}(u_{k},~\eta_{k})\), a negative correlation between
\(\eta_{k}\) and \(u_{k}\) imply that when \(\eta_{k}\) decreases,
\(u_{k}\) increases and conversely when \(\eta_{K}\) increases,
\(u_{k}\) decreases. In other words, an increased risk level is reached
quickly and a decreased risk level is reached later, respectively.

Practical situations with a positive within-cause correlation are hard
to find, i.e. where an increased risk level is associated with a late
onset and vice versa. However, a positive cross-cause correlation
between \(\eta\) and \(u\) sounds more realistic. i.e. where late onset
of one failure cause is associated with a high absolute risk of another
failure cause.

The latent effects are assumed independent across clusters and shared by
individuals within the same cluster/family.

\section{MODEL SPECIFICATION}
\label{cap:modelitself}

Our generalized linear mixed model (GLMM) is specified in the following
fashion. For two competing causes of failure, a subject \(i\), with
cluster \(j\), in the time \(t\), we have
\begin{align}
  y_{i j t} \mid \{u_{1j},~u_{2j},~\eta_{1j},~\eta_{2j}\}&\sim
  \text{Multinomial}(p_{1ijt},~p_{2ijt},~p_{3ijt})\nonumber\\
  \nonumber\\
  \begin{bmatrix} u_{1}\\u_{2}\\\eta_{1}\\\eta_{2} \end{bmatrix}&\sim
  \mathcal{N} \left(\begin{bmatrix} 0\\0\\0\\0\end{bmatrix},
  \begin{bmatrix}
    \sigma_{u_{1}}^{2}&
    \text{cov}(u_{1},~u_{2})&
    \text{cov}(u_{1},~\eta_{1})&\text{cov}(u_{1},~\eta_{2})\\
    &\sigma_{u_{2}}^{2}&
    \text{cov}(u_{2},~\eta_{1})&\text{cov}(u_{2},~\eta_{2})\\
    &&\sigma_{\eta_{1}}^{2}&\text{cov}(\eta_{1},~\eta_{2})\\
    &&&\sigma_{\eta_{2}}^{2}
  \end{bmatrix}\right)\nonumber\\
  \nonumber\\
  p_{kijt} &=
  \frac{\partial}{\partial t}F_{k} (t \mid X, u_{1}, u_{2}, \eta_{k})
  \nonumber\\
  &= \frac{\exp\{\bm{x}_{kij}\bm{\beta}_{ki} + u_{kj}\}}{
    1 + \sum_{m=1}^{K-1}\exp\{\bm{x}_{mij}\bm{\beta}_{mi} + u_{mj}\}}
  \label{eq:model}\\
  &\times w_{k}\frac{\delta}{2\delta t - 2t^{2}}~
  \phi\left(
    w_{k}
    \text{arctanh}\left(\frac{t-\delta/2}{\delta/2}\right)
    - \bm{x}_{kij}\bm{\gamma}_{ki} - \eta_{kj}
    \right),\nonumber\\ k = 1,~2.\nonumber
\end{align}

The chosen link function to represent the probabilities is given by the
derivative w.r.t. time \(t\) of the cluster-specific CIF. The choice of
a multinomial logistic regression model ensures that the sum of the
predicted cause-specific CIFs does not exceed 1.

Considering two competing causes of failure, we have a multinomial with
three classes. The third class exists to handle the censorship and its
probability is given by the complementary to reach 1. This framework in
\autoref{eq:model} results in what we call multiGLMM, a multinomial
GLMM.

For a random sample, the corresponding marginal likelihood functions in
given by
\begin{align}
  L(\bm{\theta}~;~y)
  &= \prod_{j=1}^{J}~\int_{\Re^{4}}
    \pi(y_{j} \mid \bm{r}_{j})\times\pi(\bm{r}_{j})~\text{d}\bm{r}_{j}
    \nonumber\\
  &= \prod_{j=1}^{J}~\int_{\Re^{4}}
    \Bigg\{
    \underbrace{\prod_{i=1}^{n_{j}}~\prod_{t=1}^{n_{ij}}
    \Bigg(
    \frac{(\sum_{k=1}^{K}y_{kijt})!}{y_{1ijt}!~y_{2ijt}!~y_{3ijt}!}~
    \prod_{k=1}^{K} p_{kijt}^{y_{kijt}}
    \Bigg)}_{\substack{\text{fixed effect component}}}
  \Bigg\}\times\nonumber\\
  &\hspace{2cm}\underbrace{
    (2\pi)^{-2} |\Sigma|^{-1/2} \exp
    \left\{-\frac{1}{2}\bm{r}_{j}^{\top} \Sigma^{-1} \bm{r}_{j}\right\}
    }_{\substack{\text{latent effect component}}}
    \text{d}\bm{r}_{j}\nonumber\\
  &= \prod_{j=1}^{J}~\int_{\Re^{4}}
    \Bigg\{
    \underbrace{\prod_{i=1}^{n_{j}}~\prod_{t=1}^{n_{ij}}
    \prod_{k=1}^{K} p_{kijt}^{y_{kijt}}
    }_{\substack{\text{fixed effect}}}
  \Bigg\}\underbrace{
  (2\pi)^{-2} |\Sigma|^{-1/2} \exp
  \left\{-\frac{1}{2}\bm{r}_{j}^{\top} \Sigma^{-1} \bm{r}_{j}\right\}
  }_{\substack{\text{latent effect component}}}
  \text{d}\bm{r}_{j}\label{eq:loglik},
\end{align}
where \(\bm{\theta} = [\bm{\beta}~\bm{\gamma}~\bm{w}~\bm{\sigma^{2}}~
\bm{\varrho}]^{\top}\) is the parameters vector to be maximized. In our
framework, a subject can fail from just one competing cause or get
censor, at a given time. Thus, the fraction of factorials in the fixed
effect component is made only by 0's and 1's. Finally, returning the
value 1 .The matrix \(\Sigma\) is the variance-covariance matrix, which
components are given by \(\bm{\sigma}^{2}\) and \(\bm{\varrho}\).

Now, \autoref{eq:loglik} in words. To each cluster (family) \(j\) we
have a product of two components. The fixed effect component, given by a
multinomial distribution with its probabilities specified through the
cluster-specific CIF (\autoref{eq:cif}) and, the latent effect
component, given by a multivariate Gaussian distribution.

To each subject \(i\) that composes a cluster \(j\) we have its specific
fixed effects contribution. The likelihood in \autoref{eq:loglik} is the
most general as possible, allowing for repeated measures to each
subject. Since all subjects of a given cluster shares the same latent
effect, we have just one latent effect contribution multiplying the
product of fixed effects contribution. As we don't observe the latent
effect variables, \(\bm{r}_{j}\), we integrate out in it. With two
competing causes of failure, we have four latent effects (a multivariate
Gaussian distribution in four dimensions). As consequence, for each
cluster, we approximate an integral in four dimensions. The product of
these approximated integrals results in the called marginal likelihood,
to be maximized in \(\bm{\theta}\).

% END ==================================================================
% ----------------------------------------------------------------------
\chapter{simulation study datasets}
\label{cap:datasets}
This chapter describes how to simulate from our multiGLMM, and describes
a real-based dataset used as an application example. The simulation
procedure is addressed in \autoref{cap:simu}. In \autoref{cap:data} a
simulated dataset based on the Nordic Cancer Union (NCU) twins data is
presented as an application example.

\section{SIMULATING FROM THE MODEL}
\label{cap:simu}

Being able to simulate data from a model is a key task, fundamental to
assess the finite-sample properties and the estimation procedure
liability of a given statistical model. The step-by-step describing the
simulation procedure of our multiGLMM is presented on Algorithm
\autoref{alg:algo}, following the model hierarchical structure
stipulated in Formula \autoref{eq:model}.

\begin{algorithm}[H]
 \caption{SIMULATING FROM A \(\text{multiGLMM}\) FOR CLUSTERED COMPETING
          RISKS DATA}
 \label{alg:algo}
 \begin{algorithmic}[1]
  \State
   Set \(J\), the number of clusters
  \State
   Set \(n_{j}\), the number of cluster elements
   \Comment{can be of different sizes}
  \State
   Set \(K-1\), the number of competing causes of failure
  \State
   Set the model parameter values \(\bm{\theta} =
   [\bm{\beta}~\bm{\gamma}~\bm{w}~\bm{\sigma^{2}}~\bm{\varrho}]^{\top}\)
  \State
   Sample \(J\) latent effect vectors from a
   \(\mathcal{N}_{(K-1)\times(K-1)}(\bm{0},~\Sigma(\bm{\sigma^{2}},
     \bm{\varrho}))\)
  \State
   Set \(\delta\)
   \Comment{maximum follow-up time}
  \State
   Set the failure times \(t_{ij}\)
  \State
   Compute the competing risks probabilities
   \begin{align*}
      p_{kijt}
      &= \frac{\exp\{\bm{x}_{kij}\bm{\beta}_{ki} + u_{kj}\}}{
        1 +
        \sum_{m=1}^{K-1}\exp\{\bm{x}_{mij}\bm{\beta}_{mi} + u_{mj}\}}\\
      &\times
        w_{k}\frac{\delta}{2\delta t - 2t^{2}}~
        \phi\left(
        w_{k}
        \text{arctanh}\left(\frac{t-\delta/2}{\delta/2}\right)
        - \bm{x}_{kij}\bm{\gamma}_{ki} - \eta_{kj}
        \right),\\
      p_{Kijt}
      &= 1 - \sum_{k = 1}^{K - 1} p_{kijt}, \quad k = 1,~2,~\dots,~K -1
    \end{align*}
    \State
    Sample \(J\times n_{j}\) vectors from a
    \(\text{Multinomial}(p_{1ijt},~p_{2ijt},~\dots,~p_{Kijt})\)
    \State
    If \(t_{kij} = \delta\), subject moves to class K
    \Comment{any failure at time \(\delta\) is censored}
    \State
    \Return
    To each individual, its failure/censoring time and from which
    cause-specific it is
  \end{algorithmic}
\end{algorithm}
\vspace{-1cm}
\begin{footnotesize}
  \begin{center}
    SOURCE: The author (2021).
  \end{center}
\end{footnotesize}

Sample \(\varsigma\sim\text{U}(0,~1)\)

Compute the cause-specific failure times by solving
\[
 \varsigma = \Phi[w_{k} g(t_{k}) - X^{\top}\gamma_{k} - \eta_{k}]
 \quad\text{for } t_{k}, \quad k = 1,~2,~\dots,~K - 1
\]

The model described in \autoref{eq:model} is in its most general form,
i.e. allowing for multiple measures at each subject and varying
coefficients. However, we focus on a simpler structure without
covariates, a single measure per subject, and common coefficients.
Putting in practice Algorithm \autoref{alg:algo}, we use the following
model configuration

\begin{align}
  p_{kijt}
  &= \frac{\exp\{\beta_{ki} + u_{kj}\}}{
    1 + \sum_{m=1}^{K-1}\exp\{\beta_{mi} + u_{mj}\}}\nonumber\\
  &\times w_{k}\frac{\delta}{2\delta t - 2t^{2}}~
    \phi\left(
    w_{k}
    \text{arctanh}\left(\frac{t-\delta/2}{\delta/2}\right)
    - \gamma_{ki} - \eta_{kj}
    \right),\quad k = 1,~2,\nonumber\\
  \text{with }\quad
  \bm{\beta}_{i} &= [-2~~~1.5]^{\top}\nonumber\\
  \bm{\gamma}_{i} &= [1.2~~~1]^{\top}\label{eq:modelconfig}\\
  \bm{w} &= [3~~~5]^{\top}\nonumber\\
  \bm{u}_{j} &= [0~~~0]^{\top}\quad
               \bm{\eta}_{j} = [0~~~0]^{\top}\nonumber.
\end{align}
Based on that we get the cluster-specific CIF's and failure
probabilities, its CIF derivatives (dCIF) w.r.t. time \(t\), presented
respectively in \autoref{fig:datasimucif}.

\begin{figure}[H]
  \setlength{\abovecaptionskip}{.0001pt}
  \caption{CLUSTER-SPECIFIC CUMULATIVE INCIDENCE FUNCTIONS (CIF) AND
    RESPECTIVE DERIVATIVES W.R.T. TIME (\(\text{dCIF}\)) FOR A MODEL
    WITH TWO COMPETING CAUSES OF FAILURE, WITHOUT COVARIATES AND THE
    FOLLOWING CONFIGURATION: \(\beta_{1} = -2\), \(\beta_{2} = -1.5\),
    \(\gamma_{1} = 1.2\), \(\gamma = 1\), \(w_{1} = 3\), \(w_{2} = 5\)
    AND LATENT EFFECTS FIXED AT ZERO}
  \vspace{0.2cm} \centering
  \includegraphics[width=\textwidth]{datasimucif-1.png}
  \\
  \begin{footnotesize}
    SOURCE: The author (2020).
  \end{footnotesize}
  \label{fig:datasimucif}
\end{figure}

By adding the latent structure
\[
  \begin{bmatrix} u_{1}\\u_{2}\\\eta_{1}\\\eta_{2} \end{bmatrix}
  \sim\mathcal{N} \left(
    \begin{bmatrix} 0\\0\\0\\0 \end{bmatrix},
    \begin{bmatrix}
      1&0.4&0.5&0.4\\
      &1&0.4&0.3\\
      &&1&0.4\\
      &&&1
    \end{bmatrix}\right),
\]
in \autoref{eq:modelconfig}, we generate a complete model sample with
500 clusters/pairs of twins, summarized in \autoref{fig:datasimu}.

\begin{figure}[H]
  % \vspace{0.35cm}
  \setlength{\abovecaptionskip}{.0001pt}
  \caption{SUMMARY OF A SIMULATED DATASET WITH 500 PAIRS OF TWINS. A)
    TIME BY TWIN; B) TIMES BOXPLOT; C) PROBABILITIES SCATTERPLOT D)
    \(y_{3}\)'S \%}
  \vspace{0.2cm} \centering
  \includegraphics[width=\textwidth]{datasimu-1.png}
  \\
  \vspace{0.2cm}
  \begin{footnotesize}
    SOURCE: The author (2020).
  \end{footnotesize}
  \label{fig:datasimu}
\end{figure}

\section{REAL-BASED DATASET}
\label{cap:data}

% END ==================================================================

% ----------------------------------------------------------------------
\chapter{Results}
\label{cap:results}
This chapter presents the simulation study results. We have seventy-two
simulation scenarios, as detailed in \autoref{cap:datasets}. For each
scenario we simulate 500 samples. In total, we fit 36000 models.

\section{SIMULATION STUDY}
\label{cap:simures}

Let us just recap the parameter values used
\begin{align*}
 \text{High CIF configuration}:~&\quad
 \{\beta_{1} = -2,~\beta_{2} = -1.5,~\gamma_{1} = 1,~\gamma_{2} = 1.5,~
   w_{1} = 3,~w_{2} = 4
 \};\\
 \text{Low CIF configuration}:~&\quad
 \{\beta_{1} = 3,~\beta_{2} = 2.5,~\gamma_{1} = 2.6,~\gamma_{2} = 4,~
   w_{1} = 5,~w_{2} = 10
 \}.
\end{align*}
\begin{minipage}{0.15\textwidth}
 \begin{align*}
  \sigma_{u_{1}}^{2}   &= 1\\
  \sigma_{u_{2}}^{2}   &= 0.7,\\
  \sigma_{\eta_{1}}^{2} &= 0.6\\
  \sigma_{\eta_{2}}^{2} &= 0.9
 \end{align*}
\end{minipage}%
\begin{minipage}{0.85\textwidth}
 \[
  \text{Correlation structure}~=~\begin{blockarray}{ccccc}
                                  u_{1} & u_{2} & \eta_{1} & \eta_{2}\\
                                  \begin{block}{(cccc)c}
                                   1 & 0.1 & -0.5 &  0.3 & u_{1}\\
                                     &   1 &  0.3 & -0.4 & u_{2}\\
                                     &     &    1 &  0.2 & \eta_{1}\\
                                     &     &      &    1 & \eta_{2}\\
                                  \end{block}
                                 \end{blockarray}.
 \]
\end{minipage}

\vspace{0.3cm}
\noindent
The parameter values per se are not important. What is important is to
keep in mind the behaviors implied by them, and see if the proposed
model is able to estimate the true values in several different scenarios
and measure the quality of the estimates.

The take-home message for the fixed-effect parameters, is to show that
we can construct different level CIF scenarios. The \(\bm{\beta}\)s are
responsible for the curve maximum point or plateau, being in the risk
level CIF component, the \(\bm{\gamma}\)s and \(\bm{w}\)s are
responsible for basically the curve shape, being in the failure time
trajectory level CIF component. Its interpretation is presented in
detail in \autoref{cap:model}. About the latent-effects, the chosen
covariance structure is considerably high but still acceptable. The
underlying idea was to try to build a realistic covariance scenario and
consequently be able to check how the model performs in such conditions.

In the following pages we have several graphs summarizing the estimates
bias. In each figure, we have the estimate bias and its uncertainty
described by a Wald-based confidence interval i.e., \(\pm\) 1.96 the
bias standard deviation. This is a good uncertainty representation
choice since it is symmetric. In the \autoref{cap:appendixD}, we have
the same estimates bias but with its uncertainty measure being the
corresponding 2.5 and 97.5\% bias quantiles. We chose to use these
uncertainty representations uniquely based on the point estimates
instead of the standard error computations. In several scenarios, the
model fails to compute all the standard errors, caused by Hessian
numerical instabilities.

In each of the following estimates bias graphs, the seventy-two
scenarios are accommodated. We have up to four blocks of bars, each
block representing a model. In each block we have eighteen bars, each
bar representing the 500 fits in each of the eighteen
scenarios, \(4 \times 18 \times 500 = 36000\).

Each scenario name consists of a combination of three strings
\begin{itemize}
 \item The cluster size (cs), 2, 5, and 10;
 \item The CIF configuration, high and low;
 \item The sample size, 5, 30, and 60 thousand.
\end{itemize}
We have tried to fit a total of 36000 models but not all converged. To
show these characteristic, we control the bar widths. Something specific
can be said about each parameter but let us keep the focus on the
general remarks. Starting from the fixed-effect parameters
in \autoref{fig:biassdbeta1}, \autoref{fig:biassdbeta2},
\autoref{fig:biassdgama1}, \autoref{fig:biassdgama2},
\autoref{fig:biassdw1}, and \autoref{fig:biassdw2}, we have very nice
results that already show a strong inclination towards the complete
model's choice.

With a latent structure only in the risk level or in the failure time
trajectory level, the low CIF scenarios are the ones with a much smaller
bias-variance. In general, the mean-bias is small but the variances are
high. When we have a latent structure on both levels but we still assume
the cross-correlations as zero (block-diag model), the results get a
little bit better. Nevertheless, when we assume a non-zero
cross-correlation structure (complete model), basically everything
changes for the better. The mean biases get even closer to zero, the
standard deviations decrease 50\% or more, and mainly, now the high CIF
scenarios are the ones with a much smaller bias-variance. All this is
accomplished through the consideration of the cross-correlations.

In the \textit{simpler} models, with a latent structure just in one
level, is hard to see some significant difference between the clusters
and sample sizes. With the complete model, in the other hand, the
difference is clear: as we increase the clusters and the sample sizes,
the bias-variance decreases. The mean-bias is basically always the
same. In the risk model is hard to point-out a scenario as the best or
worst. For the time model, with the scenarios \texttt{cs02-high-05k}
and \texttt{cs05-high-60k}, we get a much bigger standard deviation in
the \(\bm{\beta}\)s parameter estimates. For the block-diag model, with
the scenario \texttt{cs05-low-05k}, the standard deviations are huge for
the shape curve parameter estimates of the competing cause 1. In
the \autoref{cap:appendixD}, with the 2.5 and 97.5\% bias quantiles, the
most extreme values are removed from the uncertainty
representation. There, the main characteristic is the parameter
estimates asymmetry.

\begin{figure}[H]
 \setlength{\abovecaptionskip}{.0001pt}
 \caption{PARAMETER \(\beta_{1}\) BIAS WITH \(\pm\) 1.96 STANDARD
          DEVIATIONS}
 \vspace{0.2cm}\centering
 \includegraphics[width=\textwidth]{bias2plotsd-1.png}\\
 \begin{footnotesize}
  SOURCE: The author (2021).
 \end{footnotesize}
 \label{fig:biassdbeta1}
\end{figure}

\begin{figure}[H]
 \setlength{\abovecaptionskip}{.0001pt}
 \caption{PARAMETER \(\beta_{2}\) BIAS WITH \(\pm\) 1.96 STANDARD
         DEVIATIONS}
 \vspace{0.2cm}\centering
 \includegraphics[width=\textwidth]{bias2plotsd-2.png}\\
 \begin{footnotesize}
  SOURCE: The author (2021).
 \end{footnotesize}
 \label{fig:biassdbeta2}
\end{figure}

\begin{figure}[H]
 \setlength{\abovecaptionskip}{.0001pt}
 \caption{PARAMETER \(\gamma_{1}\) BIAS WITH \(\pm\) 1.96 STANDARD
          DEVIATIONS}
 \vspace{0.2cm}\centering
 \includegraphics[width=\textwidth]{bias2plotsd-3.png}\\
 \begin{footnotesize}
  SOURCE: The author (2021).
 \end{footnotesize}
 \label{fig:biassdgama1}
\end{figure}

\begin{figure}[H]
 \setlength{\abovecaptionskip}{.0001pt}
 \caption{PARAMETER \(\gamma_{2}\) BIAS WITH \(\pm\) 1.96 STANDARD
          DEVIATIONS}
 \vspace{0.2cm}\centering
 \includegraphics[width=\textwidth]{bias2plotsd-4.png}\\
 \begin{footnotesize}
  SOURCE: The author (2021).
 \end{footnotesize}
 \label{fig:biassdgama2}
\end{figure}

\begin{figure}[H]
 \setlength{\abovecaptionskip}{.0001pt}
 \caption{PARAMETER \(w_{1}\) BIAS WITH \(\pm\) 1.96 STANDARD DEVIATIONS}
 \vspace{0.2cm}\centering
 \includegraphics[width=\textwidth]{bias2plotsd-5.png}\\
 \begin{footnotesize}
  SOURCE: The author (2021).
 \end{footnotesize}
 \label{fig:biassdw1}
\end{figure}

\begin{figure}[H]
 \setlength{\abovecaptionskip}{.0001pt}
 \caption{PARAMETER \(w_{2}\) BIAS WITH \(\pm\) 1.96 STANDARD DEVIATIONS}
 \vspace{0.2cm}\centering
 \includegraphics[width=\textwidth]{bias2plotsd-6.png}\\
 \begin{footnotesize}
  SOURCE: The author (2021).
 \end{footnotesize}
 \label{fig:biassdw2}
\end{figure}

\begin{figure}[H]
 \setlength{\abovecaptionskip}{.0001pt}
 \caption{PARAMETER \(\log(\sigma_{1}^{2})\) BIAS WITH \(\pm\) 1.96
          STANDARD DEVIATIONS}
 \vspace{0.2cm}\centering
 \includegraphics[width=\textwidth]{bias2plotsd-7.png}\\
 \begin{footnotesize}
  SOURCE: The author (2021).
 \end{footnotesize}
 \label{fig:biassdlogs2_1}
\end{figure}

\begin{figure}[H]
 \setlength{\abovecaptionskip}{.0001pt}
 \caption{PARAMETER \(\log(\sigma_{2}^{2})\) BIAS WITH \(\pm\) 1.96
          STANDARD DEVIATIONS}
 \vspace{0.2cm}\centering
 \includegraphics[width=\textwidth]{bias2plotsd-8.png}\\
 \begin{footnotesize}
  SOURCE: The author (2021).
 \end{footnotesize}
 \label{fig:biassdlogs2_2}
\end{figure}

\begin{figure}[H]
 \setlength{\abovecaptionskip}{.0001pt}
 \caption{PARAMETER \(\log(\sigma_{3}^{2})\) BIAS WITH \(\pm\) 1.96
          STANDARD DEVIATIONS}
 \vspace{0.2cm}\centering
 \includegraphics[width=\textwidth]{bias2plotsd-9.png}\\
 \begin{footnotesize}
  SOURCE: The author (2021).
 \end{footnotesize}
 \label{fig:biassdlogs2_3}
\end{figure}

\begin{figure}[H]
 \setlength{\abovecaptionskip}{.0001pt}
 \caption{PARAMETER \(\log(\sigma_{4}^{2})\) BIAS WITH \(\pm\) 1.96
          STANDARD DEVIATIONS}
 \vspace{0.2cm}\centering
 \includegraphics[width=\textwidth]{bias2plotsd-10.png}\\
 \begin{footnotesize}
  SOURCE: The author (2021).
 \end{footnotesize}
 \label{fig:biassdlogs2_4}
\end{figure}

\begin{figure}[H]
 \setlength{\abovecaptionskip}{.0001pt}
 \caption{PARAMETER \(z(\rho_{12})\) BIAS WITH \(\pm\) 1.96 STANDARD
          DEVIATIONS}
 \vspace{0.2cm}\centering
 \includegraphics[width=\textwidth]{bias2plotsd-11.png}\\
 \begin{footnotesize}
  SOURCE: The author (2021).
 \end{footnotesize}
 \label{fig:biassdrhoz12}
\end{figure}

\begin{figure}[H]
 \setlength{\abovecaptionskip}{.0001pt}
 \caption{PARAMETER \(z(\rho_{34})\) BIAS WITH \(\pm\) 1.96 STANDARD
          DEVIATIONS}
 \vspace{0.2cm}\centering
 \includegraphics[width=\textwidth]{bias2plotsd-12.png}\\
 \begin{footnotesize}
  SOURCE: The author (2021).
 \end{footnotesize}
 \label{fig:biassdrhoz34}
\end{figure}

\begin{figure}[H]
 \setlength{\abovecaptionskip}{.0001pt}
 \caption{PARAMETERS
          \(\{z(\rho_{13}),~z(\rho_{24}),~z(\rho_{14}),~z(\rho_{23})\}\)
          BIAS WITH \(\pm\) 1.96 STANDARD DEVIATIONS}
 \vspace{0.2cm}\centering
 \includegraphics[width=\textwidth]{bias2plotsd-13.png}\\
 \begin{footnotesize}
  SOURCE: The author (2021).
 \end{footnotesize}
 \label{fig:biassdrhoz4}
\end{figure}

With the log-variances presented in \autoref{fig:biassdlogs2_1},
\autoref{fig:biassdlogs2_2}, \autoref{fig:biassdlogs2_3}, and
\autoref{fig:biassdlogs2_4}, we have instead a similar behavior through
the models. For all the models, the high CIF scenarios are the ones with
a smaller mean and bias-variances. From the risk/time model to the
block-diag model, we do not see a significant improvement in terms of
bias reduction. Such improvement, however, is clear when we look at the
complete model. Again, the magick of considering the cross-correlations.

The same said about the log-variances, can be applied to the risk
correlations in \autoref{fig:biassdrhoz12}, with one addendum: the bias
reduction is even bigger. With the time correlation in
\autoref{fig:biassdrhoz34}, at least with clusters of size 2 and 5,
we get the same behavior observed with the fixed-effect parameters i.e.,
with the simpler models, the smaller biases are observed in the low CIF
scenarios. However, with the complete model, we get the opposite. With
the cross-correlations in
\autoref{fig:biassdrhoz4}, the mean and bias-variances are much smaller
in the high CIF scenarios.

The biggest bias-variances are obtained in the log-variances. A final
remark to be made is about convergences. With the simpler models, not
all of them work, having in some scenarios (generally the ones with 60
thousand data points) a 50\(\sim\)60\% convergence rate. With the
complete model, basically, almost all fits reach convergence
(\(\sim\)95\% performance).

After looking at the parameter estimates biases, let us take a look at
the implied mean-CIF curves. To nicely accommodate all seventy-two
scenarios we split the curves by level-CIF. In \autoref{fig:cifshigh} we
have the high CIF scenario curves and in \autoref{fig:cifslow} the low
CIF scenario curves. Since for all the models we have a latent structure
for the within-cluster dependency, the inherent idea is that this also
affect the fixed-effect parameter estimates. By taking its average in
each of the seventy-two scenarios, we are able to construct the mean CIF
curves.

In \autoref{fig:cifshigh} we have all the thirty-six curves obtained in
the high CIF scenarios. It is clear that with the complete model we get
a perfect fit in all nine scenarios. The risk and time models estimate
well the curve shape parameters but they fail to learn the max
incidence. A compensation between curves is clear.

\begin{figure}[H]
 \setlength{\abovecaptionskip}{.0001pt}
 \caption{HIGH CUMULATIVE INCIDENCE FUNCTION (CIF) SCENARIO CURVES}
 \vspace{0.2cm}\centering
 \includegraphics[width=\textwidth]{cifs-1.png}\\
 \begin{footnotesize}
  SOURCE: The author (2021).
 \end{footnotesize}
 \label{fig:cifshigh}
\end{figure}

\begin{figure}[H]
 \setlength{\abovecaptionskip}{.0001pt}
 \caption{LOW CUMULATIVE INCIDENCE FUNCTION (CIF) SCENARIO CURVES}
 \vspace{0.2cm}\centering
 \includegraphics[width=\textwidth]{cifs-2.png}\\
 \begin{footnotesize}
  SOURCE: The author (2021).
 \end{footnotesize}
 \label{fig:cifslow}
\end{figure}

Still in \autoref{fig:cifshigh}, in the risk model, there is a super
estimation of \(\beta_{1}\) in all scenarios. For failure cause 2, there
is a sub estimation. With the time model, we observe the opposite
compensation but on a smaller scale. With the time model, we get much
better curves than with the risk model. The block-diag model results are
a middle term between them. For the time model, the scenario with
cluster size 10 and 60 thousand data points is a highlight. For the
block-diag model, the highlight is the scenario with cluster size 5 and
30 thousand data points.

In the low CIF scenarios in \autoref{fig:cifslow}, the estimation is
clearly more difficult. The overall fits are bad, being impossible to
select a scenario with overall good results. For one of the failure
causes, the estimation quality is not so bad. The problem is when we
look to the other. An interesting scenario is the one with cluster size
2 and 60 thousand data points. In this scenario we see the worst fits
for failure cause 1, with a negative highlight in the block-diag
configuration. However, with this same model, for failure cause 2, it is
the scenario were we better learn the true curve. An interesting
compensation phenomena. The best joint fit is still with the complete
model.

Now we look at how the latent-effect parameter estimates distribute
themselves. Given the huge number of scenarios and the fact that is
harder to estimate covariance parameters, we chose to plot the parameter
estimates just in the scenarios with better performances. By the metrics
of small bias and CIF shape learning, the scenarios with better results
are the ones with high CIF and bigger sample sizes. We have the
densities for the variance parameter estimates, in each of these
scenarios, presented
in \autoref{fig:histologs2}. In \autoref{fig:historhoz} we have the same
for the correlation parameter estimates.

An interesting result is the clear difference between risk and time
models' covariance parameter estimates. With the risk model, we have an
evident super estimation and bigger variances. With the time model we
get much better results, but still with high variances. The block-diag
model generally performs better than the risk model and worst than the
time model, showing again to be a compromise between them. Besides the
bias itself, we should also pay attention to the values. We model the
variances in the log-scale, so a value 5, in reality, implies a variance
of \(\exp(5) = 148\). Terrible. This kind of problem do not sound to
appear with the complete model.

All correlations are quite well estimated, in all three scenarios, with
the complete model. Not only the correlations but the variances
also. The lack of any considerable difference between the covariance
densities, indicates no quality divergences in the results for different
cluster sizes. The densities in \autoref{fig:histologs2} and
\autoref{fig:historhoz} are the final corroboration indicating the good
performance of the maximum likelihood method in the complete
model.

Between the four tested models, the complete model was the one with the
smallest biases, better CIF shape learning, and precisest covariance
parameter estimates. In \autoref{fig:cor2plot} we have a heat-map of the
correlations between parameter estimates for the complete model in the
scenario with clusters of size 10, high CIF, and 60 thousand data
points.

We have a little bit of everything in the parameter estimates
correlations' heat-map. Some correlations are very close to zero, but we
also have strong positive and negative correlations. We can mention some
curiosities, but nothing pathological appears to happen, at least
nothing clear.

\begin{figure}[H]
 \setlength{\abovecaptionskip}{.0001pt}
 \caption{VARIANCE PARAMETERS DENSITIES IN THE SCENARIOS OF HIGH CIF AND
          60 THOUSAND DATA POINTS}
 \vspace{0.2cm}\centering
 \includegraphics[width=\textwidth]{histologs2-1.png}\\
 \begin{footnotesize}
  SOURCE: The author (2021).
 \end{footnotesize}
 \label{fig:histologs2}
\end{figure}

\begin{figure}[H]
 \setlength{\abovecaptionskip}{.0001pt}
 \caption{CORRELATION PARAMETERS DENSITIES IN THE SCENARIOS OF HIGH CIF
          AND 60 THOUSAND DATA POINTS}
 \vspace{0.2cm}\centering
 \includegraphics[width=\textwidth]{historhoz-1.png}\\
 \begin{footnotesize}
  SOURCE: The author (2021).
 \end{footnotesize}
 \label{fig:historhoz}
\end{figure}

In \autoref{fig:cor2plot}, all fixed-effect parameters are positive
correlated, with an emphasis on the correlation between \(\beta_{1}\)
and \(\beta_{2}\), and the one of the \(\bm{\beta}\)s with
the \(\bm{w}\)s. Another interesting observation is the strong negative
correlation between the \(\bm{\beta}\)s and the risk level
log-variances, and also the (less strong) positive correlation between
the \(\bm{\beta}\)s and the failure time trajectory level
log-variances. The risk level log-variances are (strongly) positively
correlated. So do the failure time trajectory level ones, but again, not
so strong as in the risk level. The correlations between the
log-variances of different levels are negative.

\begin{figure}[H]
 \setlength{\abovecaptionskip}{.0001pt}
 \caption{COMPLETE MODEL'S PARAMETERS CORRELATION HEAT-MAP IN THE
          SCENARIO OF CLUSTER SIZE 10, HIGH CIF, AND SIXTY-THOUSAND DATA
          POINTS}
 \centering
 \includegraphics[width=\textwidth]{cor2plot-1.png}\\
 \vspace{-0.2cm}
 \begin{footnotesize}
  SOURCE: The author (2021).
 \end{footnotesize}
 \label{fig:cor2plot}
\end{figure}

% END ==================================================================

% ----------------------------------------------------------------------
\chapter{Final considerations}
\label{cap:finalc}
The general goal of this master thesis was the proposition and study of
a maximum likelihood estimation approach for the analysis of clustered
competing risks data. Focused on the probability scale, by means of the
cumulative incidence function (CIF), instead of the hazard scale usual
in the survival modeling literature \cite{kalb&prentice}. We model the
clustered competing risks on a latent-effects framework, a generalized
linear mixed model (GLMM) \cite{GLMM}, with a multinomial distribution
for the competing risks and censorship, conditioned on the
latent-effects. The within-cluster latent dependency is accommodated by
a multivariate Gaussian distribution and is modeled via its covariance
matrix parameters.

The failures by the competing causes and their respective censorships
are modeled in the probability scale, by means of the CIF
\cite{kalb&prentice, andersen12}. The CIF is accommodated in our GLMM
framework in terms of the link function \cite{GLM89}, as the product of
two functions, one responsible to model the instantaneous risk and the
other the failure time trajectory, both in a cluster-specific
fashion. The shape of these functions is described in detail
in \autoref{cap:model}. This particular GLMM formulation is what makes
our model, particular. Thus, we have what we call a multiGLMM: a
multinomial GLMM for clustered competing risks data.

The two-function product CIF formulation was taken from
\citeonline{SCHEIKE} but there they use a different framework, a
composite likelihood framework \cite{lindsay88, cox&reid04, varin11}.
Here we do a full likelihood analysis instead. A composite approach is
generally used when a full likelihood approach is impossible or
computationally impracticable. Our goal here was to assess a full
likelihood framework taking advantage of state-of-the-art computational
libraries and very efficient algorithm implementations. We have all this
with the \texttt{R} \cite{R21} package TMB \cite{TMB}.

The applications in focus here were family studies. This kind of study
is characterized by involving big samples, generally, populations. Also,
generally having a high number of small clusters, families. A maximum
likelihood approach with the use of efficiently implemented Laplace
approximations \cite{tierney,patrao} together with an automatic
differentiation (AD) \cite{corestats,nocedal&wright} routine, all via
TMB, is able to handle a considerably high number of clusters,
independent of its size. The multinomial distribution assumption, on its
own, is an excellent probabilistic choice since it can accommodate
virtually any number of competing causes of failure and its
censorship. The presence of those two characteristics in our multiGLMM
makes it an efficient and scalable modeling framework for clustered
competing risks data.

Even with our modeling framework being virtually able to handle any
number of competing causes of failure, we restrained ourselves to work
here with only two of them. With two competing causes, we have
a \(4\times4\) covariance matrix for the latent effects, which implies
ten covariance parameters, which is already a lot of parameters to be
estimated in a latent structure. Since our goal was to study the
viability of the maximum likelihood estimation method, we kept it with
two causes.

All models from the simulation study were run, in a parallelized
fashion, in one of the two following Linux systems:
\begin{description}
 \item[System 1]
  12 Intel (R) Core (TM) i7-8750H CPU @ 2.20GHz processors
  with 16GB RAM;
 \item[System 2]
  30 Intel (R) Xeon (R) CPU E5-2690 v2 @ 3.00GHz processors
  and 206GB RAM.
\end{description}

Each risk and time model run is not so time-consuming, generally never
taking more than 5 minutes. The inherent idea is that for each cluster
we are always performing two-dimension integral approximations and we
have \textit{just} three covariance parameters. With the block-diag
model, we are theoretically in four dimensions. However, since the
covariance matrix is, block-diagonal, we experienced several numerical
instability problems. The solution, as can be seen in the
\autoref{cap:blockdiagModel} (\autoref{cap:appendixD}) code, was to
split it into two two-dimension matrices, since the \(4\times4\)
covariance matrix is block-diagonal.  This simple solution solved all
numerical instability problems. The computational time was only a little
bit bigger than with the risk and time models.

Finally, the complete model. In the biggest scenario, with 60 thousand
data points and clusters of size 2 i.e., with 30 thousand four-dimension
integral approximations (ten parameters in the covariance matrix), the
model fitting takes 30 minutes, in parallel, with TMB. Before doing the
TMB implementation, to really understand what we were doing, we did a
complete \texttt{R} implementation. We wrote the marginal log-likelihood
in \texttt{R}, based on our own Laplace approximation \cite{patrao} and
Newton-Raphson implementation (the gradients, \autoref{cap:appendixA},
and Hessian, \autoref{cap:appendixB}, were computed by hand and
implemented). Running this complete \texttt{R} implementation in a
scenario with 20 thousand data points and clusters of size 2, took
around 30 hours, parallelizing it between all threads of system 1. In
summary, by using TMB we were able to increase the model size 3 times
and to decrease the computational time 60 times. An incredible
performance gain.

Still, with the complete model, we performed a Bayesian analysis via
\texttt{tmbstan} \cite{tmbstan}. \texttt{tmbstan} enables MCMC sampling
\cite{MCMC, Diaconis} from a TMB model object using Stan \cite{Stan,
RStan}. Sampling can be performed with or without a Laplace
approximation for the random effects. We performed just one Bayesian
model fitting in a modest scenario with 5 thousand data points and
clusters of size 2. It took around 1 whole week of parallelized
processing in system 1. The results were basically the same as the ones
obtained with TMB but this high computational time just reinforces the
MCMC framework limitation.

An important point to be made here is about TMB's memory consumption. As
the sample size increases, the dimension of the model matrices also
increases. This, summed to a high number of clusters (Laplace
approximations to be performed), turns out to be a computational
nightmare. For several models, even the 16GB RAM of system 1 was not
enough. The bottleneck appears to be in the AD tape, which is made in
parallel, by default, if the model fitting is in parallel. By turning
this option off (line 11 of \autoref{cap:rscript}
(\autoref{cap:appendixD}) code), we were able to save a lot of memory,
making several models practicable.

Model the CIF of clustered competing risks data is far from being
trivial or straightforward. The formulation in \autoref{eq:cif} implies
the desired curve behavior, \autoref{fig:datasimucif}. However, in
counterpart, its derivatives w.r.t. time, generates very small
probabilities for the failure competing causes, ending by concentrating
almost everything on censorship, \autoref{fig:datasimu}. For each
competing cause with poor data representativity, we have three curve
shape parameters to estimate, implying the necessity of having a lot of
data to then have enough information about the causes.

We proposed for our multiGLMM an ideally complete latent-effects
formulation i.e., correlated latent effects on both levels,
instantaneous risk and failure time trajectory. The main underlying idea
of the \autoref{cap:results} simulation study was to see in which
scenarios we would be able to learn all the involved mean and covariance
parameters. As part of that, simpler formulations were proposed i.e.,
latent-effects in only one level, or in both but without
cross-correlations. As result, we got that latent effects only in the
risk level did not work. The optimization appears to get lost as if
something is missing. Inserting latent effects only in the failure time
trajectory level returned better results, but still not satisfactorily
good. In most of the evaluated scenarios, the block-diagonal model
appeared to be in the middle of them, as a compromise. The best results
were obtained with the complete model i.e. when we consider the
cross-correlations between levels. In general, we still observe some
high variances between the parameter estimates, but given all the
problem characteristics mentioned earlier, sounds to be reasonable. On
average, the complete model works fine, mainly in the scenarios of high
CIF configuration, and also as expected, as the sample size
increases. We can also say that as the cluster size increases, the
estimates get better but we did not have very strong results supporting
that.

\section{FUTURE WORKS}
\label{cap:future}

As show in \autoref{cap:results} results, even with the complete model
specification, the parameter estimates present an excessive variance.
In terms of a traditional GLMM specification \cite{GLMM}, we do not have
a lot more to do. We are already using a smart quasi-Newton algorithm
\cite{PORTpaper}, the most efficient derivatives computation technique
(AD) \cite{peyre}, and an also efficient Laplace approximation routine
\cite{corestats, patrao}, via TMB \cite{TMB}. We could change the
Laplace approximation for an adaptative Gaussian quadrature
\cite{quadrature}, but we do not see any good reason to do that.

There are two possible paths here. We could instead of a conditional
modeling framework (GLMM/latent-effects model), employ a marginal
modeling framework. In this framework, instead of caring about the
specification of a probability distribution to the competing causes
conditioned on the latent effects, we just care about the specification
of a mean and a variance structure. This approach does not have a
likelihood function per se, but the estimation procedure tends to be
easier than with the GLMM one. A marginal modeling framework that can be
used here is the multivariate covariance generalized linear model
(McGLM) \cite{mcglm, rmcglm}. How to exactly model the CIF of clustered
competing risks data in this framework, is something to still be figured
out.

The other path is by the use of a different way of modeling the
dependence structure. Instead of a latent-effects approach, we could use
copulas \cite{copulas,semiparametricSCHEIKE,gcmr,factorcopulas}. How to
do that is something to still be figured out by us, in terms of which
kind (conditional or marginal) and version (Archimedean-, Gauss-,
Maltesian-, \(t\)-, hyperbolic-, zebra-, and elliptical-) of copula to
use, besides the estimability issue.

% END ==================================================================

% ----------------------------------------------------------------------
\setlength{\afterchapskip}{\baselineskip}
% ----------------------------------------------------------------------
\bibliography{references}
% ----------------------------------------------------------------------
\postextual
% ----------------------------------------------------------------------
\begin{apendicesenv}
\partapendices
\addcontentsline{toc}{chapter}{\hspace{2.105cm}APPENDIX}
\renewcommand{\ABNTEXchapterfontsize}{\ABNTEXsectionfont}

\chapter{ANALYTIC GRADIENT OF THE LATENT EFFECTS FOR THE JOINT
         LOG-LIKELIHOOD FUNCTION OF THE MULTINOMIAL GLMM FOR CLUSTERED
         COMPETING RISKS DATA}
\label{cap:appendixA}

The following gradient components are computed by cluster, to be used
e.g., in a Newton optimization. Subject \(i\) at cluster \(j\) and for
competing cause \(k\)

\begin{align*}
  &\frac{\partial}{\partial u_{kj}}
    \log L(\bm{\theta}\mid\bm{y}_{j}, \bm{r}_{j}) =\\
  &y_{kij}\frac{1 +
    \sum_{m \neq k}^{K-1}\exp\{\bm{x}_{mij}\bm{\beta}_{mj} + u_{mj}\}
    }{1 +
    \sum_{n = 1}^{K-1}\exp\{\bm{x}_{nij}\bm{\beta}_{nj} + u_{nj}\}} -
    \left(\sum_{m \neq k}^{K-1} y_{mij}\right)
    \frac{\exp\{\bm{x}_{kij}\bm{\beta}_{kj} + u_{kj}\}
    }{1 +
    \sum_{n = 1}^{K-1}\exp\{\bm{x}_{nij}\bm{\beta}_{nj} + u_{nj}\}}-\\
  &y_{Kij}\frac{1}{1 +
    \sum_{n = 1}^{K-1}\exp\{\bm{x}_{nij}\bm{\beta}_{nj} + u_{nj}\}
    }\Bigg(\\
  &\frac{\exp\{\bm{x}_{kij}\bm{\beta}_{kj} + u_{kj}\}
    \left(1 +
    \sum_{m \neq k}^{K-1}\exp\{\bm{x}_{mij}\bm{\beta}_{mj} + u_{mj}\}
    \right)}{
    1 + \sum_{n = 1}^{K-1}\exp\{\bm{x}_{nij}\bm{\beta}_{nj} + u_{nj}\}
    }\times\\
  &\frac{w_{k}\frac{\delta}{2\delta t - 2t^{2}}
    \phi[w_{k}\text{arctanh}\left(\frac{t-\delta/2}{\delta/2}\right)
    - \bm{x}_{kij}\bm{\gamma}_{kj} - \eta_{kj}
    ]}{1 - w_{n}\frac{\delta}{2\delta t - 2t^{2}}
    \phi[w_{n}\text{arctanh}\left(\frac{t-\delta/2}{\delta/2}\right)
    - \bm{x}_{nij}\bm{\gamma}_{nj} - \eta_{nj}]} -
    \frac{\exp\{\bm{x}_{kij}\bm{\beta}_{kj} + u_{kj}\}}{
    1 + \sum_{n = 1}^{K-1}\exp\{\bm{x}_{nij}\bm{\beta}_{nj} + u_{nj}\}}
    \times\\
  &\frac{
    \sum_{m \neq k}^{K-1}
    w_{m}\frac{\delta}{2\delta t - 2t^{2}}
    \phi[w_{m}\text{arctanh}\left(\frac{t-\delta/2}{\delta/2}\right)
    - \bm{x}_{mij}\bm{\gamma}_{mj} - \eta_{mj}]
    \exp\{\bm{x}_{mij}\bm{\beta}_{mj} + u_{mj}\}}{
    1 - w_{n}\frac{\delta}{2\delta t - 2t^{2}}
    \phi[w_{n}\text{arctanh}\left(\frac{t-\delta/2}{\delta/2}\right)
    - \bm{x}_{nij}\bm{\gamma}_{nj} - \eta_{nj}]}\Bigg) -\\
  &\bm{e_{k}^{\top}Qr_{j}},\\
  %% -------------------------------------------------------------------
  \\
  %% -------------------------------------------------------------------
  &\frac{\partial}{\partial \eta_{kj}}
  \log L(\bm{\theta}\mid\bm{y}_{j}, \bm{r}_{j}) =\\
  &y_{kij} (w_{k}\text{arctanh}\left(\frac{t-\delta/2}{\delta/2}\right)
    - \bm{x}_{kij}\bm{\gamma}_{kj} - \eta_{kj}) -\\
  &y_{Kij}\frac{\exp\{\bm{x}_{kij}\bm{\beta}_{kj} + u_{kj}\}
    }{1 + \sum_{n = 1}^{K-1}\exp\{\bm{x}_{nij}\bm{\beta}_{nj} + u_{nj}\}}
    \times\\
  &\frac{
    w_{k}\frac{\delta}{2\delta t - 2t^{2}}
    (w_{k}\text{arctanh}\left(\frac{t-\delta/2}{\delta/2}\right)
    - \bm{x}_{kij}\bm{\gamma}_{kj} - \eta_{kj})
    \phi[w_{k} \text{arctanh}\left(\frac{t-\delta/2}{\delta/2}\right)
    - \bm{x}_{kij}\bm{\gamma}_{kj} - \eta_{kj}
    ]}{1 -
    \sum_{n = 1}^{K-1}
    \frac{\exp\{\bm{x}_{nij}\bm{\beta}_{nj} + u_{nj}\}}{1 +
    \sum_{n = 1}^{K-1}\exp\{\bm{x}_{nij}\bm{\beta}_{nj} + u_{nj}\}}
    w_{n}\frac{\delta}{2\delta t - 2t^{2}}
    \phi[w_{n}\text{arctanh}\left(\frac{t-\delta/2}{\delta/2}\right)
    - \bm{x}_{nij}\bm{\gamma}_{nj} - \eta_{nj}]} -\\
  &\bm{e_{k}^{\top}Qr_{j}},
\end{align*}
with \(\bm{e_{k}^{\top}}\) begin a vector with \(1\) at the \(k\)-th
position and zero elsewhere.

\chapter{ANALYTIC HESSIAN OF THE LATENT EFFECTS FOR THE JOINT
         LOG-LIKELIHOOD FUNCTION OF THE MULTINOMIAL GLMM FOR CLUSTERED
         COMPETING RISKS DATA}
\label{cap:appendixB}

The following hessian components are computed by cluster, to be used
e.g., in a Newton optimization. Subject \(i\) at cluster \(j\) and for
competing cause \(k\)
\begin{align*}
  &\frac{\partial^{2}}{\partial u_{kj}^{2}}
    \log L(\bm{\theta}\mid\bm{y}_{j}, \bm{r}_{j}) =\\
  &-\frac{\left(\sum_{k = 1}^{K-1} y_{kij}\right)
    \exp\{\bm{x}_{kij}\bm{\beta}_{kj} + u_{kj}\}
    \left(1 +
    \sum_{m \neq k}^{K-1}\exp\{\bm{x}_{mij}\bm{\beta}_{mj} + u_{mj}\}
    \right)}{\left(1 +
    \sum_{n = 1}^{K-1}\exp\{\bm{x}_{nij}\bm{\beta}_{nj} + u_{nj}\}
    \right)^{2}} +\\
  &\frac{y_{Kij}
    \exp\{\bm{x}_{kij} \bm{\beta}_{kj} + u_{kj}\}}{
    1 + \sum_{n = 1}^{K-1}\exp\{\bm{x}_{nij} \bm{\beta}_{nj} + u_{nj}\}
    }\times\\
  &\frac{
    \sum_{m \neq k}^{K-1}w_{m}\frac{\delta}{2\delta t - 2t^{2}}
    \phi[w_{m}\text{arctanh}\left(\frac{t-\delta/2}{\delta/2}\right)
    - \bm{x}_{mij}\bm{\gamma}_{mj} - \eta_{mj}]
    \exp\{\bm{x}_{mij}\bm{\beta}_{mj} + u_{mj}\}}{1 +
    \sum_{n = 1}^{K-1}\exp\{\bm{x}_{nij}\bm{\beta}_{nj} + u_{nj}\}
    (1 - w_{n}\frac{\delta}{2\delta t - 2t^{2}}
    \phi[w_{n}\text{arctanh}\left(\frac{t-\delta/2}{\delta/2}\right)
    - \bm{x}_{nij}\bm{\gamma}_{nj} - \eta_{nj}])} -\\
  &\frac{
    y_{Kij}
    w_{k}\frac{\delta}{2\delta t - 2t^{2}}
    \phi[w_{k}\text{arctanh}\left(\frac{t-\delta/2}{\delta/2}\right)
    - \bm{x}_{kij}\bm{\gamma}_{kj} - \eta_{kj}] }{1 +
    \sum_{n = 1}^{K-1}\exp\{\bm{x}_{nij}\bm{\beta}_{nj} + u_{nj}\}
    }\times\\
  &\frac{\exp\{\bm{x}_{kij}\bm{\beta}_{kj} + u_{kj}\}
    \left(1 +
    \sum_{m \neq k}^{K-1}\exp\{\bm{x}_{mij}\bm{\beta}_{mj} + u_{mj}\}
    \right)}{1 +
    \sum_{n = 1}^{K-1}\exp\{\bm{x}_{nij}\bm{\beta}_{nj} + u_{nj}\}
    (1 - w_{n}\frac{\delta}{2\delta t - 2t^{2}}
    \phi[w_{n}\text{arctanh}\left(\frac{t-\delta/2}{\delta/2}\right)
    - \bm{x}_{nij}\bm{\gamma}_{nj} - \eta_{nj}])} -\\
  &\frac{y_{Kij}\exp\{\bm{x}_{kij}\bm{\beta}_{kj} + u_{kj}\}}{\left(1 +
    \sum_{n = 1}^{K-1}\exp\{\bm{x}_{nij}\bm{\beta}_{nj} + u_{nj}\}
    \right)^{2}}\Bigg(\\
  &\frac{\sum_{m \neq k}^{K-1}
    w_{m}\frac{\delta}{2\delta t - 2t^{2}}
    \phi[w_{m}\text{arctanh}\left(\frac{t-\delta/2}{\delta/2}\right)
    - \bm{x}_{mij}\bm{\gamma}_{mj} - \eta_{mj}]
    \exp\{\bm{x}_{mij}\bm{\beta}_{mj} + u_{mj}\}}{\left(1 +
    \sum_{n = 1}^{K-1}\exp\{\bm{x}_{nij}\bm{\beta}_{nj} + u_{nj}\}
    (1 - w_{n}\frac{\delta}{2\delta t - 2t^{2}}
    \phi[w_{n}\text{arctanh}\left(\frac{t-\delta/2}{\delta/2}\right)
    - \bm{x}_{nij}\bm{\gamma}_{nj} - \eta_{nj}])\right)^{2}}-\\
  &\frac{w_{k}\frac{\delta}{2\delta t - 2t^{2}}
    \phi[w_{k}\text{arctanh}\left(\frac{t-\delta/2}{\delta/2}\right)
    - \bm{x}_{kij}\bm{\gamma}_{kj} - \eta_{kj}]\left(1 +
    \sum_{m \neq k}^{K-1}\exp\{\bm{x}_{mij}\bm{\beta}_{mj} + u_{mj}\}
    \right)}{\left(1 +
    \sum_{n = 1}^{K-1}\exp\{\bm{x}_{nij}\bm{\beta}_{nj} + u_{nj}\}
    (1 - w_{n}\frac{\delta}{2\delta t - 2t^{2}}
    \phi[w_{n}\text{arctanh}\left(\frac{t-\delta/2}{\delta/2}\right)
    - \bm{x}_{nij}\bm{\gamma}_{nj} - \eta_{nj}])\right)^{2}}\Bigg)\\
  &\times\Bigg(\Big(1 +\\
  &\sum_{n = 1}^{K-1}\exp\{\bm{x}_{nij}\bm{\beta}_{nj} + u_{nj}\}
    (1 - w_{n}\frac{\delta}{2\delta t - 2t^{2}}
    \phi[w_{n}\text{arctanh}\left(\frac{t-\delta/2}{\delta/2}\right)
    - \bm{x}_{nij}\bm{\gamma}_{nj} - \eta_{nj}])\Big) +\\
  &\Big(1 +
    \sum_{n = 1}^{K-1}\exp\{\bm{x}_{nij} \bm{\beta}_{nj} + u_{nj}\}
    \Big)\times\\
  &(1 - w_{k}\frac{\delta}{2\delta t - 2t^{2}}
    \phi[w_{k}\text{arctanh}\left(\frac{t-\delta/2}{\delta/2}\right)
    - \bm{x}_{kij}\bm{\gamma}_{kj} - \eta_{kj}])\Bigg)
    - \bm{e_{k}^{\top}Q},
\end{align*}

\begin{align*}
  &\frac{\partial^{2}}{\partial \eta_{kj}^{2}}
    \log L(\bm{\theta}\mid\bm{y}_{j}, \bm{r}_{j}) =\\
  &- y_{kij} - y_{Kij}
    \frac{\exp\{\bm{x}_{kij} \bm{\beta}_{kj} + u_{kj}\}}{1 +
    \sum_{n = 1}^{K-1}\exp\{\bm{x}_{nij} \bm{\beta}_{nj} + u_{nj}\}}\Bigg(\\
  &w_{k}\frac{\delta}{2\delta t - 2t^{2}}
    \phi[w_{k}\text{arctanh}\left(\frac{t-\delta/2}{\delta/2}\right)
    - \bm{x}_{kij}\bm{\gamma}_{kj} - \eta_{kj}]\times\\
  &\frac{\left(
    w_{k} \text{arctanh}\left(\frac{t-\delta/2}{\delta/2}\right)
    - \bm{x}_{kij}\bm{\gamma}_{kj} - \eta_{kj}
    \right)^{2} - 1}{
    1 - \sum_{n = 1}^{K-1}
    \frac{\exp\{\bm{x}_{nij}\bm{\beta}_{nj} + u_{nj}\}}{1 +
    \sum_{n = 1}^{K-1}\exp\{\bm{x}_{nij} \bm{\beta}_{nj} + u_{nj}\}}
    w_{n}\frac{\delta}{2\delta t - 2t^{2}}
    \phi[w_{n}\text{arctanh}\left(\frac{t-\delta/2}{\delta/2}\right)
    - \bm{x}_{nij}\bm{\gamma}_{nj} - \eta_{nj}]} -\\
  &\frac{\left(
    w_{k}\frac{\delta}{2\delta t - 2t^{2}}
    (w_{k}\text{arctanh}\left(\frac{t-\delta/2}{\delta/2}\right)
    - \bm{x}_{kij}\bm{\gamma}_{kj} - \eta_{kj})
    \phi[w_{k}\text{arctanh}\left(\frac{t-\delta/2}{\delta/2}\right)
    - \bm{x}_{kij}\bm{\gamma}_{kj} - \eta_{kj}]\right)^{2}}{\left(1 -
    \sum_{n = 1}^{K-1}
    \frac{\exp\{\bm{x}_{nij} \bm{\beta}_{nj} + u_{nj}\}}{1 +
    \sum_{n = 1}^{K-1}\exp\{\bm{x}_{nij} \bm{\beta}_{nj} + u_{nj}\}}
    w_{n}\frac{\delta}{2\delta t - 2t^{2}}
    \phi[w_{n}\text{arctanh}\left(\frac{t-\delta/2}{\delta/2}\right)
    - \bm{x}_{nij}\bm{\gamma}_{nj} - \eta_{nj}]\right)^{2}}\\
  &\Bigg) - \bm{e_{k}^{\top}Q},
\end{align*}

\begin{align*}
  &\frac{\partial^{2}}{\partial u_{kj} u_{mj}}
    \log L(\bm{\theta}\mid\bm{y}_{j}, \bm{r}_{j}) =\\
  &\left(\sum_{k = 1}^{K-1} y_{kij}\right)
    \frac{
    \exp\{\bm{x}_{kij}\bm{\beta}_{kj} + u_{kj}\}
    \exp\{\bm{x}_{mij}\bm{\beta}_{mj} + u_{mj}\}}{
    \left(1 +
    \sum_{n = 1}^{K-1}\exp\{\bm{x}_{nij} \bm{\beta}_{nj} + u_{nj}\}
    \right)^{2}} +\\
  &\frac{
    y_{Kij}
    \exp\{\bm{x}_{kij}\bm{\beta}_{kj} + u_{kj}\}
    \exp\{\bm{x}_{mij} \bm{\beta}_{mj} + u_{mj}\}}{1 +
    \sum_{n = 1}^{K-1}\exp\{\bm{x}_{nij} \bm{\beta}_{nj} + u_{nj}\}}\Bigg(\\
  &\frac{
    w_{m}\frac{\delta}{2\delta t - 2t^{2}}
    \phi[w_{m}\text{arctanh}\left(\frac{t-\delta/2}{\delta/2}\right)
    - \bm{x}_{mij}\bm{\gamma}_{mj} - \eta_{mj}]}{1 +
    \sum_{n = 1}^{K-1}\exp\{\bm{x}_{nij} \bm{\beta}_{nj} + u_{nj}\}
    (1 - w_{n}\frac{\delta}{2\delta t - 2t^{2}}
    \phi[w_{n}\text{arctanh}\left(\frac{t-\delta/2}{\delta/2}\right)
    - \bm{x}_{nij}\bm{\gamma}_{nj} - \eta_{nj}])} -\\
  &\frac{
    w_{k}\frac{\delta}{2\delta t - 2t^{2}}
    \phi[w_{k}\text{arctanh}\left(\frac{t-\delta/2}{\delta/2}\right)
    - \bm{x}_{kij}\bm{\gamma}_{kj} - \eta_{kj}]}{1 +
    \sum_{n = 1}^{K-1}\exp\{\bm{x}_{nij}\bm{\beta}_{nj} + u_{nj}\}
    (1 - w_{n}\frac{\delta}{2\delta t - 2t^{2}}
    \phi[w_{n}\text{arctanh}\left(\frac{t-\delta/2}{\delta/2}\right)
    - \bm{x}_{nij}\bm{\gamma}_{nj} - \eta_{nj}])}\Bigg) -\\
  &\frac{y_{Kij}}{
    \left(1 +
    \sum_{n = 1}^{K-1}\exp\{\bm{x}_{nij}\bm{\beta}_{nj} + u_{nj}\}
    \right)^{2}}\Bigg(\exp\{\bm{x}_{kij}\bm{\beta}_{kj} + u_{kj}\}\Bigg(\\
  &\frac{
    \sum_{m \neq k}^{K-1}
    w_{m}\frac{\delta}{2\delta t - 2t^{2}}
    \phi[w_{m}\text{arctanh}\left(\frac{t-\delta/2}{\delta/2}\right)
    - \bm{x}_{mij}\bm{\gamma}_{mj} - \eta_{mj}]
    \exp\{\bm{x}_{mij} \bm{\beta}_{mj} + u_{mj}\}}{
    \left(1 + \sum_{n = 1}^{K-1}\exp\{\bm{x}_{nij}\bm{\beta}_{nj} + u_{nj}\}
    (1 - w_{n}\frac{\delta}{2\delta t - 2t^{2}}
    \phi[w_{n}\text{arctanh}\left(\frac{t-\delta/2}{\delta/2}\right)
    - \bm{x}_{nij}\bm{\gamma}_{nj} - \eta_{nj}])\right)^{2}} -\\
  &\frac{
    w_{k}\frac{\delta}{2\delta t - 2t^{2}}
    \phi[w_{k}\text{arctanh}\left(\frac{t-\delta/2}{\delta/2}\right)
    - \bm{x}_{kij}\bm{\gamma}_{kj} - \eta_{kj}]
    \left(1 +
    \sum_{m \neq k}^{K-1}\exp\{\bm{x}_{mij}\bm{\beta}_{mj} + u_{mj}\}
    \right)}{\left(1 +
    \sum_{n = 1}^{K-1}\exp\{\bm{x}_{nij} \bm{\beta}_{nj} + u_{nj}\}
    (1 - w_{n}\frac{\delta}{2\delta t - 2t^{2}}
    \phi[w_{n}\text{arctanh}\left(\frac{t-\delta/2}{\delta/2}\right)
    - \bm{x}_{nij}\bm{\gamma}_{nj} - \eta_{nj}])\right)^{2}}\Bigg)
\end{align*}
\begin{align*}
  &\Bigg)\times\Bigg(\exp\{\bm{x}_{mij}\bm{\beta}_{mj} + u_{mj}\}
    \Big(1 +\\
  &\sum_{n = 1}^{K-1}\exp\{\bm{x}_{nij}\bm{\beta}_{nj} + u_{nj}\}
    (1 - w_{n}\frac{\delta}{2\delta t - 2t^{2}}
    \phi[w_{n}\text{arctanh}\left(\frac{t-\delta/2}{\delta/2}\right)
    - \bm{x}_{nij}\bm{\gamma}_{nj} - \eta_{nj}])\Big) +\\
  &\exp\{\bm{x}_{mij}\bm{\beta}_{mj} + u_{mj}\}
    (1 - w_{m}\frac{\delta}{2\delta t - 2t^{2}}
    \phi[w_{m}\text{arctanh}\left(\frac{t-\delta/2}{\delta/2}\right)
    - \bm{x}_{mij}\bm{\gamma}_{mj} - \eta_{mj}])\Big(1 +\\
  &\sum_{n = 1}^{K-1}\exp\{\bm{x}_{nij}\bm{\beta}_{nj} + u_{nj}\}\Big)
    \Bigg) - \bm{e_{k}^{\top}Q},
\end{align*}

\begin{align*}
  &\frac{\partial^{2}}{\partial \eta_{kj} \eta_{mj}}
    \log L(\bm{\theta}\mid\bm{y}_{j}, \bm{r}_{j}) =\\
  &- y_{Kij}\frac{
    \exp\{\bm{x}_{kij}\bm{\beta}_{kj} + u_{kj}\}}{1 +
    \sum_{n = 1}^{K-1}\exp\{\bm{x}_{nij}\bm{\beta}_{nj} + u_{nj}\}}\times\\
  &\frac{w_{k}\frac{\delta}{2\delta t - 2t^{2}}
    (w_{k}\text{arctanh}\left(\frac{t-\delta/2}{\delta/2}\right)
    - \bm{x}_{kij}\bm{\gamma}_{kj} - \eta_{kj})
    \phi[w_{k}\text{arctanh}\left(\frac{t-\delta/2}{\delta/2}\right)
    - \bm{x}_{kij}\bm{\gamma}_{kj} - \eta_{kj}]}{\left(1 -
    \sum_{n = 1}^{K-1}\frac{\exp\{\bm{x}_{nij}\bm{\beta}_{nj} + u_{nj}\}
    }{1 +
    \sum_{n = 1}^{K-1}\exp\{\bm{x}_{nij}\bm{\beta}_{nj} + u_{nj}\}}
    w_{n}\frac{\delta}{2\delta t - 2t^{2}}
    \phi[w_{n}\text{arctanh}\left(\frac{t-\delta/2}{\delta/2}\right)
    - \bm{x}_{nij}\bm{\gamma}_{nj} - \eta_{nj}]\right)^{2}}\times\\
  &\frac{\exp\{\bm{x}_{mij}\bm{\beta}_{mj} + u_{mj}\}}{1 +
    \sum_{n = 1}^{K-1}\exp\{\bm{x}_{nij}\bm{\beta}_{nj} + u_{nj}\}}
    w_{m}\frac{\delta}{2\delta t - 2t^{2}}
    (w_{m}\text{arctanh}\left(\frac{t-\delta/2}{\delta/2}\right)
    - \bm{x}_{mij}\bm{\gamma}_{mj} - \eta_{mj})\times\\
  &\phi[w_{m}\text{arctanh}\left(\frac{t-\delta/2}{\delta/2}\right)
    - \bm{x}_{mij}\bm{\gamma}_{mj} - \eta_{mj}] - \bm{e_{k}^{\top}Q},
\end{align*}

\begin{align*}
  &\frac{\partial^{2}}{\partial \eta_{kj} u_{kj}}
    \log L(\bm{\theta}\mid\bm{y}_{j}, \bm{r}_{j}) =\\
  &y_{Kij}
    \frac{\exp\{\bm{x}_{kij}\bm{\beta}_{kj} + u_{kj}\}}{1 +
    \sum_{n = 1}^{K-1}\exp\{\bm{x}_{nij}\bm{\beta}_{nj} + u_{nj}\}}\times\\
  &\frac{
    w_{k}\frac{\delta}{2\delta t - 2t^{2}}
    (w_{k}\text{arctanh}\left(\frac{t-\delta/2}{\delta/2}\right)
    - \bm{x}_{kij}\bm{\gamma}_{kj} - \eta_{kj})
    \phi[w_{k}\text{arctanh}\left(\frac{t-\delta/2}{\delta/2}\right)
    - \bm{x}_{kij}\bm{\gamma}_{kj} - \eta_{kj})]}{
    \left(1 -
    \sum_{n = 1}^{K-1}\frac{\exp\{\bm{x}_{nij}\bm{\beta}_{nj} + u_{nj}\}}{
    1 +
    \sum_{n = 1}^{K-1}\exp\{\bm{x}_{nij}\bm{\beta}_{nj} + u_{nj}\}}
    w_{n}\frac{\delta}{2\delta t - 2t^{2}}
    \phi[w_{n}\text{arctanh}\left(\frac{t-\delta/2}{\delta/2}\right)
    - \bm{x}_{nij}\bm{\gamma}_{nj} - \eta_{nj}]\right)^{2}}\times\\
  &\Bigg(
    \sum_{n \neq k}^{K-1}
    \frac{
    \exp\{\bm{x}_{nij}\bm{\beta}_{nj} + u_{nj}\}
    \exp\{\bm{x}_{kij}\bm{\beta}_{kj} + u_{kj}\}}{
    \left(1 +
    \sum_{n = 1}^{K-1}\exp\{\bm{x}_{nij}\bm{\beta}_{nj} + u_{nj}\}
    \right)^{2}}\times\\
  &w_{n}\frac{\delta}{2\delta t - 2t^{2}}
    \phi[w_{n}\text{arctanh}\left(\frac{t-\delta/2}{\delta/2}\right)
    - \bm{x}_{nij}\bm{\gamma}_{nj} - \eta_{nj}] -\\
  &\frac{\exp\{\bm{x}_{kij}\bm{\beta}_{kj} + u_{kj}\}
    \left(
    \left(1 +
    \sum_{n = 1}^{K-1}\exp\{\bm{x}_{nij}\bm{\beta}_{nj} + u_{nj}\}
    \right) - \exp\{\bm{x}_{kij} \bm{\beta}_{kj} + u_{kj}\}
    \right)}{
    \left(1 +
    \sum_{n = 1}^{K-1}\exp\{\bm{x}_{nij}\bm{\beta}_{nj} + u_{nj}\}
    \right)^{2}}\times\\
  &w_{k}\frac{\delta}{2\delta t - 2t^{2}}
    \phi[w_{k}\text{arctanh}\left(\frac{t-\delta/2}{\delta/2}\right)
    - \bm{x}_{kij}\bm{\gamma}_{kj} - \eta_{kj}]\Bigg) -
\end{align*}
\begin{align*}
  &y_{Kij}
    \frac{
    \frac{\exp\{\bm{x}_{kij}\bm{\beta}_{kj} + u_{kj}\}
    \left(
    \left(1 +
    \sum_{n = 1}^{K-1}\exp\{\bm{x}_{nij}\bm{\beta}_{ni} + u_{nj}\}
    \right) - \exp\{\bm{x}_{kij} \bm{\beta}_{kj} + u_{kj}\}
    \right)}{
    \left(1 +
    \sum_{n = 1}^{K-1}\exp\{\bm{x}_{nij}\bm{\beta}_{nj} + u_{nj}\}
    \right)^{2}}}{1 -
    \sum_{n = 1}^{K-1}\frac{\exp\{\bm{x}_{nij}\bm{\beta}_{nj} + u_{nj}\}}{
    1 + \sum_{n = 1}^{K-1}\exp\{\bm{x}_{nij}\bm{\beta}_{nj} + u_{nj}\}}
    w_{n}\frac{\delta}{2\delta t - 2t^{2}}
    \phi[w_{n}\text{arctanh}\left(\frac{t-\delta/2}{\delta/2}\right)
    - \bm{x}_{nij}\bm{\gamma}_{nj} - \eta_{nj}]}\times\\
  &w_{k}\frac{\delta}{2\delta t - 2t^{2}}
    (w_{k}\text{arctanh}\left(\frac{t-\delta/2}{\delta/2}\right)
    - \bm{x}_{kij}\bm{\gamma}_{kj} - \eta_{kj})\times\\
  &\phi[w_{k}\text{arctanh}\left(\frac{t-\delta/2}{\delta/2}\right)
    - \bm{x}_{kij}\bm{\gamma}_{kj} - \eta_{kj}] - \bm{e_{k}^{\top}Q},
\end{align*}

\begin{align*}
  &\frac{\partial^{2}}{\partial \eta_{kj} u_{mj}}
    \log L(\bm{\theta}\mid\bm{y}_{j}, \bm{r}_{j}) =\\
  &y_{Kij}
    \frac{\exp\{\bm{x}_{kij}\bm{\beta}_{kj} + u_{kj}\}
    \exp\{\bm{x}_{mij}\bm{\beta}_{mj} + u_{mj}\}}{
    \left(1 +
    \sum_{n = 1}^{K-1}\exp\{\bm{x}_{nij}\bm{\beta}_{nj} + u_{nj}\}
    \right)^{2}}\times\\
  &\frac{
    w_{k}\frac{\delta}{2\delta t - 2t^{2}}
    (w_{k} \text{arctanh}\left(\frac{t-\delta/2}{\delta/2}\right)
    - \bm{x}_{kij}\bm{\gamma}_{kj} - \eta_{kj})
    \phi[w_{k}\text{arctanh}\left(\frac{t-\delta/2}{\delta/2}\right)
    - \bm{x}_{kij}\bm{\gamma}_{kj} - \eta_{kj})]}{1 - \sum_{n = 1}^{K-1}
    \frac{\exp\{\bm{x}_{nij}\bm{\beta}_{nj} + u_{nj}\}}{1 +
    \sum_{n = 1}^{K-1}\exp\{\bm{x}_{nij}\bm{\beta}_{nj} + u_{nj}\}}
    w_{n}\frac{\delta}{2\delta t - 2t^{2}}
    \phi[w_{n}\text{arctanh}\left(\frac{t-\delta/2}{\delta/2}\right)
    - \bm{x}_{nij}\bm{\gamma}_{nj} - \eta_{nj}]} +\\
  &y_{Kij}
    \frac{\exp\{\bm{x}_{kij}\bm{\beta}_{kj} + u_{kj}\}}{1 +
    \sum_{n = 1}^{K-1}\exp\{\bm{x}_{nij}\bm{\beta}_{nj} + u_{nj}\}}\times\\
  &\frac{
    w_{k}\frac{\delta}{2\delta t - 2t^{2}}
    (w_{k}\text{arctanh}\left(\frac{t-\delta/2}{\delta/2}\right)
    - \bm{x}_{kij}\bm{\gamma}_{kj} - \eta_{kj})
    \phi[w_{k}\text{arctanh}\left(\frac{t-\delta/2}{\delta/2}\right)
    - \bm{x}_{kij}\bm{\gamma}_{kj} - \eta_{kj})]}{
    \left(1 - \sum_{n = 1}^{K-1}
    \frac{\exp\{\bm{x}_{nij}\bm{\beta}_{nj} + u_{nj}\}}{1 +
    \sum_{n = 1}^{K-1}\exp\{\bm{x}_{nij}\bm{\beta}_{nj} + u_{nj}\}}
    w_{n}\frac{\delta}{2\delta t - 2t^{2}}
    \phi[w_{n}\text{arctanh}\left(\frac{t-\delta/2}{\delta/2}\right)
    - \bm{x}_{nij}\bm{\gamma}_{nj} - \eta_{nj}]\right)^{2}}\times\\
  &\Bigg(
    \sum_{n \neq m}^{K-1}\frac{
    \exp\{\bm{x}_{nij}\bm{\beta}_{nj} + u_{nj}\}
    \exp\{\bm{x}_{mij}\bm{\beta}_{mj} + u_{mj}\}}{
    \left(1 +
    \sum_{n = 1}^{K-1}\exp\{\bm{x}_{nij}\bm{\beta}_{nj} + u_{nj}\}
    \right)^{2}}\times\\
  &w_{n}\frac{\delta}{2\delta t - 2t^{2}}
    \phi[w_{n}\text{arctanh}\left(\frac{t-\delta/2}{\delta/2}\right)
    - \bm{x}_{nij}\bm{\gamma}_{nj} - \eta_{nj}] -\\
  &\frac{\exp\{\bm{x}_{mij}\bm{\beta}_{mj} + u_{mj}\}
    \left(
    \left(1 +
    \sum_{n = 1}^{K-1}\exp\{\bm{x}_{nij}\bm{\beta}_{nj} + u_{nj}\}
    \right) - \exp\{\bm{x}_{mij} \bm{\beta}_{mj} + u_{mj}\}
    \right)}{
    \left(1 + \sum_{n = 1}^{K-1}\exp\{\bm{x}_{nij}\bm{\beta}_{nj} + u_{nj}\}
    \right)^{2}}\times\\
  &w_{m}\frac{\delta}{2\delta t - 2t^{2}}
    \phi[w_{m}\text{arctanh}\left(\frac{t-\delta/2}{\delta/2}\right)
    - \bm{x}_{mij}\bm{\gamma}_{mj} - \eta_{mj}]\Bigg) - \bm{e_{k}^{\top}Q},
\end{align*}
with \(\bm{e_{k}^{\top}}\) begin a vector with \(1\) at the \(k\)-th
position and zero elsewhere.

\chapter{\texttt{R} CODE TO SIMULATE FROM A \(\text{multiGLMM}\) WITH
         TWO COMPETING CAUSES AND CLUSTERS OF SIZE TWO. FOR MORE
         INFORMATION CHECK SECTION \ref{cap:simu}}
\label{cap:appendixC}

\lstinputlisting[firstline=97,lastline=153]{datasets.Rmd}
\vspace{-0.5cm}
\begin{center}
 \begin{footnotesize}
  SOURCE: The author (2021).
 \end{footnotesize}
\end{center}

\end{apendicesenv}
% ----------------------------------------------------------------------
% \begin{anexosenv}
% \partanexos
% \addcontentsline{toc}{chapter}{\hspace{2.105cm}ANNEX}
% \renewcommand{\ABNTEXchapterfontsize}{\ABNTEXsectionfont}
% \end{anexosenv}
%-----------------------------------------------------------------------
\phantompart
\printindex
%-----------------------------------------------------------------------
\end{document}
% END ==================================================================

% ----------------------------------------------------------------------
% pacote para fazer o checkmark
\usepackage{pifont} % http://ctan.org/pkg/pifont
\newcommand{\cmark}{\ding{51}}%
\newcommand{\xmark}{\ding{55}}%
% ----------------------------------------------------------------------
\usepackage{amsmath}
\usepackage{blkarray}
\usepackage{amsfonts}
\usepackage{amssymb}
\usepackage{pdfpages}
% \usepackage{times}
% \usepackage{helvet}
% \renewcommand{\familydefault}{\sfdefault}
% ----------------------------------------------------------------------
\NewDocumentCommand\cc{+u{\cc}}{\ignorespaces}
% ----------------------------------------------------------------------
% controle do espaçamento entre um parágrafo e outro:
\setlength{\parskip}{0.2cm} % tente também \onelineskip
% ----------------------------------------------------------------------
\titulo{MODELING THE CUMULATIVE INCIDENCE FUNCTION OF CLUSTERED
  COMPETING RISKS DATA: A MULTINOMIAL GLMM APPROACH}
\autor{HENRIQUE APARECIDO LAUREANO}
\data{2021}
\instituicao{FEDERAL UNIVERSITY OF PARANÁ}
\orientador{Prof. PhD Wagner Hugo Bonat}
\coorientador{Prof. PhD Paulo Justiniano Ribeiro Jr}
\tipotrabalho{Dissertação (mestrado)}
\preambulo{\small{Thesis presented to the Graduate Program of Numerical
    Methods in Engineering, Concentration Area in Mathematical
    Programming: Statistical Methods Applied in Engineering, Federal
    University of Paran\'{a}, as part of the requirements to the
    obtention of the Master's Degree in Sciences.}}
% ----------------------------------------------------------------------
% informações do PDF
\makeatletter
\hypersetup{
  % pagebackref=true,
  pdftitle={\@title},
  pdfauthor={\@author},
  pdfsubject={\imprimirpreambulo},
  % pdfkeywords = {}{}{}{},
  colorlinks=true, % false: boxed links; true: colored links
  linkcolor=blue, % color of internal links
  citecolor=blue, % color of links to bibliography
  filecolor=magenta, % color of file links
  urlcolor=blue,
  bookmarksdepth=4
}
\addto\captionsenglish{
  % adjusts names from abnTeX2
  \renewcommand{\folhaderostoname}{Title page}
  \renewcommand{\epigraphname}{Epigraph}
  \renewcommand{\dedicatorianame}{Dedication}
  \renewcommand{\errataname}{Errata sheet}
  \renewcommand{\agradecimentosname}{Acknowledgements}
  \renewcommand{\anexoname}{ANNEX}
  \renewcommand{\anexosname}{Annex}
  \renewcommand{\apendicename}{APPENDIX}
  \renewcommand{\apendicesname}{Appendix}
  \renewcommand{\orientadorname}{Supervisor:}
  \renewcommand{\coorientadorname}{Co-supervisor:}
  \renewcommand{\folhadeaprovacaoname}{Approval}
  \renewcommand{\resumoname}{Abstract}
  \renewcommand{\listadesiglasname}{List of abbreviations and acronyms}
  \renewcommand{\listadesimbolosname}{List of symbols}
  \renewcommand{\fontename}{Source}
  \renewcommand{\notaname}{Note}
  % adjusts names used by \autoref
  \renewcommand{\pageautorefname}{page}
  \renewcommand{\chapterautorefname}{Chapter}
  \renewcommand{\sectionautorefname}{Section}
  \renewcommand{\subsectionautorefname}{subsection}
  \renewcommand{\subsubsectionautorefname}{subsubsection}
  \renewcommand{\paragraphautorefname}{subsubsubsection}
}
\makeatother
% ----------------------------------------------------------------------
\graphicspath{{figures/}}
% ----------------------------------------------------------------------
\begin{document}
\selectlanguage{english}
% adequando o uppercase titulo dos elementos nas suas respectivas
% legendas
\renewcommand{\tablename}{TABLE }
\renewcommand{\figurename}{FIGURE }
% ----------------------------------------------------------------------
\frenchspacing % retira espaço extra obsoleto entre as frases
% ----------------------------------------------------------------------
% capa
\tikz[remember picture,overlay] \node[opacity=1,inner sep=0pt] at
(current page.center){
  \includegraphics[width=\paperwidth,
  height=\paperheight]{Figuras/ufpr_bg}};
% ----------------------------------------------------------------------
\imprimircapa
% ----------------------------------------------------------------------
% folha de rosto
\imprimirfolhaderosto
% ----------------------------------------------------------------------
% \begin{dedicatoria}
%   \vspace*{\fill}
%   ...
%   \vspace*{\fill}
% \end{dedicatoria}
% ----------------------------------------------------------------------
% ficha catalográfica

% \begin{fichacatalografica}
%   \includepdf{ficha.pdf}
% \end{fichacatalografica}
% ----------------------------------------------------------------------
% inserir folha de aprovação
% \begin{folhadeaprovacao}
%   \includepdf{termo.pdf}
% \end{folhadeaprovacao}
\begin{folhadeaprovacao}
 \begin{center}
   {\ABNTEXchapterfont\large\imprimirautor}

   \vspace*{\fill}\vspace*{\fill}
   \begin{center}
     \ABNTEXchapterfont\bfseries\large\imprimirtitulo
   \end{center}
   \vspace*{\fill}

    \hspace{.45\textwidth}
    \begin{minipage}{.5\textwidth}
       \imprimirpreambulo
    \end{minipage}
   \vspace*{\fill}
 \end{center}

 Master thesis approved. XXX XX, 2021.

  \assinatura{\textbf{\imprimirorientador}\\ Supervisor}
  \assinatura{\textbf{Prof. PhD Paulo Justiniano Ribeiro Jr}\\
    Co-supervisor}
  \assinatura{\textbf{Prof. PhD \(\dots\)}\\
    Internal Examinator - PPGMNE}
  \assinatura{\textbf{Prof. PhD \(\dots\)}\\
    Internal Examinator - PPGMNE}
  \assinatura{\textbf{Prof. PhD \(\dots\)}\\External Examiner - }

  \begin{center}
   \vspace*{0.5cm}
   {\large CURITIBA}
   \par
   {\large\imprimirdata}
   \vspace*{1cm}
 \end{center}

\end{folhadeaprovacao}
% ----------------------------------------------------------------------
\begin{dedicatoria}
  \vspace*{22.7cm}
  \begin{flushright}
    \begin{minipage}[H]{4.5cm}
      {To Celita and Olivio}
    \end{minipage}
  \end{flushright}
\end{dedicatoria}
% ----------------------------------------------------------------------
\begin{agradecimentos}
 As Moro once said, I am thankful for everything and everyone.

 We do not go up by ourselves. I am grateful to everyone that help me in
 something or was patient with me in these last two years I am
 especially grateful to LEG's core for inspiring and indirectly
 motivating me since my bachelor's. I am even more especially grateful
 to my advisor, Professor Wagner Hugo Bonat.
\end{agradecimentos}

\begin{epigrafe}
  \vspace*{\fill}
  \begin{flushright}
    \textit{"It's not supposed to be easy."\\
             (Gregg Popovich)}
              % on Sao Antonio Spurs \(\times\) Oklahoma City Thunder,
              % first game of the 2012 Western Conference Finais
  \end{flushright}
\end{epigrafe}
% ----------------------------------------------------------------------
\newpage
\setlength{\absparsep}{18pt} % ajusta o espaçamento dos parágrafos do
                             % resumo
\setlength{\abstitleskip}{1cm} % adiciona mais um cm após o 'titulo' do
                               % resumo para ficar com 2cm
\begin{resumo}[]
  \vspace{-2cm}
  \begin{center}
    \bfseries{\large{\textsf{ABSTRACT}}}
  \end{center}
  \vspace{0.3cm}
  
  Clustered competing risks data is a special case of failure time
  data. Besides the cluster structure which implies a latent
  within-cluster dependence between its elements, this kind of data is
  characterized by 1) multiple causes/variables competing to be the one
  responsible for the occurrence of an event, a failure; and 2)
  censorship, when the event of interest happens or not for none of the
  competing causes, in the study period. To handle this type of data, we
  propose a generalized linear mixed model (GLMM) i.e., a latent-effects
  framework, instead of a usual survival model. In survival analysis,
  the modeling is usually done by means of the hazard rate, and the
  within-cluster dependence accommodation ends by generating a
  complicated likelihood function, sometimes intractable. We, on the
  other hand, model the clustered competing causes in the probability
  scale, in terms of the cumulative incidence function (CIF) of each
  competing cause. In our framework, we suppose a multinomial
  probability distribution for the competing causes and censorship,
  conditioned on the latent effects. The latent effects are accommodated
  via a multivariate Gaussian distribution and are modeled by the
  parameters of its covariance matrix. The probability distributions are
  connected via CIF, modeled here following \citeonline{SCHEIKE}
  specification, based on its decomposition as the product of an
  instantaneous risk level function with a trajectory time level
  function. The latent effects are inserted in those level functions. To
  make the model parameters estimation the most efficient as possible,
  we use the template model builder (TMB) \cite{TMB}. With this
  \texttt{R} \cite{R21} package, we have 1) the log-likelihood function
  written in \texttt{C++}; 2) access to efficient linear algebra
  libraries; 3) efficient Laplace approximation implementation for the
  latent-effects; and 4) an automatic differentiation (AD) routine, the
  state-of-the-art in derivatives computation. To check the estimability
  of our model a large simulation study is performed, based on different
  latent structure formulations, with the aim to verify which one is
  most adequate to real scenarios. The model presents to be of difficult
  estimation, with our results converging to a latent structure where
  the risk and trajectory time levels are correlated. In scenarios with
  high CIF the model exhibits the better results, but still with an
  excessive variance, showing that improvements are necessary.

  \vfill
  \textbf{Keywords}: Clustered competing risks.
                     Within-cluster dependence.
                     Multinomial generalized linear mixed model (GLMM).
                     TMB: Template Model Builder.
                     Laplace approximation.
                     Automatic differentiation (AD).
\end{resumo}
% ----------------------------------------------------------------------
\newpage
\setlength{\absparsep}{18pt} % ajusta o espaçamento dos parágrafos do
                             % resumo
\setlength{\abstitleskip}{1cm} % adiciona mais um cm após o 'titulo' do
                               % resumo para ficar com 2cm
\begin{resumo}[]
  \begin{otherlanguage*}{brazil}
    \vspace{-2cm}
    \begin{center}
      \bfseries{\large{\textsf{RESUMO}}}
    \end{center}
    \vspace{0.3cm}

    Dados de riscos competitivos agrupados s\~{a}o um caso especial de
    dados de tempo de falha. Al\'{e}m da estrutura de grupo que implica
    uma depend\^{e}ncia latente intra-grupo entre seus elementos, esse
    tipo de dado \'{e} caracterizado por 1) m\'{u}ltiplas
    causas/vari\'{a}veis ​​competindo para ser a respons\'{a}vel pela
    ocorr\^{e}ncia de um evento, uma falha; e 2) censura, quando o
    evento de interesse ocorre ou n\~{a}o por nenhuma das causas
    concorrentes, no per\'{i}odo de estudo. Para lidar com este tipo de
    dado, propomos um modelo linear generalizado misto (GLMM), ou seja,
    um modelo de efeitos latentes/aleat\'{o}rios, em vez de um modelo de
    sobreviv\^{e}ncia usual. Em an\'{a}lise de sobreviv\^{e}ncia, a
    modelagem \'{e} usualmente feita por meio da taxa de risco, e a
    acomodação da depend\^{e}ncia intra-grupo acaba por gerar uma
    complicada fun\c{c}\~{a}o de verossimilhan\c{c}a, \`{a}s vezes
    intrat\'{a}vel. N\'{o}s, por outro lado, modelamos as causas
    competidoras agrupadas na escala da probabilidada, por meio da
    fun\c{c}\~{a}o de incid\^{e}ncia acumulada (CIF, em ingl\^{e}s) de
    cada causa competidora. Em nossa modelagem, supomos uma
    distribui\c{c}\~{a}o de probabilidade multinomial para as causas
    competidoras e censura, condicionado aos efeitos latentes. Os
    efeitos latentes são acomodados por meio de uma distribui\c{c}\~{a}o
    Gaussiana multivariada e s\~{a}o modelados via os par\^{a}metros de
    sua matriz de covari\^{a}ncia. As distribui\c{c}\~{o}es de
    probabilidade s\~{a}o conectadas por meio da CIF, modeladas aqui
    seguindo a especifica\c{c}\~{a}o em \citeonline{SCHEIKE}, com base
    em sua decomposi\c{c}\~{a}o como o produto de uma fun\c{c}\~{a}o de
    n\'{i}vel de risco instant\^{a}neo com uma fun\c{c}\~{a}o de
    n\'{i}vel de tempo de trajet\'{o}ria. Os efeitos latentes são
    inseridos nestas fun\c{c}\~{o}es. Para tornar a estimativa dos
    par\^{a}metros do modelo o mais eficiente poss\'{i}vel, usamos o
    template model builder (TMB) \cite{TMB}. Com este pacote \texttt{R}
    \cite{R21}, temos 1) a fun\c{c}\~{a}o de log-verossimilhan\c{c}a
    escrita em \texttt{C++}; 2) acesso a eficientes bibliotecas de
    \'{a}lgebra linear; 3) implementa\c{c}\~{a}o eficiente da
    aproxima\c{c}\~{a}o de Laplace para os efeitos latentes; e 4) uma
    rotina computacional de diferencia\c{c}\~{a}o autom\'{a}tica, o
    estado da arte em computa\c{c}\~{a}o de derivadas. Para verificar a
    estimabilidade do nosso modelo \'{e} realizado um amplo estudo de
    simula\c{c}\~{a}o, baseado em diferentes formula\c{c}\~{o}es de
    estruturas latentes, com o objetivo de verificar qual delas \'{e} a
    mais adequada a um cen\'{a}rio real. O modelo se apresenta de
    dif\'{i}cil estimati\c{c}\~{a}o, com nossos resultados convergindo
    para uma estrutura latente onde os n\'{i}veis de risco e de
    trajet\'{o}ria est\~{a}o correlacionados. Em cen\'{a}rios de CIF
    alta o modelo apresenta os melhores resultados, mas ainda com uma
    excessiva variablidade, mostrando que melhorias s\~{a}o
    necess\'{a}rias.
    
    \vfill
    \textbf{Palavras-chave}: Riscos competitivos agrupados.
                             Depend\^{e}ncia intra-cluster.
                             Modelo linear generalizado misto
                             multinomial (MLGM).
                             TMB: Template Model Builder.
                             Aproxima\c{c}\~{a}o de Laplace.
                             Diferencia\c{c}\~{a}o autom\'{a}tica.
  \end{otherlanguage*}
\end{resumo}
% ----------------------------------------------------------------------
\pdfbookmark[0]{\listfigurename}{lof}
\listoffigures*
\cleardoublepage
% ----------------------------------------------------------------------
%% \pdfbookmark[0]{\listtablename}{lot}
%% \listoftables*
%% \cleardoublepage
% ----------------------------------------------------------------------
\makeatletter
\renewcommand\numberline[1]{
	\leftskip -0.7em
	\rightskip 1.6em
	\parfillskip -\rightskip
	\parindent 0em
	\@tempdima 2.0em
	\vspace{0em}
  \advance\leftskip \@tempdima \null\nobreak\hskip -\leftskip
	ALGORITHM \normalfont #1 ~~ }
\makeatother
% ----------------------------------------------------------------------
\pdfbookmark[0]{\listalgorithmname}{loa}
\listofalgorithms
\cleardoublepage
% ----------------------------------------------------------------------
\makeatletter
\def\numberline#1{\hb@xt@\@tempdima{#1\hfil}}
\makeatother
% ----------------------------------------------------------------------
% \begin{siglas}
% \item[Fig.] Area of the $i^{th}$ component
% \item[456] Isto é um número
% \item[123] Isto é outro número
% \item[lauro cesar] este é o meu nome
% \end{siglas}
% ----------------------------------------------------------------------
% \begin{simbolos}
% \item[\(\mathbb{E}(\cdot)\)] The mathematical expectation of a random
%   variable \(\cdot\)
% \end{simbolos}
% ----------------------------------------------------------------------
\pdfbookmark[0]{\contentsname}{toc}
\tableofcontents*
\cleardoublepage
% ----------------------------------------------------------------------
\makepagestyle{abntheadings}
\makeevenhead{abntheadings}{\ABNTEXfontereduzida\thepage}{}{}
\makeoddhead{abntheadings}{}{}{\ABNTEXfontereduzida\thepage}
\makeheadrule{abntheadings}{\textwidth}{0in}
% ----------------------------------------------------------------------
\textual
% ----------------------------------------------------------------------
\chapter{Introduction}
\label{cap:intro}
Consider a cluster of random variables. Each random variable represents
the time until some event occurs. The random variables that compose the
cluster are assumed to be correlated, i.e., the method for the analysis
is flexible enough to be able to verify if this happens to that data. In
this thesis, the cluster is a family - more precisely, a part of a
family, i.e., a pair of twins; the random variables are the time until
the occurrence (or not) of an event in each twin; and the event under
focus is the occurrence of cancer.

When we deal with random variables, in the context of a statistical
model - a response of interest; that represents the time until some
event occurs, such events are generically referred to as
\textit{failures}. From this, we get the name of the field of study:
failure time data~\cite{kalb&prentice}. These events may not necessarily
consist of a failure, however, the major areas of application of the
methods that will be discussed here are biomedical studies and
industrial life testing. Thus, this name sounds appropriate.

Independent of the area of application the methods are the same, but the
name of what you're doing is different. In industrial life testing
applications, you perform what is called a reliability analysis; in
biomedical studies, you perform what is called a survival analysis. In
this thesis, we'll maintain our focus on the latter.

Generally speaking, the term survival analysis is applied when we deal
with a univariate set, i.e., we have one response variable. As an
example,

When we want to study the possible relation between the components (seen
as random variables) of a dataset in an associative manner, the perhaps
most common scientific way of doing that is by fitting a linear model.
More generally, a generalized linear model (GLM). To fit a GLM to a
dataset we basically need to select a probability distribution for the
so-called response variable \(Y_{i}\) (or dependent variable); and how
we'll approach the possible relation between the response and the other
variables \(\mathbf{X}_{i}\) (independent variables or covariates).

Generalized linear models~\cite{GLM72} allow for response distribution
other than normal, which configures the so-called linear models. In a
GLM we model the mean, \(\mu_{i}\), of the response random variable, and
it has the basic structure
\[
 g(\mu_{i}) = \mathbf{X}_{i} \bm{\beta},
\]
where \(\mu_{i} \equiv \mathbb{E}(Y_{i})\), \(g\) is a smooth monotonic
``link function'', \(\mathbf{X}_{i}\) is the \(i^\text{th}\) row of a
model matrix \(\mathbf{X}\), and \(\bm{\beta}\) is a vector of unknown
parameters. In addition, a GLM usually makes the distributional
assumption that the \(Y_{i}\) are independent and
\[
  Y_{i} \sim \text{some exponential family distribution}.
\]

The \textit{exponential family} of distributions includes many
distributions that are useful for practical modelling, such as the
Poisson (for counting data), binomial (dichotomic data), gamma
(continuous but positive) and normal (continuous data) distributions.
The comprehensive reference for GLMs is~\citeonline{GLM89}.

However broad it may be the range of models that can be constructed
thanks to the generality of the GLM framework, is plausible that for
some applications a specific modeling framework come up. This is the
case when the response variable is the time until some event occurs.
Such events are generically referred to as \textit{failures}, and its
major areas of use are biomedical studies and industrial life testing.
In the latter the commonly used name is reliability and in the former is
survival. In this thesis, we focus on the survival side.

A survival model consists

\section{GOALS}

\subsection{General goals}

Propor um modelo de regressão para análise de variáveis respostas
limitadas multivariada.

\subsection{Specific goals}

\begin{enumerate}
\item Estudar o desempenho do algoritmo NORTA (\emph{NORmal To
    Anything}) para simular variáveis aleatórias beta correlacionadas.

\item Especificar o modelo usando suposições de primeiro e segundo
  momentos.

\item Usar as funções de estimação quase-score e Pearson para estimar os
  parâmetros de regressão e dispersão, respectivamente.

\item Delinear estudos de simulação para explorar a flexibilidade do
  modelo para lidar com dados limitados em estudos longitudinais, além
  de checar propriedades dos estimadores em estudos com múltiplas
  respostas correlacionadas.

\item Adaptar técnicas de diagnóstico para o modelo proposto, como
  DFFITS, DFBETAS, distância de Cook e o gráfico de probabilidade
  meio-normal com envelope simulado.

\item Aplicar o modelo proposto em dois conjuntos de dados.
\end{enumerate}

\section{JUSTIFICATION}

\section{LIMITATION}

Este trabalho se restringe a propor um novo modelo de regressão para
análise de variáveis respostas limitadas multivariada. Para motivar o
novo modelo, serão apresentadas aplicações em dois conjuntos de dados,
que não são facilmente manipulados pelos métodos estatísticos
existentes. Portanto, testes de hipóteses e de comparações múltiplas
multivariados não serão desenvolvidos no decorrer deste trabalho.

\section{THESIS ORGANIZATION}

Esta dissertação contém seis capítulos incluindo esta introdução.
O~\autoref{cap:aplicacoes} descreve os dois conjuntos de dados que serão
usados como exemplos de aplicação no novo modelo.
O~\autoref{cap:fundamentacaoteorica} apresenta a revisão bibliográfica
que motivou este trabalho, introduz o modelo de regressão beta
(univariado), apresenta o algoritmo NORTA (\textit{NORmal To Anything})
usado nos estudos de simulação e discute brevemente as medidas de
bondade de ajuste usadas no trabalho. O~\autoref{cap:multivariatemodel}
propõe o modelo de regressão quase-beta multivariado, apresenta o método
usado para estimação e inferência e adapta técnicas de diagnóstico.
No~\autoref{cap:resultados} são apresentados os resultados de três
estudos de simulação, além da análise dos dados apresentados
no~\autoref{cap:aplicacoes}. Finalmente, o~\autoref{cap:considefinais}
discute as principais contribuições desta dissertação, além de
apresentar as conclusões seguidas por sugestões para futuros trabalhos.

% END ==================================================================
% ----------------------------------------------------------------------
\chapter{Generalized linear mixed models: formulation, optimization, and
  implementation}
\label{cap:methods}
This chapter presents a systematic review of the main theoretical
aspects involved in the construction, estimation and implementation of a
generalized linear mixed model (GLMM). We start in \autoref{cap:joint}
with the model construction framework, concluding with the so-called
joint likelihood function. \autoref{cap:laplace} address the integration
of that joint likelihood, a necessary and fundamental step in our
modeling approach, resulting in a marginal likelihood function.
\autoref{cap:opt} discusses available alternatives for the optimization
of the marginal distribution obtained through that integration.
\autoref{cap:ad} talks about automatic differentiation, the most
efficent manner of computing derivatives, and a key point for us. Last
but not least, in \autoref{cap:tmb} we present the computational tool
used to peform all the discussed procedure, the TMB: Template Model
Builder. A very exciting \texttt{R} \cite{R18} package developed
by~\citeonline{TMB}.

\section{JOINT LIKELIHOOD}
\label{cap:joint}

A standard, univariate, GLMM models an \(n\)-vector of exponential
family random variables, \(\mathbf{Y}\), with conditional expected
value, \(\bm{\mu} \equiv \mathbb{E}(\mathbf{Y} \mid \mathbf{X},
\mathbf{u})\), via a linear predictor equation expressed by
\begin{equation}
  g(\bm{\mu}) = \mathbf{X} \bm{\beta} + \mathbf{Zu}, \quad
  \mathbf{u} \sim \mathcal{N}(\mathbf{0}, \bm{\Sigma}).
  \label{eq:gmu}
\end{equation}

That is, a GLMM is a generalized linear model (GLM) in which the linear
predictor depends on some Gaussian latent effects, \(\mathbf{u}\), times
a latent effects model matrix \(\mathbf{Z}\). The idea embedded in that
matrix is exemplified in~\autoref{eq:Zu}. Suppose, e.g., three
individuals and each has two measures. This configures a simple repeated
measures context, the focus of this work. It is reasonable to admit that
each individual has a particular latent effect value. Consequently,
\begin{equation}
  \mathbf{Zu} = \begin{bmatrix}
                 1 & 0 & 0\\
                 1 & 0 & 0\\
                 0 & 1 & 0\\
                 0 & 1 & 0\\
                 0 & 0 & 1\\
                 0 & 0 & 1\\
                \end{bmatrix} \begin{bmatrix}
                               u_{1}\\
                               u_{2}\\
                               u_{3}\\
                              \end{bmatrix} = \begin{bmatrix}
                                               u_{1}\\
                                               u_{1}\\
                                               u_{2}\\
                                               u_{2}\\
                                               u_{3}\\
                                               u_{3}\\
                                              \end{bmatrix},
  \label{eq:Zu}
\end{equation}
where \(\mathbf{u}^{\top} = [u_{1}~u_{2}~u_{3}]\) and \(\mathbf{Z}\) has
the role of projecting the values of \(\mathbf{u}\) to match the number
of measures.

We model this mean structure into a combination of probability
distributions. It's a combination since we have to assume probabilistic
structures for the observed and non-observed, latent, data. For each
observed variable \(y_{ij}\), we have a probability distribution of the
exponential family, denoted by \(f(y_{ij} \mid \mathbf{u}_{i},
\bm{\theta})\). For the non-observed latent effect we have, generally, a
(multivariate) Gaussian distribution, denoted by \(f(\mathbf{u}_{i} \mid
\bm{\Sigma})\). For each individual or unity under study, \(i\), and to
each measure, \(j\), we have the product of these probability densities,
a likelihood contribution.

We want to estimate the parameter vector \(\bm{\theta} =
[\bm{\beta}~\bm{\Sigma}]^{\top}\) of~\autoref{eq:gmu}. Besides the role
of emphasizing the fact that \(\bm{\mu}\) is a function of
\(\bm{\theta}\) and that we want to estimate \(\bm{\theta}\), the
likelihood function ties the probability densities. i.e., the likelihood
is the product of the probability densities product for each subject.
Since the \(Y_{i}\) are mutually independent, the likelihood of
\(\bm{\theta}\) is
\begin{equation}
  L(\bm{\theta} \mid \mathbf{y}, \mathbf{u}) =
  \prod_{i=1}^{n}~\prod_{j=1}^{n_{i}}~
  f(y_{ij} \mid \mathbf{u}_{i}, \bm{\beta}, \bm{\Sigma})~
  f(\mathbf{u}_{i} \mid \bm{\Sigma}).
  \label{eq:joint}
\end{equation}

From standard probability theory is easy to see that in the right-hand
side (r.h.s.) we have a joint density, consequently,~\autoref{eq:joint}
represents what is called a joint likelihood function. What makes
working with this joint likelihood problematic is that we don't have all
the information necessary to just optimize it and get the desired
parameter estimates. The latent effect \(\mathbf{u}\) is
\textit{latent}, we don't observe it. To handle with this we basically
have two available paths.

\section{LAPLACE APPROXIMATION}
\label{cap:laplace}

To deal with the joint likelihood in~\autoref{eq:joint}~we have a choice
to make. Be or not to be Bayesian. Each choice has its own difficulties,
advantages, and characteristics.

The Bayesian path assumes that all \(\bm{\theta}\) components are random
variables. With all parameters being treated as random variables, and
since we don't observe them, what the Bayesian framework does is try to
compute the mode of each ``parameter'' marginal distribution via a
sampling algorithm, called MCMC: Markov Chain Monte Carlo~\cite{MCMC,
  Diaconis}. The advantage is that we can reach an MCMC algorithm to
basically any statistical model, the disadvantages are that this
approach is very time consuming and we have to propose prior
distributions to each ``parameter''. These prior proposals are not
always easy to make, and the resulting marginal distributions can be
very depending on it.

A Bayesian approach can be applied in basically any context. However, in
complex scenarios they can be the only available method to maximize the
likelihood. This isn't the case here. We have a joint density where one
of the random variables isn't observed, but we're not interested in it,
only in the variance parameters inherent in it. Again, from standard
probability theory, if we have a joint density we can just integrate out
the undesired variable. This results in
\begin{equation}
  \begin{aligned}
    L(\bm{\theta} \mid \mathbf{y}) &=
    \prod_{i=1}^{n}~\int_{\mathcal{R}^{\mathbf{u}_{i}}}
    \left\{
      \prod_{j=1}^{n_{i}}~
      f(y_{ij} \mid \mathbf{u}_{i}, \bm{\beta}, \bm{\Sigma})~
      f(\mathbf{u}_{i} \mid \bm{\Sigma})
    \right\} \text{d} \mathbf{u}_{i}\\
    &= \prod_{i=1}^{n}~\int_{\mathcal{R}^{\mathbf{u}_{i}}}~
    f(\mathbf{y}_{i}, \mathbf{u}_{i} \mid \bm{\theta})~
    \text{d} \mathbf{u}_{i},
    \label{eq:generalmarginal}
  \end{aligned}
\end{equation}
a marginal density that keeps the parameters of the integrated variable.

If the response distribution in our mixed model is Gaussian, is
analytically tractable to integrate \(\mathbf{u}\) out of the joint
density. Consequently, is possible to evaluate the likelihood exactly.
This is the case of linear mixed models and the main difference to the
GLMMs. When the response distribution isn't Gaussian, generally, isn't
anymore analytically tractable to integrate out the latent effect. So
what do we do? Well, basically we have two options.

We can avoid the integrals in~\autoref{eq:generalmarginal}, replacing it
by integrals that are sometimes more analytically tractable. This can be
performed via an algorithm called EM:
Expectation-Maximization~\cite{EM77}. This method is considered a little
bit naive and generally isn't recommended if you have a better option.
Our better option consists in take advantage of the exponential family
structure and the fact that we're dealing with Gaussian latent effects.
These ideas converge to what is called, \textit{Laplace
  approximation}~\cite{molenberghs&verbeke, LA4H, tierney, corestats}.

If the integral is analytically intractable, we can approximate it to
obtain a tractable closed-form expression, allowing the numerical
maximization of the marginal likelihood~\cite{patrao}. The Laplace
approximation has been designed to approximate integrals in the form
\begin{equation}
  \int_{\mathcal{R}^{\mathbf{u}_{i}}}
  \exp\{Q(\mathbf{u}_{i})\} \text{d} \mathbf{u}_{i}
  \approx (2\pi)^{n_{\mathbf{u}}/2}~
  |{Q}''(\mathbf{\hat{u}}_{i})|^{-1/2}~\exp\{Q(\mathbf{\hat{u}}_{i})\},
  \label{eq:laplace}
\end{equation}
where \(Q(\mathbf{u}_{i})\) is a known, unimodal bounded function and
\(\mathbf{\hat{u}}_{i}\) is the value for which \(Q(\mathbf{u}_{i})\) is
maximized. As~\citeonline{corestats}~shows, a Laplace approximation
consists of a second order Taylor expansion of \(\log f(\mathbf{y}_{i},
\mathbf{u}_{i} \mid \bm{\theta})\), about \(\mathbf{\hat{u}}_{i}\), that
gives
\[
  \log f(\mathbf{y}_{i}, \mathbf{u}_{i} \mid \bm{\theta}) \approx
  \log f(\mathbf{y}_{i}, \mathbf{\hat{u}}_{i} \mid \bm{\theta}) -
  \frac{1}{2}
  (\mathbf{u}_{i} - \mathbf{\hat{u}}_{i})^{\top}\mathbf{H}~
  (\mathbf{u}_{i} - \mathbf{\hat{u}}_{i}),
\]
where \(\mathbf{H} = - \nabla_{u}^{2} \log f(\mathbf{y}_{i},
\mathbf{\hat{u}}_{i} \mid \bm{\theta})\). Hence, we can approximate the
joint by
\begin{equation}
  f(\mathbf{y}_{i}, \mathbf{u}_{i} \mid \bm{\theta}) \approx
  f(\mathbf{y}_{i}, \mathbf{\hat{u}}_{i} \mid \bm{\theta})~\exp
  \left\{- \frac{1}{2}
    (\mathbf{u}_{i} - \mathbf{\hat{u}}_{i})^{\top}\mathbf{H}~
    (\mathbf{u}_{i} - \mathbf{\hat{u}}_{i})
  \right\}.
  \label{eq:taylor}
\end{equation}

From here we start to take advantage of the points mentioned above.
First, the fact that we're working with Gaussian distributed latent
effects. In~\autoref{eq:taylor}~we see the core of a Gaussian density,
that complete is
\[
  \int_{\mathcal{R}^{\mathbf{u}_{i}}}
  \frac{1}{(2 \pi)^{n_{\mathbf{u}}/2}~|\mathbf{H}^{-1}|^{1/2}}~\exp
  \left\{- \frac{1}{2}
    (\mathbf{u}_{i} - \mathbf{\hat{u}}_{i})^{\top}\mathbf{H}~
    (\mathbf{u}_{i} - \mathbf{\hat{u}}_{i})
  \right\} \text{d} \mathbf{u}_{i} = 1,
\]
and integrates to 1. Integrating~\autoref{eq:taylor}, it follows that
\begin{align*}
  \int_{\mathcal{R}^{\mathbf{u}_{i}}}
  f(\mathbf{y}_{i}, \mathbf{u}_{i} \mid \bm{\theta})
  \text{d} \mathbf{u}_{i}
  &\approx f(\mathbf{y}_{i}, \mathbf{\hat{u}}_{i} \mid \bm{\theta})
    \int_{\mathcal{R}^{\mathbf{u}_{i}}} \exp
    \left\{- \frac{1}{2}
    (\mathbf{u}_{i} - \mathbf{\hat{u}}_{i})^{\top}\mathbf{H}~
    (\mathbf{u}_{i} - \mathbf{\hat{u}}_{i})
    \right\} \text{d} \mathbf{u}_{i}\\
  &= (2 \pi)^{n_{\mathbf{u}}/2}~|\mathbf{H}|^{-1/2}~
    f(\mathbf{y}_{i}, \mathbf{\hat{u}}_{i} \mid \bm{\theta}),
\end{align*}
i.e., we get~\autoref{eq:laplace}, a first order Laplace approximation
to the integral. Careful accounting of the approximation error shows it
to generally be \(\mathcal{O}(n^{-1})\) where \(n\) is the sample size,
assuming a fixed length for \(\mathbf{u}_{i}\)~\cite{corestats}.

The second advantage of a Laplace approximation approach in a GLMM is
the exponential family structure. In a usual GLMM, the response follows
a one-parameter exponential family distribution, that can be written as
\[
  f(\mathbf{y}_{i} \mid \mathbf{u}_{i}, \bm{\theta}) = \exp
  \left\{\mathbf{y}_{i}^{\top}
    (\mathbf{X}_{i}\bm{\beta} + \mathbf{Z}_{i}\mathbf{u}_{i}) -
    \mathbf{1}_{i}^{\top}
    b(\mathbf{X}_{i}\bm{\beta} + \mathbf{Z}_{i}\mathbf{u}_{i}) +
    \mathbf{1}_{i}^{\top} c(\mathbf{y}_{i})
  \right\},
\]
where \(b(\cdot)\) and \(c(\cdot)\) are known functions. This general
and easy to compute expression, together with a (multivariate) Gaussian
distribution, highlights the convenience of the Laplace method. The
\(Q(\mathbf{u}_{i})\) function to be maximized can then be expressed as
\begin{equation}
  \begin{aligned}
    Q(\mathbf{u}_{i}) &= \mathbf{y}_{i}^{\top}
    (\mathbf{X}_{i}\bm{\beta} + \mathbf{Z}_{i}\mathbf{u}_{i}) -
    \mathbf{1}_{i}^{\top}
    b(\mathbf{X}_{i}\bm{\beta} + \mathbf{Z}_{i}\mathbf{u}_{i}) +
    \mathbf{1}_{i}^{\top} c(\mathbf{y}_{i})\\
    &- \frac{n_{\mathbf{u}}}{2} \log (2 \pi) -
    \frac{1}{2} \log |\bm{\Sigma}| -
    \frac{1}{2} \mathbf{u}_{i}^{\top} \bm{\Sigma}^{-1}~\mathbf{u}_{i}.
  \end{aligned}
\end{equation}

The approximation in~\autoref{eq:laplace} requires the maximum
\(\mathbf{\hat{u}}_{i}\) of the function \(Q(\mathbf{u}_{i})\). Since we
assume a Gaussian distribution with a known mean for the latent effects,
we have the perfect initial guess for a gradient-based method as the
Newton-Raphson (NR) algorithm. The NR method consists of an iterative
scheme as follows:
\[
  \mathbf{u}_{i}^{(k+1)} = \mathbf{u}_{i}^{(k)} -
  {Q}''(\mathbf{u}_{i}^{(k)})^{-1}~{Q}'(\mathbf{u}_{i}^{(k)}),
\]
until convergence, which gives \(\mathbf{\hat{u}}_{i}\). At this stage,
all parameters are considered known.~\citeonline{patrao}~presents the
generic expressions for the derivatives required by the NR method, given
by the following:
\begin{equation}
  \begin{aligned}
    {Q}'(\mathbf{u}_{i}^{(k)}) &= \{\mathbf{y}_{i} -
    {b}'(\mathbf{X}_{i}\bm{\beta} +
    \mathbf{Z}_{i}\mathbf{u}_{i}^{(k)})\}^{\top} -
    {\mathbf{u}_{i}^{(k)}}^{\top} \bm{\Sigma}^{-1},\\
    {Q}''(\mathbf{u}_{i}^{(k)}) &=
    - \text{diag}\{{b}''(\mathbf{X}_{i}\bm{\beta} +
    \mathbf{Z}_{i}\mathbf{u}_{i}^{(k)})\} - \bm{\Sigma}^{-1}.
  \end{aligned}
  \nonumber
\end{equation}

Finally, the marginal log-likelihood returned by the Laplace
approximation for each invividual or unit under study, is as follows:
\begin{equation}
  \begin{aligned}
    l(\bm{\theta} \mid \mathbf{y}_{i}) =
    \log L(\bm{\theta} \mid \mathbf{y}_{i}) &=
    \frac{n}{2} \log (2 \pi) - \frac{1}{2} \log
    \left|
      \text{diag}\{{b}''(\mathbf{X}_{i}\bm{\beta} +
      \mathbf{Z}_{i}\mathbf{\hat{u}}_{i})\} + \bm{\Sigma}^{-1}
    \right|\\
    &+ \mathbf{y}_{i}^{\top}
    (\mathbf{X}_{i} \bm{\beta} + \mathbf{Z}_{i} \mathbf{\hat{u}}_{i}) -
    \mathbf{1}_{i}^{\top}
    b(\mathbf{X}_{i}\bm{\beta} + \mathbf{Z}_{i} \mathbf{\hat{u}}_{i}) +
    \mathbf{1}_{i}^{\top} c(\mathbf{y}_{i})\\
    &- \frac{n_{\mathbf{u}}}{2} \log (2 \pi) -
    \frac{1}{2} \log |\bm{\Sigma}| - \frac{1}{2}
    \mathbf{\hat{u}}_{i}^{\top}\bm{\Sigma}^{-1}~\mathbf{\hat{u}}_{i},
  \end{aligned}
  \nonumber
\end{equation}
that can now be numerically maximized over the model parameters.

\section{MARGINAL LIKELIHOOD OPTIMIZATION}
\label{cap:opt}

At this point is clear that we have two optimizations to be made. An
``inside'' and an ``outside'' optimization. The inside optimization is
performed into the Laplace approximation layer via a Newton-Raphson
algorithm, a Newton's method. The outside optimization is made with the
Laplace approximation outputs, i.e., the maximization step
of~\autoref{eq:generalmarginal}'s marginal log-likelihood over its
parameters. This task is usually performed via a quasi-Newton method. We
focus here on two of the most traditional ones, the
Broyden-Fletcher-Goldfarb-Shanno (BFGS) algorithm and the PORT routines.

The inside optimization is the joint log-likelihood numerical
maximization w.r.t. its latent effects. This is kind of a simple task
since all model parameters are considered as fixed and we ``know'' that
the latent effects are distributed with zero mean, i.e., we have the
perfect initial guess. In this context, the use of a Newton's method is
straightforward. When we talk about the outside optimization it is a
completely different scenario. It is not straightforward to find a good
initial guess or reach convergence, so more robust methods are a good
choice.

In optimization, Newton methods are algorithms for finding local maxima
and minima of functions, i.e., the search for the zeroes of the gradient
of that function. Newton methods are characterized by the use of a
symmetric matrix of function's second derivatives, the Hessian matrix.
Quasi-Newton methods are based on Newton's method and are seen as an
alternative to it. They can be used if the Hessian is unavailable or is
too expensive to compute at every iteration.

As shown in~\citeonline{nocedal&wright}, chief advantages of
quasi-Newton methods over Newton's method are that the Hessian matrix
doesn't need to be computed, its approximated; and it also doesn't need
to be inverted. Newton's method requires the Hessian to be inverted,
typically by solving a system of linear equations - often quite costly.
In contrast, quasi-Newton methods usually generate an estimate of it
directly. As in Newton's method, they use a second-order approximation
to find the minimum of a function \(f(\mathbf{x})\). The Taylor series
of \(f(\mathbf{x})\) around an iterate is
\[
  f(\mathbf{x}_{k} + \Delta\mathbf{x}) \approx
  f(\mathbf{x}_{k}) + \nabla f(\mathbf{x}_{k})^{\top} \Delta\mathbf{x} +
  \frac{1}{2} \Delta\mathbf{x}^{\top} \mathbf{B}~\Delta\mathbf{x},
\]
where \(\nabla f(\cdot)\) is the gradient, and \(\mathbf{B}\) an
approximation to the Hessian matrix. The gradient of this approximation
w.r.t. \(\Delta\mathbf{x}\) is
\[
  \nabla f(\mathbf{x}_{k} + \Delta\mathbf{x}) \approx
  \nabla f(\mathbf{x}_{k}) + \mathbf{B}~\Delta\mathbf{x},
\]
setting this gradient to zero provides the Newton step:
\[
  \Delta\mathbf{x} = - \mathbf{B}^{-1} \nabla f(\mathbf{x}_{k}).
\]

The Hessian approximation \(\mathbf{B}\) is chosen to satisfy
\[
  \nabla f(\mathbf{x}_{k} + \Delta\mathbf{x}) =
  \nabla f(\mathbf{x}_{k}) + \mathbf{B}~\Delta\mathbf{x},
\]
which is called the \textit{secant} equation, the Taylor series of the
gradient itself. Solving for \(\mathbf{B}\) and applying the Newton's
step with the updated value is equivalent to the \textit{secant} method.
Quasi-Newton methods are a generalization of the secant method to find
the root of the first derivative for multidimensional problems. The
various quasi-Newton methods differ in their choice of the solution to
the secant equation.

In a general quasi-Newton method, the unknown \(\mathbf{x}_{k}\) is
updated applying the Newton's step calculated using the current
approximate Hessian matrix \(\mathbf{B}_{k}\) in the following fashion:
\begin{itemize}
\item \(\Delta \mathbf{x}_{k} = -\alpha_{k}\mathbf{B}_{k}^{-1}\nabla
  f(\mathbf{x}_{k})\), with \(\alpha\) chosen to satisfy the so called
  Wolfe conditions~\cite[p.~34]{nocedal&wright};

\item \(\mathbf{x}_{k+1} = \mathbf{x}_{k} + \Delta\mathbf{x}_{k}\);

\item The gradient computed at the new point \(\nabla
  f(\mathbf{x}_{k+1})\), and \(\mathbf{y}_{k} = \nabla
  f(\mathbf{x}_{k+1}) - \nabla f(\mathbf{x}_{k})\) is used to update the
  approximate Hessian \(\mathbf{B}_{k+1}\), or directly its inverse
  \(\mathbf{H}_{k+1} = \mathbf{B}_{k+1}^{-1}\).
\end{itemize}

The most popular quasi-Newton method is the BFGS algorithm, named for
its discoverers Broyden, Fletcher, Goldfarb, and Shanno. It has the
following update formula
\begin{align*}
  \mathbf{B}_{k+1} &= \mathbf{B}_{k} +
                     \frac{\mathbf{y}_{k}\mathbf{y}_{k}^{\top}}{
                     \mathbf{y}_{k}^{\top}\Delta\mathbf{x}_{k}} -
                     \frac{\mathbf{B}_{k}\Delta\mathbf{x}_{k}
                     (\mathbf{B}_{k}\Delta\mathbf{x}_{k})^{\top}}{
                     \Delta\mathbf{x}_{k}^{\top}\mathbf{B}_{k}
                     \Delta\mathbf{x}_{k}},\\
  \mathbf{H}_{k+1} = \mathbf{B}_{k+1}^{-1}
                   &= \left(
                     \mathbf{I} -
                     \frac{\Delta\mathbf{x}_{k}\mathbf{y}_{k}^{\top}}{
                     \mathbf{y}_{k}^{\top}\Delta\mathbf{x}_{k}}
                     \right) \mathbf{H}_{k}
                     \left(
                     \mathbf{I} -
                     \frac{\mathbf{y}_{k}\Delta\mathbf{x}_{k}^{\top}}{
                     \mathbf{y}_{k}^{\top}\Delta\mathbf{x}_{k}}
                     \right) +
                     \frac{\Delta\mathbf{x}_{k}
                     \Delta\mathbf{x}_{k}^{\top}}{
                     \mathbf{y}_{k}^{\top}\Delta\mathbf{x}_{k}}.
\end{align*}

Another quasi-Newton method, popular in statistical data analysis, is
the one based on PORT routines~\url{http://www.netlib.org/port/}. A
Fortran mathematical subroutine library designed to be \textit{portable}
over different types of computers, and developed by David Gay in the
Bell Labs~\cite{PORTreport}. It is a quasi-Newton adaptive nonlinear
least-squares algorithm~\cite{PORTpaper} with the following update
formula
\begin{align*}
  \mathbf{B}_{k+1} &= \mathbf{B}_{k}\\
                   &+ \frac{
                     \left(\mathbf{y}_{k} -
                     \mathbf{B}_{k}\Delta\mathbf{x}_{k}\right)
                     \Delta\mathbf{x}_{k}^{\top}\mathbf{B}_{k} +
                     \mathbf{B}_{k}\Delta\mathbf{x}_{k}
                     \left(\mathbf{y}_{k} -
                     \mathbf{B}_{k}\Delta\mathbf{x}_{k}\right)^{\top}}{
                     \Delta\mathbf{x}_{k}^{\top}\mathbf{B}_{k}
                     \Delta\mathbf{x}_{k}}\\
                   &- \frac{\Delta\mathbf{x}_{k}^{\top}
                     \left(\mathbf{y}_{k} -
                     \mathbf{B}_{k}\Delta\mathbf{x}_{k}\right)
                     \mathbf{B}_{k}\Delta\mathbf{x}_{k}
                     \Delta\mathbf{x}_{k}^{\top}\mathbf{B}_{k}}{
                     \left(\Delta\mathbf{x}_{k}^{\top}\mathbf{B}_{k}
                     \Delta\mathbf{x}_{k}\right)^{\top}
                     \Delta\mathbf{x}_{k}^{\top}\mathbf{B}_{k}
                     \Delta\mathbf{x}_{k}}.
\end{align*}

As~\citeonline{nocedal&wright} points out, each quasi-Newton method
iteration can be performed at a cost of \(\mathcal{O}(n^{2})\)
arithmetic operations (plus the cost of function and gradient
evaluations); there are no \(\mathcal{O}(n^{3})\) operations such as
linear system solves or matrix-matrix operations. For the BFGS algorithm
is known that the rate of convergence is superlinear, but this is a
valid assumption to any quasi-Newton method, which is fast enough for
most practical purposes. Even though Newton's method converges more
rapidly, quadratically, its cost per iteration usually is higher,
because of its need for second derivatives and solution of a linear
system.

In this thesis, the used BFGS implementation is the one in the
\texttt{R}~\cite{R18}~function \texttt{optim()}, and the PORT routine
used is the one implemented in the \texttt{R} function
\texttt{nlminb()}.

\section{AUTOMATIC DIFFERENTIATION}
\label{cap:ad}

Computing gradients, \(\nabla f(\mathbf{x})\), are a fundamental and
crucial task, but also the main computational bottleneck for any Newton
and quasi-Newton method. Consequently, these computations are
fundamental to the development of this thesis. We choose to use the most
efficient manner of computing gradients, and one of the best scientific
computing techniques, the \textit{automatic differentiation} (AD)
procedure. AD has two modes, the so-called forward and reverse mode. We
will talk a bit about both, but we will use only the reverse mode. The
reason can be illustraded by a simple example, given later.

Automatic differentiation, also called algorithmic differentiation or
computational differentiation, is a set of techniques to numerically and
recursively evaluate the derivative of a function specified by a
computer program. AD techniques are based on the observation that any
function, no matter how complicated, is evaluated by performing a
sequence of simple elementary operations involving just one or two
arguments at a time. Derivatives of arbitrary order can be computed
automatically, automatized and accurately to working precision. Most of
the information in this section was taken of \citeonline{peyre}, but
\citeonline[p.~120]{corestats} and \citeonline[p.~204]{nocedal&wright}
are also very good references.

The most common differentiation approaches are finite differences (FD)
and symbolic calculus. Considering a function \(f: \mathbb{R}^{p}
\rightarrow \mathbb{R}\) and the goal of deriving a method to evaluate
\(\nabla f: \mathbb{R}^{p} \rightarrow \mathbb{R}^{p}\), the
approximation of this vector field via FD would require \(p + 1\)
evaluations of \(f\). The same task via reverse mode AD has in most
cases a cost proportional to a single evaluation of \(f\). AD is similar
to symbolic calculus in the sense that it provides an exact gradient
computation, up to machine precision. However, symbolic calculus does
not takes into account the underlying algorithm which compute the
function, while AD factorizes the computation of the derivative
according to an efficient algorithm.

The use of AD is inherent to the use of a computational graph,
\autoref{fig:compgraph}. Assuming that \(f\) is implemented in an
algorithm, the goal is to compute the derivatives
\begin{align*}
  &\frac{\partial f(\mathbf{x})}{\partial\mathbf{x}_{k}} \in
  \mathbb{R}^{n_{t} \times n_{k}},\\
  &\text{for a numerical algorithm
         (succession of functions) of the form}\\
  &\forall~k = s + 1, \dots, t, \quad
    \mathbf{x}_{k} = f_{k}(\mathbf{x}_{1}, \dots, \mathbf{x}_{k-1}),
\end{align*}
where \(f_{k}\) is a function which only depends on the previous
variables.

\begin{figure}[H]
  % \vspace{0.35cm}
  \setlength{\abovecaptionskip}{.0001pt}
  \caption{A COMPUTATIONAL GRAPH}
  \vspace{0.425cm} \centering
  \includegraphics[width=.8\textwidth]{computational_graph.png}
  \\
  \vspace{0.45cm}
  \begin{footnotesize}
    SOURCE:~\citeonline[p.~31]{peyre}.
  \end{footnotesize}
  \label{fig:compgraph}
\end{figure}

The computational graph, \autoref{fig:compgraph}, has the role of
represent the linking of the variables involved in \(f_{k}\) to
\(\mathbf{x}_{k}\). The evaluation of \(f(\mathbf{x})\) corresponds to a
forward traversal of this graph. Now, how exactly we evaluate \(f\)
through the graph? Via one of the AD modes.

\subsection{Forward Mode}

The forward mode correspond to the usual way of computing differentials.
The method initialize with the derivative of the input nodes
\[
  \frac{\partial \mathbf{x}_{1}}{\partial \mathbf{x}_{1}} =
  \text{Id}_{n_{1} \times n_{1}}, \quad
  \frac{\partial \mathbf{x}_{2}}{\partial \mathbf{x}_{1}} =
  \mathbf{0}_{n_{2} \times n_{1}}, \quad
  \frac{\partial \mathbf{x}_{s}}{\partial \mathbf{x}_{1}} =
  \mathbf{0}_{n_{s} \times n_{1}},
\]
and then iteratively make use of the following recursion formula
\begin{align*}
  &\forall~k = s + 1, \dots, t,\\
  &\frac{\partial\mathbf{x}_{k}}{\partial\mathbf{x}_{1}} =
    \sum_{l~\in~\text{father}(k)}
    \frac{\partial\mathbf{x}_{k}}{\partial\mathbf{x}_{l}} \times
    \frac{\partial\mathbf{x}_{l}}{\partial\mathbf{x}_{1}} =
    \sum_{l~\in~\text{father}(k)}
    \frac{\partial}{\partial\mathbf{x}_{l}}
    f_{k}(\mathbf{x}_{1}, \dots, \mathbf{x}_{k-1}) \times
    \frac{\partial\mathbf{x}_{l}}{\partial\mathbf{x}_{1}}.
\end{align*}

The notation ``father(\(k\))'' denotes the nodes \(l < k\) of the graph
that are connected to \(k\). We make use of \citeonline[p.~32]{peyre}'s
simple example.

\noindent\textbf{Example.}\hspace{.5cm}
Consider the function
\[
  f(x, y) = y\log(x) + \sqrt{y\log(x)}
\]
with the corresponding computational graph being displayed in
\autoref{fig:excompgraph}.

\begin{figure}[H]
  % \vspace{0.35cm}
  \setlength{\abovecaptionskip}{.0001pt}
  \caption{EXAMPLE OF A SIMPLE COMPUTATIONAL GRAPH}
  \vspace{0.425cm} \centering
  \includegraphics[width=.8\textwidth]{ex-computational_graph.png}
  \\
  \vspace{0.45cm}
  \begin{footnotesize}
    SOURCE:~\citeonline[p.~33]{peyre}.
  \end{footnotesize}
  \label{fig:excompgraph}
\end{figure}

The forward mode iterations to compute the derivative w.r.t. \(x\),
following the computational graph, is given by
\begin{align*}
  \frac{\partial x}{\partial x} &= 1, \quad
  \frac{\partial y}{\partial x} = 0\\
  \frac{\partial a}{\partial x} &=
  \frac{\partial a}{\partial x} \frac{\partial x}{\partial x} =
  \frac{1}{x} \frac{\partial x}{\partial x} \qquad
  &\{x \mapsto a = \log(x)\}\\
  \frac{\partial b}{\partial x} &=
  \frac{\partial b}{\partial a} \frac{\partial a}{\partial x} +
  \frac{\partial b}{\partial y} \frac{\partial y}{\partial x} =
  y \frac{\partial a}{\partial x} + 0 \qquad
  &\{(y, a) \mapsto b = ya\}\\
  \frac{\partial c}{\partial x} &=
  \frac{\partial c}{\partial b} \frac{\partial b}{\partial x} =
  \frac{1}{2\sqrt{b}} \frac{\partial b}{\partial x} \qquad
  &\{b \mapsto c = \sqrt{b}\}\\
  \frac{\partial f}{\partial x} &=
  \frac{\partial f}{\partial b} \frac{\partial b}{\partial x} +
  \frac{\partial f}{\partial c} \frac{\partial c}{\partial x} =
  1 \frac{\partial b}{\partial x} + 1 \frac{\partial c}{\partial x}
  \qquad &\{(b, c) \mapsto f = b + c\}
\end{align*}

To compute the derivative w.r.t. \(y\) we run another forward process
\begin{align*}
  \frac{\partial x}{\partial y} &= 0, \quad
  \frac{\partial y}{\partial y} = 1\\
  \frac{\partial a}{\partial y} &=
  \frac{\partial a}{\partial x} \frac{\partial x}{\partial y} = 0 \qquad
  &\{x \mapsto a = \log(x)\}\\
  \frac{\partial b}{\partial y} &=
  \frac{\partial b}{\partial a} \frac{\partial a}{\partial y} +
  \frac{\partial b}{\partial y} \frac{\partial y}{\partial y} =
  0 + a \frac{\partial y}{\partial y}\qquad
  &\{(y, a) \mapsto b = ya\}\\
  \frac{\partial c}{\partial y} &=
  \frac{\partial c}{\partial b} \frac{\partial b}{\partial y} =
  \frac{1}{2\sqrt{b}} \frac{\partial b}{\partial y} \qquad
  &\{b \mapsto c = \sqrt{b}\}\\
  \frac{\partial f}{\partial y} &=
  \frac{\partial f}{\partial b} \frac{\partial b}{\partial y} +
  \frac{\partial f}{\partial c} \frac{\partial c}{\partial y} =
  1 \frac{\partial b}{\partial y} + 1 \frac{\partial c}{\partial y}
  \qquad &\{(b, c) \mapsto f = b + c\}
\end{align*}

\subsection{Reverse Mode}

Instead of evaluating the differentials for all the input nodes, which
is problematic for a large number of nodes, the reverse mode evaluates
the differentials of the output node w.r.t. all the inner nodes.

The method initialize with the derivative of the final node
\[
  \frac{\partial \mathbf{x}_{t}}{\partial \mathbf{x}_{t}} =
  \text{Id}_{n_{y} \times n_{y}},
\]
and then, from the last to the first node, iteratively make use of the
following recursion formula
\begin{align*}
  &\forall~k = t - 1, t - 2, \dots, 1,\\
  &\frac{\partial\mathbf{x}_{t}}{\partial\mathbf{x}_{k}} =
    \sum_{m~\in~\text{son}(k)}
    \frac{\partial\mathbf{x}_{t}}{\partial\mathbf{x}_{m}} \times
    \frac{\partial\mathbf{x}_{m}}{\partial\mathbf{x}_{k}} =
    \sum_{m~\in~\text{son}(k)}
    \frac{\partial\mathbf{x}_{t}}{\partial\mathbf{x}_{m}} \times
    \frac{\partial}{\partial\mathbf{x}_{k}}
    f_{m}(\mathbf{x}_{1}, \dots, \mathbf{x}_{m}).
\end{align*}

The notation ``son(\(k\))'' denotes the nodes \(m < k\) of the graph
that are connected to \(k\). To be clear, the same simple example.

\noindent\textbf{Example.}\hspace{.5cm}
Consider, again, the function
\[
  f(x, y) = y\log(x) + \sqrt{y\log(x)}.
\]

The iterations of the reverse mode is given by
\begin{align*}
  \frac{\partial f}{\partial f} &= 1\\
  \frac{\partial f}{\partial c} &=
  \frac{\partial f}{\partial f} \frac{\partial f}{\partial c} =
  \frac{\partial f}{\partial f} 1\qquad &\{c \mapsto f = b + c\}\\
  \frac{\partial f}{\partial b} &=
  \frac{\partial f}{\partial c} \frac{\partial c}{\partial b} +
  \frac{\partial f}{\partial f} \frac{\partial f}{\partial b} =
  \frac{\partial f}{\partial c} \frac{1}{2\sqrt{b}} +
  \frac{\partial f}{\partial f} 1\qquad
  &\{b \mapsto c = \sqrt{b},~b \mapsto f = b + c\}\\
  \frac{\partial f}{\partial a} &=
  \frac{\partial f}{\partial b} \frac{\partial b}{\partial a} =
  \frac{\partial f}{\partial b} y\qquad &\{a \mapsto b = ya\}\\
  \frac{\partial f}{\partial y} &=
  \frac{\partial f}{\partial b} \frac{\partial b}{\partial y} =
  \frac{\partial f}{\partial b} a \qquad &\{y \mapsto b = ya\}\\
  \frac{\partial f}{\partial x} &=
  \frac{\partial f}{\partial a} \frac{\partial a}{\partial x} =
  \frac{\partial f}{\partial a} \frac{1}{x} \qquad
  &\{x \mapsto a = \log(x)\}
\end{align*}

This is the advantage of reverse mode over the forward mode. A single
traversal over the computational graph allows to compute both
derivatives w.r.t. \(x, y\), while the forward mode necessities two
processes.

An drawback of the reverse mode is the need to store the entire
computational graph, which is needed for the reverse sweep. In
principle, storage of this graph is not too difficult to implement.
However, the main benefit of AD is higher accuracy, and in many
applications the cost is not critical.


\section{TMB: TEMPLATE MODEL BUILDER}
\label{cap:tmb}

Note that the goal of AD is not to define an efficient computational
graph, it is up to the user to provide it. However, computing an
efficient graph associated to a mathematical formula is a complicated
combinatorial problem. Thus, since our goal is to be able to fit our
desired statistical models, a computational tool able to efficiently
define and implement this computational graph is make necessary. To
solve this and many other tasks, we have the Template Model Builder
(TMB) \cite{TMB}.

TMB \url{ http://tmb-project.org} is an \texttt{R} \cite{R18} package
for fitting statistical latent variable models to data, inpired by AD
Model Builder (ADMB) \cite{ADMB}. ADMB is a statistical application for
fitting nonlinear statistical models and solve optimization problems,
that implements AD using \texttt{C++} classes and a native template
language. Unlike most \texttt{R} packages, in TMB the model is
formulated in \texttt{C++}. This characteristic provides great
flexibility, but requires some familiarity with the
\texttt{C}/\texttt{C++} programming language. With TMB a user should be
able to quickly implement complex latent effect models through simple
\texttt{C++} templates.

In this chapter we describe step-by-step all the processes involved in
the creation and parameter estimation of a GLMM. With the TMB, all this
is put in practice in an efficient and robust fashion.

A user needs to provide just the joint likelihood function writing in a
\texttt{C++} template. If the model presents latent effects, during the
compilation the latent effects will be integrated out via an efficient
Laplace approximation routine, with a Newton algorithm inside, and the
marginal log-likelihood gradient will be also computed. These marginal
log-likelihood will be returned into an \texttt{R} object, that can then
be optimized using the user's favorite quasi-Newton routine, available
in \texttt{R}. To do all that, TMB combines some state of the art
software

\begin{itemize}
\item \texttt{CppAD}, a \texttt{C++} AD package
  \url{https://coin-or.github.io/CppAD/};
\item \texttt{Eigen} \cite{eigen}, a \texttt{C++} templated
  matrix-vector library;
\item \texttt{CHOLMOD}, sparse matrix routines available from
  \texttt{R}, used to obtain an efficient implementation of the Laplace
  approximation with exact derivatives
  \url{https://developer.nvidia.com/cholmod};
\item Parallelism through \texttt{BLAS}: Basic Linear Algebra
  Subprograms \url{http://www.netlib.org/blas/}.
\end{itemize}

Also, some of its key characteristics are

\begin{itemize}
\item TMB employs AD to calculate first and second order derivatives of
  the likelihood function or any objective function in \texttt{C++};
\item The objective function, and its derivatives, can be called from
  \texttt{R}. Hence, parameter estimation via \texttt{optim()} or
  \texttt{nlminb()} is easy to be performed;
\item Standard deviations of any parameter, or derived parameter, can be
  obtained via the \textit{delta method}.
\end{itemize}

Here we focus on GLMMs, but basically any statistical model with a
latent structure (or not), linear (or not), can be fitted with TMB. In
times of \textit{big data}, and with the TMB's authors having a
professional preference for state-of-space and spatial models, TMB has
also automatic sparseness detection.and some other nice built tools. Pre
and post-processing of data should be done in \texttt{R}.

A TMB Users' mailing list exists, and it is extremely helpful for taking
doubts and questions \url{https://groups.google.com/g/tmb-users}. Also,
a very didactic and comprehensive documentation with several examples is
available online
\url{https://kaskr.github.io/adcomp/_book/Tutorial.html}.

% END ==================================================================
% ----------------------------------------------------------------------
\chapter{\(\text{multiGLMM}\): a multinomial GLMM for clustered
  competing risks data}
\label{cap:model}
We are handling with a complex survival data structure, the clustered
competing risks setting. But we are using a general statistical modeling
framework, the generalized linear mixed models (GLMMs), that was not
made for this purpose.

To model competing risks data, one has to choose in which scale to work.
We can work on the hazard scale dealing with the cause-specific hazard
or on the probability scale dealing with the cause-specific cumulative
incidence function (CIF). With the correct link function, we can make an
appropriate GLMM to work on that probability scale.

Our focus in this thesis is to be able to deal with complex family
studies, where there is generally a strong interest in describing age at
disease onset in the scenarios of within-cluster dependence. The
distribution of age at disease onset is directly described by the
cause-specific CIF. To make a GLMM work for this type of data we need to
accommodate the cause-specific CIFs and the censorings. Assuming the
conditional distribution for our model response as multinomial already
deals with both left-truncation and right-censoring, avoiding the
specification of a censoring distribution. The cause-specific CIFs can
be modeled via the link function of our, then, multinomial GLMM
(multiGLMM). The multinomial distribution also guarantees that the CIFs
of all causes are modeled.

Our choice of a general framework tries to make the inference of this
complex model, easier. Besides, taking advantage of all the procedures
mentioned in the previous chapter.

\section{CLUSTER-SPECIFIC CUMULATIVE INCIDENCE FUNCTION
  (CIF)}
\label{cap:cif}

Consider that the observed follow-up time of an individual is given by
\(T = \min(T^{\ast},~C)\), where \(T^{\ast}\) denote the failure time
and \(C\) denote the censoring time. Given the possible covariates \(X\)
(that can be time-dependent), for a cause-specific of failure \(k\) the
CIF is defined as
\begin{align*}
  F_{k}(t \mid X) &= \mathbb{P}[T \leq t, K = k \mid X]\\
                  &= \int_{0}^{t} f_{k}(z \mid X)~\text{d}z\\
                  &= \int_{0}^{t} \lambda_{k}(z \mid X)~S(z \mid X)
                    ~\text{d}z, \quad t > 0, \quad k = 1, \dots, K.
\end{align*}
where \(f_{k}(t \mid X)\) is the (sub)density for the time to a type
\(k\) failure. This is the general definition of a CIF, and to define it
we need to define the functions that compose the subdensity.

The first is the cause-specific hazard function or process
\[
  \lambda_{k}(t \mid X) =
  \lim_{h \rightarrow 0}~h^{-1}
  \mathbb{P}[t \leq T < t + h, K = k \mid T \geq t, X],
  \quad t > 0, \quad k = 1, \dots, K.
\]

In words, the cause-specific hazard function, \(\lambda_{k}(t \mid X)\),
represents the instantaneous rate for failures of type \(k\) at time
\(t\) given \(X\) and all other failure types (competing causes). If
we sum up all cause-specific hazard function we get the overall hazard
function,
\[
  \lambda(t \mid X) = \sum_{k=1}^{K}\lambda_{k}(t \mid X).
\]

From the overall hazard function we arrive in the overall survival function,
\[
  S(t \mid X) =
  \mathbb{P}[T > t \mid X] =
  \exp\left\{-\int_{0}^{t} \lambda(z \mid X)~\text{d}z\right\},
\]
the second function that compose the subdensity \(f_{k}(t \mid X)\). A
comprehensive reference for all these definitions is the book of
\citeonline{kalb&prentice}.

Until this point, we were talking about a general CIF's definition. We
need now a precise framework telling how to take into consideration our
clustered/family structure. We use the same CIF specification of
\citeonline{SCHEIKE}, i.e. the approach that motivated this thesis.

For two competing causes of failure, the cause-specific CIFs are
specified in the following manner,
\begin{equation}
  F_{k} (t \mid X, u_{1}, u_{2}, \eta_{k}) =
  \underbrace{\pi_{k}(X, u_{1}, u_{2})}_{
    \substack{\text{cluster-specific}\\\text{risk level}}}\times
  \underbrace{\Phi[w_{k} g(t) - X^{\top}\gamma_{k} - \eta_{k}]}_{
    \substack{\text{cluster-specific}\\\text{failure time trajectory}}
  }, \quad t > 0, \quad k = 1,~2.
  \label{eq:cif}
\end{equation}
i.e. as a product of a cluster-specific risk level and a
cluster-specific failure time trajectory, resulting in a
cluster-specific CIF.

What makes the components in \autoref{eq:cif} cluster-specific are
\(\bm{u} = \{u_{1}, u_{2}\}\) and \(\bm{\eta} = \{\eta_{1},
\eta_{2}\}\), Gaussian distributed latent effects with zero mean and
potentially correlated, i.e.
\[
  \begin{bmatrix} u_{1}\\u_{2}\\\eta_{1}\\\eta_{2} \end{bmatrix} \sim
  \mathcal{N} \left(\begin{bmatrix} 0\\0\\0\\0\end{bmatrix},
    \begin{bmatrix}
      \sigma_{u_{1}}^{2}&
      \text{cov}(u_{1},~u_{2})&
      \text{cov}(u_{1},~\eta_{1})&\text{cov}(u_{1},~\eta_{2})\\
      &\sigma_{u_{2}}^{2}&
      \text{cov}(u_{2},~\eta_{1})&\text{cov}(u_{2},~\eta_{2})\\
      &&\sigma_{\eta_{1}}^{2}&\text{cov}(\eta_{1},~\eta_{2})\\
      &&&\sigma_{\eta_{2}}^{2}
    \end{bmatrix}\right).
\]

The cluster-specific survival function is given as \(S(t \mid X, \bm{u},
\bm{\eta}) = 1 - F_{1} (t \mid X, \bm{u}, \eta_{1}) - F_{2} (t \mid X,
\bm{u}, \eta_{2})\).

Since we use the same CIF specification of \citeonline{SCHEIKE}, the
following descriptions and details are essentially the same encountered
in the paper.

Focusing first on the second component of \autoref{eq:cif}. The
cluster-specific failure time trajectory
\[
  \Phi[w_{k} g(t) - X^{\top}\gamma_{k} - \eta_{k}],
  \quad t > 0, \quad k = 1, ~2,
\]
where \(\Phi(\cdot)\) is the cumulative distribution function of a
standard Gaussian distribution.

Instead of \(w_{k} g(t)\), in \citeonline{SCHEIKE} is specified
\(\alpha_{k}(g(t))\), where \(\alpha_{k}(\cdot)\) are monotonically
increasing functions known up to a finite-dimensional parameter vector,
\(w_{k}\). Examples are monotonically increasing B-spline or piecewise
lienar functions. However, to try to simplify the model structure we
consider just the finite-dimensional parameter vector. The bottom line
is that the authors do the same approach in their applications.

With regard to the function \(g(t)\), it plays a crucial role since the
separation of the CIF in \autoref{eq:cif} is only possible with it. A
time \(t\) transformation given by
\[
  g(t) = \text{arctanh}\left(\frac{t - \delta/2}{\delta/2}\right),
  \quad t\in~]0,~\delta[, \quad g(t)\in~]-\infty,~\infty[,
\]
where \(\delta\) depends on the data and cannot exceed the maximum
observed follow-up time \(\tau\), i.e. \(\delta \leq \tau\). With this
transformation, based on a Fisher transformation, the value of the
cluster-specific failure time trajectory is equal 1, at time \(\delta\).
Consequently, \(F_{k} (\delta \mid X, \bm{u}, \eta_{k}) = \pi_{k}(X \mid
\bm{u})\) and, we can interpret \(\pi_{1}(X \mid \bm{u})\) and
\(\pi_{2}(X \mid \bm{u})\) as the cause-specific cluster-specific risk
levels, at time \(\delta\).

The cluster-specific risk levels are modeled by a multinomial logistic
regression model with latent effects, i.e.
\begin{equation}
  \pi_{k}(X, \bm{u}) =
  \frac{\exp\{X^{\top}\beta_{k} + u_{k}\}}{1 +
    \exp\{X^{\top}\beta_{1} + u_{1}\} +
    \exp\{X^{\top}\beta_{2} + u_{2}\}}, \quad k = 1,~2,
  \label{eq:risklevel}
\end{equation}
where \(\beta_{k}\)'s are the parameters responsible for quantifying the
impact of the covariates in the cause-specific risk levels. For
individuals from the same chuster/family, at the same time point, the
\(\beta_{k}\)s have the well-known odds ratio interpretation.

The \(\gamma_{k}\)'s are the parameters responsible for quantifying the
impact of the covariates in the cause-specific failure time
trajectories, i.e. the shape of the cumulative incidence, and
consequently how quickly the cluster-specific risk levels observed at
time \(\delta\) are reached. The fact that \(\gamma_{k}\) enters
negatively in the cluster-specific failure time trajectory makes that a
negative value causes an advance towards the cluster-specific risk
level, whereas a covariate with a positive effect causes a delay.

Within-cluster dependence is induced by the latent effects in \(\bm{u}\)
and \(\bm{\eta}\), but they don't have an easy interpretation. To help
in the discussion, \autoref{fig:cif} illustrates the cluster-specific
CIF for a given failure cause, let's call it failure cause 1 (in total
we have two).

\begin{figure}[H]
  % \vspace{0.35cm}
  \setlength{\abovecaptionskip}{.0001pt}
  \caption{ILLUSTRATION OF A CLUSTER-SPECIFIC CUMULATIVE INCIDENCE
    FUNCTION (CIF), PROPOSED BY \citeonline{SCHEIKE}, FOR A GIVEN
    FAILURE CAUSE 1. FROM A CONFIGURATION WITH \(X = 1\) FOR ALL
    SUBJECTS AND WITH \(\beta_{1} = -1.9\), \(\beta_{2} = -0.2\),
    \(\gamma_{1} = 1\), \(w_{1} = 3\) AND \(u_{2} = 0\). THE VARIATION
    BETWEEN FRAMES IS GIVEN BY THE LATENT EFFECTS \(u_{1}\) AND
    \(\eta_{1}\)}
  \vspace{0.425cm} \centering
  \includegraphics[width=\textwidth]{cif-1.png}
  \\
  \vspace{0.45cm}
  \begin{footnotesize}
    SOURCE: The author (2020).
  \end{footnotesize}
  \label{fig:cif}
\end{figure}

The latent effects \(u_{1}\) and \(u_{2}\) always appear together in the
cluster-specific risk level, as consequency they have a joint effect on
the cumulative incidence of both causes. Nevertheless, as we can see in
\autoref{fig:cif}, an increase in \(u_{k}\) will increase the risk of
failure from cause \(k\) and vice versa. The interpretation of
\(\text{cov}(\eta_{1},~\eta_{2})\) and \(\text{cov}(u_{1},~u_{2})\) is
more or less straightforward. With regard to
\(\text{cov}(u_{k},~\eta_{k})\), a negative correlation between
\(\eta_{k}\) and \(u_{k}\) imply that when \(\eta_{k}\) decreases,
\(u_{k}\) increases and conversely when \(\eta_{K}\) increases,
\(u_{k}\) decreases. In other words, an increased risk level is reached
quickly and a decreased risk level is reached later, respectively.

Practical situations with a positive within-cause correlation are hard
to find, i.e. where an increased risk level is associated with a late
onset and vice versa. However, a positive cross-cause correlation
between \(\eta\) and \(u\) sounds more realistic. i.e. where late onset
of one failure cause is associated with a high absolute risk of another
failure cause.

The latent effects are assumed independent across clusters and shared by
individuals within the same cluster/family.

\section{MODEL SPECIFICATION}
\label{cap:modelitself}

Our generalized linear mixed model (GLMM) is specified in the following
fashion. For two competing causes of failure, a subject \(i\), with
cluster \(j\), in the time \(t\), we have
\begin{align}
  y_{i j t} \mid \{u_{1j},~u_{2j},~\eta_{1j},~\eta_{2j}\}&\sim
  \text{Multinomial}(p_{1ijt},~p_{2ijt},~p_{3ijt})\nonumber\\
  \nonumber\\
  \begin{bmatrix} u_{1}\\u_{2}\\\eta_{1}\\\eta_{2} \end{bmatrix}&\sim
  \mathcal{N} \left(\begin{bmatrix} 0\\0\\0\\0\end{bmatrix},
  \begin{bmatrix}
    \sigma_{u_{1}}^{2}&
    \text{cov}(u_{1},~u_{2})&
    \text{cov}(u_{1},~\eta_{1})&\text{cov}(u_{1},~\eta_{2})\\
    &\sigma_{u_{2}}^{2}&
    \text{cov}(u_{2},~\eta_{1})&\text{cov}(u_{2},~\eta_{2})\\
    &&\sigma_{\eta_{1}}^{2}&\text{cov}(\eta_{1},~\eta_{2})\\
    &&&\sigma_{\eta_{2}}^{2}
  \end{bmatrix}\right)\nonumber\\
  \nonumber\\
  p_{kijt} &=
  \frac{\partial}{\partial t}F_{k} (t \mid X, u_{1}, u_{2}, \eta_{k})
  \nonumber\\
  &= \frac{\exp\{\bm{x}_{kij}\bm{\beta}_{ki} + u_{kj}\}}{
    1 + \sum_{m=1}^{K-1}\exp\{\bm{x}_{mij}\bm{\beta}_{mi} + u_{mj}\}}
  \label{eq:model}\\
  &\times w_{k}\frac{\delta}{2\delta t - 2t^{2}}~
  \phi\left(
    w_{k}
    \text{arctanh}\left(\frac{t-\delta/2}{\delta/2}\right)
    - \bm{x}_{kij}\bm{\gamma}_{ki} - \eta_{kj}
    \right),\nonumber\\ k = 1,~2.\nonumber
\end{align}

The chosen link function to represent the probabilities is given by the
derivative w.r.t. time \(t\) of the cluster-specific CIF. The choice of
a multinomial logistic regression model ensures that the sum of the
predicted cause-specific CIFs does not exceed 1.

Considering two competing causes of failure, we have a multinomial with
three classes. The third class exists to handle the censorship and its
probability is given by the complementary to reach 1. This framework in
\autoref{eq:model} results in what we call multiGLMM, a multinomial
GLMM.

For a random sample, the corresponding marginal likelihood functions in
given by
\begin{align}
  L(\bm{\theta}~;~y)
  &= \prod_{j=1}^{J}~\int_{\Re^{4}}
    \pi(y_{j} \mid \bm{r}_{j})\times\pi(\bm{r}_{j})~\text{d}\bm{r}_{j}
    \nonumber\\
  &= \prod_{j=1}^{J}~\int_{\Re^{4}}
    \Bigg\{
    \underbrace{\prod_{i=1}^{n_{j}}~\prod_{t=1}^{n_{ij}}
    \Bigg(
    \frac{(\sum_{k=1}^{K}y_{kijt})!}{y_{1ijt}!~y_{2ijt}!~y_{3ijt}!}~
    \prod_{k=1}^{K} p_{kijt}^{y_{kijt}}
    \Bigg)}_{\substack{\text{fixed effect component}}}
  \Bigg\}\times\nonumber\\
  &\hspace{2cm}\underbrace{
    (2\pi)^{-2} |\Sigma|^{-1/2} \exp
    \left\{-\frac{1}{2}\bm{r}_{j}^{\top} \Sigma^{-1} \bm{r}_{j}\right\}
    }_{\substack{\text{latent effect component}}}
    \text{d}\bm{r}_{j}\nonumber\\
  &= \prod_{j=1}^{J}~\int_{\Re^{4}}
    \Bigg\{
    \underbrace{\prod_{i=1}^{n_{j}}~\prod_{t=1}^{n_{ij}}
    \prod_{k=1}^{K} p_{kijt}^{y_{kijt}}
    }_{\substack{\text{fixed effect}}}
  \Bigg\}\underbrace{
  (2\pi)^{-2} |\Sigma|^{-1/2} \exp
  \left\{-\frac{1}{2}\bm{r}_{j}^{\top} \Sigma^{-1} \bm{r}_{j}\right\}
  }_{\substack{\text{latent effect component}}}
  \text{d}\bm{r}_{j}\label{eq:loglik},
\end{align}
where \(\bm{\theta} = [\bm{\beta}~\bm{\gamma}~\bm{w}~\bm{\sigma^{2}}~
\bm{\varrho}]^{\top}\) is the parameters vector to be maximized. In our
framework, a subject can fail from just one competing cause or get
censor, at a given time. Thus, the fraction of factorials in the fixed
effect component is made only by 0's and 1's. Finally, returning the
value 1 .The matrix \(\Sigma\) is the variance-covariance matrix, which
components are given by \(\bm{\sigma}^{2}\) and \(\bm{\varrho}\).

Now, \autoref{eq:loglik} in words. To each cluster (family) \(j\) we
have a product of two components. The fixed effect component, given by a
multinomial distribution with its probabilities specified through the
cluster-specific CIF (\autoref{eq:cif}) and, the latent effect
component, given by a multivariate Gaussian distribution.

To each subject \(i\) that composes a cluster \(j\) we have its specific
fixed effects contribution. The likelihood in \autoref{eq:loglik} is the
most general as possible, allowing for repeated measures to each
subject. Since all subjects of a given cluster shares the same latent
effect, we have just one latent effect contribution multiplying the
product of fixed effects contribution. As we don't observe the latent
effect variables, \(\bm{r}_{j}\), we integrate out in it. With two
competing causes of failure, we have four latent effects (a multivariate
Gaussian distribution in four dimensions). As consequence, for each
cluster, we approximate an integral in four dimensions. The product of
these approximated integrals results in the called marginal likelihood,
to be maximized in \(\bm{\theta}\).

% END ==================================================================
% ----------------------------------------------------------------------
\chapter{simulation study datasets}
\label{cap:datasets}
This chapter describes how to simulate from our multiGLMM, and describes
a real-based dataset used as an application example. The simulation
procedure is addressed in \autoref{cap:simu}. In \autoref{cap:data} a
simulated dataset based on the Nordic Cancer Union (NCU) twins data is
presented as an application example.

\section{SIMULATING FROM THE MODEL}
\label{cap:simu}

Being able to simulate data from a model is a key task, fundamental to
assess the finite-sample properties and the estimation procedure
liability of a given statistical model. The step-by-step describing the
simulation procedure of our multiGLMM is presented on Algorithm
\autoref{alg:algo}, following the model hierarchical structure
stipulated in Formula \autoref{eq:model}.

\begin{algorithm}[H]
 \caption{SIMULATING FROM A \(\text{multiGLMM}\) FOR CLUSTERED COMPETING
          RISKS DATA}
 \label{alg:algo}
 \begin{algorithmic}[1]
  \State
   Set \(J\), the number of clusters
  \State
   Set \(n_{j}\), the number of cluster elements
   \Comment{can be of different sizes}
  \State
   Set \(K-1\), the number of competing causes of failure
  \State
   Set the model parameter values \(\bm{\theta} =
   [\bm{\beta}~\bm{\gamma}~\bm{w}~\bm{\sigma^{2}}~\bm{\varrho}]^{\top}\)
  \State
   Sample \(J\) latent effect vectors from a
   \(\mathcal{N}_{(K-1)\times(K-1)}(\bm{0},~\Sigma(\bm{\sigma^{2}},
     \bm{\varrho}))\)
  \State
   Set \(\delta\)
   \Comment{maximum follow-up time}
  \State
   Set the failure times \(t_{ij}\)
  \State
   Compute the competing risks probabilities
   \begin{align*}
      p_{kijt}
      &= \frac{\exp\{\bm{x}_{kij}\bm{\beta}_{ki} + u_{kj}\}}{
        1 +
        \sum_{m=1}^{K-1}\exp\{\bm{x}_{mij}\bm{\beta}_{mi} + u_{mj}\}}\\
      &\times
        w_{k}\frac{\delta}{2\delta t - 2t^{2}}~
        \phi\left(
        w_{k}
        \text{arctanh}\left(\frac{t-\delta/2}{\delta/2}\right)
        - \bm{x}_{kij}\bm{\gamma}_{ki} - \eta_{kj}
        \right),\\
      p_{Kijt}
      &= 1 - \sum_{k = 1}^{K - 1} p_{kijt}, \quad k = 1,~2,~\dots,~K -1
    \end{align*}
    \State
    Sample \(J\times n_{j}\) vectors from a
    \(\text{Multinomial}(p_{1ijt},~p_{2ijt},~\dots,~p_{Kijt})\)
    \State
    If \(t_{kij} = \delta\), subject moves to class K
    \Comment{any failure at time \(\delta\) is censored}
    \State
    \Return
    To each individual, its failure/censoring time and from which
    cause-specific it is
  \end{algorithmic}
\end{algorithm}
\vspace{-1cm}
\begin{footnotesize}
  \begin{center}
    SOURCE: The author (2021).
  \end{center}
\end{footnotesize}

Sample \(\varsigma\sim\text{U}(0,~1)\)

Compute the cause-specific failure times by solving
\[
 \varsigma = \Phi[w_{k} g(t_{k}) - X^{\top}\gamma_{k} - \eta_{k}]
 \quad\text{for } t_{k}, \quad k = 1,~2,~\dots,~K - 1
\]

The model described in \autoref{eq:model} is in its most general form,
i.e. allowing for multiple measures at each subject and varying
coefficients. However, we focus on a simpler structure without
covariates, a single measure per subject, and common coefficients.
Putting in practice Algorithm \autoref{alg:algo}, we use the following
model configuration

\begin{align}
  p_{kijt}
  &= \frac{\exp\{\beta_{ki} + u_{kj}\}}{
    1 + \sum_{m=1}^{K-1}\exp\{\beta_{mi} + u_{mj}\}}\nonumber\\
  &\times w_{k}\frac{\delta}{2\delta t - 2t^{2}}~
    \phi\left(
    w_{k}
    \text{arctanh}\left(\frac{t-\delta/2}{\delta/2}\right)
    - \gamma_{ki} - \eta_{kj}
    \right),\quad k = 1,~2,\nonumber\\
  \text{with }\quad
  \bm{\beta}_{i} &= [-2~~~1.5]^{\top}\nonumber\\
  \bm{\gamma}_{i} &= [1.2~~~1]^{\top}\label{eq:modelconfig}\\
  \bm{w} &= [3~~~5]^{\top}\nonumber\\
  \bm{u}_{j} &= [0~~~0]^{\top}\quad
               \bm{\eta}_{j} = [0~~~0]^{\top}\nonumber.
\end{align}
Based on that we get the cluster-specific CIF's and failure
probabilities, its CIF derivatives (dCIF) w.r.t. time \(t\), presented
respectively in \autoref{fig:datasimucif}.

\begin{figure}[H]
  \setlength{\abovecaptionskip}{.0001pt}
  \caption{CLUSTER-SPECIFIC CUMULATIVE INCIDENCE FUNCTIONS (CIF) AND
    RESPECTIVE DERIVATIVES W.R.T. TIME (\(\text{dCIF}\)) FOR A MODEL
    WITH TWO COMPETING CAUSES OF FAILURE, WITHOUT COVARIATES AND THE
    FOLLOWING CONFIGURATION: \(\beta_{1} = -2\), \(\beta_{2} = -1.5\),
    \(\gamma_{1} = 1.2\), \(\gamma = 1\), \(w_{1} = 3\), \(w_{2} = 5\)
    AND LATENT EFFECTS FIXED AT ZERO}
  \vspace{0.2cm} \centering
  \includegraphics[width=\textwidth]{datasimucif-1.png}
  \\
  \begin{footnotesize}
    SOURCE: The author (2020).
  \end{footnotesize}
  \label{fig:datasimucif}
\end{figure}

By adding the latent structure
\[
  \begin{bmatrix} u_{1}\\u_{2}\\\eta_{1}\\\eta_{2} \end{bmatrix}
  \sim\mathcal{N} \left(
    \begin{bmatrix} 0\\0\\0\\0 \end{bmatrix},
    \begin{bmatrix}
      1&0.4&0.5&0.4\\
      &1&0.4&0.3\\
      &&1&0.4\\
      &&&1
    \end{bmatrix}\right),
\]
in \autoref{eq:modelconfig}, we generate a complete model sample with
500 clusters/pairs of twins, summarized in \autoref{fig:datasimu}.

\begin{figure}[H]
  % \vspace{0.35cm}
  \setlength{\abovecaptionskip}{.0001pt}
  \caption{SUMMARY OF A SIMULATED DATASET WITH 500 PAIRS OF TWINS. A)
    TIME BY TWIN; B) TIMES BOXPLOT; C) PROBABILITIES SCATTERPLOT D)
    \(y_{3}\)'S \%}
  \vspace{0.2cm} \centering
  \includegraphics[width=\textwidth]{datasimu-1.png}
  \\
  \vspace{0.2cm}
  \begin{footnotesize}
    SOURCE: The author (2020).
  \end{footnotesize}
  \label{fig:datasimu}
\end{figure}

\section{REAL-BASED DATASET}
\label{cap:data}

% END ==================================================================

% ----------------------------------------------------------------------
\chapter{Results}
\label{cap:results}
This chapter presents the simulation study results. We have seventy-two
simulation scenarios, as detailed in \autoref{cap:datasets}. For each
scenario we simulate 500 samples. In total, we fit 36000 models.

\section{SIMULATION STUDY}
\label{cap:simures}

Let us just recap the parameter values used
\begin{align*}
 \text{High CIF configuration}:~&\quad
 \{\beta_{1} = -2,~\beta_{2} = -1.5,~\gamma_{1} = 1,~\gamma_{2} = 1.5,~
   w_{1} = 3,~w_{2} = 4
 \};\\
 \text{Low CIF configuration}:~&\quad
 \{\beta_{1} = 3,~\beta_{2} = 2.5,~\gamma_{1} = 2.6,~\gamma_{2} = 4,~
   w_{1} = 5,~w_{2} = 10
 \}.
\end{align*}
\begin{minipage}{0.15\textwidth}
 \begin{align*}
  \sigma_{u_{1}}^{2}   &= 1\\
  \sigma_{u_{2}}^{2}   &= 0.7,\\
  \sigma_{\eta_{1}}^{2} &= 0.6\\
  \sigma_{\eta_{2}}^{2} &= 0.9
 \end{align*}
\end{minipage}%
\begin{minipage}{0.85\textwidth}
 \[
  \text{Correlation structure}~=~\begin{blockarray}{ccccc}
                                  u_{1} & u_{2} & \eta_{1} & \eta_{2}\\
                                  \begin{block}{(cccc)c}
                                   1 & 0.1 & -0.5 &  0.3 & u_{1}\\
                                     &   1 &  0.3 & -0.4 & u_{2}\\
                                     &     &    1 &  0.2 & \eta_{1}\\
                                     &     &      &    1 & \eta_{2}\\
                                  \end{block}
                                 \end{blockarray}.
 \]
\end{minipage}

\vspace{0.3cm}
\noindent
The parameter values per se are not important. What is important is to
keep in mind the behaviors implied by them, and see if the proposed
model is able to estimate the true values in several different scenarios
and measure the quality of the estimates.

The take-home message for the fixed-effect parameters, is to show that
we can construct different level CIF scenarios. The \(\bm{\beta}\)s are
responsible for the curve maximum point or plateau, being in the risk
level CIF component, the \(\bm{\gamma}\)s and \(\bm{w}\)s are
responsible for basically the curve shape, being in the failure time
trajectory level CIF component. Its interpretation is presented in
detail in \autoref{cap:model}. About the latent-effects, the chosen
covariance structure is considerably high but still acceptable. The
underlying idea was to try to build a realistic covariance scenario and
consequently be able to check how the model performs in such conditions.

In the following pages we have several graphs summarizing the estimates
bias. In each figure, we have the estimate bias and its uncertainty
described by a Wald-based confidence interval i.e., \(\pm\) 1.96 the
bias standard deviation. This is a good uncertainty representation
choice since it is symmetric. In the \autoref{cap:appendixD}, we have
the same estimates bias but with its uncertainty measure being the
corresponding 2.5 and 97.5\% bias quantiles. We chose to use these
uncertainty representations uniquely based on the point estimates
instead of the standard error computations. In several scenarios, the
model fails to compute all the standard errors, caused by Hessian
numerical instabilities.

In each of the following estimates bias graphs, the seventy-two
scenarios are accommodated. We have up to four blocks of bars, each
block representing a model. In each block we have eighteen bars, each
bar representing the 500 fits in each of the eighteen
scenarios, \(4 \times 18 \times 500 = 36000\).

Each scenario name consists of a combination of three strings
\begin{itemize}
 \item The cluster size (cs), 2, 5, and 10;
 \item The CIF configuration, high and low;
 \item The sample size, 5, 30, and 60 thousand.
\end{itemize}
We have tried to fit a total of 36000 models but not all converged. To
show these characteristic, we control the bar widths. Something specific
can be said about each parameter but let us keep the focus on the
general remarks. Starting from the fixed-effect parameters
in \autoref{fig:biassdbeta1}, \autoref{fig:biassdbeta2},
\autoref{fig:biassdgama1}, \autoref{fig:biassdgama2},
\autoref{fig:biassdw1}, and \autoref{fig:biassdw2}, we have very nice
results that already show a strong inclination towards the complete
model's choice.

With a latent structure only in the risk level or in the failure time
trajectory level, the low CIF scenarios are the ones with a much smaller
bias-variance. In general, the mean-bias is small but the variances are
high. When we have a latent structure on both levels but we still assume
the cross-correlations as zero (block-diag model), the results get a
little bit better. Nevertheless, when we assume a non-zero
cross-correlation structure (complete model), basically everything
changes for the better. The mean biases get even closer to zero, the
standard deviations decrease 50\% or more, and mainly, now the high CIF
scenarios are the ones with a much smaller bias-variance. All this is
accomplished through the consideration of the cross-correlations.

In the \textit{simpler} models, with a latent structure just in one
level, is hard to see some significant difference between the clusters
and sample sizes. With the complete model, in the other hand, the
difference is clear: as we increase the clusters and the sample sizes,
the bias-variance decreases. The mean-bias is basically always the
same. In the risk model is hard to point-out a scenario as the best or
worst. For the time model, with the scenarios \texttt{cs02-high-05k}
and \texttt{cs05-high-60k}, we get a much bigger standard deviation in
the \(\bm{\beta}\)s parameter estimates. For the block-diag model, with
the scenario \texttt{cs05-low-05k}, the standard deviations are huge for
the shape curve parameter estimates of the competing cause 1. In
the \autoref{cap:appendixD}, with the 2.5 and 97.5\% bias quantiles, the
most extreme values are removed from the uncertainty
representation. There, the main characteristic is the parameter
estimates asymmetry.

\begin{figure}[H]
 \setlength{\abovecaptionskip}{.0001pt}
 \caption{PARAMETER \(\beta_{1}\) BIAS WITH \(\pm\) 1.96 STANDARD
          DEVIATIONS}
 \vspace{0.2cm}\centering
 \includegraphics[width=\textwidth]{bias2plotsd-1.png}\\
 \begin{footnotesize}
  SOURCE: The author (2021).
 \end{footnotesize}
 \label{fig:biassdbeta1}
\end{figure}

\begin{figure}[H]
 \setlength{\abovecaptionskip}{.0001pt}
 \caption{PARAMETER \(\beta_{2}\) BIAS WITH \(\pm\) 1.96 STANDARD
         DEVIATIONS}
 \vspace{0.2cm}\centering
 \includegraphics[width=\textwidth]{bias2plotsd-2.png}\\
 \begin{footnotesize}
  SOURCE: The author (2021).
 \end{footnotesize}
 \label{fig:biassdbeta2}
\end{figure}

\begin{figure}[H]
 \setlength{\abovecaptionskip}{.0001pt}
 \caption{PARAMETER \(\gamma_{1}\) BIAS WITH \(\pm\) 1.96 STANDARD
          DEVIATIONS}
 \vspace{0.2cm}\centering
 \includegraphics[width=\textwidth]{bias2plotsd-3.png}\\
 \begin{footnotesize}
  SOURCE: The author (2021).
 \end{footnotesize}
 \label{fig:biassdgama1}
\end{figure}

\begin{figure}[H]
 \setlength{\abovecaptionskip}{.0001pt}
 \caption{PARAMETER \(\gamma_{2}\) BIAS WITH \(\pm\) 1.96 STANDARD
          DEVIATIONS}
 \vspace{0.2cm}\centering
 \includegraphics[width=\textwidth]{bias2plotsd-4.png}\\
 \begin{footnotesize}
  SOURCE: The author (2021).
 \end{footnotesize}
 \label{fig:biassdgama2}
\end{figure}

\begin{figure}[H]
 \setlength{\abovecaptionskip}{.0001pt}
 \caption{PARAMETER \(w_{1}\) BIAS WITH \(\pm\) 1.96 STANDARD DEVIATIONS}
 \vspace{0.2cm}\centering
 \includegraphics[width=\textwidth]{bias2plotsd-5.png}\\
 \begin{footnotesize}
  SOURCE: The author (2021).
 \end{footnotesize}
 \label{fig:biassdw1}
\end{figure}

\begin{figure}[H]
 \setlength{\abovecaptionskip}{.0001pt}
 \caption{PARAMETER \(w_{2}\) BIAS WITH \(\pm\) 1.96 STANDARD DEVIATIONS}
 \vspace{0.2cm}\centering
 \includegraphics[width=\textwidth]{bias2plotsd-6.png}\\
 \begin{footnotesize}
  SOURCE: The author (2021).
 \end{footnotesize}
 \label{fig:biassdw2}
\end{figure}

\begin{figure}[H]
 \setlength{\abovecaptionskip}{.0001pt}
 \caption{PARAMETER \(\log(\sigma_{1}^{2})\) BIAS WITH \(\pm\) 1.96
          STANDARD DEVIATIONS}
 \vspace{0.2cm}\centering
 \includegraphics[width=\textwidth]{bias2plotsd-7.png}\\
 \begin{footnotesize}
  SOURCE: The author (2021).
 \end{footnotesize}
 \label{fig:biassdlogs2_1}
\end{figure}

\begin{figure}[H]
 \setlength{\abovecaptionskip}{.0001pt}
 \caption{PARAMETER \(\log(\sigma_{2}^{2})\) BIAS WITH \(\pm\) 1.96
          STANDARD DEVIATIONS}
 \vspace{0.2cm}\centering
 \includegraphics[width=\textwidth]{bias2plotsd-8.png}\\
 \begin{footnotesize}
  SOURCE: The author (2021).
 \end{footnotesize}
 \label{fig:biassdlogs2_2}
\end{figure}

\begin{figure}[H]
 \setlength{\abovecaptionskip}{.0001pt}
 \caption{PARAMETER \(\log(\sigma_{3}^{2})\) BIAS WITH \(\pm\) 1.96
          STANDARD DEVIATIONS}
 \vspace{0.2cm}\centering
 \includegraphics[width=\textwidth]{bias2plotsd-9.png}\\
 \begin{footnotesize}
  SOURCE: The author (2021).
 \end{footnotesize}
 \label{fig:biassdlogs2_3}
\end{figure}

\begin{figure}[H]
 \setlength{\abovecaptionskip}{.0001pt}
 \caption{PARAMETER \(\log(\sigma_{4}^{2})\) BIAS WITH \(\pm\) 1.96
          STANDARD DEVIATIONS}
 \vspace{0.2cm}\centering
 \includegraphics[width=\textwidth]{bias2plotsd-10.png}\\
 \begin{footnotesize}
  SOURCE: The author (2021).
 \end{footnotesize}
 \label{fig:biassdlogs2_4}
\end{figure}

\begin{figure}[H]
 \setlength{\abovecaptionskip}{.0001pt}
 \caption{PARAMETER \(z(\rho_{12})\) BIAS WITH \(\pm\) 1.96 STANDARD
          DEVIATIONS}
 \vspace{0.2cm}\centering
 \includegraphics[width=\textwidth]{bias2plotsd-11.png}\\
 \begin{footnotesize}
  SOURCE: The author (2021).
 \end{footnotesize}
 \label{fig:biassdrhoz12}
\end{figure}

\begin{figure}[H]
 \setlength{\abovecaptionskip}{.0001pt}
 \caption{PARAMETER \(z(\rho_{34})\) BIAS WITH \(\pm\) 1.96 STANDARD
          DEVIATIONS}
 \vspace{0.2cm}\centering
 \includegraphics[width=\textwidth]{bias2plotsd-12.png}\\
 \begin{footnotesize}
  SOURCE: The author (2021).
 \end{footnotesize}
 \label{fig:biassdrhoz34}
\end{figure}

\begin{figure}[H]
 \setlength{\abovecaptionskip}{.0001pt}
 \caption{PARAMETERS
          \(\{z(\rho_{13}),~z(\rho_{24}),~z(\rho_{14}),~z(\rho_{23})\}\)
          BIAS WITH \(\pm\) 1.96 STANDARD DEVIATIONS}
 \vspace{0.2cm}\centering
 \includegraphics[width=\textwidth]{bias2plotsd-13.png}\\
 \begin{footnotesize}
  SOURCE: The author (2021).
 \end{footnotesize}
 \label{fig:biassdrhoz4}
\end{figure}

With the log-variances presented in \autoref{fig:biassdlogs2_1},
\autoref{fig:biassdlogs2_2}, \autoref{fig:biassdlogs2_3}, and
\autoref{fig:biassdlogs2_4}, we have instead a similar behavior through
the models. For all the models, the high CIF scenarios are the ones with
a smaller mean and bias-variances. From the risk/time model to the
block-diag model, we do not see a significant improvement in terms of
bias reduction. Such improvement, however, is clear when we look at the
complete model. Again, the magick of considering the cross-correlations.

The same said about the log-variances, can be applied to the risk
correlations in \autoref{fig:biassdrhoz12}, with one addendum: the bias
reduction is even bigger. With the time correlation in
\autoref{fig:biassdrhoz34}, at least with clusters of size 2 and 5,
we get the same behavior observed with the fixed-effect parameters i.e.,
with the simpler models, the smaller biases are observed in the low CIF
scenarios. However, with the complete model, we get the opposite. With
the cross-correlations in
\autoref{fig:biassdrhoz4}, the mean and bias-variances are much smaller
in the high CIF scenarios.

The biggest bias-variances are obtained in the log-variances. A final
remark to be made is about convergences. With the simpler models, not
all of them work, having in some scenarios (generally the ones with 60
thousand data points) a 50\(\sim\)60\% convergence rate. With the
complete model, basically, almost all fits reach convergence
(\(\sim\)95\% performance).

After looking at the parameter estimates biases, let us take a look at
the implied mean-CIF curves. To nicely accommodate all seventy-two
scenarios we split the curves by level-CIF. In \autoref{fig:cifshigh} we
have the high CIF scenario curves and in \autoref{fig:cifslow} the low
CIF scenario curves. Since for all the models we have a latent structure
for the within-cluster dependency, the inherent idea is that this also
affect the fixed-effect parameter estimates. By taking its average in
each of the seventy-two scenarios, we are able to construct the mean CIF
curves.

In \autoref{fig:cifshigh} we have all the thirty-six curves obtained in
the high CIF scenarios. It is clear that with the complete model we get
a perfect fit in all nine scenarios. The risk and time models estimate
well the curve shape parameters but they fail to learn the max
incidence. A compensation between curves is clear.

\begin{figure}[H]
 \setlength{\abovecaptionskip}{.0001pt}
 \caption{HIGH CUMULATIVE INCIDENCE FUNCTION (CIF) SCENARIO CURVES}
 \vspace{0.2cm}\centering
 \includegraphics[width=\textwidth]{cifs-1.png}\\
 \begin{footnotesize}
  SOURCE: The author (2021).
 \end{footnotesize}
 \label{fig:cifshigh}
\end{figure}

\begin{figure}[H]
 \setlength{\abovecaptionskip}{.0001pt}
 \caption{LOW CUMULATIVE INCIDENCE FUNCTION (CIF) SCENARIO CURVES}
 \vspace{0.2cm}\centering
 \includegraphics[width=\textwidth]{cifs-2.png}\\
 \begin{footnotesize}
  SOURCE: The author (2021).
 \end{footnotesize}
 \label{fig:cifslow}
\end{figure}

Still in \autoref{fig:cifshigh}, in the risk model, there is a super
estimation of \(\beta_{1}\) in all scenarios. For failure cause 2, there
is a sub estimation. With the time model, we observe the opposite
compensation but on a smaller scale. With the time model, we get much
better curves than with the risk model. The block-diag model results are
a middle term between them. For the time model, the scenario with
cluster size 10 and 60 thousand data points is a highlight. For the
block-diag model, the highlight is the scenario with cluster size 5 and
30 thousand data points.

In the low CIF scenarios in \autoref{fig:cifslow}, the estimation is
clearly more difficult. The overall fits are bad, being impossible to
select a scenario with overall good results. For one of the failure
causes, the estimation quality is not so bad. The problem is when we
look to the other. An interesting scenario is the one with cluster size
2 and 60 thousand data points. In this scenario we see the worst fits
for failure cause 1, with a negative highlight in the block-diag
configuration. However, with this same model, for failure cause 2, it is
the scenario were we better learn the true curve. An interesting
compensation phenomena. The best joint fit is still with the complete
model.

Now we look at how the latent-effect parameter estimates distribute
themselves. Given the huge number of scenarios and the fact that is
harder to estimate covariance parameters, we chose to plot the parameter
estimates just in the scenarios with better performances. By the metrics
of small bias and CIF shape learning, the scenarios with better results
are the ones with high CIF and bigger sample sizes. We have the
densities for the variance parameter estimates, in each of these
scenarios, presented
in \autoref{fig:histologs2}. In \autoref{fig:historhoz} we have the same
for the correlation parameter estimates.

An interesting result is the clear difference between risk and time
models' covariance parameter estimates. With the risk model, we have an
evident super estimation and bigger variances. With the time model we
get much better results, but still with high variances. The block-diag
model generally performs better than the risk model and worst than the
time model, showing again to be a compromise between them. Besides the
bias itself, we should also pay attention to the values. We model the
variances in the log-scale, so a value 5, in reality, implies a variance
of \(\exp(5) = 148\). Terrible. This kind of problem do not sound to
appear with the complete model.

All correlations are quite well estimated, in all three scenarios, with
the complete model. Not only the correlations but the variances
also. The lack of any considerable difference between the covariance
densities, indicates no quality divergences in the results for different
cluster sizes. The densities in \autoref{fig:histologs2} and
\autoref{fig:historhoz} are the final corroboration indicating the good
performance of the maximum likelihood method in the complete
model.

Between the four tested models, the complete model was the one with the
smallest biases, better CIF shape learning, and precisest covariance
parameter estimates. In \autoref{fig:cor2plot} we have a heat-map of the
correlations between parameter estimates for the complete model in the
scenario with clusters of size 10, high CIF, and 60 thousand data
points.

We have a little bit of everything in the parameter estimates
correlations' heat-map. Some correlations are very close to zero, but we
also have strong positive and negative correlations. We can mention some
curiosities, but nothing pathological appears to happen, at least
nothing clear.

\begin{figure}[H]
 \setlength{\abovecaptionskip}{.0001pt}
 \caption{VARIANCE PARAMETERS DENSITIES IN THE SCENARIOS OF HIGH CIF AND
          60 THOUSAND DATA POINTS}
 \vspace{0.2cm}\centering
 \includegraphics[width=\textwidth]{histologs2-1.png}\\
 \begin{footnotesize}
  SOURCE: The author (2021).
 \end{footnotesize}
 \label{fig:histologs2}
\end{figure}

\begin{figure}[H]
 \setlength{\abovecaptionskip}{.0001pt}
 \caption{CORRELATION PARAMETERS DENSITIES IN THE SCENARIOS OF HIGH CIF
          AND 60 THOUSAND DATA POINTS}
 \vspace{0.2cm}\centering
 \includegraphics[width=\textwidth]{historhoz-1.png}\\
 \begin{footnotesize}
  SOURCE: The author (2021).
 \end{footnotesize}
 \label{fig:historhoz}
\end{figure}

In \autoref{fig:cor2plot}, all fixed-effect parameters are positive
correlated, with an emphasis on the correlation between \(\beta_{1}\)
and \(\beta_{2}\), and the one of the \(\bm{\beta}\)s with
the \(\bm{w}\)s. Another interesting observation is the strong negative
correlation between the \(\bm{\beta}\)s and the risk level
log-variances, and also the (less strong) positive correlation between
the \(\bm{\beta}\)s and the failure time trajectory level
log-variances. The risk level log-variances are (strongly) positively
correlated. So do the failure time trajectory level ones, but again, not
so strong as in the risk level. The correlations between the
log-variances of different levels are negative.

\begin{figure}[H]
 \setlength{\abovecaptionskip}{.0001pt}
 \caption{COMPLETE MODEL'S PARAMETERS CORRELATION HEAT-MAP IN THE
          SCENARIO OF CLUSTER SIZE 10, HIGH CIF, AND SIXTY-THOUSAND DATA
          POINTS}
 \centering
 \includegraphics[width=\textwidth]{cor2plot-1.png}\\
 \vspace{-0.2cm}
 \begin{footnotesize}
  SOURCE: The author (2021).
 \end{footnotesize}
 \label{fig:cor2plot}
\end{figure}

% END ==================================================================

% ----------------------------------------------------------------------
\chapter{Final considerations}
\label{cap:finalc}
The general goal of this master thesis was the proposition and study of
a maximum likelihood estimation approach for the analysis of clustered
competing risks data. Focused on the probability scale, by means of the
cumulative incidence function (CIF), instead of the hazard scale usual
in the survival modeling literature \cite{kalb&prentice}. We model the
clustered competing risks on a latent-effects framework, a generalized
linear mixed model (GLMM) \cite{GLMM}, with a multinomial distribution
for the competing risks and censorship, conditioned on the
latent-effects. The within-cluster latent dependency is accommodated by
a multivariate Gaussian distribution and is modeled via its covariance
matrix parameters.

The failures by the competing causes and their respective censorships
are modeled in the probability scale, by means of the CIF
\cite{kalb&prentice, andersen12}. The CIF is accommodated in our GLMM
framework in terms of the link function \cite{GLM89}, as the product of
two functions, one responsible to model the instantaneous risk and the
other the failure time trajectory, both in a cluster-specific
fashion. The shape of these functions is described in detail
in \autoref{cap:model}. This particular GLMM formulation is what makes
our model, particular. Thus, we have what we call a multiGLMM: a
multinomial GLMM for clustered competing risks data.

The two-function product CIF formulation was taken from
\citeonline{SCHEIKE} but there they use a different framework, a
composite likelihood framework \cite{lindsay88, cox&reid04, varin11}.
Here we do a full likelihood analysis instead. A composite approach is
generally used when a full likelihood approach is impossible or
computationally impracticable. Our goal here was to assess a full
likelihood framework taking advantage of state-of-the-art computational
libraries and very efficient algorithm implementations. We have all this
with the \texttt{R} \cite{R21} package TMB \cite{TMB}.

The applications in focus here were family studies. This kind of study
is characterized by involving big samples, generally, populations. Also,
generally having a high number of small clusters, families. A maximum
likelihood approach with the use of efficiently implemented Laplace
approximations \cite{tierney,patrao} together with an automatic
differentiation (AD) \cite{corestats,nocedal&wright} routine, all via
TMB, is able to handle a considerably high number of clusters,
independent of its size. The multinomial distribution assumption, on its
own, is an excellent probabilistic choice since it can accommodate
virtually any number of competing causes of failure and its
censorship. The presence of those two characteristics in our multiGLMM
makes it an efficient and scalable modeling framework for clustered
competing risks data.

Even with our modeling framework being virtually able to handle any
number of competing causes of failure, we restrained ourselves to work
here with only two of them. With two competing causes, we have
a \(4\times4\) covariance matrix for the latent effects, which implies
ten covariance parameters, which is already a lot of parameters to be
estimated in a latent structure. Since our goal was to study the
viability of the maximum likelihood estimation method, we kept it with
two causes.

All models from the simulation study were run, in a parallelized
fashion, in one of the two following Linux systems:
\begin{description}
 \item[System 1]
  12 Intel (R) Core (TM) i7-8750H CPU @ 2.20GHz processors
  with 16GB RAM;
 \item[System 2]
  30 Intel (R) Xeon (R) CPU E5-2690 v2 @ 3.00GHz processors
  and 206GB RAM.
\end{description}

Each risk and time model run is not so time-consuming, generally never
taking more than 5 minutes. The inherent idea is that for each cluster
we are always performing two-dimension integral approximations and we
have \textit{just} three covariance parameters. With the block-diag
model, we are theoretically in four dimensions. However, since the
covariance matrix is, block-diagonal, we experienced several numerical
instability problems. The solution, as can be seen in the
\autoref{cap:blockdiagModel} (\autoref{cap:appendixD}) code, was to
split it into two two-dimension matrices, since the \(4\times4\)
covariance matrix is block-diagonal.  This simple solution solved all
numerical instability problems. The computational time was only a little
bit bigger than with the risk and time models.

Finally, the complete model. In the biggest scenario, with 60 thousand
data points and clusters of size 2 i.e., with 30 thousand four-dimension
integral approximations (ten parameters in the covariance matrix), the
model fitting takes 30 minutes, in parallel, with TMB. Before doing the
TMB implementation, to really understand what we were doing, we did a
complete \texttt{R} implementation. We wrote the marginal log-likelihood
in \texttt{R}, based on our own Laplace approximation \cite{patrao} and
Newton-Raphson implementation (the gradients, \autoref{cap:appendixA},
and Hessian, \autoref{cap:appendixB}, were computed by hand and
implemented). Running this complete \texttt{R} implementation in a
scenario with 20 thousand data points and clusters of size 2, took
around 30 hours, parallelizing it between all threads of system 1. In
summary, by using TMB we were able to increase the model size 3 times
and to decrease the computational time 60 times. An incredible
performance gain.

Still, with the complete model, we performed a Bayesian analysis via
\texttt{tmbstan} \cite{tmbstan}. \texttt{tmbstan} enables MCMC sampling
\cite{MCMC, Diaconis} from a TMB model object using Stan \cite{Stan,
RStan}. Sampling can be performed with or without a Laplace
approximation for the random effects. We performed just one Bayesian
model fitting in a modest scenario with 5 thousand data points and
clusters of size 2. It took around 1 whole week of parallelized
processing in system 1. The results were basically the same as the ones
obtained with TMB but this high computational time just reinforces the
MCMC framework limitation.

An important point to be made here is about TMB's memory consumption. As
the sample size increases, the dimension of the model matrices also
increases. This, summed to a high number of clusters (Laplace
approximations to be performed), turns out to be a computational
nightmare. For several models, even the 16GB RAM of system 1 was not
enough. The bottleneck appears to be in the AD tape, which is made in
parallel, by default, if the model fitting is in parallel. By turning
this option off (line 11 of \autoref{cap:rscript}
(\autoref{cap:appendixD}) code), we were able to save a lot of memory,
making several models practicable.

Model the CIF of clustered competing risks data is far from being
trivial or straightforward. The formulation in \autoref{eq:cif} implies
the desired curve behavior, \autoref{fig:datasimucif}. However, in
counterpart, its derivatives w.r.t. time, generates very small
probabilities for the failure competing causes, ending by concentrating
almost everything on censorship, \autoref{fig:datasimu}. For each
competing cause with poor data representativity, we have three curve
shape parameters to estimate, implying the necessity of having a lot of
data to then have enough information about the causes.

We proposed for our multiGLMM an ideally complete latent-effects
formulation i.e., correlated latent effects on both levels,
instantaneous risk and failure time trajectory. The main underlying idea
of the \autoref{cap:results} simulation study was to see in which
scenarios we would be able to learn all the involved mean and covariance
parameters. As part of that, simpler formulations were proposed i.e.,
latent-effects in only one level, or in both but without
cross-correlations. As result, we got that latent effects only in the
risk level did not work. The optimization appears to get lost as if
something is missing. Inserting latent effects only in the failure time
trajectory level returned better results, but still not satisfactorily
good. In most of the evaluated scenarios, the block-diagonal model
appeared to be in the middle of them, as a compromise. The best results
were obtained with the complete model i.e. when we consider the
cross-correlations between levels. In general, we still observe some
high variances between the parameter estimates, but given all the
problem characteristics mentioned earlier, sounds to be reasonable. On
average, the complete model works fine, mainly in the scenarios of high
CIF configuration, and also as expected, as the sample size
increases. We can also say that as the cluster size increases, the
estimates get better but we did not have very strong results supporting
that.

\section{FUTURE WORKS}
\label{cap:future}

As show in \autoref{cap:results} results, even with the complete model
specification, the parameter estimates present an excessive variance.
In terms of a traditional GLMM specification \cite{GLMM}, we do not have
a lot more to do. We are already using a smart quasi-Newton algorithm
\cite{PORTpaper}, the most efficient derivatives computation technique
(AD) \cite{peyre}, and an also efficient Laplace approximation routine
\cite{corestats, patrao}, via TMB \cite{TMB}. We could change the
Laplace approximation for an adaptative Gaussian quadrature
\cite{quadrature}, but we do not see any good reason to do that.

There are two possible paths here. We could instead of a conditional
modeling framework (GLMM/latent-effects model), employ a marginal
modeling framework. In this framework, instead of caring about the
specification of a probability distribution to the competing causes
conditioned on the latent effects, we just care about the specification
of a mean and a variance structure. This approach does not have a
likelihood function per se, but the estimation procedure tends to be
easier than with the GLMM one. A marginal modeling framework that can be
used here is the multivariate covariance generalized linear model
(McGLM) \cite{mcglm, rmcglm}. How to exactly model the CIF of clustered
competing risks data in this framework, is something to still be figured
out.

The other path is by the use of a different way of modeling the
dependence structure. Instead of a latent-effects approach, we could use
copulas \cite{copulas,semiparametricSCHEIKE,gcmr,factorcopulas}. How to
do that is something to still be figured out by us, in terms of which
kind (conditional or marginal) and version (Archimedean-, Gauss-,
Maltesian-, \(t\)-, hyperbolic-, zebra-, and elliptical-) of copula to
use, besides the estimability issue.

% END ==================================================================

% ----------------------------------------------------------------------
\setlength{\afterchapskip}{\baselineskip}
% ----------------------------------------------------------------------
\bibliography{references}
% ----------------------------------------------------------------------
\postextual
% ----------------------------------------------------------------------
\begin{apendicesenv}
\partapendices
\addcontentsline{toc}{chapter}{\hspace{2.105cm}APPENDIX}
\renewcommand{\ABNTEXchapterfontsize}{\ABNTEXsectionfont}

\chapter{ANALYTIC GRADIENT OF THE LATENT EFFECTS FOR THE JOINT
         LOG-LIKELIHOOD FUNCTION OF THE MULTINOMIAL GLMM FOR CLUSTERED
         COMPETING RISKS DATA}
\label{cap:appendixA}

The following gradient components are computed by cluster, to be used
e.g., in a Newton optimization. Subject \(i\) at cluster \(j\) and for
competing cause \(k\)

\begin{align*}
  &\frac{\partial}{\partial u_{kj}}
    \log L(\bm{\theta}\mid\bm{y}_{j}, \bm{r}_{j}) =\\
  &y_{kij}\frac{1 +
    \sum_{m \neq k}^{K-1}\exp\{\bm{x}_{mij}\bm{\beta}_{mj} + u_{mj}\}
    }{1 +
    \sum_{n = 1}^{K-1}\exp\{\bm{x}_{nij}\bm{\beta}_{nj} + u_{nj}\}} -
    \left(\sum_{m \neq k}^{K-1} y_{mij}\right)
    \frac{\exp\{\bm{x}_{kij}\bm{\beta}_{kj} + u_{kj}\}
    }{1 +
    \sum_{n = 1}^{K-1}\exp\{\bm{x}_{nij}\bm{\beta}_{nj} + u_{nj}\}}-\\
  &y_{Kij}\frac{1}{1 +
    \sum_{n = 1}^{K-1}\exp\{\bm{x}_{nij}\bm{\beta}_{nj} + u_{nj}\}
    }\Bigg(\\
  &\frac{\exp\{\bm{x}_{kij}\bm{\beta}_{kj} + u_{kj}\}
    \left(1 +
    \sum_{m \neq k}^{K-1}\exp\{\bm{x}_{mij}\bm{\beta}_{mj} + u_{mj}\}
    \right)}{
    1 + \sum_{n = 1}^{K-1}\exp\{\bm{x}_{nij}\bm{\beta}_{nj} + u_{nj}\}
    }\times\\
  &\frac{w_{k}\frac{\delta}{2\delta t - 2t^{2}}
    \phi[w_{k}\text{arctanh}\left(\frac{t-\delta/2}{\delta/2}\right)
    - \bm{x}_{kij}\bm{\gamma}_{kj} - \eta_{kj}
    ]}{1 - w_{n}\frac{\delta}{2\delta t - 2t^{2}}
    \phi[w_{n}\text{arctanh}\left(\frac{t-\delta/2}{\delta/2}\right)
    - \bm{x}_{nij}\bm{\gamma}_{nj} - \eta_{nj}]} -
    \frac{\exp\{\bm{x}_{kij}\bm{\beta}_{kj} + u_{kj}\}}{
    1 + \sum_{n = 1}^{K-1}\exp\{\bm{x}_{nij}\bm{\beta}_{nj} + u_{nj}\}}
    \times\\
  &\frac{
    \sum_{m \neq k}^{K-1}
    w_{m}\frac{\delta}{2\delta t - 2t^{2}}
    \phi[w_{m}\text{arctanh}\left(\frac{t-\delta/2}{\delta/2}\right)
    - \bm{x}_{mij}\bm{\gamma}_{mj} - \eta_{mj}]
    \exp\{\bm{x}_{mij}\bm{\beta}_{mj} + u_{mj}\}}{
    1 - w_{n}\frac{\delta}{2\delta t - 2t^{2}}
    \phi[w_{n}\text{arctanh}\left(\frac{t-\delta/2}{\delta/2}\right)
    - \bm{x}_{nij}\bm{\gamma}_{nj} - \eta_{nj}]}\Bigg) -\\
  &\bm{e_{k}^{\top}Qr_{j}},\\
  %% -------------------------------------------------------------------
  \\
  %% -------------------------------------------------------------------
  &\frac{\partial}{\partial \eta_{kj}}
  \log L(\bm{\theta}\mid\bm{y}_{j}, \bm{r}_{j}) =\\
  &y_{kij} (w_{k}\text{arctanh}\left(\frac{t-\delta/2}{\delta/2}\right)
    - \bm{x}_{kij}\bm{\gamma}_{kj} - \eta_{kj}) -\\
  &y_{Kij}\frac{\exp\{\bm{x}_{kij}\bm{\beta}_{kj} + u_{kj}\}
    }{1 + \sum_{n = 1}^{K-1}\exp\{\bm{x}_{nij}\bm{\beta}_{nj} + u_{nj}\}}
    \times\\
  &\frac{
    w_{k}\frac{\delta}{2\delta t - 2t^{2}}
    (w_{k}\text{arctanh}\left(\frac{t-\delta/2}{\delta/2}\right)
    - \bm{x}_{kij}\bm{\gamma}_{kj} - \eta_{kj})
    \phi[w_{k} \text{arctanh}\left(\frac{t-\delta/2}{\delta/2}\right)
    - \bm{x}_{kij}\bm{\gamma}_{kj} - \eta_{kj}
    ]}{1 -
    \sum_{n = 1}^{K-1}
    \frac{\exp\{\bm{x}_{nij}\bm{\beta}_{nj} + u_{nj}\}}{1 +
    \sum_{n = 1}^{K-1}\exp\{\bm{x}_{nij}\bm{\beta}_{nj} + u_{nj}\}}
    w_{n}\frac{\delta}{2\delta t - 2t^{2}}
    \phi[w_{n}\text{arctanh}\left(\frac{t-\delta/2}{\delta/2}\right)
    - \bm{x}_{nij}\bm{\gamma}_{nj} - \eta_{nj}]} -\\
  &\bm{e_{k}^{\top}Qr_{j}},
\end{align*}
with \(\bm{e_{k}^{\top}}\) begin a vector with \(1\) at the \(k\)-th
position and zero elsewhere.

\chapter{ANALYTIC HESSIAN OF THE LATENT EFFECTS FOR THE JOINT
         LOG-LIKELIHOOD FUNCTION OF THE MULTINOMIAL GLMM FOR CLUSTERED
         COMPETING RISKS DATA}
\label{cap:appendixB}

The following hessian components are computed by cluster, to be used
e.g., in a Newton optimization. Subject \(i\) at cluster \(j\) and for
competing cause \(k\)
\begin{align*}
  &\frac{\partial^{2}}{\partial u_{kj}^{2}}
    \log L(\bm{\theta}\mid\bm{y}_{j}, \bm{r}_{j}) =\\
  &-\frac{\left(\sum_{k = 1}^{K-1} y_{kij}\right)
    \exp\{\bm{x}_{kij}\bm{\beta}_{kj} + u_{kj}\}
    \left(1 +
    \sum_{m \neq k}^{K-1}\exp\{\bm{x}_{mij}\bm{\beta}_{mj} + u_{mj}\}
    \right)}{\left(1 +
    \sum_{n = 1}^{K-1}\exp\{\bm{x}_{nij}\bm{\beta}_{nj} + u_{nj}\}
    \right)^{2}} +\\
  &\frac{y_{Kij}
    \exp\{\bm{x}_{kij} \bm{\beta}_{kj} + u_{kj}\}}{
    1 + \sum_{n = 1}^{K-1}\exp\{\bm{x}_{nij} \bm{\beta}_{nj} + u_{nj}\}
    }\times\\
  &\frac{
    \sum_{m \neq k}^{K-1}w_{m}\frac{\delta}{2\delta t - 2t^{2}}
    \phi[w_{m}\text{arctanh}\left(\frac{t-\delta/2}{\delta/2}\right)
    - \bm{x}_{mij}\bm{\gamma}_{mj} - \eta_{mj}]
    \exp\{\bm{x}_{mij}\bm{\beta}_{mj} + u_{mj}\}}{1 +
    \sum_{n = 1}^{K-1}\exp\{\bm{x}_{nij}\bm{\beta}_{nj} + u_{nj}\}
    (1 - w_{n}\frac{\delta}{2\delta t - 2t^{2}}
    \phi[w_{n}\text{arctanh}\left(\frac{t-\delta/2}{\delta/2}\right)
    - \bm{x}_{nij}\bm{\gamma}_{nj} - \eta_{nj}])} -\\
  &\frac{
    y_{Kij}
    w_{k}\frac{\delta}{2\delta t - 2t^{2}}
    \phi[w_{k}\text{arctanh}\left(\frac{t-\delta/2}{\delta/2}\right)
    - \bm{x}_{kij}\bm{\gamma}_{kj} - \eta_{kj}] }{1 +
    \sum_{n = 1}^{K-1}\exp\{\bm{x}_{nij}\bm{\beta}_{nj} + u_{nj}\}
    }\times\\
  &\frac{\exp\{\bm{x}_{kij}\bm{\beta}_{kj} + u_{kj}\}
    \left(1 +
    \sum_{m \neq k}^{K-1}\exp\{\bm{x}_{mij}\bm{\beta}_{mj} + u_{mj}\}
    \right)}{1 +
    \sum_{n = 1}^{K-1}\exp\{\bm{x}_{nij}\bm{\beta}_{nj} + u_{nj}\}
    (1 - w_{n}\frac{\delta}{2\delta t - 2t^{2}}
    \phi[w_{n}\text{arctanh}\left(\frac{t-\delta/2}{\delta/2}\right)
    - \bm{x}_{nij}\bm{\gamma}_{nj} - \eta_{nj}])} -\\
  &\frac{y_{Kij}\exp\{\bm{x}_{kij}\bm{\beta}_{kj} + u_{kj}\}}{\left(1 +
    \sum_{n = 1}^{K-1}\exp\{\bm{x}_{nij}\bm{\beta}_{nj} + u_{nj}\}
    \right)^{2}}\Bigg(\\
  &\frac{\sum_{m \neq k}^{K-1}
    w_{m}\frac{\delta}{2\delta t - 2t^{2}}
    \phi[w_{m}\text{arctanh}\left(\frac{t-\delta/2}{\delta/2}\right)
    - \bm{x}_{mij}\bm{\gamma}_{mj} - \eta_{mj}]
    \exp\{\bm{x}_{mij}\bm{\beta}_{mj} + u_{mj}\}}{\left(1 +
    \sum_{n = 1}^{K-1}\exp\{\bm{x}_{nij}\bm{\beta}_{nj} + u_{nj}\}
    (1 - w_{n}\frac{\delta}{2\delta t - 2t^{2}}
    \phi[w_{n}\text{arctanh}\left(\frac{t-\delta/2}{\delta/2}\right)
    - \bm{x}_{nij}\bm{\gamma}_{nj} - \eta_{nj}])\right)^{2}}-\\
  &\frac{w_{k}\frac{\delta}{2\delta t - 2t^{2}}
    \phi[w_{k}\text{arctanh}\left(\frac{t-\delta/2}{\delta/2}\right)
    - \bm{x}_{kij}\bm{\gamma}_{kj} - \eta_{kj}]\left(1 +
    \sum_{m \neq k}^{K-1}\exp\{\bm{x}_{mij}\bm{\beta}_{mj} + u_{mj}\}
    \right)}{\left(1 +
    \sum_{n = 1}^{K-1}\exp\{\bm{x}_{nij}\bm{\beta}_{nj} + u_{nj}\}
    (1 - w_{n}\frac{\delta}{2\delta t - 2t^{2}}
    \phi[w_{n}\text{arctanh}\left(\frac{t-\delta/2}{\delta/2}\right)
    - \bm{x}_{nij}\bm{\gamma}_{nj} - \eta_{nj}])\right)^{2}}\Bigg)\\
  &\times\Bigg(\Big(1 +\\
  &\sum_{n = 1}^{K-1}\exp\{\bm{x}_{nij}\bm{\beta}_{nj} + u_{nj}\}
    (1 - w_{n}\frac{\delta}{2\delta t - 2t^{2}}
    \phi[w_{n}\text{arctanh}\left(\frac{t-\delta/2}{\delta/2}\right)
    - \bm{x}_{nij}\bm{\gamma}_{nj} - \eta_{nj}])\Big) +\\
  &\Big(1 +
    \sum_{n = 1}^{K-1}\exp\{\bm{x}_{nij} \bm{\beta}_{nj} + u_{nj}\}
    \Big)\times\\
  &(1 - w_{k}\frac{\delta}{2\delta t - 2t^{2}}
    \phi[w_{k}\text{arctanh}\left(\frac{t-\delta/2}{\delta/2}\right)
    - \bm{x}_{kij}\bm{\gamma}_{kj} - \eta_{kj}])\Bigg)
    - \bm{e_{k}^{\top}Q},
\end{align*}

\begin{align*}
  &\frac{\partial^{2}}{\partial \eta_{kj}^{2}}
    \log L(\bm{\theta}\mid\bm{y}_{j}, \bm{r}_{j}) =\\
  &- y_{kij} - y_{Kij}
    \frac{\exp\{\bm{x}_{kij} \bm{\beta}_{kj} + u_{kj}\}}{1 +
    \sum_{n = 1}^{K-1}\exp\{\bm{x}_{nij} \bm{\beta}_{nj} + u_{nj}\}}\Bigg(\\
  &w_{k}\frac{\delta}{2\delta t - 2t^{2}}
    \phi[w_{k}\text{arctanh}\left(\frac{t-\delta/2}{\delta/2}\right)
    - \bm{x}_{kij}\bm{\gamma}_{kj} - \eta_{kj}]\times\\
  &\frac{\left(
    w_{k} \text{arctanh}\left(\frac{t-\delta/2}{\delta/2}\right)
    - \bm{x}_{kij}\bm{\gamma}_{kj} - \eta_{kj}
    \right)^{2} - 1}{
    1 - \sum_{n = 1}^{K-1}
    \frac{\exp\{\bm{x}_{nij}\bm{\beta}_{nj} + u_{nj}\}}{1 +
    \sum_{n = 1}^{K-1}\exp\{\bm{x}_{nij} \bm{\beta}_{nj} + u_{nj}\}}
    w_{n}\frac{\delta}{2\delta t - 2t^{2}}
    \phi[w_{n}\text{arctanh}\left(\frac{t-\delta/2}{\delta/2}\right)
    - \bm{x}_{nij}\bm{\gamma}_{nj} - \eta_{nj}]} -\\
  &\frac{\left(
    w_{k}\frac{\delta}{2\delta t - 2t^{2}}
    (w_{k}\text{arctanh}\left(\frac{t-\delta/2}{\delta/2}\right)
    - \bm{x}_{kij}\bm{\gamma}_{kj} - \eta_{kj})
    \phi[w_{k}\text{arctanh}\left(\frac{t-\delta/2}{\delta/2}\right)
    - \bm{x}_{kij}\bm{\gamma}_{kj} - \eta_{kj}]\right)^{2}}{\left(1 -
    \sum_{n = 1}^{K-1}
    \frac{\exp\{\bm{x}_{nij} \bm{\beta}_{nj} + u_{nj}\}}{1 +
    \sum_{n = 1}^{K-1}\exp\{\bm{x}_{nij} \bm{\beta}_{nj} + u_{nj}\}}
    w_{n}\frac{\delta}{2\delta t - 2t^{2}}
    \phi[w_{n}\text{arctanh}\left(\frac{t-\delta/2}{\delta/2}\right)
    - \bm{x}_{nij}\bm{\gamma}_{nj} - \eta_{nj}]\right)^{2}}\\
  &\Bigg) - \bm{e_{k}^{\top}Q},
\end{align*}

\begin{align*}
  &\frac{\partial^{2}}{\partial u_{kj} u_{mj}}
    \log L(\bm{\theta}\mid\bm{y}_{j}, \bm{r}_{j}) =\\
  &\left(\sum_{k = 1}^{K-1} y_{kij}\right)
    \frac{
    \exp\{\bm{x}_{kij}\bm{\beta}_{kj} + u_{kj}\}
    \exp\{\bm{x}_{mij}\bm{\beta}_{mj} + u_{mj}\}}{
    \left(1 +
    \sum_{n = 1}^{K-1}\exp\{\bm{x}_{nij} \bm{\beta}_{nj} + u_{nj}\}
    \right)^{2}} +\\
  &\frac{
    y_{Kij}
    \exp\{\bm{x}_{kij}\bm{\beta}_{kj} + u_{kj}\}
    \exp\{\bm{x}_{mij} \bm{\beta}_{mj} + u_{mj}\}}{1 +
    \sum_{n = 1}^{K-1}\exp\{\bm{x}_{nij} \bm{\beta}_{nj} + u_{nj}\}}\Bigg(\\
  &\frac{
    w_{m}\frac{\delta}{2\delta t - 2t^{2}}
    \phi[w_{m}\text{arctanh}\left(\frac{t-\delta/2}{\delta/2}\right)
    - \bm{x}_{mij}\bm{\gamma}_{mj} - \eta_{mj}]}{1 +
    \sum_{n = 1}^{K-1}\exp\{\bm{x}_{nij} \bm{\beta}_{nj} + u_{nj}\}
    (1 - w_{n}\frac{\delta}{2\delta t - 2t^{2}}
    \phi[w_{n}\text{arctanh}\left(\frac{t-\delta/2}{\delta/2}\right)
    - \bm{x}_{nij}\bm{\gamma}_{nj} - \eta_{nj}])} -\\
  &\frac{
    w_{k}\frac{\delta}{2\delta t - 2t^{2}}
    \phi[w_{k}\text{arctanh}\left(\frac{t-\delta/2}{\delta/2}\right)
    - \bm{x}_{kij}\bm{\gamma}_{kj} - \eta_{kj}]}{1 +
    \sum_{n = 1}^{K-1}\exp\{\bm{x}_{nij}\bm{\beta}_{nj} + u_{nj}\}
    (1 - w_{n}\frac{\delta}{2\delta t - 2t^{2}}
    \phi[w_{n}\text{arctanh}\left(\frac{t-\delta/2}{\delta/2}\right)
    - \bm{x}_{nij}\bm{\gamma}_{nj} - \eta_{nj}])}\Bigg) -\\
  &\frac{y_{Kij}}{
    \left(1 +
    \sum_{n = 1}^{K-1}\exp\{\bm{x}_{nij}\bm{\beta}_{nj} + u_{nj}\}
    \right)^{2}}\Bigg(\exp\{\bm{x}_{kij}\bm{\beta}_{kj} + u_{kj}\}\Bigg(\\
  &\frac{
    \sum_{m \neq k}^{K-1}
    w_{m}\frac{\delta}{2\delta t - 2t^{2}}
    \phi[w_{m}\text{arctanh}\left(\frac{t-\delta/2}{\delta/2}\right)
    - \bm{x}_{mij}\bm{\gamma}_{mj} - \eta_{mj}]
    \exp\{\bm{x}_{mij} \bm{\beta}_{mj} + u_{mj}\}}{
    \left(1 + \sum_{n = 1}^{K-1}\exp\{\bm{x}_{nij}\bm{\beta}_{nj} + u_{nj}\}
    (1 - w_{n}\frac{\delta}{2\delta t - 2t^{2}}
    \phi[w_{n}\text{arctanh}\left(\frac{t-\delta/2}{\delta/2}\right)
    - \bm{x}_{nij}\bm{\gamma}_{nj} - \eta_{nj}])\right)^{2}} -\\
  &\frac{
    w_{k}\frac{\delta}{2\delta t - 2t^{2}}
    \phi[w_{k}\text{arctanh}\left(\frac{t-\delta/2}{\delta/2}\right)
    - \bm{x}_{kij}\bm{\gamma}_{kj} - \eta_{kj}]
    \left(1 +
    \sum_{m \neq k}^{K-1}\exp\{\bm{x}_{mij}\bm{\beta}_{mj} + u_{mj}\}
    \right)}{\left(1 +
    \sum_{n = 1}^{K-1}\exp\{\bm{x}_{nij} \bm{\beta}_{nj} + u_{nj}\}
    (1 - w_{n}\frac{\delta}{2\delta t - 2t^{2}}
    \phi[w_{n}\text{arctanh}\left(\frac{t-\delta/2}{\delta/2}\right)
    - \bm{x}_{nij}\bm{\gamma}_{nj} - \eta_{nj}])\right)^{2}}\Bigg)
\end{align*}
\begin{align*}
  &\Bigg)\times\Bigg(\exp\{\bm{x}_{mij}\bm{\beta}_{mj} + u_{mj}\}
    \Big(1 +\\
  &\sum_{n = 1}^{K-1}\exp\{\bm{x}_{nij}\bm{\beta}_{nj} + u_{nj}\}
    (1 - w_{n}\frac{\delta}{2\delta t - 2t^{2}}
    \phi[w_{n}\text{arctanh}\left(\frac{t-\delta/2}{\delta/2}\right)
    - \bm{x}_{nij}\bm{\gamma}_{nj} - \eta_{nj}])\Big) +\\
  &\exp\{\bm{x}_{mij}\bm{\beta}_{mj} + u_{mj}\}
    (1 - w_{m}\frac{\delta}{2\delta t - 2t^{2}}
    \phi[w_{m}\text{arctanh}\left(\frac{t-\delta/2}{\delta/2}\right)
    - \bm{x}_{mij}\bm{\gamma}_{mj} - \eta_{mj}])\Big(1 +\\
  &\sum_{n = 1}^{K-1}\exp\{\bm{x}_{nij}\bm{\beta}_{nj} + u_{nj}\}\Big)
    \Bigg) - \bm{e_{k}^{\top}Q},
\end{align*}

\begin{align*}
  &\frac{\partial^{2}}{\partial \eta_{kj} \eta_{mj}}
    \log L(\bm{\theta}\mid\bm{y}_{j}, \bm{r}_{j}) =\\
  &- y_{Kij}\frac{
    \exp\{\bm{x}_{kij}\bm{\beta}_{kj} + u_{kj}\}}{1 +
    \sum_{n = 1}^{K-1}\exp\{\bm{x}_{nij}\bm{\beta}_{nj} + u_{nj}\}}\times\\
  &\frac{w_{k}\frac{\delta}{2\delta t - 2t^{2}}
    (w_{k}\text{arctanh}\left(\frac{t-\delta/2}{\delta/2}\right)
    - \bm{x}_{kij}\bm{\gamma}_{kj} - \eta_{kj})
    \phi[w_{k}\text{arctanh}\left(\frac{t-\delta/2}{\delta/2}\right)
    - \bm{x}_{kij}\bm{\gamma}_{kj} - \eta_{kj}]}{\left(1 -
    \sum_{n = 1}^{K-1}\frac{\exp\{\bm{x}_{nij}\bm{\beta}_{nj} + u_{nj}\}
    }{1 +
    \sum_{n = 1}^{K-1}\exp\{\bm{x}_{nij}\bm{\beta}_{nj} + u_{nj}\}}
    w_{n}\frac{\delta}{2\delta t - 2t^{2}}
    \phi[w_{n}\text{arctanh}\left(\frac{t-\delta/2}{\delta/2}\right)
    - \bm{x}_{nij}\bm{\gamma}_{nj} - \eta_{nj}]\right)^{2}}\times\\
  &\frac{\exp\{\bm{x}_{mij}\bm{\beta}_{mj} + u_{mj}\}}{1 +
    \sum_{n = 1}^{K-1}\exp\{\bm{x}_{nij}\bm{\beta}_{nj} + u_{nj}\}}
    w_{m}\frac{\delta}{2\delta t - 2t^{2}}
    (w_{m}\text{arctanh}\left(\frac{t-\delta/2}{\delta/2}\right)
    - \bm{x}_{mij}\bm{\gamma}_{mj} - \eta_{mj})\times\\
  &\phi[w_{m}\text{arctanh}\left(\frac{t-\delta/2}{\delta/2}\right)
    - \bm{x}_{mij}\bm{\gamma}_{mj} - \eta_{mj}] - \bm{e_{k}^{\top}Q},
\end{align*}

\begin{align*}
  &\frac{\partial^{2}}{\partial \eta_{kj} u_{kj}}
    \log L(\bm{\theta}\mid\bm{y}_{j}, \bm{r}_{j}) =\\
  &y_{Kij}
    \frac{\exp\{\bm{x}_{kij}\bm{\beta}_{kj} + u_{kj}\}}{1 +
    \sum_{n = 1}^{K-1}\exp\{\bm{x}_{nij}\bm{\beta}_{nj} + u_{nj}\}}\times\\
  &\frac{
    w_{k}\frac{\delta}{2\delta t - 2t^{2}}
    (w_{k}\text{arctanh}\left(\frac{t-\delta/2}{\delta/2}\right)
    - \bm{x}_{kij}\bm{\gamma}_{kj} - \eta_{kj})
    \phi[w_{k}\text{arctanh}\left(\frac{t-\delta/2}{\delta/2}\right)
    - \bm{x}_{kij}\bm{\gamma}_{kj} - \eta_{kj})]}{
    \left(1 -
    \sum_{n = 1}^{K-1}\frac{\exp\{\bm{x}_{nij}\bm{\beta}_{nj} + u_{nj}\}}{
    1 +
    \sum_{n = 1}^{K-1}\exp\{\bm{x}_{nij}\bm{\beta}_{nj} + u_{nj}\}}
    w_{n}\frac{\delta}{2\delta t - 2t^{2}}
    \phi[w_{n}\text{arctanh}\left(\frac{t-\delta/2}{\delta/2}\right)
    - \bm{x}_{nij}\bm{\gamma}_{nj} - \eta_{nj}]\right)^{2}}\times\\
  &\Bigg(
    \sum_{n \neq k}^{K-1}
    \frac{
    \exp\{\bm{x}_{nij}\bm{\beta}_{nj} + u_{nj}\}
    \exp\{\bm{x}_{kij}\bm{\beta}_{kj} + u_{kj}\}}{
    \left(1 +
    \sum_{n = 1}^{K-1}\exp\{\bm{x}_{nij}\bm{\beta}_{nj} + u_{nj}\}
    \right)^{2}}\times\\
  &w_{n}\frac{\delta}{2\delta t - 2t^{2}}
    \phi[w_{n}\text{arctanh}\left(\frac{t-\delta/2}{\delta/2}\right)
    - \bm{x}_{nij}\bm{\gamma}_{nj} - \eta_{nj}] -\\
  &\frac{\exp\{\bm{x}_{kij}\bm{\beta}_{kj} + u_{kj}\}
    \left(
    \left(1 +
    \sum_{n = 1}^{K-1}\exp\{\bm{x}_{nij}\bm{\beta}_{nj} + u_{nj}\}
    \right) - \exp\{\bm{x}_{kij} \bm{\beta}_{kj} + u_{kj}\}
    \right)}{
    \left(1 +
    \sum_{n = 1}^{K-1}\exp\{\bm{x}_{nij}\bm{\beta}_{nj} + u_{nj}\}
    \right)^{2}}\times\\
  &w_{k}\frac{\delta}{2\delta t - 2t^{2}}
    \phi[w_{k}\text{arctanh}\left(\frac{t-\delta/2}{\delta/2}\right)
    - \bm{x}_{kij}\bm{\gamma}_{kj} - \eta_{kj}]\Bigg) -
\end{align*}
\begin{align*}
  &y_{Kij}
    \frac{
    \frac{\exp\{\bm{x}_{kij}\bm{\beta}_{kj} + u_{kj}\}
    \left(
    \left(1 +
    \sum_{n = 1}^{K-1}\exp\{\bm{x}_{nij}\bm{\beta}_{nj} + u_{nj}\}
    \right) - \exp\{\bm{x}_{kij} \bm{\beta}_{kj} + u_{kj}\}
    \right)}{
    \left(1 +
    \sum_{n = 1}^{K-1}\exp\{\bm{x}_{nij}\bm{\beta}_{nj} + u_{nj}\}
    \right)^{2}}}{1 -
    \sum_{n = 1}^{K-1}\frac{\exp\{\bm{x}_{nij}\bm{\beta}_{nj} + u_{nj}\}}{
    1 + \sum_{n = 1}^{K-1}\exp\{\bm{x}_{nij}\bm{\beta}_{nj} + u_{nj}\}}
    w_{n}\frac{\delta}{2\delta t - 2t^{2}}
    \phi[w_{n}\text{arctanh}\left(\frac{t-\delta/2}{\delta/2}\right)
    - \bm{x}_{nij}\bm{\gamma}_{nj} - \eta_{nj}]}\times\\
  &w_{k}\frac{\delta}{2\delta t - 2t^{2}}
    (w_{k}\text{arctanh}\left(\frac{t-\delta/2}{\delta/2}\right)
    - \bm{x}_{kij}\bm{\gamma}_{kj} - \eta_{kj})\times\\
  &\phi[w_{k}\text{arctanh}\left(\frac{t-\delta/2}{\delta/2}\right)
    - \bm{x}_{kij}\bm{\gamma}_{kj} - \eta_{kj}] - \bm{e_{k}^{\top}Q},
\end{align*}

\begin{align*}
  &\frac{\partial^{2}}{\partial \eta_{kj} u_{mj}}
    \log L(\bm{\theta}\mid\bm{y}_{j}, \bm{r}_{j}) =\\
  &y_{Kij}
    \frac{\exp\{\bm{x}_{kij}\bm{\beta}_{kj} + u_{kj}\}
    \exp\{\bm{x}_{mij}\bm{\beta}_{mj} + u_{mj}\}}{
    \left(1 +
    \sum_{n = 1}^{K-1}\exp\{\bm{x}_{nij}\bm{\beta}_{nj} + u_{nj}\}
    \right)^{2}}\times\\
  &\frac{
    w_{k}\frac{\delta}{2\delta t - 2t^{2}}
    (w_{k} \text{arctanh}\left(\frac{t-\delta/2}{\delta/2}\right)
    - \bm{x}_{kij}\bm{\gamma}_{kj} - \eta_{kj})
    \phi[w_{k}\text{arctanh}\left(\frac{t-\delta/2}{\delta/2}\right)
    - \bm{x}_{kij}\bm{\gamma}_{kj} - \eta_{kj})]}{1 - \sum_{n = 1}^{K-1}
    \frac{\exp\{\bm{x}_{nij}\bm{\beta}_{nj} + u_{nj}\}}{1 +
    \sum_{n = 1}^{K-1}\exp\{\bm{x}_{nij}\bm{\beta}_{nj} + u_{nj}\}}
    w_{n}\frac{\delta}{2\delta t - 2t^{2}}
    \phi[w_{n}\text{arctanh}\left(\frac{t-\delta/2}{\delta/2}\right)
    - \bm{x}_{nij}\bm{\gamma}_{nj} - \eta_{nj}]} +\\
  &y_{Kij}
    \frac{\exp\{\bm{x}_{kij}\bm{\beta}_{kj} + u_{kj}\}}{1 +
    \sum_{n = 1}^{K-1}\exp\{\bm{x}_{nij}\bm{\beta}_{nj} + u_{nj}\}}\times\\
  &\frac{
    w_{k}\frac{\delta}{2\delta t - 2t^{2}}
    (w_{k}\text{arctanh}\left(\frac{t-\delta/2}{\delta/2}\right)
    - \bm{x}_{kij}\bm{\gamma}_{kj} - \eta_{kj})
    \phi[w_{k}\text{arctanh}\left(\frac{t-\delta/2}{\delta/2}\right)
    - \bm{x}_{kij}\bm{\gamma}_{kj} - \eta_{kj})]}{
    \left(1 - \sum_{n = 1}^{K-1}
    \frac{\exp\{\bm{x}_{nij}\bm{\beta}_{nj} + u_{nj}\}}{1 +
    \sum_{n = 1}^{K-1}\exp\{\bm{x}_{nij}\bm{\beta}_{nj} + u_{nj}\}}
    w_{n}\frac{\delta}{2\delta t - 2t^{2}}
    \phi[w_{n}\text{arctanh}\left(\frac{t-\delta/2}{\delta/2}\right)
    - \bm{x}_{nij}\bm{\gamma}_{nj} - \eta_{nj}]\right)^{2}}\times\\
  &\Bigg(
    \sum_{n \neq m}^{K-1}\frac{
    \exp\{\bm{x}_{nij}\bm{\beta}_{nj} + u_{nj}\}
    \exp\{\bm{x}_{mij}\bm{\beta}_{mj} + u_{mj}\}}{
    \left(1 +
    \sum_{n = 1}^{K-1}\exp\{\bm{x}_{nij}\bm{\beta}_{nj} + u_{nj}\}
    \right)^{2}}\times\\
  &w_{n}\frac{\delta}{2\delta t - 2t^{2}}
    \phi[w_{n}\text{arctanh}\left(\frac{t-\delta/2}{\delta/2}\right)
    - \bm{x}_{nij}\bm{\gamma}_{nj} - \eta_{nj}] -\\
  &\frac{\exp\{\bm{x}_{mij}\bm{\beta}_{mj} + u_{mj}\}
    \left(
    \left(1 +
    \sum_{n = 1}^{K-1}\exp\{\bm{x}_{nij}\bm{\beta}_{nj} + u_{nj}\}
    \right) - \exp\{\bm{x}_{mij} \bm{\beta}_{mj} + u_{mj}\}
    \right)}{
    \left(1 + \sum_{n = 1}^{K-1}\exp\{\bm{x}_{nij}\bm{\beta}_{nj} + u_{nj}\}
    \right)^{2}}\times\\
  &w_{m}\frac{\delta}{2\delta t - 2t^{2}}
    \phi[w_{m}\text{arctanh}\left(\frac{t-\delta/2}{\delta/2}\right)
    - \bm{x}_{mij}\bm{\gamma}_{mj} - \eta_{mj}]\Bigg) - \bm{e_{k}^{\top}Q},
\end{align*}
with \(\bm{e_{k}^{\top}}\) begin a vector with \(1\) at the \(k\)-th
position and zero elsewhere.

\chapter{\texttt{R} CODE TO SIMULATE FROM A \(\text{multiGLMM}\) WITH
         TWO COMPETING CAUSES AND CLUSTERS OF SIZE TWO. FOR MORE
         INFORMATION CHECK SECTION \ref{cap:simu}}
\label{cap:appendixC}

\lstinputlisting[firstline=97,lastline=153]{datasets.Rmd}
\vspace{-0.5cm}
\begin{center}
 \begin{footnotesize}
  SOURCE: The author (2021).
 \end{footnotesize}
\end{center}

\chapter{\texttt{C++} CODES FOR THE TMB IMPLEMENTATION OF THE
         \(\text{multiGLMM}\) COMPLETE MODEL'S SPECIAL CASES}
\label{cap:appendixD}

\section{\texttt{C++} CODE FOR THE TMB IMPLEMENTATION OF A
         \(\text{multiGLMM}\) WITH A \(2\times2\) LATENT STRUCTURE ON
         THE RISK LEVEL}
\label{cap:riskModel}

\lstinputlisting[language=C++]{modules/riskModel.cpp}
\vspace{-0.5cm}
\begin{center}
 \begin{footnotesize}
  SOURCE: The author (2021).
 \end{footnotesize}
\end{center}

\section{\texttt{C++} CODE FOR THE TMB IMPLEMENTATION OF A
         \(\text{multiGLMM}\) WITH A \(2\times2\) LATENT STRUCTURE ON
         THE TRAJECTORY TIME LEVEL}
\label{cap:timeModel}

\lstinputlisting[language=C++]{modules/timeModel.cpp}
\vspace{-0.5cm}
\begin{center}
 \begin{footnotesize}
  SOURCE: The author (2021).
 \end{footnotesize}
\end{center}

\section{\texttt{C++} CODE FOR THE TMB IMPLEMENTATION OF A
         \(\text{multiGLMM}\) WITH A BLOCK-DIAG \(4\times4\) LATENT
         STRUCTURE}
\label{cap:blockdiagModel}

\lstinputlisting[language=C++]{modules/blockdiagModel.cpp}
\vspace{-0.5cm}
\begin{center}
 \begin{footnotesize}
  SOURCE: The author (2021).
 \end{footnotesize}
\end{center}

\section{\texttt{R} CODE SHOWING HOW TO LOAD AND FIT THE
         \(\text{multiGLMM}\) VERSIONS}
\label{cap:rscript}

\lstinputlisting{modules/runModel.R}
\vspace{-0.5cm}
\begin{center}
 \begin{footnotesize}
  SOURCE: The author (2021).
 \end{footnotesize}
\end{center}

\chapter{MODEL PARAMETERS BIAS WITH 2.5\% AND 97.5\% QUANTILES}
\label{cap:appendixE}
\vspace{-0.65cm}
\begin{figure}[H]
 \setlength{\abovecaptionskip}{.0001pt}
 \caption{PARAMETER \(\beta_{1}\) BIAS WITH 2.5\% AND 97.5\% QUANTILES}
 \vspace{0.2cm}\centering
 \includegraphics[width=\textwidth]{bias2plot-1.png}\\
 \begin{footnotesize}
  SOURCE: The author (2021).
 \end{footnotesize}
 \label{fig:biasbeta1}
\end{figure}

\begin{figure}[H]
 \setlength{\abovecaptionskip}{.0001pt}
 \caption{PARAMETER \(\beta_{2}\) BIAS WITH 2.5\% AND 97.5\% QUANTILES}
 \vspace{0.2cm}\centering
 \includegraphics[width=\textwidth]{bias2plot-2.png}\\
 \begin{footnotesize}
  SOURCE: The author (2021).
 \end{footnotesize}
 \label{fig:biasbeta2}
\end{figure}

\begin{figure}[H]
 \setlength{\abovecaptionskip}{.0001pt}
 \caption{PARAMETER \(\gamma_{1}\) BIAS WITH 2.5\% AND 97.5\% QUANTILES}
 \vspace{0.2cm}\centering
 \includegraphics[width=\textwidth]{bias2plot-3.png}\\
 \begin{footnotesize}
  SOURCE: The author (2021).
 \end{footnotesize}
 \label{fig:biasgama1}
\end{figure}

\begin{figure}[H]
 \setlength{\abovecaptionskip}{.0001pt}
 \caption{PARAMETER \(\gamma_{2}\) BIAS WITH 2.5\% AND 97.5\% QUANTILES}
 \vspace{0.2cm}\centering
 \includegraphics[width=\textwidth]{bias2plot-4.png}\\
 \begin{footnotesize}
  SOURCE: The author (2021).
 \end{footnotesize}
 \label{fig:biasgama2}
\end{figure}

\begin{figure}[H]
 \setlength{\abovecaptionskip}{.0001pt}
 \caption{PARAMETER \(w_{1}\) BIAS WITH 2.5\% AND 97.5\% QUANTILES}
 \vspace{0.2cm}\centering
 \includegraphics[width=\textwidth]{bias2plot-5.png}\\
 \begin{footnotesize}
  SOURCE: The author (2021).
 \end{footnotesize}
 \label{fig:biasw1}
\end{figure}

\begin{figure}[H]
 \setlength{\abovecaptionskip}{.0001pt}
 \caption{PARAMETER \(w_{2}\) BIAS WITH 2.5\% AND 97.5\% QUANTILES}
 \vspace{0.2cm}\centering
 \includegraphics[width=\textwidth]{bias2plot-6.png}\\
 \begin{footnotesize}
  SOURCE: The author (2021).
 \end{footnotesize}
 \label{fig:biasw2}
\end{figure}

\begin{figure}[H]
 \setlength{\abovecaptionskip}{.0001pt}
 \caption{PARAMETER \(\log(\sigma_{1}^{2})\) BIAS WITH 2.5\% AND 97.5\%
          QUANTILES}
 \vspace{0.2cm}\centering
 \includegraphics[width=\textwidth]{bias2plot-7.png}\\
 \begin{footnotesize}
  SOURCE: The author (2021).
 \end{footnotesize}
 \label{fig:biaslogs2_1}
\end{figure}

\begin{figure}[H]
 \setlength{\abovecaptionskip}{.0001pt}
 \caption{PARAMETER \(\log(\sigma_{2}^{2})\) BIAS WITH 2.5\% AND 97.5\%
          QUANTILES}
 \vspace{0.2cm}\centering
 \includegraphics[width=\textwidth]{bias2plot-8.png}\\
 \begin{footnotesize}
  SOURCE: The author (2021).
 \end{footnotesize}
 \label{fig:biaslogs2_2}
\end{figure}

\begin{figure}[H]
 \setlength{\abovecaptionskip}{.0001pt}
 \caption{PARAMETER \(\log(\sigma_{3}^{2})\) BIAS WITH 2.5\% AND 97.5\%
          QUANTILES}
 \vspace{0.2cm}\centering
 \includegraphics[width=\textwidth]{bias2plot-9.png}\\
 \begin{footnotesize}
  SOURCE: The author (2021).
 \end{footnotesize}
 \label{fig:biaslogs2_3}
\end{figure}

\begin{figure}[H]
 \setlength{\abovecaptionskip}{.0001pt}
 \caption{PARAMETER \(\log(\sigma_{4}^{2})\) BIAS WITH 2.5\% AND 97.5\%
          QUANTILES}
 \vspace{0.2cm}\centering
 \includegraphics[width=\textwidth]{bias2plot-10.png}\\
 \begin{footnotesize}
  SOURCE: The author (2021).
 \end{footnotesize}
 \label{fig:biaslogs2_4}
\end{figure}

\begin{figure}[H]
 \setlength{\abovecaptionskip}{.0001pt}
 \caption{PARAMETER \(z(\rho_{12})\) BIAS WITH 2.5\% AND 97.5\%
          QUANTILES}
 \vspace{0.2cm}\centering
 \includegraphics[width=\textwidth]{bias2plot-11.png}\\
 \begin{footnotesize}
  SOURCE: The author (2021).
 \end{footnotesize}
 \label{fig:biasrhoz12}
\end{figure}

\begin{figure}[H]
 \setlength{\abovecaptionskip}{.0001pt}
 \caption{PARAMETER \(z(\rho_{34})\) BIAS WITH 2.5\% AND 97.5\%
          QUANTILES}
 \vspace{0.2cm}\centering
 \includegraphics[width=\textwidth]{bias2plot-12.png}\\
 \begin{footnotesize}
  SOURCE: The author (2021).
 \end{footnotesize}
 \label{fig:biasrhoz34}
\end{figure}

\begin{figure}[H]
 \setlength{\abovecaptionskip}{.0001pt}
 \caption{PARAMETERS
          \(\{z(\rho_{13}),~z(\rho_{24}),~z(\rho_{14}),~z(\rho_{23})\}\)
          BIAS WITH 2.5\% AND 97.5\% QUANTILES}
 \vspace{0.2cm}\centering
 \includegraphics[width=\textwidth]{bias2plot-13.png}\\
 \begin{footnotesize}
  SOURCE: The author (2021).
 \end{footnotesize}
 \label{fig:biasrhoz4}
\end{figure}

\end{apendicesenv}
% ----------------------------------------------------------------------
% \begin{anexosenv}
% \partanexos
% \addcontentsline{toc}{chapter}{\hspace{2.105cm}ANNEX}
% \renewcommand{\ABNTEXchapterfontsize}{\ABNTEXsectionfont}
% \end{anexosenv}
%-----------------------------------------------------------------------
\phantompart
\printindex
%-----------------------------------------------------------------------
\end{document}
% END ==================================================================
